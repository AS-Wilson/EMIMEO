\documentclass[colorlinks,12pt,a4paper,normalphoto,withhyper,ragged2e]{altareport}


%%%%%%%%%%%%%%%%%%%%%%%%%%%%%%%%%%%%%%%%%%%
%%%%%%%%%% DEFAULT PACKAGES & SETTINGS %%%%%%%%%%
\usepackage[utf8]{inputenc}
\usepackage{setspace} %1.5 line spacing
\usepackage{notoccite} %% Citation numbering
\usepackage{lscape} %% Landscape table
\usepackage{caption} %% Adds a newline in the table caption

%% The paracol package lets you typeset columns of text in parallel
\usepackage{paracol}
\usepackage[none]{hyphenat}

%% Document and Theme Fonts
\usepackage[T1]{fontenc}
\usepackage{paratype}
\usepackage[defaultsans]{lato}
%\usepackage[sfdefault,light,condensed]{roboto}
%\usepackage[rm]{roboto}
%\usepackage[defaultsans]{lato}
%\usepackage{sourcesanspro}
%\usepackage[rm]{merriweather}

\setlength{\intextsep}{4pt} % Set defualt spacing around floats

%%%%%%%%%%%%%%%%%%%%%%%%%%%%%%%%%%%%%%%%%%%


%%%%%%%%%%%%%%%%%%%%%%%%%%%%%%%%%%%%%%%%%%%
%%%%%%%%%% THEMES %%%%%%%%%%

%% Standard theme options are below, leave blank for B&W / no colours (BoringDefault). Note the theme will be set to default if you enter a non-exsistant theme name.
\SetTheme{UNILIM}
%% UNILIM
%% PastelBlue
%% GreenAndGold
%% Purple
%% PastelRed
%% BoringDefault (Leave blank / enter anything not found above)

%%%%%%%%%%%%%%%%%%%%%%%%%%%%%%%%%%%%%%%%%%%






%%%%%%%%%%%%%%%%%%%%%%%%%%%%%%%%%%%%%%%%%%%
%%%%%%%%%% DOCUMENT SPECIFIC PACKAGES %%%%%%%%%%

% Reduce space around captions
% \captionsetup{aboveskip=5pt, belowskip=5pt}
%%%%%%%%%%%%%%%%%%%%%%%%%%%%%%%%%%%%%%%%%%%




%%%%%%%%%%%%%%%%%%%%%%%%%%%%%%%%%%%%%%%%%%%
%%%%%%%%%% USEFUL SETTINGS %%%%%%%%%%
%% Change some font sizes, this will override the defaults
\renewcommand{\ReportTitleFont}{\Huge\rmfamily\bfseries} %% Title Page - Main Title
\renewcommand{\ReportSubTitleFont}{\huge\bfseries} %% Title Page - Sub-Title
\renewcommand{\ReportSectionFont}{\LARGE\rmfamily\bfseries} %% Section Title
\renewcommand{\ReportSubSectionFont}{\large\bfseries} %% SubSection Title
\renewcommand{\FootNoteFont}{\footnotesize} %% Footnotes and Header/Footer

%% Change the bullets for itemize and rating marker
\renewcommand{\itemmarker}{{\small\textbullet}}
\renewcommand{\ratingmarker}{\faCircle}

%% Change the page layout
\geometry{left=1.5cm,right=1.5cm,top=3cm,bottom=3cm,columnsep=8mm}
\onehalfspace   % 1.5 line spacing

\definecolor{CommentGreen}{HTML}{228B22}
%%%%%%%%%%%%%%%%%%%%%%%%%%%%%%%%%%%%%%%%%%%




%%%%%%%%%%%%%%%%%%%%%%%%%%%%%%%%%%%%%%%%%%%
%\include{references.bib}

%%%%%%%%%% TITLE PAGE INFO %%%%%%%%%%
\ReportTitle{Français}
\SubTitle{Folklore et Mythes Irlandais}
\Author{Andrew Simon Wilson \& Alejandro Dominguez Costa}
\ReportDate{\today}
\FacultyOrLocation{EMIMEO Programme}
\ModCoord{Prof. Katherine Corbie}

%%%%%%%%%%%%%%%%%%%%%%%%%%%%%%%%%%%%%%%%%%%

\begin{document}

\MakeReportTitlePage


%%%%% CONTENTS %%%%%
\pagenumbering{roman} % Start roman numbering
\setcounter{page}{1}


%%%%%%%%%% YOUR NAME, PROFESSION, PORTRAIT, CONTACT INFO, SOCIAL MEDIA ETC. %%%%%%%%%%
\name{Andrew Simon Wilson, BEng}
\tagline{Post-graduate Master's Student, Erasmus Mundus JMD - EMIMEO Programme}

\personalinfo{
  \email{andrew.wilson@etu.unilim.fr}
  \linkedin{andrew-simon-wilson} 
  \github{AS-Wilson}
  \phone{+44 7930 560 383}
}

%% You can add multiple photos on the left or right
% \photoR{3cm}{Images/a-wilson-potrait.jpg}
% \photoL{3cm}{Yacht_High,Suitcase_High}


\section*{Author Details}
\makeauthordetails




%% Table of contents print level -1: part, 0: chapter, 1: section, 2:sub-section, 3:sub-sub-section, etc.
\setcounter{tocdepth}{2} 
\tableofcontents %% Prints a list of all sections based on the above command
%\listoffigures %% Prints a list of all figures in the report
%\listoftables %% Prints a list of all tables in the report

\newpage
\pagenumbering{arabic} % Start document numbering - roman numbering


%%%%%%%%%% DOCUMENT CONTENT BEGINS HERE %%%%%%%%%%

%%%%% INTRO %%%%%
\section{Introduction}
Le folklore et mythes d'Irlande est riche et pleine, il y a beaucoup d'histoires mais je voudrais presenter le plus célèbre. Chaque histoire essaye a enseigner une leçon ou fournit un divertissement. Vous avez deja entendu quelque (le main rouge d'Ulster et (le main rouge d'Ulster quatre fois plus)). \linebreak

Avant la fin de ma présentation vous allez savoir de ``Ces Noinden Ulad'' (\textit{la malédiction de Macha}), Fionn mac Cumhaill et Cú Culainn. \linebreak




\section{Ces Noinden Ulad - \textit{la malédiction de Macha}}
Cette histoire vient de le cycle Ulster (un groupe d'histoire dans le folklore d'Irlande qu'est très célèbre et le plus connu), c'est un histoire acienne et un peu fémeniste (vous allez a voir). \linebreak

Il y avait un homme appelé Crunden et il était agriculteur avec trois fille et sa femme était mort. Sans la femme et deja avoir besoin a provider pour les enfants, il va a le champs chaque jour a travailler et il retournée chaque jour a un maison trés sale. \linebreak 

Mais un jour il retournée a trouver Macha au coin du feu, la déesse belle de Ulster, elle a nettoyé la maison, s'occuper les enfants et cuisiné le dîner. Alors Macha racontait Crunden qu'elle est Macha et il va l'épouser, Crunden n'était pas un homme argumentatif donc il a dit ``d'accord'' alors ils se sont mariés et Macha tombe enceinte. \linebreak

La vie a continué comme ça pendant quelque temp et finalement Crunden a réalisé que elle était de un autre monde, elle pouvait courir comme un cheval, elle avait une grande force et elle semblait avoir d’autres grands pouvoirs. \linebreak

Un jours il y avait un grand fête, fait par le Chieftain, pour tous d'Ulster. Crunden était excité et voulu aller mais Macha n’irait pas et prévenu Crunden que si il a parlé de elle, il etait aller a causer des problèmes pour eux deux. Crunden a promis qu'il ne va dire rien de elle et il est allé à la fête. \linebreak






Crunden ate and drank, along with all the other people at the feast, but he remembered Macha’s warning, and when the other men began boasting about the beauty of their wives, he kept his mouth shut. When the other men started boasting about the cooking of their wives, Crunden bit his tongue. But when the king boasted that no creature in Ireland was faster than his new horses, Crunden could not keep quiet any longer and bragged aloud that his wife was so swift, she would beat the king’s horses in a race.

Stung by this, King Connor ordered his men to seize the boastful farmer. He demanded that Crunden send for his wife, and if she did not come to prove the truth of his statement, Crunden would pay for his lie with his life.

Men were sent to Crunden’s house, but when Macha opened the door, they could see that she was heavily pregnant. Nonetheless, they told her what her husband had said, and that if she did not make good his boast, he would pay for it with his life. Macha agreed to go with them, with a bad grace.

When she came before the king, Macha begged him to consider her condition, and postpone the race until after she had given birth and had time to recover. But the king had been brooding on the insult Crunden had given him, and he refused her plea. Then Macha turned to all the warriors of Ulster, the Craobh Rua, or Red Branch, assembled there, and asked them to intercede, to protect her. She reminded them that each one of them was born of a woman, and that it was not right for them to put her in this position. But none of them stepped forward for her, none would plead with the king. They had been drinking at the feast, they were eager to see this race, and see their king put the boastful farmer in his place.

Something about Macha must have given King Connor pause, because before the race, he had his charioteer strip back all the decorations on his chariot, all the cushions and cloths that made the ride easier, till the king’s chariot was barely a plank of wood with wheels, as light as it could possibly be. He then stripped off his armour and heavy cloak till he stood in his lightest linen tunic, and dismissed his sister Deichtre, who was his charioteer, and took the reins of the chariot himself. Macha waited.

The race was held on the grass outside of the king’s fort, where there were no stones or uneven ground to trip the horses or foul the wheels. All the men of Ulster gathered there to watch, as the king and Macha raced.

The king raced his matched horses, and they ran as swift as the wind, moving in perfect unison, pulling him so fast he felt he was flying. But if the king raced as fast as the wind, Macha ran faster. She outpaced the wind itself. Her feet seemed barely to touch the ground. But as she ran, the birth pains came on Macha, and she began to scream.

All the people watching felt suddenly that this was not the great sport and entertainment they had thought it was.

Screaming in agony, Macha ran the course, and crossed the finish line with her belly protruding in front of the noses of Connor’s horses. Then, having won the race, she collapsed onto the grass, and in a rush of blood, her twins were born, still and dead. She gathered them into her arms, and put a curse on all the warriors of Ulster.

For failing to use their strength to defend her in her time of need, Macha declared that their strength would become useless to them. Whenever they needed it most, their strength would desert them, and for nine days and nine nights, they would endure the pains of a woman in childbirth. This curse would last for nine generations: each fighting-man of Ulster, as soon as he was old enough to grow a beard, would come under the curse.

With that, Macha gathered her dead twins, leaped over the heads of those watching, and ran off, never to be seen again. And from that day forth, the fort of the King of Ulster was known as Emain Macha; the Twins of Macha.




\section{Fionn mac Cumhaill - \textit{Finn McCool}}




\section{Cú Culainn}





%\newpage
%\setstretch{1}  % Reduce bibliography line spacing
%\bibliographystyle{IEEETran}
%\bibliography{references.bib}
\end{document}
