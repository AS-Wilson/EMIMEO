\documentclass[colorlinks,11pt,a4paper,normalphoto,withhyper,ragged2e]{altareport}


%%%%%%%%%%%%%%%%%%%%%%%%%%%%%%%%%%%%%%%%%%%
%%%%%%%%%% DEFAULT PACKAGES & SETTINGS %%%%%%%%%%
\usepackage[utf8]{inputenc}
\usepackage{setspace} %1.5 line spacing
\usepackage{notoccite} %% Citation numbering
\usepackage{lscape} %% Landscape table
\usepackage{caption} %% Adds a newline in the table caption

%% The paracol package lets you typeset columns of text in parallel
\usepackage{paracol}
\usepackage[none]{hyphenat}

%% Document and Theme Fonts
\usepackage[T1]{fontenc}
\usepackage{paratype}
\usepackage[defaultsans]{lato}
%\usepackage[sfdefault,light,condensed]{roboto}
%\usepackage[rm]{roboto}
%\usepackage[defaultsans]{lato}
%\usepackage{sourcesanspro}
%\usepackage[rm]{merriweather}

\setlength{\intextsep}{4pt} % Set defualt spacing around floats

%%%%%%%%%%%%%%%%%%%%%%%%%%%%%%%%%%%%%%%%%%%


%%%%%%%%%%%%%%%%%%%%%%%%%%%%%%%%%%%%%%%%%%%
%%%%%%%%%% THEMES %%%%%%%%%%

%% Standard theme options are below, leave blank for B&W / no colours (BoringDefault). Note the theme will be set to default if you enter a non-exsistant theme name.
\SetTheme{UNILIM}
%% UNILIM
%% PastelBlue
%% GreenAndGold
%% Purple
%% PastelRed
%% BoringDefault (Leave blank / enter anything not found above)

%%%%%%%%%%%%%%%%%%%%%%%%%%%%%%%%%%%%%%%%%%%






%%%%%%%%%%%%%%%%%%%%%%%%%%%%%%%%%%%%%%%%%%%
%%%%%%%%%% DOCUMENT SPECIFIC PACKAGES %%%%%%%%%%

\usepackage{amssymb}
\usepackage{amsfonts}
\usepackage{mathtools}

\usepackage{pythontex} % Run python code in this latex doc

%%%%%%%%% Karnaugh Map Package & Settings %%%%%%%%%
\usepackage[export]{adjustbox}

\usetikzlibrary{matrix,calc}
\usepackage{karnaugh-map}

\colorlet{LightRed}{red!60!}
\colorlet{LightBlue}{blue!60!}
\colorlet{LightYellow}{yellow!60!}
\colorlet{LightGreen}{green!60!}
\colorlet{LightOrange}{orange!60!}

%%%%%%%%% MATLAB Language Settings %%%%%%%%%
\usepackage[numbered,framed]{matlab-prettifier} % To add code listings from matlab
\lstMakeShortInline[style=Matlab-editor]" %% This makes " an escape character to write in matlab editor font

%%%%%%%%% Arduino Language Settings %%%%%%%%%
\lstset{%
  language = Octave,
  backgroundcolor=\color{white},   
  basicstyle=\color{body}\footnotesize\ttfamily,       
  %breakatwhitespace=false,         
  breaklines=true,                                
  commentstyle=\color{CommentGreen},
  columns=fullflexible,
  escapeinside={\%*}{*)},
  extendedchars=true,
  frame=leftline,
  keepspaces=true,
  keywordstyle=\color{LightBlue},
  numbersep=5pt,
  numberstyle=\footnotesize\color{gray},
  rulecolor=\color{black},
  rulesepcolor=\color{black},
  showtabs=true,
  stringstyle=\color{LightBlue},
  tabsize=2,                       
  title=\lstname,
  emphstyle=\bfseries\color{LightOrange}%  style for emph={} 
} 

%%%%% language/example specific settings: %%%%%
\lstdefinestyle{Arduino}{%
    language = C++,
    keywords={void, int, boolean, char, unsigned, long, uint32_t, volatile, byte, uint8_t, HIGH, OUTPUT, LOW, INPUT},%                 define keywords
    morecomment=[l]{//},%             treat // as comments
    morecomment=[s]{/*}{*/},%         define /* ... */ comments
    emph={WiFi, Serial, IPAddress, Adafruit_NeoPixel, setFrequency, println, print, delay, digitalWrite, pinMode, available, digitalPinToInterrupt, attachInterrupt, detachInterrupt, analogRead, NEO_GRB, NEO_KHZ800, noInterrupts, interrupts, strstr, endPacket, beginPacket, remoteIP, remotePort, APClientMacAddress, Color}%        keywords to emphasize
}


%%%%% Settings for python pgf graphs %%%%%
\usepackage{pgfplots}
\usetikzlibrary{arrows.meta}

\pgfplotsset{compat=newest,
    width=6cm,
    height=3cm,
    scale only axis=true,
    max space between ticks=25pt,
    try min ticks=5,
    every axis/.style={
        axis y line=left,
        axis x line=bottom,
        axis line style={thick,->,>=latex, shorten >=-.4cm}
    },
    every axis plot/.append style={thick},
    tick style={black, thick}
}
\tikzset{
    semithick/.style={line width=0.8pt},
}

\usepgfplotslibrary{groupplots}
\usepgfplotslibrary{dateplot}


% Reduce space around captions
% \captionsetup{aboveskip=5pt, belowskip=5pt}
%%%%%%%%%%%%%%%%%%%%%%%%%%%%%%%%%%%%%%%%%%%




%%%%%%%%%%%%%%%%%%%%%%%%%%%%%%%%%%%%%%%%%%%
%%%%%%%%%% USEFUL SETTINGS %%%%%%%%%%
%% Change some font sizes, this will override the defaults
\renewcommand{\ReportTitleFont}{\Huge\rmfamily\bfseries} %% Title Page - Main Title
\renewcommand{\ReportSubTitleFont}{\huge\bfseries} %% Title Page - Sub-Title
\renewcommand{\ReportSectionFont}{\LARGE\rmfamily\bfseries} %% Section Title
\renewcommand{\ReportSubSectionFont}{\large\bfseries} %% SubSection Title
\renewcommand{\FootNoteFont}{\footnotesize} %% Footnotes and Header/Footer

%% Change the bullets for itemize and rating marker
\renewcommand{\itemmarker}{{\small\textbullet}}
\renewcommand{\ratingmarker}{\faCircle}

%% Change the page layout
\geometry{left=1.5cm,right=1.5cm,top=3cm,bottom=3cm,columnsep=8mm}
\onehalfspace   % 1.5 line spacing

\definecolor{CommentGreen}{HTML}{228B22}
%%%%%%%%%%%%%%%%%%%%%%%%%%%%%%%%%%%%%%%%%%%




%%%%%%%%%%%%%%%%%%%%%%%%%%%%%%%%%%%%%%%%%%%
\include{references.bib}

%%%%%%%%%% TITLE PAGE INFO %%%%%%%%%%
\ReportTitle{Fundamentals of Coherent Photonics}
\SubTitle{Lab Reports}
\Author{Andrew Simon Wilson \& Alejandro Dominguez Costa}
\ReportDate{\today}
\FacultyOrLocation{EMIMEO Programme}
\ModCoord{Dr. Philippe Di Bin \& Dr. Rafael Jamier}

%%%%%%%%%%%%%%%%%%%%%%%%%%%%%%%%%%%%%%%%%%%


\newcommand*\circled[1]{\tikz[baseline=(char.base)]{
            \node[shape=circle,draw,inner sep=0.5pt] (char) {#1};}}


\begin{document}

\MakeReportTitlePage


%%%%% CONTENTS %%%%%
\pagenumbering{roman} % Start roman numbering
\setcounter{page}{1}


%%%%%%%%%% YOUR NAME, PROFESSION, PORTRAIT, CONTACT INFO, SOCIAL MEDIA ETC. %%%%%%%%%%
\name{Andrew Simon Wilson, BEng}
\tagline{Post-graduate Master's Student, Erasmus Mundus JMD - EMIMEO Programme}

\personalinfo{
  \email{andrew.wilson@etu.unilim.fr}
  \linkedin{andrew-simon-wilson} 
  \github{AS-Wilson}
  \phone{+44 7930 560 383}
}

%% You can add multiple photos on the left or right
% \photoR{3cm}{Images/a-wilson-potrait.jpg}
% \photoL{3cm}{Yacht_High,Suitcase_High}


\section*{Author Details}
\makeauthordetails




\section*{Co-Author Details}
\name{Alejandro Dominguez Costa}
\tagline{Post-graduate Master's Student, Erasmus Mundus JMD - EMIMEO Programme}

\personalinfo{
  \email{aledomcos11@gmail.com}
  %\linkedin{LinkedIn Homepage} 
  %\github{GitHub Account Homepage}
  \phone{+34 622 61 19 50}
}
\makeauthordetails




%% Table of contents print level -1: part, 0: chapter, 1: section, 2:sub-section, 3:sub-sub-section, etc.
\setcounter{tocdepth}{2} 
\tableofcontents %% Prints a list of all sections based on the above command
%\listoffigures %% Prints a list of all figures in the report
%\listoftables %% Prints a list of all tables in the report




%%%%%%%%%% DOCUMENT CONTENT BEGINS HERE %%%%%%%%%%

%%%%% INTRO %%%%%
\section*{Introduction}
A lot of time has been spent developing the template used to make this {\LaTeX} document, I want others to benefit from this work so the source code for this template is available on GitHub \cite{JenningsWilson2021}.
\newpage
\pagenumbering{arabic} % Start document numbering - roman numbering








%\begin{pycode}
%import numpy as np
%
%n_clad = 1.453
%n_core_MMF = 1.456
%MMF_radius = 25e-6
%lamb_s = 800e-9
%lamb_l = 
%
%
%NA = np.round_((np.sqrt((n_core[i]**2) - (n_clad**2))), decimals=2)
%MMF_FRID = np.round_( (((n_core_MMF**2)-(n_clad**2)) / (2*(n_core_MMF**2))), decimals=4)
%MMF_NSF = np.round_( (((2 * np.pi * MMF_radius) / (lamb_s)) * (np.sqrt((n_core_MMF**2) - (n_clad**2)))), decimals=2)
%
%
%\end{pycode}








\section{Lab One - Optical Fibres Splices \& Losses}

%%%%% %%%%% %%%%% %%%%% %%%%% %%%%% %%%%% %%%%% %%%%% %%%%% %%%%% %%%%% %%%%% %%%%% %%%%% %%%%%
\subsection{Measurement One: Observation and Cleaning of the FC-PC Connectors}
Given that the core diameter is 125$\mu$m the core is estimated to be around 50$\mu$m in diameter (25$\mu$m in radius) meaning this is likely a multi-mode fibre. \linebreak

%Given Equations 1 \& 2 the weak guidance conditions are met ($\Delta =  (<0.01)$) so only LP modes will propagate and the fibre parameter can be calculated to be between x and y (given that the wavelength of visible light is in the range x and y) meaning that this should be a multi-mode fibre.d
%
%%\py{NA[2]}
%
%\setlength{\jot}{4ex}
%\begin{gather}
%	\Delta = \frac{n_{core}^2 - n_{clad}^2}{2 \cdot n_{core}^2}
%	\label{eqn:frac_ri_diff}\\
%	V = \frac{2 \pi a}{\lambda}\sqrt{n_{core}^2 - n_{clad}^2} 
%	\label{eqn:normalised}
%\end{gather}

After touching the end of the connector with ones finger, dirt and oil can be seen and the light is not propagating clearly. \linebreak

Given that the core diameter is 125$\mu$m the core is estimated to be around 5~10$\mu$m in diameter meaning this is likely a single-mode fibre. \linebreak



%%%%% %%%%% %%%%% %%%%% %%%%% %%%%% %%%%% %%%%% %%%%% %%%%% %%%%% %%%%% %%%%% %%%%% %%%%% %%%%%
\subsection{Measurement Two: Light Injection into an Optical Fibre}



%%%%% %%%%% %%%%% %%%%% %%%%% %%%%% %%%%% %%%%% %%%%% %%%%% %%%%% %%%%% %%%%% %%%%% %%%%% %%%%%
\subsection{Measurement Three: Study of the Operation of the Thorlabs Optical Power Metre}

\subsubsection{Calibration of the Power Metre}

\paragraph{Note the calibration wavelength on power meter, justify its value.}

\paragraph{Note the value of the power on meter, in dBm and mW.}

\paragraph{Change the calibration wavelength, and measure the power in dBm and mW for each.}

\paragraph{Return the calibration wavelength to 633nm and check power in dBm seems correct.}	


\subsubsection{Measurement of Power Variation or Losses Using the ``Power Difference $\Delta$'' Mode of the Power Metre}

\paragraph{Press $\Delta$ on the power metre, why is the power now ``0dB'', why have the units changed from ``dBm'' to ``dB''?}

\paragraph{Comment on the evolution of the power as the launching conditions of light into the optical fiber are modified slightly.} (by playing, for instance, with one of the 3-axes of the fiber holder).



%%%%% %%%%% %%%%% %%%%% %%%%% %%%%% %%%%% %%%%% %%%%% %%%%% %%%%% %%%%% %%%%% %%%%% %%%%% %%%%%
\subsection{Measurement Four: Realisation of a Splice between Two Multi-mode Fibres}

\paragraph{After splicing fibre no.2 and no.1, measure the optical power at the output of the second fibre, $P_3$ (dBm).}

\paragraph{Deduce the insertion loss, noted $A_1$(dB), of the fibre made by ``fiber splice + fiber no.2''.}



%%%%% %%%%% %%%%% %%%%% %%%%% %%%%% %%%%% %%%%% %%%%% %%%%% %%%%% %%%%% %%%%% %%%%% %%%%% %%%%%
\subsection{Measurement Five: Measurement of the Linear Propagation Loss as well as the Fibre Length Using an OTDR}

\paragraph{Measure, for the two working wavelengths of the OTDR (850nm \& 1300nm); the linear loss (dB/km) of each fiber drum, the length of each fiber, the splicing loss (dB).}



%%%%% %%%%% %%%%% %%%%% %%%%% %%%%% %%%%% %%%%% %%%%% %%%%% %%%%% %%%%% %%%%% %%%%% %%%%% %%%%%
\subsection{Measurement Six: Measurement of the Linear Propagation Loss Using the Cut-Back Technique}

\paragraph{Measure the power at the output of fibre No.2. If the value measured is not equal to $P_3$, re-optimize the launching conditions into Fibre No.1.}

\paragraph{After cutting the fibre 15cm downstream of the splice, measure the optical power, $P_2$ (dBm).}

\paragraph{Knowing the length of fibre No.2 (obtained with the OTDR), calculate for $\lambda=633nm$ the average linear propagation loss (dB/km), compare this calculated value with that obtained with the OTDR.}




%%%%% %%%%% %%%%% %%%%% %%%%% %%%%% %%%%% %%%%% %%%%% %%%%% %%%%% %%%%% %%%%% %%%%% %%%%% %%%%%
\subsection{Measurement Seven: Measurement of the Splicing Loss Using the Cut-Back Technique}

\paragraph{After cutting the fibre 1cm upstream of the splice, measure the optical power(dBm), ensure this power corresponds to $P_1$.}

\paragraph{Deduce the insertion loss of the splice, i.e. the splicing loss, expressed in dB. Compare the measured value with that of the OTDR.}




%%%%% %%%%% %%%%% %%%%% %%%%% %%%%% %%%%% %%%%% %%%%% %%%%% %%%%% %%%%% %%%%% %%%%% %%%%% %%%%%
\section{Lab Two - YAG:ND LASER}









\newpage
\setstretch{1}  % Reduce bibliography line spacing
\bibliographystyle{IEEETran}
\bibliography{references.bib}
\end{document}
