%%%%%%%%%%% PAM %%%%%%%%%%
%%%%%%%%%%% PAM %%%%%%%%%%
%%%%%%%%%%% PAM %%%%%%%%%%
%%%%%%%%%%% PAM %%%%%%%%%%
%%%%%%%%%%% PAM %%%%%%%%%%



Considering PAM base-band modulation, describe the Nyquist pulses and the role of the roll-off parameter.



Pulse Amplitude Modulation (PAM) is a baseband modulation technique where the amplitude of pulses is varied according to the message signal. The Nyquist pulse, in the context of PAM, refers to a pulse shape that satisfies the Nyquist criterion for zero intersymbol interference (ISI). The Nyquist criterion ensures that pulses can be transmitted without overlap, allowing for reliable data transmission.

In PAM, the Nyquist pulse is often represented by a sinc pulse, which is the sinc function:

p(t)=sinc(tT)p(t)=sinc(Tt​)

Here:

p(t)p(t) is the Nyquist pulse,
tt is time,
TT is the symbol period.

The role of the roll-off parameter (αα) comes into play when considering a modified version of the Nyquist pulse to account for practical limitations in communication systems. The roll-off parameter is associated with pulse shaping, and it influences the shape of the pulse and the spectral efficiency of the signal.

The modified pulse shape, often used in practice, is given by the raised cosine pulse:

p(t)=sin⁡(πt/T)πt/T⋅cos⁡(παt/T)1−(2αt/T)2p(t)=πt/Tsin(πt/T)​⋅1−(2αt/T)2cos(παt/T)​

Here:

αα is the roll-off factor,
TT is the symbol period.

The raised cosine pulse allows for some control over the bandwidth of the transmitted signal. The roll-off parameter αα determines how fast the pulse transitions from its peak to zero. A larger αα value results in a slower roll-off and a narrower bandwidth, while a smaller αα value leads to a faster roll-off and a wider bandwidth.

The choice of the roll-off parameter is a trade-off between spectral efficiency and susceptibility to intersymbol interference. A smaller roll-off (αα) results in a more bandwidth-efficient signal but may increase the likelihood of intersymbol interference. On the other hand, a larger roll-off provides better immunity to intersymbol interference but at the cost of increased bandwidth usage.

In summary, in PAM baseband modulation, the Nyquist pulse is often represented by a sinc pulse, and the roll-off parameter (αα) in the context of pulse shaping, particularly with raised cosine pulses, allows for control over the trade-off between spectral efficiency and intersymbol interference.



