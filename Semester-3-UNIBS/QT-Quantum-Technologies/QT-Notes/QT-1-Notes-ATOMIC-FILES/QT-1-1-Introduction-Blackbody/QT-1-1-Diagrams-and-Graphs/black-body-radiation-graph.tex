%% BLACKBODY RADIATION GRAPH WITH APPROXIMATIONS
%%%% https://tikz.net/blackbody/ %%%%
%%%% https://tikz.net/blackbody_plots/ %%%%
\documentclass{standalone}

%%%% https://tikz.net/blackbody/ %%%%
%%%% https://tikz.net/blackbody_plots/ %%%%


\IfStandalone{\def\datapath{../../../}}{\def\datapath{}}

\input{\datapath QT-Notes-Preambles/QT-Notes-Diagrams-and-Graphs-DEFAULT-Preamble.tex}


%%%%%%%%%% CUSTOM COLORS %%%%%%%%%%
% See https://tikz.net/blackbody_color/
\definecolor{1000K}{rgb}{1,0.0337,0}
\definecolor{2000K}{rgb}{1,0.2647,0.0033}
\definecolor{3000K}{rgb}{1,0.4870,0.1411}
\definecolor{4000K}{rgb}{1,0.6636,0.3583}
\definecolor{5000K}{rgb}{1,0.7992,0.6045}
\definecolor{6000K}{rgb}{1,0.9019,0.8473}
\definecolor{8000K}{rgb}{0.7874,0.8187,1}
\definecolor{10000K}{rgb}{0.6268,0.7039,1}
\pgfdeclareverticalshading{rainbow}{100bp}{
	color(0bp)=(red); color(25bp)=(red); color(35bp)=(yellow);
	color(45bp)=(green); color(55bp)=(cyan); color(65bp)=(blue);
	color(75bp)=(violet); color(100bp)=(violet)
}
\colorlet{myred}{red!70!black}
\colorlet{mydarkgreen}{green!55!black}


% PLANCK & RAYLEIGH-JEANS
% 2hc^2/lambda^5 = 2 * 6.62607015e-34 * 299792458^2
%                = 1.191042972e-16
%    W.m -> kW.nm: 1.191042972e26
%  hc/k lambda T = 6.62607015e-34*299792458/(1.38064852e-23)
%                = 0.01438777378
%         m -> nm: 0.01438777378e9
% 2ckT/lambda^4  = 2 * 299792458 * 1.38064852e-23
%                = 8.278160269e-15
%    W.m -> kW.nm: 8.278160269e18
\pgfmathdeclarefunction{planck}{2}{%
	\pgfmathparse{1.191042972e26/(#1^5)/(exp(0.01439e9/(#1*#2))-1)}%
}
\pgfmathdeclarefunction{rayleighjeans}{2}{%
	\pgfmathparse{8.278160269e18*#2/(#1^4)}%
}
\pgfmathdeclarefunction{wien}{2}{%
	\pgfmathparse{1.191042972e26/(#1^5)*exp(-0.01439e9/(#1*#2))}%
}
\pgfmathdeclarefunction{lampeak}{1}{% % Wien's displacement law
	\pgfmathparse{2.898e6/#1}%
}

% redraw axis on top
\makeatletter \newcommand{\pgfplotsdrawaxis}{\pgfplots@draw@axis} \makeatother
\pgfplotsset{axis line on top/.style={after end axis/.append code={\pgfplotsdrawaxis}}
}


%%%%%%%%%% TIKZ & TIKZ SETTINGS %%%%%%%%%%
\usetikzlibrary{arrows.meta}


\usetikzlibrary{decorations.pathmorphing,decorations.markings,calc} % for random steps & snake
\usetikzlibrary{arrows.meta} % for arrow size
\tikzset{>=latex} % for LaTeX arrow head
\tikzstyle{radiation}=[-{Latex[length=2,width=1.5]},red!95!black!50,opacity=0.7,very thin,decorate,decoration={snake,amplitude=0.7,segment length=2,post length=2}]

\begin{document}
	% BLACK BODY - 3000, 4000, 5000K
	\begin{tikzpicture}
		\message{^^JBlack body}
		\def\N{60}
		\def\xmax{3100}
		\def\ymax{1.32e10}
		\def\tick#1#2{\draw[thick] (#1+.01*\ymax) -- (#1-.01*\ymax) node[below=-.5pt,scale=0.75,angle=45] {#2};}
		\begin{axis}[
			every axis plot/.style={
				mark=none,samples=\N,domain=5:\xmax,smooth},
			xmin=(-.05*\xmax), xmax=(1.05*\xmax),
			ymin=(-.08*\ymax), ymax=(1.08*\ymax),
			restrict y to domain=0:\ymax,
			axis lines=middle,
			axis line style=thick,
			%enlargelimits=upper, % extend the axes a bit to the right and top
			tick style={black,thick},
			tick label style={scale=0.7},
			x tick label style={rotate=-45},
%			xtick style={draw=none},
%			xticklabels={rotate=45},
			xlabel={Wavelength $\lambda$ [nm]},
			ylabel={Power $P$ [kW/srm$^2$nm]},
			xlabel style={at={(rel axis cs:0.5,0)},below=4pt,font=\small},
			ylabel style={at={(rel axis cs:-0.1,0.5)},rotate=90},
			width=9cm, height=7cm,
			%clip=false
			tick scale binop=\times,
			every y tick scale label/.style={at={(rel axis cs:0,1)},anchor=south}]
			]
			
			% RAINBOW
			\shade[shading=rainbow,shading angle=90,opacity=0.5] (380,0) rectangle (740,\ymax);
			\node[above=-1pt,scale=0.8] at (200,\ymax) {\strut UV}; % 10 - 400 nm
			\node[above=-1pt,scale=0.8] at (570,\ymax) {\strut optical}; % 380 - 740 nm
			\node[above=-1pt,scale=0.8] at (920,\ymax) {\strut IR}; % 740 - 1050 nm
			
			% PLANCK
			\addplot[very thick,red]    {planck(x,3000)};
			\addplot[very thick,orange] {planck(x,4000)};
			\addplot[very thick,samples=3*\N,blue] {planck(x,5000)};
			%\addplot[dashed,thick,red,domain=1000:4000]    {rayleighjeans(x,3000)};
			%\addplot[dashed,thick,orange,domain=1000:4000] {rayleighjeans(x,4000)};
			\addplot[dashed,thick,blue,domain=1000:4000]   {rayleighjeans(x,5000)};
			%\addplot[dashed,thick,red,domain=1000:4000]    {wien(x,3000)};
			%\addplot[dashed,thick,orange,domain=1000:4000] {wien(x,4000)};
			%\addplot[dashed,thick,blue,domain=1000:4000]   {wien(x,5000)};
			
			%% MAXIMUM (Wien's displacement law)
			%\addplot[ScienceGreen,very thin,variable=T,domain=2500:6000]
			%  ({lampeak(T)},{planck(lampeak(T),T)});
			
			% LABELS
			\node[above=0pt,scale=0.75,red] at (1150,{planck(1150,3000)}) {\SI{3000}{K}};
			\node[above right=-1pt,scale=0.75,orange!80!black] at (740,{planck(740,4000)}) {\SI{4000}{K}};
			\node[above right=-1pt,scale=0.75,blue] at (800,{planck(800,5000)}) {\SI{5000}{K}};
			\node[above right=-1pt,scale=0.75,blue] at (1500,{rayleighjeans(1500,5000)}) {\SI{5000}{K} Rayleigh-Jeans};
			
			%% TICKS
			%\tick{500,0}{500}
			%\tick{1000,0}{1000}
			%\tick{1500,0}{1500}
			%\tick{2000,0}{2000}
			%\tick{2500,0}{2500}
			%\tick{3000,0}{3000}
			
		\end{axis}
	\end{tikzpicture}
\end{document}