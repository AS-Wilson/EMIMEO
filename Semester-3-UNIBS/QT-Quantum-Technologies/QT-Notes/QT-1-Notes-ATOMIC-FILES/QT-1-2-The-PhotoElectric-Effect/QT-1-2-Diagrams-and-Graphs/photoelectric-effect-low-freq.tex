%% PHOTOELECTRIC EFFECT LOW FREQUENCY
%%%% https://tikz.net/photoelectric_effect/ %%%%
\documentclass{standalone}


%%%% https://tikz.net/photoelectric_effect/ %%%%


\IfStandalone{\def\datapath{../../../}}{\def\datapath{}}

\input{\datapath QT-Notes-Preambles/QT-Notes-Diagrams-and-Graphs-DEFAULT-Preamble.tex}

% Circuits
\usepackage{circuitikz}
%% Specifications
\ctikzset{bipoles/thickness=1.2}

% Styles
\tikzset{>=latex}

% Tikz Library
\usetikzlibrary{angles,quotes}

% Define Color
\tikzstyle{bigphoton}=[-{Latex[length=8,width=6]},red!95!black!50,opacity=0.85,very thin,decorate,decoration={snake,amplitude=2.8,segment length=8,post length=8}]



%%%% https://tikz.net/function_average/ %%%%
\usepackage{physics}
\usepackage[outline]{contour} % glow around text
\contourlength{1.0pt}

\tikzset{>=latex} % for LaTeX arrow head
\colorlet{myred_}{red!85!black}
\colorlet{myblue_}{blue!80!black}
\colorlet{mydarkred_}{myred_!80!black}
\colorlet{mydarkblue_}{myblue_!60!black}
\tikzstyle{xline}=[myblue_,thick]
\def\tick#1#2{\draw[thick] (#1) ++ (#2:0.09) --++ (#2-180:0.18)}
\tikzstyle{myarr_}=[myblue_!50,-{Latex[length=3,width=2]}]
\def\N{100}


\begin{document}
	\begin{tikzpicture}[scale=1, line width=1]
		% Grid
		%				\draw[help lines] (0,0) grid (11,11);
		
		% Circuits		
		\draw (4,6) -- (2,6) to[rmeterwa, t=$A$] ++(0,-4) -- (2,2) to[controlled voltage source, l_={
			\begin{tabular}{c}
				\Large $V$ \\
				\footnotesize (Controlled Voltage Source)
		\end{tabular}}] ++(6,0) -- (8,6) -- (6,6);
		\filldraw (4,6) circle [radius=0.05];
		
		% Dashed Arrows
		\draw[color=ScienceGreen,opacity=0,dashed,<-] (4,6) -- +(1.7,1)coordinate(A);
		\draw[color=ScienceGreen,opacity=0,dashed,<-] (4,6)coordinate(B) -- +(1.7,-1)coordinate(C);
		
		% Arc		
		\pic [draw, angle radius = 2cm, line width = 5] {angle=C--B--A};
		
		% Photocell
		\draw[color=black, line width=5, opacity=0] (5,6) circle [radius=1.8];
		
		% Laser
		\draw[fill=yellow, line width=1, rotate=12] (2.8,9.31) -- ++(1,-1) -- ++(0.1,0.1) -- ++(-1,1) -- +(-0.1,-0.1);
		%% Rays
		\draw[bigphoton,ScienceRed!60] (2,9) -- (5.7,5);
		\draw[bigphoton,ScienceRed!60] (2,9) -- (5.7,6.95);
		
		
		% Nodes
		\node at (5.5,2.5) {\large $+$};
		\node at (4.5,2.5) {\large $-$};
		\node at (3.8,5.15) {
			\begin{tabular}{c}
				\large $C$ \\
				\scriptsize (Collector)
		\end{tabular}};
		\node at (7,6.5) {
			\begin{tabular}{l}
				\large $P$ \\
				\scriptsize (Photocathode)
		\end{tabular}};
		\node at (2.4,10) {
			\begin{tabular}{c}
				\large Light Source \\
				\small (Laser)
		\end{tabular}};
		\node at (2.15,7.2) {
			\begin{tabular}{c}
				\large Light \\
				\scriptsize (Individual Photons) \\
				\normalsize $ \nu_1 $, $\nu_1 < \nu_2$
		\end{tabular}};
	\end{tikzpicture}
\end{document}