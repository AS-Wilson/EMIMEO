%%%%%%%%%% QT-1-3-Bohrs-Model-of-Hydrogen-ATOMIC-FILE.tex %%%%%%%%%%
%%%%%%%%%% 	QT NOTES - BOHR'S MODEL OF HYDROGEN SUBFILE	   %%%%%%%%%%
\documentclass[../../Quantum-Technologies-Notes]{subfiles}

\ReportTitle{Section Three - Bohr's Model of Hydrogen}



\begin{document}
	
	\ifSubfilesClassLoaded{ \pagestyle{fancy} }
	
	
	\section{Bohr’s Model of the Hydrogen Atom}
		
		\subsection{Bohr’s Hypothesis - Angular Momentum is Quantised}
			In 1913 Niels Bohr took a visit to Ernest Rutherford, Hans Geiger, and Ernest Marsden's lab in Manchester as a result of this visit he offered an answer to some strange and, at the time, unexplained measurements they had made involving electrons. \linebreak
			Bohr hypothesized that the energies of the atom are quantised (along with a number of other properties) and that as a consequence this would result in a particular model or interpretation of the hydrogen atom. Any transition up to a higher energy level requires absorption of the appropriate frequency photon, and any down transition the emission of a photon of similar frequency. Explicitly, Bohr's hypothesis is derived below, then written in Equation \ref{eqn:bohrs_hypthesis}.
			
			\begin{gather}
					\text{\small Classically, Angular Momentum would be } \int L_{\theta} \;  d\theta = \Planckconst n \nonumber\\
					\text{\small Bohr's Hypothesis } \implies \int L_{\theta} \; d\theta  =  L_{\theta} \oint^{2\pi}_0 d\theta  =  L_{\theta} \cdot 2 \pi \nonumber\\
					\label{eqn:bohrs_hypthesis} \therefore \; L_{\theta}  =  \frac{\Planckconst}{2 \pi} n  =  \hbar n \; \; \; \text{Where } n \in \mathbb{N}
				\end{gather}
			
			\vspace{5mm}
			
			Thus, in quantum physics, the angular momentum is quantised and an integer multiple of Planck's reduced constant, $\hbar$. We will soon look at the implications of this hypothesis and how this helped Bohr construct his model of the Hydrogen atom. \linebreak
			
			
		\subsection{Bohr’s Model - An Intro for the Unfamiliar and Recap for the Familiar}
		
			Hydrogen Atoms are a very simple system consisting of a single proton (with a positive charge) and a single electron (with a negative charge). We say that the electron orbits with \textit{tangential} velocity, $V_R$, it sweeps and angle, $\theta$, and it orbits with a radius, $r$. As a quick reminder to classical physics, the \textit{angular momentum} of the electron with thus be:
		
			\begin{equation*}
					\text{\small Angular Momentum, }\overrightarrow{L}  =  \text{radius, }\overrightarrow{r}  \wedge  \text{\small Momentum, }\overrightarrow{P}
				\end{equation*}
			
			In the case of a single particle, in circular motion; Momentum has the scalar value of $\overrightarrow{P} = p = m \cdot V_R$ and the Equation becomes:
			
			\begin{equation*}
					\overrightarrow{L}  =  r \cdot m \cdot V_R \cdot \hat{\theta} = L_\theta \hat{\theta}
				\end{equation*}
			
			And we eventually obtain the result as per Equation \ref{eqn:bohrs_hypthesis} given Bohr's hypothesis. But let's elaborate the consequences fully, and provide some simple proof... \linebreak
			
			
			\textbf{\LARGE\color{heading}TODO: Nice Bohr Model and Angular Momentum Diagram} % TODO: Nice Bohr Model and Angular Momentum Diagram
			
			
		\subsection{Force Balance - The Bohr Radius}
			In order to have electrons orbiting a central nucleus they must be under a ``centrifugal'' force.
			
			\begin{equation*}
					\text{Force, }F  =  \text{mass, }m  \cdot  \frac{\text{Instant. Velocity, }V_R^2}{\text{radius, }r}
				\end{equation*}
			
			\vspace{5mm}
			
			As we assume the only attractive/repellent forces in play here must be electromotive, the centrifugal force must therefore be equal to the electromotive forces. Given our previous definitions regarding E.M.F we can state:
			
			\begin{gather*}
					\text{Force, }F  =  \text{mass, }m  \cdot  \frac{\text{Instant. Velocity, }V_R^2}{\text{radius, }r}  =  \frac{e^2}{r^2} \\
					\therefore \;  r  =  \frac{e^2}{m  V_R^2} \\
					\because \; V_R  =  \frac{L_\theta}{m  \cdot  r},  \; \; \; \frac{L_\theta^2}{m \cdot r^3}  =  \frac{e^2}{r^2} \; \to \; \therefore \; r  =  \frac{L_\theta}{m \cdot e^2} \\
					\because \; L_\theta  =  \hbar n \; :
				\end{gather*}
			
			\begin{equation} \label{eqn:bohr_model_radii_n}
					 r_n  =  \frac{\hbar^2}{m \cdot e^2} \cdot n^2  \; \; \; \text{\small OR, } \; \; \; r_n  =  r_0 \cdot n^2 \; \; \; \text{Where } n \in \mathbb{N}	
				\end{equation}
			
			
			$r_n$ for the value $n=1$ in Equation \ref{eqn:bohr_model_radii_n} above is usually termed $r_0$ and is known as the "Bohr Radius" ($r_0 \approx 0.5$\r{A}), it is the inner most orbit radius an electron can have in a hydrogen atom. All other orbits are square integer multiples of this value, giving us a way of calculating the other possible orbits. It should be apparent this means that all electron orbit radii are \textbf{quantised}, we will explore the implications of this discovery in the following subsections. Table \ref{tab:bohr_model_radii} below shows some of the radii: \linebreak
			
			\begin{table}[h!]
					\color{body}
					\centering
					\SetTblrInner{rowsep=2.5mm}
					\begin{tblr}{width=6cm,colspec={|X[1,c,m]|X[1,c,m]|}}
							\hline
							{\small Orbit One, }$r_1$ & "$r_0$" $\approx 0.5${\small \r{A}}\\	
							\hline
							{\small Orbit Two, }$r_2$ & $r_0 \times 4 \approx 2${\small \r{A}} \\
							\hline
							{\small Orbit Three, }$r_3$ & $r_0 \times 9 \approx 4.5${\small \r{A}} \\
							\hline
						\end{tblr}
					\caption{\label{tab:bohr_model_radii}\textit{The first three Bohr model radii}}
				\end{table}
			
			\pagebreak
			
			
		\subsection{Velocity}
			As the velocity of the "orbiting" particle is dependent on the angular momentum it will be discretized also. The result of the first orbit is derived below in Equation \ref{eqn:bohr_model_velocity_1}, with the general case found in Equation \ref{eqn:bohr_model_velocity_n}.
			
			\begin{gather*}
					\text{\small Recall:} \\
					V_R = \frac{L_\theta}{m  \cdot  r} = \frac{\hbar \cdot n}{\frac{\hbar^2}{m \cdot e^2} \cdot n^2 \cdot m} = \frac{e^2}{\hbar \cdot n} \\
					\therefore \; \text{For the first orbit we have: } \; V_{R,1} = \frac{e^2}{\hbar} \\
					\text{\small But there is a more "convenient"/beautiful definition:} \\
					V_{R,1} = \frac{e^2}{\hbar \cdot n} = \frac{e^2}{\hbar} \; \; \text{\small (Multiply by $\frac{c}{c}$ for a nice result)} \; V_{R,1} = \frac{e^2}{\hbar \cdot c} \cdot c \dots
				\end{gather*}
			
			\begin{equation} \label{eqn:bohr_model_velocity_1}
					V_{R,1} \equiv \alpha \cdot c \approx 3.8 \cdot 10^6 \; m/s
				\end{equation}
			
			\begin{equation} \label{eqn:bohr_model_velocity_n}
					V_{R,n} = \frac{\alpha \cdot c}{n} \; \; \; \text{Where } n \in \mathbb{N}
				\end{equation}
			
			
			Where $\alpha$ is the Fine Structure constant ($\alpha = \frac{1}{137}$), and there are other names for the orbits; $V_{R,1}$ is also called the "ground state orbit", $V_{R,2}$ is also known as the "first excited state orbit". \linebreak
			
			\pagebreak
			
			
		\subsection{Energy and Energy Levels}
			We will now show one of the most important implications of the quantization of angular momentum, which is that all electrons of an atom have a specific "energy level" corresponding to each orbit. As usual, the derivation is below, with the first orbit value found in Equation \ref{eqn:bohr_model_energy_1} and the general case in Equation \ref{eqn:bohr_model_energy_n}:
			
			
			\begin{gather*}
					\text{\small Recall:} \\
					\text{\small We have energy from an attractive force, } \; E_q = - \frac{e^2}{r} \\
					\text{\small Also kinetic energy, } \; E_k = \frac{1}{2} m \cdot V_R^2 \\
					E = \frac{1}{2} \cdot m \cdot V_R^2 - \frac{e^2}{r} \\
					\text{\small Sub $V_R$ \& $r$ from Equations \ref{eqn:bohr_model_velocity_n} \& \ref{eqn:bohr_model_radii_n}:} \\
					\therefore \; E = \frac{1}{2} \cdot m \cdot \left(\frac{e^2}{\hbar \cdot n}\right)^2 - \frac{e^2}{\frac{\hbar^2}{m \cdot e^2} \cdot n^2}
					E = \frac{1}{2} \cdot \frac{m \cdot e^4}{\hbar^2 \cdot n^2} - \frac{m e^4}{\hbar^2 \cdot n^2}
				\end{gather*}
			
			\begin{gather} \label{eqn:bohr_model_energy_n}
					E_n = - \frac{1}{2} \cdot \frac{m \cdot e^4}{\hbar^2 \cdot n^2} = - \frac{m \cdot e^4}{2 \cdot \hbar^2} \cdot \frac{1}{n^2} = - \frac{R}{n^2} \\ 
					\text{\small Where $R=\frac{m \cdot e^4}{2 \cdot \hbar^2}$ is the Rydberg Constant and $n \in \mathbb{N}$} \nonumber
				\end{gather}
			
			\begin{gather}
					E_1 = - \frac{m \cdot e^4}{2 \cdot \hbar^2} \cdot \frac{1}{1^2} = - \frac{R}{1^2} = - \frac{m \cdot |e|^4}{2 \cdot \hbar^2} = -R \nonumber\\
					\label{eqn:bohr_model_energy_1} E_1 \approx -13.6 \; \text{\small eV } = 13.6 \cdot 1.602 \cdot 10^{-19} \; \text{\small Joules (J)}
				\end{gather}
			
			
			
			
			\begin{figure}[!h]
					\centering
					%% BOHR HYDROGEN MODEL, JUST ENERGY LEVELS, NO TRANSITIONS
					\includestandalone{QT-1-3-Diagrams-and-Graphs/bohr-model-energy-levels}
				\caption{Something... TODO } % TODO: Energy levels graph
				\label{fig:bohr_model_energy_levels}
				\end{figure}
			
			
			
			\begin{figure}[!h]
					\centering
					%% BOHR HYDROGEN MODEL, ENERGY LEVELS, WITH TRANSITION PHOTONS
					\includestandalone{QT-1-3-Diagrams-and-Graphs/bohr-model-energy-transitions}
				\caption{Something... TODO } % TODO: Energy levels graph
				\label{fig:bohr_model_energy_transitions}
				\end{figure}
			
			
			\pagebreak
\end{document}