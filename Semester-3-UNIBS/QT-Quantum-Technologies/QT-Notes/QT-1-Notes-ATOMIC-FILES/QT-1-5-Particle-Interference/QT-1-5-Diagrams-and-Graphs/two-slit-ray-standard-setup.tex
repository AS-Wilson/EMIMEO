%% TWO SLIT - SETUP, UNOBSEVRED, RAY PATH, WITH ELECTRON GUN

% https://tikz.net/optics_twoslit/
% https://www.overleaf.com/learn/latex/Counters
% https://tex.stackexchange.com/questions/29989/how-does-the-counter-in-tikz-foreach-work
% https://tex.stackexchange.com/questions/45848/rotate-node-text-and-use-relative-positioning-in-tikz
% https://tex.stackexchange.com/questions/625022/using-tikz-to-plot-angle-notations-and-various-arrows
% https://tex.stackexchange.com/questions/449675/how-to-draw-a-coil-such-that-you-can-see-if-its-right-or-left-handed
\documentclass{standalone}


% https://tikz.net/optics_twoslit/
% https://www.overleaf.com/learn/latex/Counters
% https://tex.stackexchange.com/questions/29989/how-does-the-counter-in-tikz-foreach-work
% https://tex.stackexchange.com/questions/45848/rotate-node-text-and-use-relative-positioning-in-tikz
% https://tex.stackexchange.com/questions/625022/using-tikz-to-plot-angle-notations-and-various-arrows
% https://tex.stackexchange.com/questions/449675/how-to-draw-a-coil-such-that-you-can-see-if-its-right-or-left-handed



\IfStandalone{\def\datapath{../../../}}{\def\datapath{}}

\input{\datapath QT-Notes-Preambles/QT-Notes-Diagrams-and-Graphs-DEFAULT-Preamble.tex}


\usetikzlibrary{calc}
\usetikzlibrary{arrows,arrows.meta,math}
\usetikzlibrary{decorations.markings}
\usetikzlibrary{angles,quotes} % for pic (angle labels)
\usetikzlibrary{fadings}
\tikzset{>=latex} % for LaTeX arrow head
\contourlength{1.4pt}

\colorlet{myshadow}{blue!30!black!90}
\tikzstyle{wave}=[ScienceBlue,thick]
\tikzstyle{mydashed}=[black!70,dashed,thin]
\tikzstyle{mymeas}=[{Latex[length=3,width=2]}-{Latex[length=3,width=2]},thin]
\tikzstyle{mysmallarr}=[-{Latex[length=3,width=2]}]

\usetikzlibrary{decorations.pathmorphing}

\newcommand\rightAngle[4]{
	\pgfmathanglebetweenpoints{\pgfpointanchor{#2}{center}}{\pgfpointanchor{#3}{center}}
	\coordinate (tmpRA) at ($(#2)+(\pgfmathresult+45:#4)$);
	\draw[white,line width=0.6] ($(#2)!(tmpRA)!(#1)$) -- (tmpRA) -- ($(#2)!(tmpRA)!(#3)$);
	\draw[ScienceRed] ($(#2)!(tmpRA)!(#1)$) -- (tmpRA) -- ($(#2)!(tmpRA)!(#3)$);
}
\newcommand\lineend[2]{
	\def\w{0.1} \def\c{30}
	\draw[ScienceGreen] (#1)++(#2:\w) to[out=#2-180-\c,in=#2+\c] (#1)
	to[out=#2+\c-180,in=#2-\c]++ (#2-180:\w);
}
\def\tick#1#2{\draw[thick] (#1) ++ (#2:0.1) --++ (#2-180:0.2)}

% INTERFERENCE FADING
\begin{tikzfadingfrompicture}[name=interference]
	\def\lambd{0.5} % wavelength
	\foreach \r in {1,...,15}
	\foreach \j in {1,...,25}
	\path [line width=\lambd*\j,draw=transparent!0,opacity=0.04]
	(0,0) circle (\lambd*\r); %(0:\r) arc (0:180:\r);
\end{tikzfadingfrompicture}

% INTERFERENCE FADING
\begin{tikzfadingfrompicture}[name=interference]
	\def\lambd{0.5} % wavelength
	\foreach \r in {1,...,15}
	\foreach \j in {1,...,25}
	\path [line width=\lambd*\j,draw=transparent!0,opacity=0.04]
	(0,0) circle (\lambd*\r); %(0:\r) arc (0:180:\r);
\end{tikzfadingfrompicture}


\newcommand\DETECTOR[4]{
	\def\Width{#1}
	\def\Height{#2}
	\def\x{#3}
	\def\y{#4}
	
	\path[draw,shape=coordinate]
	(\x-0.5*\Width,\y-0.5*\Height) coordinate(BotL) 
	(\x+0.5*\Width,\y-0.5*\Height) coordinate(BotR)
	(\x+0.5*\Width,\y+0.5*\Height) coordinate(TopR) 
	(\x-0.5*\Width,\y+0.5*\Height) coordinate(TopL)
	(\x-0.5*\Width,\y+0.15*\Height) coordinate(EntryTop)
	(\x+0.1*\Width,\y+0.15*\Height) coordinate(InnerTop)
	(\x+0.1*\Width,\y-0.15*\Height) coordinate(InnerBot)
	(\x-0.5*\Width,\y-0.15*\Height) coordinate(EntryBot);
	\filldraw[color=black,fill=darkgray!25] (BotL) -- (BotR) -- (TopR) -- (TopL) -- (EntryTop) -- (InnerTop) -- (InnerBot) -- (EntryBot) -- (BotL);
}


\newcommand\DETECTORATSLITTOP[4]{
	\def\Width{#1}
	\def\Height{#2}
	\def\x{#3}
	\def\y{#4}
	
	\path[draw,shape=coordinate]
	(\x-0.5*\Width,\y-0.5*\Height) coordinate(BotL)
	(\x-0.5*\Width,\y+0.5*\Height) coordinate(TopL)
	(\x+0.5*\Width,\y+0.5*\Height) coordinate(TopR) 
	(\x+0.5*\Width,\y-0.5*\Height) coordinate(BotR)
	
	(\x+0.2*\Width,\y-0.5*\Height) coordinate(EntryR)
	(\x+0.2*\Width,\y+0.15*\Height) coordinate(InnerR)
	(\x-0.2*\Width,\y+0.15*\Height) coordinate(InnerL)
	(\x-0.2*\Width,\y-0.5*\Height) coordinate(EntryL);
	\filldraw[color=black,fill=darkgray!25] (BotL) -- (TopL) -- (TopR) -- (BotR) -- (EntryR) -- (InnerR) -- (InnerL) -- (EntryL) -- (BotL);
}


\newcommand\DETECTORATSLITBOTTOM[4]{
	\def\Width{#1}
	\def\Height{#2}
	\def\x{#3}
	\def\y{#4}
	
	\path[draw,shape=coordinate]
	(\x-0.5*\Width,\y+0.5*\Height) coordinate(TopL)
	(\x-0.5*\Width,\y-0.5*\Height) coordinate(BotL)
	(\x+0.5*\Width,\y-0.5*\Height) coordinate(BotR)
	(\x+0.5*\Width,\y+0.5*\Height) coordinate(TopR) 	
	
	(\x+0.2*\Width,\y+0.5*\Height) coordinate(EntryR)
	(\x+0.2*\Width,\y-0.15*\Height) coordinate(InnerR)
	(\x-0.2*\Width,\y-0.15*\Height) coordinate(InnerL)
	(\x-0.2*\Width,\y+0.5*\Height) coordinate(EntryL);
	\filldraw[color=black,fill=darkgray!25] (TopL) -- (BotL) -- (BotR) -- (TopR) -- (EntryR) -- (InnerR) -- (InnerL) -- (EntryL) -- (TopL);
}


% Electron Gun
\newcommand\ElectronGun[4]{
	\def\Width{#1}
	\def\Height{#2}
	\def\x{#3}
	\def\y{#4}
	
	\path[draw,shape=coordinate]
	(\x+0.5*\Width,\y-0.15*\Height) coordinate(ExitBot)
	(\x+0.5*\Width,\y-0.5*\Height) coordinate(BotR)
	(\x-0.5*\Width,\y-0.5*\Height) coordinate(BotL)
	(\x-0.5*\Width,\y+0.5*\Height) coordinate(TopL)
	(\x+0.5*\Width,\y+0.5*\Height) coordinate(TopR) 
	(\x+0.5*\Width,\y+0.15*\Height) coordinate(ExitTop);
	\filldraw[color=black,fill=darkgray!25] (ExitBot) -- (BotR) -- (BotL) -- (TopL) -- (TopR) -- (ExitTop);
	
	\draw[decoration={coil,aspect=0.4,segment length=\Height*3.333,amplitude=\Width*2},decorate,color=ScienceRed,fill=darkgray!25] (\x-0.3*\Height,\y+0.4*\Height) -- (\x-0.3*\Height,\y-0.4*\Height);
	
	\draw[color=ScienceRed] (\x-0.3*\Height,\y+0.4*\Height) -- (\x-\Width*0.6,\y+0.4*\Height);
	\draw[color=ScienceRed] (\x-0.3*\Height,\y-0.4*\Height) -- (\x-\Width*0.6,\y-0.4*\Height);
}






\begin{document}
	\begin{tikzpicture}
		
		\def\L{9}       % distance between walls
		\def\H{6}       % total wall height
		\def\f{0.9}       % fractional height of projection point
		\def\ang{atan((\f*\H+\a)/\L/2)} % theta
		\def\t{0.1}      % wall thickness
		\def\a{2}       % slit distance
		\def\d{0.20}      % slit size
		\coordinate (T) at (0,\a/2);
		\coordinate (B) at (0,-\a/2);
		\coordinate (L) at (0,0);
		\coordinate (R) at (\L,0);
		\coordinate (P) at (\L-0.2,\f*\H/2);
		\coordinate (M) at ($(B)!(T)!(P)$);
		
		
		% LINES
		\draw[ScienceGreen,thick] (T) -- (P) coordinate[pos=0.3,above=5] (k11) coordinate[pos=0.4,above=5] (k12) coordinate[pos=0.35,above=4] (k1); % TOP RAY
		\draw[->,ScienceGreen] (k11) -- (k12);
		\draw (k1) node[color=ScienceGreen,anchor=south] {\small $\overrightarrow{k}_1$};
		
		\draw[ScienceGreen,thick] (B) -- (P) coordinate[pos=0.45,below=5] (k21) coordinate[pos=0.55,below=5] (k22) coordinate[pos=0.525,below=6] (k2);% BOTTOM RAY
		\draw[->,ScienceGreen] (k21) -- (k22);
		\draw (k2) node[color=ScienceGreen,anchor=north] {\small $\overrightarrow{k}_2$};
		
		\draw[ScienceGreen,thick] (T) -- (P) node[pos=0.75,above=-1] {\small $\overrightarrow{r}_1$};
		\draw[ScienceGreen,thick] (B) -- (P) node[pos=0.75,below=3,right=6] {\small $\overrightarrow{r}_2$}; %right=6,below right=-4
		\draw[dashed] (L) -- (P);
		\draw[dashed,black!60] (L) -- (\L*0.22,0);% Line from middle left to theta angle starting point
		\draw[dashed,black!60] (\L*0.3,0) -- ([shift={(25*\t,0)}]R); % continuation of horizontal centreline
		\draw[dashed,black!60] ([shift={(2*\t,0)}]P) -- ([shift={(25*\t,0)}]P); % Top dashed line for the "y" measurement
		
		
		% ANGLES
		\draw pic[mysmallarr,"\small $\theta$",ScienceRed,draw=ScienceRed,angle radius=38,angle eccentricity=1.14]
		{angle = R--L--P};
		
		
		% MEASURES
		\draw[<->,black] (0,-0.47*\H) --++ (\L,0) node[midway,fill=white,inner sep=3] {\small $L$};
		\draw[<->,black] (-4*\t,-\a/2) --++ (0,\a) node[midway,fill=white,inner sep=3] {\small $d$};
		
		\draw[<->,black] ([shift={(20*\t,0)}]\L,\f*\H/2) -- ([shift={(20*\t,0)}]R)
		node[midway,fill=white,inner sep=3]{\small y};
		
		
		% WALL
		\fill[ScienceBlue!30!]
		(0,\a/2-\d/2) rectangle (-\t,-\a/2+\d/2)
		(0,\a/2+\d/2) rectangle (-\t,\H/2)
		(0,-\a/2-\d/2) rectangle (-\t,-\H/2)
		(\L,-\H/2) rectangle (\L+\t,\H/2);
		\fill[ScienceGreen!80!black] (P) circle (0.6*\t) node[right=1,above left=-1.5] {P};
		
		
		% Detector
		\DETECTOR{0.2}{0.75}{\L-\t}{\f*\H/2}
		\draw[color=darkgray!85] (\L-\t,\f*\H/2) node[color=darkgray!75, above = 20,left = 0.25] {\small Detector};
		
		\draw[<->,darkgray!75] ([shift={(4*\t,3*\t)}]P) -- ([shift={(4*\t,-3*\t)}]P) node[midway,right=3,fill=white,inner sep=3,rotate=90,anchor=north]{\scriptsize Moving};
		\fill[color=darkgray!75] ([shift={(4*\t,0)}]P) circle (0.4*\t);
		
		% Electron Gun
		\ElectronGun{2}{0.75}{-\L*0.65}{0}
		\draw[color=darkgray!85] (-\L*0.65,0) node[color=darkgray!75, above = 8] {\small Electron Gun};
		
		% Electron Beams
		%https://wiki.physik.uzh.ch/cms/latex:tikz
		\def\R{\L*0.35} % radius/length of lines
		\def\r{\L*0.05}
		
		\foreach \t/\e/\b in {0/0.65/0.35,10/0.7/0.65,20/0.75/0.72,30/0.8/0.78,40/0.85/0.84)}{
			\draw[->,dashed,ScienceGreen!75,thick]
			({-\L*0.5+\r*cos(\t)},{0+\r*sin(\t)}) -- ({-\L*0.5+\R*cos(\t)},{0+\R*sin(\t)});
			\draw[fill=ScienceGreen!60] ({-\L*0.5+\R*cos(\t)*\e},{0+\R*sin(\t)*\e}) circle [radius = 0.1];
			\node[anchor=center] at ({-\L*0.5+\R*cos(\t)*\e},{0+\R*sin(\t)*\b}) {\tiny -};
		}
		
		\foreach \t/\e/\b in {10/0.6/0.65,20/0.55/0.58,30/0.495/0.515,40/0.44/0.455}{
			\draw[->,dashed,ScienceGreen!75,thick]
			({-\L*0.5+\r*cos(\t)},{0-\r*sin(\t)}) -- ({-\L*0.5+\R*cos(\t)},{0-\R*sin(\t)});
			\draw[fill=ScienceGreen!60] ({-\L*0.5+\R*cos(\t)*\e},{0-\R*sin(\t)*\e}) circle [radius = 0.1];
			\node[anchor=center] at ({-\L*0.5+\R*cos(\t)*\e},{0-\R*sin(\t)*\b}) {\tiny -};
		}
		
		
		
		
	\end{tikzpicture}
\end{document}