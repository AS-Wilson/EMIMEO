%% TWO SLIT - UNOBSEVRED, FINAL WAVE-LIKE BEHAVIOUR

% https://tikz.net/optics_twoslit/
% https://www.overleaf.com/learn/latex/Counters
% https://tex.stackexchange.com/questions/29989/how-does-the-counter-in-tikz-foreach-work
% https://tex.stackexchange.com/questions/45848/rotate-node-text-and-use-relative-positioning-in-tikz
% https://tex.stackexchange.com/questions/625022/using-tikz-to-plot-angle-notations-and-various-arrows
% https://tex.stackexchange.com/questions/449675/how-to-draw-a-coil-such-that-you-can-see-if-its-right-or-left-handed
\documentclass{standalone}


% https://tikz.net/optics_twoslit/
% https://www.overleaf.com/learn/latex/Counters
% https://tex.stackexchange.com/questions/29989/how-does-the-counter-in-tikz-foreach-work
% https://tex.stackexchange.com/questions/45848/rotate-node-text-and-use-relative-positioning-in-tikz
% https://tex.stackexchange.com/questions/625022/using-tikz-to-plot-angle-notations-and-various-arrows
% https://tex.stackexchange.com/questions/449675/how-to-draw-a-coil-such-that-you-can-see-if-its-right-or-left-handed



\IfStandalone{\def\datapath{../../../}}{\def\datapath{}}

\input{\datapath QT-Notes-Preambles/QT-Notes-Diagrams-and-Graphs-DEFAULT-Preamble.tex}


\usetikzlibrary{calc}
\usetikzlibrary{arrows,arrows.meta,math}
\usetikzlibrary{decorations.markings}
\usetikzlibrary{angles,quotes} % for pic (angle labels)
\usetikzlibrary{fadings}
\tikzset{>=latex} % for LaTeX arrow head
\contourlength{1.4pt}

\colorlet{myshadow}{blue!30!black!90}
\tikzstyle{wave}=[ScienceBlue,thick]
\tikzstyle{mydashed}=[black!70,dashed,thin]
\tikzstyle{mymeas}=[{Latex[length=3,width=2]}-{Latex[length=3,width=2]},thin]
\tikzstyle{mysmallarr}=[-{Latex[length=3,width=2]}]

\usetikzlibrary{decorations.pathmorphing}

\newcommand\rightAngle[4]{
	\pgfmathanglebetweenpoints{\pgfpointanchor{#2}{center}}{\pgfpointanchor{#3}{center}}
	\coordinate (tmpRA) at ($(#2)+(\pgfmathresult+45:#4)$);
	\draw[white,line width=0.6] ($(#2)!(tmpRA)!(#1)$) -- (tmpRA) -- ($(#2)!(tmpRA)!(#3)$);
	\draw[ScienceRed] ($(#2)!(tmpRA)!(#1)$) -- (tmpRA) -- ($(#2)!(tmpRA)!(#3)$);
}
\newcommand\lineend[2]{
	\def\w{0.1} \def\c{30}
	\draw[ScienceGreen] (#1)++(#2:\w) to[out=#2-180-\c,in=#2+\c] (#1)
	to[out=#2+\c-180,in=#2-\c]++ (#2-180:\w);
}
\def\tick#1#2{\draw[thick] (#1) ++ (#2:0.1) --++ (#2-180:0.2)}

% INTERFERENCE FADING
\begin{tikzfadingfrompicture}[name=interference]
	\def\lambd{0.5} % wavelength
	\foreach \r in {1,...,15}
	\foreach \j in {1,...,25}
	\path [line width=\lambd*\j,draw=transparent!0,opacity=0.04]
	(0,0) circle (\lambd*\r); %(0:\r) arc (0:180:\r);
\end{tikzfadingfrompicture}

% INTERFERENCE FADING
\begin{tikzfadingfrompicture}[name=interference]
	\def\lambd{0.5} % wavelength
	\foreach \r in {1,...,15}
	\foreach \j in {1,...,25}
	\path [line width=\lambd*\j,draw=transparent!0,opacity=0.04]
	(0,0) circle (\lambd*\r); %(0:\r) arc (0:180:\r);
\end{tikzfadingfrompicture}


\newcommand\DETECTOR[4]{
	\def\Width{#1}
	\def\Height{#2}
	\def\x{#3}
	\def\y{#4}
	
	\path[draw,shape=coordinate]
	(\x-0.5*\Width,\y-0.5*\Height) coordinate(BotL) 
	(\x+0.5*\Width,\y-0.5*\Height) coordinate(BotR)
	(\x+0.5*\Width,\y+0.5*\Height) coordinate(TopR) 
	(\x-0.5*\Width,\y+0.5*\Height) coordinate(TopL)
	(\x-0.5*\Width,\y+0.15*\Height) coordinate(EntryTop)
	(\x+0.1*\Width,\y+0.15*\Height) coordinate(InnerTop)
	(\x+0.1*\Width,\y-0.15*\Height) coordinate(InnerBot)
	(\x-0.5*\Width,\y-0.15*\Height) coordinate(EntryBot);
	\filldraw[color=black,fill=darkgray!25] (BotL) -- (BotR) -- (TopR) -- (TopL) -- (EntryTop) -- (InnerTop) -- (InnerBot) -- (EntryBot) -- (BotL);
}


\newcommand\DETECTORATSLITTOP[4]{
	\def\Width{#1}
	\def\Height{#2}
	\def\x{#3}
	\def\y{#4}
	
	\path[draw,shape=coordinate]
	(\x-0.5*\Width,\y-0.5*\Height) coordinate(BotL)
	(\x-0.5*\Width,\y+0.5*\Height) coordinate(TopL)
	(\x+0.5*\Width,\y+0.5*\Height) coordinate(TopR) 
	(\x+0.5*\Width,\y-0.5*\Height) coordinate(BotR)
	
	(\x+0.2*\Width,\y-0.5*\Height) coordinate(EntryR)
	(\x+0.2*\Width,\y+0.15*\Height) coordinate(InnerR)
	(\x-0.2*\Width,\y+0.15*\Height) coordinate(InnerL)
	(\x-0.2*\Width,\y-0.5*\Height) coordinate(EntryL);
	\filldraw[color=black,fill=darkgray!25] (BotL) -- (TopL) -- (TopR) -- (BotR) -- (EntryR) -- (InnerR) -- (InnerL) -- (EntryL) -- (BotL);
}


\newcommand\DETECTORATSLITBOTTOM[4]{
	\def\Width{#1}
	\def\Height{#2}
	\def\x{#3}
	\def\y{#4}
	
	\path[draw,shape=coordinate]
	(\x-0.5*\Width,\y+0.5*\Height) coordinate(TopL)
	(\x-0.5*\Width,\y-0.5*\Height) coordinate(BotL)
	(\x+0.5*\Width,\y-0.5*\Height) coordinate(BotR)
	(\x+0.5*\Width,\y+0.5*\Height) coordinate(TopR) 	
	
	(\x+0.2*\Width,\y+0.5*\Height) coordinate(EntryR)
	(\x+0.2*\Width,\y-0.15*\Height) coordinate(InnerR)
	(\x-0.2*\Width,\y-0.15*\Height) coordinate(InnerL)
	(\x-0.2*\Width,\y+0.5*\Height) coordinate(EntryL);
	\filldraw[color=black,fill=darkgray!25] (TopL) -- (BotL) -- (BotR) -- (TopR) -- (EntryR) -- (InnerR) -- (InnerL) -- (EntryL) -- (TopL);
}


% Electron Gun
\newcommand\ElectronGun[4]{
	\def\Width{#1}
	\def\Height{#2}
	\def\x{#3}
	\def\y{#4}
	
	\path[draw,shape=coordinate]
	(\x+0.5*\Width,\y-0.15*\Height) coordinate(ExitBot)
	(\x+0.5*\Width,\y-0.5*\Height) coordinate(BotR)
	(\x-0.5*\Width,\y-0.5*\Height) coordinate(BotL)
	(\x-0.5*\Width,\y+0.5*\Height) coordinate(TopL)
	(\x+0.5*\Width,\y+0.5*\Height) coordinate(TopR) 
	(\x+0.5*\Width,\y+0.15*\Height) coordinate(ExitTop);
	\filldraw[color=black,fill=darkgray!25] (ExitBot) -- (BotR) -- (BotL) -- (TopL) -- (TopR) -- (ExitTop);
	
	\draw[decoration={coil,aspect=0.4,segment length=\Height*3.333,amplitude=\Width*2},decorate,color=ScienceRed,fill=darkgray!25] (\x-0.3*\Height,\y+0.4*\Height) -- (\x-0.3*\Height,\y-0.4*\Height);
	
	\draw[color=ScienceRed] (\x-0.3*\Height,\y+0.4*\Height) -- (\x-\Width*0.6,\y+0.4*\Height);
	\draw[color=ScienceRed] (\x-0.3*\Height,\y-0.4*\Height) -- (\x-\Width*0.6,\y-0.4*\Height);
}






\begin{document}
	\begin{tikzpicture}[
		nodal/.style={ScienceGreen!50,dashed,thin},
		declare function={
			xMaximaPosition(\Integer) = \ScreenDistL*tan(asin((\Integer*\lambdaDB)/\dSlitSeparationDistance)); % Function to calculate position of screen MAXIMA - "x" in experiment, \y in our tikz diagram - value
			xMinimaPosition(\Integer) = \ScreenDistL*tan(asin(((\Integer+0.5)*\lambdaDB)/\dSlitSeparationDistance)); % Function to calculate position of screen MINIMA - "x" in experiment, \y in our tikz diagram - value
			xRecalculate(\x) = \ScreenDistL-(((\x-\WallTotalSizeinX/2)*\ScreenDistL)/\x); % function to recaluclate - "x" in experiment, \y in our tikz diagram - for maxima/minima in the case we hit the screen out side the limits of the tikz diagram
		}]
		
		\def\ScreenDistL{8}       						% L -- Distance between walls
		\def\WallTotalSizeinX{5}       					% X -- Total wall height
		\def\WallThickness{0.1}      					% t -- wall thickness
		
		\def\dSlitSeparationDistance{1}      			% d -- Centre to centre slit distance
		\def\lambdaDB{0.1}  							% De Broglie Wavelength of electrons
		\def\kVectorNumber{2*3.14159265/\lambdaDB} 		% k -- vectornumber / wavenumber
		
		\def\SlitAperture{0.22}      					% b -- Slit aperture width / size
		
		\pgfmathparse{int(\ScreenDistL/\lambdaDB*2)} % Had to make calculation like this cause tik / pgf was being weird
		\def\NumberOfWavesN{\pgfmathresult}        		% Number of slit emanating and electron gun waves to produce
		
		\def\nMaxAndnMinLinesPerSide{2}					% For doing the right number of MAXIMA/MINIMA lines
		
		\def\ang{60}									% Angle or creating all limited spherical waves
		
		\def\xPositionDetector{0.9}       				% fractional height of projection point, makes point P stand off detector a bit and look neater
		\def\nSamples{1000}								% Number of samples to compute for the intenstiy graph
		
		
		%% Coordinates for the wall
		\coordinate (LeftWallUpper) at (-\WallThickness/2,\WallTotalSizeinX/2);
		\coordinate (LeftWallLower) at (\WallThickness/2,-\WallTotalSizeinX/2);
		
		\coordinate (FirstSlitCentre) at (0,\dSlitSeparationDistance/2);
		\coordinate (FirstSlitUpper) at (\WallThickness/2,\dSlitSeparationDistance/2+\SlitAperture/2);
		\coordinate (FirstSlitLower) at (-\WallThickness/2,\dSlitSeparationDistance/2-\SlitAperture/2);
		
		\coordinate (SecondSlitCentre) at (0,-\dSlitSeparationDistance/2);
		\coordinate (SecondSlitUpper) at (\WallThickness/2,-\dSlitSeparationDistance/2+\SlitAperture/2);
		\coordinate (SecondSlitLower) at (-\WallThickness/2,-\dSlitSeparationDistance/2-\SlitAperture/2);
		
		\coordinate (RightWallUpper) at (\ScreenDistL-\WallThickness/2,\WallTotalSizeinX/2);
		\coordinate (RightWallLower) at (\ScreenDistL+\WallThickness/2,-\WallTotalSizeinX/2);
		
		
		%% WALL
		\fill[ScienceBlue!40]
			(LeftWallUpper) rectangle (FirstSlitUpper)
			(FirstSlitLower) rectangle (SecondSlitUpper)
			(SecondSlitLower) rectangle (LeftWallLower)
			(RightWallUpper) rectangle (RightWallLower);
		
		
					
		%% ELECTRON GUN SYMBOL
		\ElectronGun{2}{0.75}{-\ScreenDistL*0.65}{0}
		\draw[color=darkgray!85] (-\ScreenDistL*0.65,0) node[color=darkgray!75, above = 8] {\small Electron Gun};
		
		
		
		%% RADIAL WAVES EMANATING FROM THE SLITS
		\begin{scope}
			\clip (LeftWallUpper) rectangle (RightWallLower);
			%\clip (-\t/2,0.7*\a) -- (0.6*\L,\H/2) -- (\L,\H/2) --
			%      (\L,-\H/2) -- (0.6*\L,-\H/2) -- (-\t/2,-0.7*\a) -- cycle;
		
			% WAVES
			\foreach \i [evaluate={\R=\i*\lambdaDB/2;}] in {1,...,\NumberOfWavesN}{
				\ifodd\i
					\draw[ScienceGreen!90,line width=0.55] (0,\dSlitSeparationDistance/2)++(\ang:\R) arc (\ang:-\ang:\R);
					\draw[ScienceYellow!90,line width=0.55] (0,-\dSlitSeparationDistance/2)++(\ang:\R) arc (\ang:-\ang:\R);
				\else
					\draw[ScienceGreen!60,line width=0.075] (0,\dSlitSeparationDistance/2)++(\ang:\R) arc (\ang:-\ang:\R);
					\draw[ScienceYellow!60,line width=0.075] (0,-\dSlitSeparationDistance/2)++(\ang:\R) arc (\ang:-\ang:\R);
				\fi
			}
		\end{scope}
		
		
		
		%% ELECTRON GUN WAVES / ELECTRON WAVES 
		\begin{scope}
			\clip (-\ScreenDistL*0.65,\WallTotalSizeinX/2) rectangle ([shift={(-\WallThickness,0)}]LeftWallLower);
	
			% INITIAL ELECTRON WAVES
			\foreach \i [
				evaluate={
					\R=(\i*\lambdaDB/2);
					\ad=(exp(\i*0.01));
					\arc=26-(exp(\i*0.01))/2;
					\antarc=-26+(exp(\i*0.01))/2;
				}] in {1,...,\NumberOfWavesN}{
					\ifodd\i
						\draw[ScienceBlue!90,line width=0.55] (-\ScreenDistL*0.55,0)++(\arc:\R) arc (\arc:-\arc:\R);
					\else
						\draw[ScienceBlue!60,line width=0.075] (-\ScreenDistL*0.55,0)++(\arc:\R) arc (\arc:-\arc:\R);
					\fi
				}
		\end{scope}
		
		
		
		%% MINIMA AND MAXIMA LINES (NODAL LINES)
		\begin{scope}
			% First MAXIMA -- Max Intensity, for some reason pgf/tikz is being weird
			\draw[nodal,ScienceRed] (0,0) -- (\ScreenDistL,0);
			
			% Draw the first to MINIMA, for some reason pgf/tikz is being weird
			\draw[nodal,ScienceOrange] (0,0) -- (\ScreenDistL,{\ScreenDistL*tan(asin(((0.5)*\lambdaDB)/\dSlitSeparationDistance))});
			\draw[nodal,ScienceOrange] (0,0) -- (\ScreenDistL,{-\ScreenDistL*tan(asin(((0.5)*\lambdaDB)/\dSlitSeparationDistance))});
			
			\foreach \nNat [evaluate={
				\xminima=xMinimaPosition(\nNat);
				\xmaxima=xMaximaPosition(\nNat);
				\newxmaxima=xRecalculate(\xmaxima);
				\newxminima=xRecalculate(\xminima);
			}] in {1,...,\nMaxAndnMinLinesPerSide}{
				\pgfmathparse{\xminima>\WallTotalSizeinX*0.5 ? int(1) : int(0)}
				\ifnum\pgfmathresult=1
				%					\pgfmathparse{xRecalculate(\xminima,asin(\xminima))}
					\coordinate (MinimaLeft) at (\newxminima,{\WallTotalSizeinX/2});
					\coordinate (MinimaRight) at (\newxminima,-{\WallTotalSizeinX/2});
				\else
					\coordinate (MinimaLeft) at (\ScreenDistL,\xminima);
					\coordinate (MinimaRight) at (\ScreenDistL,-\xminima);
				\fi
				
				\pgfmathparse{\xmaxima>\WallTotalSizeinX*0.5 ? int(1) : int(0)}
				\ifnum\pgfmathresult=1
	%					\pgfmathparse{xRecalculate(\xmaxima,asin(\xmaxima))}
					\coordinate (MaximaLeft) at (\newxmaxima,{\WallTotalSizeinX/2}); 
					\coordinate (MaximaRight) at (\newxmaxima,-{\WallTotalSizeinX/2});
				\else
					\coordinate (MaximaLeft) at (\ScreenDistL,\xmaxima);
					\coordinate (MaximaRight) at (\ScreenDistL,-\xmaxima);
				\fi
				
				\draw[nodal,ScienceOrange] (0,0) -- (MinimaLeft);
				\draw[nodal,ScienceOrange] (0,0) -- (MinimaRight);
				
				\draw[nodal,ScienceRed] (0,0) -- (MaximaLeft);
				\draw[nodal,ScienceRed] (0,0) -- (MaximaRight);
			}
		\end{scope}
		
		
		
		
		 % INTENSITY FUNCTION PLOT
		\begin{scope}[shift={(1.05*\ScreenDistL,0)}]
			
			\pgfkeys{/pgf/trig format=rad} %% Took me ages to figure tht none of the equations were working cause tex works in degrees by defaut, this sets all trig functions to radians by default
			
			\draw[->,thick] (-0.08*2,0) -- (1.3*2,0) node[right=-2] {$\expval{N}$}; % I axis
			\draw[->,thick] (0,-0.52*\WallTotalSizeinX) -- (0,0.54*\WallTotalSizeinX) node[right] {$x$}; % y axis
			
			
			%% http://www.physicsbootcamp.org/section-diffraction-through-a-double-slit-light.html#subsection-Derivation-of-the-Intensity-Formula-Double-Slit
			%% http://www.physicsbootcamp.org/section-diffraction-through-a-double-slit-light.html
			%%%% Equation Explanation
			% a = slit sep distance = \dSlitSeparationDistance
			% b = slit width = \SlitAperture
			% A = (pi * a * x) / (L * Lambda) = (pi * \dSlitSeparationDistance * \y) / (\ScreenDistL * \lambdaDB)
			% B = (pi * b * x) / (L * Lambda) = (pi * \SlitAperture * \y) / (\ScreenDistL * \lambdaDB)
			% I = I_0 * (sin(B)/B)^2 * cos^2(A) = 1 * 2 * (( sin( (pi * \SlitAperture * sin(atan(\y/\ScreenDistL))) / \lambdaDB) / ((pi * \SlitAperture * sin(atan(\y/\ScreenDistL))) / \lambdaDB) )^2 *( (1 + cos(2 *((pi * \dSlitSeparationDistance * sin( atan(\y/\ScreenDistL) )) / \lambdaDB))) / 2 ))
			\draw[SciencePurple,thick,variable=\y,smooth,samples=\nSamples,domain=-\WallTotalSizeinX/2:\WallTotalSizeinX/2] 
			plot({2 * (
				( sin( (pi * \SlitAperture * sin(atan(\y/\ScreenDistL))) / \lambdaDB) / ((pi * \SlitAperture * sin(atan(\y/\ScreenDistL))) / \lambdaDB) )^2 
				*
				( (1 + cos(2 *((pi * \dSlitSeparationDistance * sin( atan(\y/\ScreenDistL) )) / \lambdaDB))) / 2 )
				)
				},\y);
			
		\end{scope}
		
		
		% DETECTOR SYMBOL
		\coordinate (Dectector) at (\ScreenDistL-\WallThickness,\xPositionDetector*\WallTotalSizeinX/2); % placment coordinate for detector
		\DETECTOR{0.2}{0.75}{\ScreenDistL-\WallThickness}{\xPositionDetector*\WallTotalSizeinX/2} % Detector Symbol itself
		
		% draw the arrow and text to show that the detector is moving along the screen, then draw the wee circle for the arrow for nice look
		\draw[<->,darkgray!75] ([shift={(2*\WallThickness,3*\WallThickness)}]Dectector) -- ([shift={(2*\WallThickness,-3*\WallThickness)}]Dectector) node[above=24,right=3,inner sep=3,anchor=south]{\scriptsize Moving}; 
		\fill[color=darkgray!75] ([shift={(2*\WallThickness,0)}]Dectector) circle (0.4*\WallThickness);
		
		
		
	\end{tikzpicture}
\end{document}