% THE 1D QUANTUM WELL -- WAVE FUNCTION SOLUTIONS
% https://tex.stackexchange.com/questions/44252/best-way-to-draw-a-quantum-harmonic-oscillator

\documentclass{standalone}



\IfStandalone{\def\datapath{../../../}}{\def\datapath{}}

\input{\datapath QT-Notes-Preambles/QT-Notes-Diagrams-and-Graphs-DEFAULT-Preamble.tex}

% https://tex.stackexchange.com/questions/441917/is-there-a-simple-way-to-use-stick-figures-into-pgf-tikz-drawings
\usepackage{tikzsymbols}
\usetikzlibrary{shapes.symbols}

\usetikzlibrary{calc}
\usetikzlibrary{arrows,arrows.meta,math}
\usetikzlibrary{decorations.markings}
\usetikzlibrary{angles,quotes} % for pic (angle labels)
\usetikzlibrary{fadings}

% https://tikz.dev/library-3d
\usetikzlibrary{3d}

% https://latexdraw.com/draw-a-sphere-in-latex-using-tikz/
\usepackage{tikz-3dplot}

% Define Color
\tikzstyle{bigphoton}=[-{Latex[length=8,width=6]},red!95!black!50,opacity=0.85,very thin,decorate,decoration={snake,amplitude=2.8,segment length=8,post length=8}]

\contourlength{1.4pt}




\begin{document}
	\begin{tikzpicture}
		\def\xmin{0} % min x axis
		\def\xmax{6}     % max x axis
		\def\L{5}
		\def\BaseE{0.5}
		\def\nSamples{100}
		
		
		\draw[->,thick,dashed] ({\xmin-0.2},0) -- ({\xmax+0.2},0) node[below=1] {\scriptsize $x$};
		
		\draw[->,thick,dashed] (\xmin,-0.2) node[below] {\scriptsize $0$} -- (\xmin,{\BaseE*12}) node[above,left] {\scriptsize $E$, \tiny$V(0\sim\infty)=\infty$} {};
		\draw[->,thick,dashed] (\L,-0.2) node[below] {\scriptsize $L$} -- (\L,{\BaseE*12}) node[above,right] {\scriptsize $E$, \tiny$V(0\sim\infty)=\infty$} {};
		
		\draw ({\L/2},-0.1) node[below] {\tiny $\frac{L}{2}$} -- ({\L/2},0.05) {};
		
		
		\draw ({\xmin-0.15},\BaseE) node[left] {\tiny $E_1$} -- ({\xmin+0.1},\BaseE) {};
		\draw ({\xmin-0.15},{\BaseE*4}) node[left] {\tiny $E_2$} -- ({\xmin+0.1},{\BaseE*4}) {};
		\draw ({\xmin-0.15},{\BaseE*9}) node[left] {\tiny $E_3$} -- ({\xmin+0.1},{\BaseE*9}) {};
		
		
		\draw[ScienceRed,thick] ({\L-0.1},\BaseE) -- ({\L+0.15},\BaseE) node[right=1] {\tiny $\Psi_1(x)$} {};
		\draw[ScienceOrange,thick] ({\L-0.1},{\BaseE*4}) -- ({\L+0.15},{\BaseE*4}) node[right=1] {\tiny $\Psi_2(x)$} {};
		\draw[ScienceGreen,thick] ({\L-0.1},{\BaseE*9}) -- ({\L+0.15},{\BaseE*9}) node[right=1] {\tiny $\Psi_3(x)$}{};
		
		
		
		% INTENSITY FUNCTION PLOT
		\begin{scope}[shift={(0,\BaseE)}]
			\pgfkeys{/pgf/trig format=rad} %% Took me ages to figure tht none of the equations were working cause tex works in degrees by defaut, this sets all trig functions to radians by default
			\draw[thin,dashed] ({\xmin-0.1},0) -- ({\L+0.1},0) {}; % 0 prob, x Axis
			
			\fill[ScienceRed!10,thick,variable=\x,smooth,samples=\nSamples,domain=\xmin:\L] plot(\x,{ (sqrt(2/\L) * (sin((pi / \L) * \x))) });
			\draw[ScienceRed!90,thick,variable=\x,smooth,samples=\nSamples,domain=\xmin:\L] plot(\x,{ (sqrt(2/\L) * (sin((pi / \L) * \x))) });
		\end{scope}
		
		\begin{scope}[shift={(0,{\BaseE*4})}]
			\pgfkeys{/pgf/trig format=rad} %% Took me ages to figure tht none of the equations were working cause tex works in degrees by defaut, this sets all trig functions to radians by default
			\draw[thin,dashed] ({\xmin-0.1},0) -- ({\L+0.1},0) {}; % 0 prob, x Axis
			
			\fill[ScienceOrange!10,thick,variable=\x,smooth,samples=\nSamples,domain=\xmin:\L] plot(\x,{ (sqrt(2/\L) * (sin((pi / \L) * 2 * \x))) });
			\draw[ScienceOrange!90,thick,variable=\x,smooth,samples=\nSamples,domain=\xmin:\L] plot(\x,{ (sqrt(2/\L) * (sin((pi / \L) * 2 * \x))) });
		\end{scope}
		
		\begin{scope}[shift={(0,{\BaseE*9})}]
			\pgfkeys{/pgf/trig format=rad} %% Took me ages to figure tht none of the equations were working cause tex works in degrees by defaut, this sets all trig functions to radians by default
			\draw[thin,dashed] ({\xmin-0.1},0) -- ({\L+0.1},0) {}; % 0 prob, x Axis
			
			\fill[ScienceGreen!10,thick,variable=\x,smooth,samples=\nSamples,domain=\xmin:\L] plot(\x,{ (sqrt(2/\L) * (sin((pi / \L) * 3 * \x))) });
			\draw[ScienceGreen!90,thick,variable=\x,smooth,samples=\nSamples,domain=\xmin:\L] plot(\x,{ (sqrt(2/\L) * (sin((pi / \L) * 3 * \x))) });
		\end{scope}
		
	\end{tikzpicture}
\end{document}