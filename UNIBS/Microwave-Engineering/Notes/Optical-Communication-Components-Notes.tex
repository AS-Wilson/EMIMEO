\documentclass[colorlinks,11pt,a4paper,normalphoto,withhyper,ragged2e]{altareport}


%%%%%%%%%%%%%%%%%%%%%%%%%%%%%%%%%%%%%%%%%%%
%%%%%%%%%% DEFAULT PACKAGES & SETTINGS %%%%%%%%%%
\usepackage[utf8]{inputenc}
\usepackage{setspace} %1.5 line spacing
\usepackage{notoccite} %% Citation numbering
\usepackage{lscape} %% Landscape table
\usepackage{caption} %% Adds a newline in the table caption

%% The paracol package lets you typeset columns of text in parallel
\usepackage{paracol}
\usepackage[none]{hyphenat}

%% Document and Theme Fonts
\usepackage[T1]{fontenc}
\usepackage{paratype}
\usepackage[defaultsans]{lato}
%\usepackage[sfdefault,light,condensed]{roboto}
%\usepackage[rm]{roboto}
%\usepackage[defaultsans]{lato}
%\usepackage{sourcesanspro}
%\usepackage[rm]{merriweather}

\setlength{\intextsep}{4pt} % Set defualt spacing around floats

\captionsetup{font=footnotesize} % Make Captions a sensible size

%%%%%%%%%%%%%%%%%%%%%%%%%%%%%%%%%%%%%%%%%%%


%%%%%%%%%%%%%%%%%%%%%%%%%%%%%%%%%%%%%%%%%%%
%%%%%%%%%% THEMES %%%%%%%%%%

%% Standard theme options are below, leave blank for B&W / no colours (BoringDefault). Note the theme will be set to default if you enter a non-exsistant theme name.
\SetTheme{UNIBS}
%% UNIBS
%% UNILIM
%% PastelBlue
%% GreenAndGold
%% Purple
%% PastelRed
%% BoringDefault (Leave blank / enter anything not found above)

%%%%%%%%%%%%%%%%%%%%%%%%%%%%%%%%%%%%%%%%%%%






%%%%%%%%%%%%%%%%%%%%%%%%%%%%%%%%%%%%%%%%%%%
%%%%%%%%%% DOCUMENT SPECIFIC PACKAGES %%%%%%%%%%

\usepackage{amssymb}
\usepackage{amsfonts}
\usepackage{mathtools}
\usepackage{relsize}

\usepackage{pythontex} % Run python code in this latex doc

%%%%%%%%% Karnaugh Map Package & Settings %%%%%%%%%
\usepackage[export]{adjustbox}

\usetikzlibrary{matrix,calc}
\usepackage{karnaugh-map}

\colorlet{LightRed}{red!60!}
\colorlet{LightBlue}{blue!60!}
\colorlet{LightYellow}{yellow!60!}
\colorlet{LightGreen}{green!60!}
\colorlet{LightOrange}{orange!60!}

%%%%%%%%% MATLAB Language Settings %%%%%%%%%
\usepackage[numbered,framed]{matlab-prettifier} % To add code listings from matlab
\lstMakeShortInline[style=Matlab-editor]" %% This makes " an escape character to write in matlab editor font


%%%%% Settings for python pgf graphs %%%%%
\usepackage{pgfplots}
\usetikzlibrary{arrows.meta}

\pgfplotsset{compat=newest,
    width=6cm,
    height=3cm,
    scale only axis=true,
    max space between ticks=25pt,
    try min ticks=5,
    every axis/.style={
        axis y line=left,
        axis x line=bottom,
        axis line style={thick,->,>=latex, shorten >=-.4cm}
    },
    every axis plot/.append style={thick},
    tick style={black, thick}
}
\tikzset{
    semithick/.style={line width=0.8pt},
}

\usepgfplotslibrary{groupplots}
\usepgfplotslibrary{dateplot}


% Reduce space around captions
% \captionsetup{aboveskip=5pt, belowskip=5pt}
%%%%%%%%%%%%%%%%%%%%%%%%%%%%%%%%%%%%%%%%%%%




%%%%%%%%%%%%%%%%%%%%%%%%%%%%%%%%%%%%%%%%%%%
%%%%%%%%%% USEFUL SETTINGS %%%%%%%%%%
%% Change some font sizes, this will override the defaults
\renewcommand{\ReportTitleFont}{\Huge\rmfamily\bfseries} %% Title Page - Main Title
\renewcommand{\ReportSubTitleFont}{\huge\bfseries} %% Title Page - Sub-Title
\renewcommand{\ReportSectionFont}{\LARGE\rmfamily\bfseries} %% Section Title
\renewcommand{\ReportSubSectionFont}{\large\bfseries} %% SubSection Title
\renewcommand{\FootNoteFont}{\footnotesize} %% Footnotes and Header/Footer

%% Change the bullets for itemize and rating marker
\renewcommand{\itemmarker}{{\small\textbullet}}
\renewcommand{\ratingmarker}{\faCircle}

%% Change the page layout
\geometry{left=1.5cm,right=1.5cm,top=3cm,bottom=3cm,columnsep=8mm}
\onehalfspace   % 1.5 line spacing

\definecolor{CommentGreen}{HTML}{228B22}
%%%%%%%%%%%%%%%%%%%%%%%%%%%%%%%%%%%%%%%%%%%




%%%%%%%%%%%%%%%%%%%%%%%%%%%%%%%%%%%%%%%%%%%
%\include{references.bib}

%%%%%%%%%% TITLE PAGE INFO %%%%%%%%%%
\ReportTitle{Optical Communication Components}
\SubTitle{Notes \& Tutorials}
\Author{Andrew Simon Wilson}
\ReportDate{\today}
\FacultyOrLocation{EMIMEO Programme}
\ModCoord{Prof. Constantion De Angelis}

%%%%%%%%%%%%%%%%%%%%%%%%%%%%%%%%%%%%%%%%%%%


\newcommand*\circled[1]{\tikz[baseline=(char.base)]{
            \node[shape=circle,draw,inner sep=0.5pt] (char) {#1};}}


\begin{document}

\MakeReportTitlePage


%%%%% CONTENTS %%%%%
\pagenumbering{roman} % Start roman numbering
\setcounter{page}{1}


%%%%%%%%%% YOUR NAME, PROFESSION, PORTRAIT, CONTACT INFO, SOCIAL MEDIA ETC. %%%%%%%%%%
\name{Andrew Simon Wilson, BEng}
\tagline{Post-graduate Master's Student - EMIMEO Programme}

\personalinfo{
  \email{andrew.wilson@protonmail.com}
  \linkedin{andrew-simon-wilson} 
  \github{AS-Wilson}
  \phone{+44 7930 560 383}
}

%% You can add multiple photos on the left or right
% \photoR{3cm}{Images/a-wilson-potrait.jpg}
% \photoL{3cm}{Yacht_High,Suitcase_High}


\section*{Author Details}
\makeauthordetails

%% Table of contents print level -1: part, 0: chapter, 1: section, 2:sub-section, 3:sub-sub-section, etc.
\setcounter{tocdepth}{2} 
\tableofcontents %% Prints a list of all sections based on the above command
%\listoffigures %% Prints a list of all figures in the report
%\listoftables %% Prints a list of all tables in the report




%%%%%%%%%% DOCUMENT CONTENT BEGINS HERE %%%%%%%%%%

%%%%% INTRO %%%%%
\section*{Introduction}
I wrote this document for the students studying Optical Communication Networks to have a nice set of notes, and correct reference code and graphs for the module. I hope that it is sufficient for this task and it helps all of your studies. \linebreak
I spent have spent a lot of time developing the template used to make this {\LaTeX} document, I want others to benefit from this work so the source code for this template is available on GitHub \cite{latex_template_github}.
\newpage
\pagenumbering{arabic} % Start document numbering - roman numbering




\section{Introduction}

	\subsection{Introduction}
	The main goal of this class is the investigation of the evolution of the optical pulses which propagate in an optical fibre. \linebreak
	
	We will consider the propagation of pulses from two perspectives:
	\begin{enumerate}[leftmargin=1cm]
		\item Theoretically
		\item Via numerical simulation, using software such as MATLAB, Python, or C/C++
	\end{enumerate}
	
	\vspace{5mm}
	
	We will analyse and understand a number of different optical effects and regimes / models. These include:
	\begin{itemize}[leftmargin=1cm]
		\item Group Velocity Dispersion (GVD, this is a linear effect)
		\item Self-Phase Modulation (this is a non-linear effect)
		\item Optical, Self-Trapped Waves (Solitons)
		\item Abnormal, Extreme Waves
		\item Optical Shocks
	\end{itemize}
	
	\vspace{5mm}
	
	The assessment for the course will consist of a piece of coursework on the numerical dynamics of optical pulses propagating under different linear and non-linear regimes.
	After this coursework is handed in, a type of oral examination will be performed with each student about their coursework. \linebreak
	Each of you is invited to work in groups of 2-3 persons, each with different jobs. \linebreak
	
	Finally, should the lectures leave you with any confusion, or you wish to study further, the suggested textbook for this course to aid in study (if this is needed) is Non-linear Fibre Optics by Govind P. Agrawal \cite{nl_fibre_optics}.


	\pagebreak


	\subsection{The Non Linear Schr\"{o}dinger Equation (3+3D)}
	The first step to considering the propagation of optical pules in a fibre is to know that an optical fibre is a non-linear and dispersive medium and that any propagation with this waveguide is governed by the fundamental and universal modal of optical wave dynamics; the Non Linear Schr\"{o}dinger Equation (NLSE). \linebreak
	
	In Optical Communication Components Prof. Constantino De Angelis will analytically study the properties of the NLSE. This course, however, will consider different regimes from theoretical viewpoints, as well as numerical simulation of each of these regimes to understand the linear and non-liner effects and their uses. \linebreak
	
	So, without further ado, the Non Linear Schr\"{o}dinger Equation (NLSE) (3+3D) is given by Equation \ref{eq:nlse}:
	\begin{equation} \label{eq:nlse}
       j \frac{\partial A(r,t)}{\partial z} + \frac{1}{2 \beta} \frac{\partial^2 A(r,t)}{\partial x^2} + \frac{1}{2 \beta} \frac{\partial^2 A(r,t)}{\partial y^2} - \frac{ \beta ''}{2} \frac{\partial^2 A(r,t)}{\partial t^2} + \chi^{(3)} |A(r,t)|^2 A(r,t) = 0
	\end{equation}
	
	\begin{align}
		\text{Where:}& \nonumber\\
		r & = (x, y, z) \text{, this is the 3-dimensional spatial coordinates} \nonumber\\
		t & \text{, is the time coordinate} \nonumber\\
		A & (r,t) \text{, is the slowly varying (compared to the carrier signal) envelope of the signal} \nonumber\\
		E & (r,t) = Re[A(r,t) e^{i(\omega_0t + \beta_0z)}] \text{, is the electrical field of the pulse and the carrier signal} \nonumber\\
		e&^{{i(\omega_0t + \beta_0z)}} \text{, is the optical carrier signal at the angular frequency } \omega_0 \text{ and `wavenumber' } \beta_0 \nonumber
	\end{align}
	
	
	\pagebreak
	
	
	\subsubsection{A Quick Detour - The Gaussian Pulse}
	The type of communication signal model that we will deal with most often initially in the course is that of the Pulsed Gaussian or Gaussian Pulse. This signal is the combination of a lower frequency Gaussian (this is where the information is really communicated) and a higher frequency carrier sine/cosine component, these individual signals are shown separately in Figure \ref{fig:gaussian_decon}. \linebreak
	
	Please note that we will give the Gaussian as a function most simply described by Equation \ref{eq:gaussian_time_def}:
	
	\begin{equation} \label{eq:gaussian_time_def}
		f(t) = I e^{-\frac{t^2}{2t^2_0}}
	\end{equation}
	
	{\footnotesize Please note $t_0$ is expressed here as the half-waist duration at an intensity of $\frac{1}{e}$, there are other definitions which are useful for other purposes} \linebreak
	
	If one has a linear system (or one that can be approximated as such a type of system) these two signals can be super-positioned to create a Gaussian pulse, which is shown in Figure \ref{fig:gaussian_demo} \linebreak
	
	\begin{figure}[h]
		\centering
		\scalebox{0.85}{%% Creator: Matplotlib, PGF backend
%%
%% To include the figure in your LaTeX document, write
%%   \input{<filename>.pgf}
%%
%% Make sure the required packages are loaded in your preamble
%%   \usepackage{pgf}
%%
%% Also ensure that all the required font packages are loaded; for instance,
%% the lmodern package is sometimes necessary when using math font.
%%   \usepackage{lmodern}
%%
%% Figures using additional raster images can only be included by \input if
%% they are in the same directory as the main LaTeX file. For loading figures
%% from other directories you can use the `import` package
%%   \usepackage{import}
%%
%% and then include the figures with
%%   \import{<path to file>}{<filename>.pgf}
%%
%% Matplotlib used the following preamble
%%   \usepackage[T1]{fontenc} \usepackage{mathpazo}
%%
\begingroup%
\makeatletter%
\begin{pgfpicture}%
\pgfpathrectangle{\pgfpointorigin}{\pgfqpoint{6.104713in}{2.137631in}}%
\pgfusepath{use as bounding box, clip}%
\begin{pgfscope}%
\pgfsetbuttcap%
\pgfsetmiterjoin%
\definecolor{currentfill}{rgb}{1.000000,1.000000,1.000000}%
\pgfsetfillcolor{currentfill}%
\pgfsetlinewidth{0.000000pt}%
\definecolor{currentstroke}{rgb}{1.000000,1.000000,1.000000}%
\pgfsetstrokecolor{currentstroke}%
\pgfsetdash{}{0pt}%
\pgfpathmoveto{\pgfqpoint{0.000000in}{0.000000in}}%
\pgfpathlineto{\pgfqpoint{6.104713in}{0.000000in}}%
\pgfpathlineto{\pgfqpoint{6.104713in}{2.137631in}}%
\pgfpathlineto{\pgfqpoint{0.000000in}{2.137631in}}%
\pgfpathlineto{\pgfqpoint{0.000000in}{0.000000in}}%
\pgfpathclose%
\pgfusepath{fill}%
\end{pgfscope}%
\begin{pgfscope}%
\pgfsetbuttcap%
\pgfsetmiterjoin%
\definecolor{currentfill}{rgb}{0.933333,0.933333,0.933333}%
\pgfsetfillcolor{currentfill}%
\pgfsetlinewidth{0.000000pt}%
\definecolor{currentstroke}{rgb}{0.000000,0.000000,0.000000}%
\pgfsetstrokecolor{currentstroke}%
\pgfsetstrokeopacity{0.000000}%
\pgfsetdash{}{0pt}%
\pgfpathmoveto{\pgfqpoint{0.470474in}{0.429287in}}%
\pgfpathlineto{\pgfqpoint{3.188288in}{0.429287in}}%
\pgfpathlineto{\pgfqpoint{3.188288in}{1.924069in}}%
\pgfpathlineto{\pgfqpoint{0.470474in}{1.924069in}}%
\pgfpathlineto{\pgfqpoint{0.470474in}{0.429287in}}%
\pgfpathclose%
\pgfusepath{fill}%
\end{pgfscope}%
\begin{pgfscope}%
\pgfpathrectangle{\pgfqpoint{0.470474in}{0.429287in}}{\pgfqpoint{2.717814in}{1.494783in}}%
\pgfusepath{clip}%
\pgfsetbuttcap%
\pgfsetroundjoin%
\pgfsetlinewidth{0.501875pt}%
\definecolor{currentstroke}{rgb}{0.698039,0.698039,0.698039}%
\pgfsetstrokecolor{currentstroke}%
\pgfsetdash{{1.850000pt}{0.800000pt}}{0.000000pt}%
\pgfpathmoveto{\pgfqpoint{0.980064in}{0.429287in}}%
\pgfpathlineto{\pgfqpoint{0.980064in}{1.924069in}}%
\pgfusepath{stroke}%
\end{pgfscope}%
\begin{pgfscope}%
\pgfsetbuttcap%
\pgfsetroundjoin%
\definecolor{currentfill}{rgb}{0.180392,0.180392,0.180392}%
\pgfsetfillcolor{currentfill}%
\pgfsetlinewidth{0.803000pt}%
\definecolor{currentstroke}{rgb}{0.180392,0.180392,0.180392}%
\pgfsetstrokecolor{currentstroke}%
\pgfsetdash{}{0pt}%
\pgfsys@defobject{currentmarker}{\pgfqpoint{0.000000in}{-0.048611in}}{\pgfqpoint{0.000000in}{0.000000in}}{%
\pgfpathmoveto{\pgfqpoint{0.000000in}{0.000000in}}%
\pgfpathlineto{\pgfqpoint{0.000000in}{-0.048611in}}%
\pgfusepath{stroke,fill}%
}%
\begin{pgfscope}%
\pgfsys@transformshift{0.980064in}{0.429287in}%
\pgfsys@useobject{currentmarker}{}%
\end{pgfscope}%
\end{pgfscope}%
\begin{pgfscope}%
\definecolor{textcolor}{rgb}{0.180392,0.180392,0.180392}%
\pgfsetstrokecolor{textcolor}%
\pgfsetfillcolor{textcolor}%
\pgftext[x=0.980064in,y=0.332064in,,top]{\color{textcolor}\rmfamily\fontsize{9.000000}{10.800000}\selectfont \(\displaystyle {0}\)}%
\end{pgfscope}%
\begin{pgfscope}%
\pgfpathrectangle{\pgfqpoint{0.470474in}{0.429287in}}{\pgfqpoint{2.717814in}{1.494783in}}%
\pgfusepath{clip}%
\pgfsetbuttcap%
\pgfsetroundjoin%
\pgfsetlinewidth{0.501875pt}%
\definecolor{currentstroke}{rgb}{0.698039,0.698039,0.698039}%
\pgfsetstrokecolor{currentstroke}%
\pgfsetdash{{1.850000pt}{0.800000pt}}{0.000000pt}%
\pgfpathmoveto{\pgfqpoint{1.659518in}{0.429287in}}%
\pgfpathlineto{\pgfqpoint{1.659518in}{1.924069in}}%
\pgfusepath{stroke}%
\end{pgfscope}%
\begin{pgfscope}%
\pgfsetbuttcap%
\pgfsetroundjoin%
\definecolor{currentfill}{rgb}{0.180392,0.180392,0.180392}%
\pgfsetfillcolor{currentfill}%
\pgfsetlinewidth{0.803000pt}%
\definecolor{currentstroke}{rgb}{0.180392,0.180392,0.180392}%
\pgfsetstrokecolor{currentstroke}%
\pgfsetdash{}{0pt}%
\pgfsys@defobject{currentmarker}{\pgfqpoint{0.000000in}{-0.048611in}}{\pgfqpoint{0.000000in}{0.000000in}}{%
\pgfpathmoveto{\pgfqpoint{0.000000in}{0.000000in}}%
\pgfpathlineto{\pgfqpoint{0.000000in}{-0.048611in}}%
\pgfusepath{stroke,fill}%
}%
\begin{pgfscope}%
\pgfsys@transformshift{1.659518in}{0.429287in}%
\pgfsys@useobject{currentmarker}{}%
\end{pgfscope}%
\end{pgfscope}%
\begin{pgfscope}%
\definecolor{textcolor}{rgb}{0.180392,0.180392,0.180392}%
\pgfsetstrokecolor{textcolor}%
\pgfsetfillcolor{textcolor}%
\pgftext[x=1.659518in,y=0.332064in,,top]{\color{textcolor}\rmfamily\fontsize{9.000000}{10.800000}\selectfont \(\displaystyle {2}\)}%
\end{pgfscope}%
\begin{pgfscope}%
\pgfpathrectangle{\pgfqpoint{0.470474in}{0.429287in}}{\pgfqpoint{2.717814in}{1.494783in}}%
\pgfusepath{clip}%
\pgfsetbuttcap%
\pgfsetroundjoin%
\pgfsetlinewidth{0.501875pt}%
\definecolor{currentstroke}{rgb}{0.698039,0.698039,0.698039}%
\pgfsetstrokecolor{currentstroke}%
\pgfsetdash{{1.850000pt}{0.800000pt}}{0.000000pt}%
\pgfpathmoveto{\pgfqpoint{2.338971in}{0.429287in}}%
\pgfpathlineto{\pgfqpoint{2.338971in}{1.924069in}}%
\pgfusepath{stroke}%
\end{pgfscope}%
\begin{pgfscope}%
\pgfsetbuttcap%
\pgfsetroundjoin%
\definecolor{currentfill}{rgb}{0.180392,0.180392,0.180392}%
\pgfsetfillcolor{currentfill}%
\pgfsetlinewidth{0.803000pt}%
\definecolor{currentstroke}{rgb}{0.180392,0.180392,0.180392}%
\pgfsetstrokecolor{currentstroke}%
\pgfsetdash{}{0pt}%
\pgfsys@defobject{currentmarker}{\pgfqpoint{0.000000in}{-0.048611in}}{\pgfqpoint{0.000000in}{0.000000in}}{%
\pgfpathmoveto{\pgfqpoint{0.000000in}{0.000000in}}%
\pgfpathlineto{\pgfqpoint{0.000000in}{-0.048611in}}%
\pgfusepath{stroke,fill}%
}%
\begin{pgfscope}%
\pgfsys@transformshift{2.338971in}{0.429287in}%
\pgfsys@useobject{currentmarker}{}%
\end{pgfscope}%
\end{pgfscope}%
\begin{pgfscope}%
\definecolor{textcolor}{rgb}{0.180392,0.180392,0.180392}%
\pgfsetstrokecolor{textcolor}%
\pgfsetfillcolor{textcolor}%
\pgftext[x=2.338971in,y=0.332064in,,top]{\color{textcolor}\rmfamily\fontsize{9.000000}{10.800000}\selectfont \(\displaystyle {4}\)}%
\end{pgfscope}%
\begin{pgfscope}%
\pgfpathrectangle{\pgfqpoint{0.470474in}{0.429287in}}{\pgfqpoint{2.717814in}{1.494783in}}%
\pgfusepath{clip}%
\pgfsetbuttcap%
\pgfsetroundjoin%
\pgfsetlinewidth{0.501875pt}%
\definecolor{currentstroke}{rgb}{0.698039,0.698039,0.698039}%
\pgfsetstrokecolor{currentstroke}%
\pgfsetdash{{1.850000pt}{0.800000pt}}{0.000000pt}%
\pgfpathmoveto{\pgfqpoint{3.018425in}{0.429287in}}%
\pgfpathlineto{\pgfqpoint{3.018425in}{1.924069in}}%
\pgfusepath{stroke}%
\end{pgfscope}%
\begin{pgfscope}%
\pgfsetbuttcap%
\pgfsetroundjoin%
\definecolor{currentfill}{rgb}{0.180392,0.180392,0.180392}%
\pgfsetfillcolor{currentfill}%
\pgfsetlinewidth{0.803000pt}%
\definecolor{currentstroke}{rgb}{0.180392,0.180392,0.180392}%
\pgfsetstrokecolor{currentstroke}%
\pgfsetdash{}{0pt}%
\pgfsys@defobject{currentmarker}{\pgfqpoint{0.000000in}{-0.048611in}}{\pgfqpoint{0.000000in}{0.000000in}}{%
\pgfpathmoveto{\pgfqpoint{0.000000in}{0.000000in}}%
\pgfpathlineto{\pgfqpoint{0.000000in}{-0.048611in}}%
\pgfusepath{stroke,fill}%
}%
\begin{pgfscope}%
\pgfsys@transformshift{3.018425in}{0.429287in}%
\pgfsys@useobject{currentmarker}{}%
\end{pgfscope}%
\end{pgfscope}%
\begin{pgfscope}%
\definecolor{textcolor}{rgb}{0.180392,0.180392,0.180392}%
\pgfsetstrokecolor{textcolor}%
\pgfsetfillcolor{textcolor}%
\pgftext[x=3.018425in,y=0.332064in,,top]{\color{textcolor}\rmfamily\fontsize{9.000000}{10.800000}\selectfont \(\displaystyle {6}\)}%
\end{pgfscope}%
\begin{pgfscope}%
\definecolor{textcolor}{rgb}{0.180392,0.180392,0.180392}%
\pgfsetstrokecolor{textcolor}%
\pgfsetfillcolor{textcolor}%
\pgftext[x=1.829381in,y=0.150823in,,top]{\color{textcolor}\rmfamily\fontsize{10.800000}{12.960000}\selectfont Time (Seconds)}%
\end{pgfscope}%
\begin{pgfscope}%
\pgfpathrectangle{\pgfqpoint{0.470474in}{0.429287in}}{\pgfqpoint{2.717814in}{1.494783in}}%
\pgfusepath{clip}%
\pgfsetbuttcap%
\pgfsetroundjoin%
\pgfsetlinewidth{0.501875pt}%
\definecolor{currentstroke}{rgb}{0.698039,0.698039,0.698039}%
\pgfsetstrokecolor{currentstroke}%
\pgfsetdash{{1.850000pt}{0.800000pt}}{0.000000pt}%
\pgfpathmoveto{\pgfqpoint{0.470474in}{0.497218in}}%
\pgfpathlineto{\pgfqpoint{3.188288in}{0.497218in}}%
\pgfusepath{stroke}%
\end{pgfscope}%
\begin{pgfscope}%
\pgfsetbuttcap%
\pgfsetroundjoin%
\definecolor{currentfill}{rgb}{0.180392,0.180392,0.180392}%
\pgfsetfillcolor{currentfill}%
\pgfsetlinewidth{0.803000pt}%
\definecolor{currentstroke}{rgb}{0.180392,0.180392,0.180392}%
\pgfsetstrokecolor{currentstroke}%
\pgfsetdash{}{0pt}%
\pgfsys@defobject{currentmarker}{\pgfqpoint{-0.048611in}{0.000000in}}{\pgfqpoint{-0.000000in}{0.000000in}}{%
\pgfpathmoveto{\pgfqpoint{-0.000000in}{0.000000in}}%
\pgfpathlineto{\pgfqpoint{-0.048611in}{0.000000in}}%
\pgfusepath{stroke,fill}%
}%
\begin{pgfscope}%
\pgfsys@transformshift{0.470474in}{0.497218in}%
\pgfsys@useobject{currentmarker}{}%
\end{pgfscope}%
\end{pgfscope}%
\begin{pgfscope}%
\definecolor{textcolor}{rgb}{0.180392,0.180392,0.180392}%
\pgfsetstrokecolor{textcolor}%
\pgfsetfillcolor{textcolor}%
\pgftext[x=0.206378in, y=0.451999in, left, base]{\color{textcolor}\rmfamily\fontsize{9.000000}{10.800000}\selectfont \(\displaystyle {\ensuremath{-}2}\)}%
\end{pgfscope}%
\begin{pgfscope}%
\pgfpathrectangle{\pgfqpoint{0.470474in}{0.429287in}}{\pgfqpoint{2.717814in}{1.494783in}}%
\pgfusepath{clip}%
\pgfsetbuttcap%
\pgfsetroundjoin%
\pgfsetlinewidth{0.501875pt}%
\definecolor{currentstroke}{rgb}{0.698039,0.698039,0.698039}%
\pgfsetstrokecolor{currentstroke}%
\pgfsetdash{{1.850000pt}{0.800000pt}}{0.000000pt}%
\pgfpathmoveto{\pgfqpoint{0.470474in}{0.836944in}}%
\pgfpathlineto{\pgfqpoint{3.188288in}{0.836944in}}%
\pgfusepath{stroke}%
\end{pgfscope}%
\begin{pgfscope}%
\pgfsetbuttcap%
\pgfsetroundjoin%
\definecolor{currentfill}{rgb}{0.180392,0.180392,0.180392}%
\pgfsetfillcolor{currentfill}%
\pgfsetlinewidth{0.803000pt}%
\definecolor{currentstroke}{rgb}{0.180392,0.180392,0.180392}%
\pgfsetstrokecolor{currentstroke}%
\pgfsetdash{}{0pt}%
\pgfsys@defobject{currentmarker}{\pgfqpoint{-0.048611in}{0.000000in}}{\pgfqpoint{-0.000000in}{0.000000in}}{%
\pgfpathmoveto{\pgfqpoint{-0.000000in}{0.000000in}}%
\pgfpathlineto{\pgfqpoint{-0.048611in}{0.000000in}}%
\pgfusepath{stroke,fill}%
}%
\begin{pgfscope}%
\pgfsys@transformshift{0.470474in}{0.836944in}%
\pgfsys@useobject{currentmarker}{}%
\end{pgfscope}%
\end{pgfscope}%
\begin{pgfscope}%
\definecolor{textcolor}{rgb}{0.180392,0.180392,0.180392}%
\pgfsetstrokecolor{textcolor}%
\pgfsetfillcolor{textcolor}%
\pgftext[x=0.206378in, y=0.791726in, left, base]{\color{textcolor}\rmfamily\fontsize{9.000000}{10.800000}\selectfont \(\displaystyle {\ensuremath{-}1}\)}%
\end{pgfscope}%
\begin{pgfscope}%
\pgfpathrectangle{\pgfqpoint{0.470474in}{0.429287in}}{\pgfqpoint{2.717814in}{1.494783in}}%
\pgfusepath{clip}%
\pgfsetbuttcap%
\pgfsetroundjoin%
\pgfsetlinewidth{0.501875pt}%
\definecolor{currentstroke}{rgb}{0.698039,0.698039,0.698039}%
\pgfsetstrokecolor{currentstroke}%
\pgfsetdash{{1.850000pt}{0.800000pt}}{0.000000pt}%
\pgfpathmoveto{\pgfqpoint{0.470474in}{1.176671in}}%
\pgfpathlineto{\pgfqpoint{3.188288in}{1.176671in}}%
\pgfusepath{stroke}%
\end{pgfscope}%
\begin{pgfscope}%
\pgfsetbuttcap%
\pgfsetroundjoin%
\definecolor{currentfill}{rgb}{0.180392,0.180392,0.180392}%
\pgfsetfillcolor{currentfill}%
\pgfsetlinewidth{0.803000pt}%
\definecolor{currentstroke}{rgb}{0.180392,0.180392,0.180392}%
\pgfsetstrokecolor{currentstroke}%
\pgfsetdash{}{0pt}%
\pgfsys@defobject{currentmarker}{\pgfqpoint{-0.048611in}{0.000000in}}{\pgfqpoint{-0.000000in}{0.000000in}}{%
\pgfpathmoveto{\pgfqpoint{-0.000000in}{0.000000in}}%
\pgfpathlineto{\pgfqpoint{-0.048611in}{0.000000in}}%
\pgfusepath{stroke,fill}%
}%
\begin{pgfscope}%
\pgfsys@transformshift{0.470474in}{1.176671in}%
\pgfsys@useobject{currentmarker}{}%
\end{pgfscope}%
\end{pgfscope}%
\begin{pgfscope}%
\definecolor{textcolor}{rgb}{0.180392,0.180392,0.180392}%
\pgfsetstrokecolor{textcolor}%
\pgfsetfillcolor{textcolor}%
\pgftext[x=0.310752in, y=1.131453in, left, base]{\color{textcolor}\rmfamily\fontsize{9.000000}{10.800000}\selectfont \(\displaystyle {0}\)}%
\end{pgfscope}%
\begin{pgfscope}%
\pgfpathrectangle{\pgfqpoint{0.470474in}{0.429287in}}{\pgfqpoint{2.717814in}{1.494783in}}%
\pgfusepath{clip}%
\pgfsetbuttcap%
\pgfsetroundjoin%
\pgfsetlinewidth{0.501875pt}%
\definecolor{currentstroke}{rgb}{0.698039,0.698039,0.698039}%
\pgfsetstrokecolor{currentstroke}%
\pgfsetdash{{1.850000pt}{0.800000pt}}{0.000000pt}%
\pgfpathmoveto{\pgfqpoint{0.470474in}{1.516398in}}%
\pgfpathlineto{\pgfqpoint{3.188288in}{1.516398in}}%
\pgfusepath{stroke}%
\end{pgfscope}%
\begin{pgfscope}%
\pgfsetbuttcap%
\pgfsetroundjoin%
\definecolor{currentfill}{rgb}{0.180392,0.180392,0.180392}%
\pgfsetfillcolor{currentfill}%
\pgfsetlinewidth{0.803000pt}%
\definecolor{currentstroke}{rgb}{0.180392,0.180392,0.180392}%
\pgfsetstrokecolor{currentstroke}%
\pgfsetdash{}{0pt}%
\pgfsys@defobject{currentmarker}{\pgfqpoint{-0.048611in}{0.000000in}}{\pgfqpoint{-0.000000in}{0.000000in}}{%
\pgfpathmoveto{\pgfqpoint{-0.000000in}{0.000000in}}%
\pgfpathlineto{\pgfqpoint{-0.048611in}{0.000000in}}%
\pgfusepath{stroke,fill}%
}%
\begin{pgfscope}%
\pgfsys@transformshift{0.470474in}{1.516398in}%
\pgfsys@useobject{currentmarker}{}%
\end{pgfscope}%
\end{pgfscope}%
\begin{pgfscope}%
\definecolor{textcolor}{rgb}{0.180392,0.180392,0.180392}%
\pgfsetstrokecolor{textcolor}%
\pgfsetfillcolor{textcolor}%
\pgftext[x=0.310752in, y=1.471180in, left, base]{\color{textcolor}\rmfamily\fontsize{9.000000}{10.800000}\selectfont \(\displaystyle {1}\)}%
\end{pgfscope}%
\begin{pgfscope}%
\pgfpathrectangle{\pgfqpoint{0.470474in}{0.429287in}}{\pgfqpoint{2.717814in}{1.494783in}}%
\pgfusepath{clip}%
\pgfsetbuttcap%
\pgfsetroundjoin%
\pgfsetlinewidth{0.501875pt}%
\definecolor{currentstroke}{rgb}{0.698039,0.698039,0.698039}%
\pgfsetstrokecolor{currentstroke}%
\pgfsetdash{{1.850000pt}{0.800000pt}}{0.000000pt}%
\pgfpathmoveto{\pgfqpoint{0.470474in}{1.856125in}}%
\pgfpathlineto{\pgfqpoint{3.188288in}{1.856125in}}%
\pgfusepath{stroke}%
\end{pgfscope}%
\begin{pgfscope}%
\pgfsetbuttcap%
\pgfsetroundjoin%
\definecolor{currentfill}{rgb}{0.180392,0.180392,0.180392}%
\pgfsetfillcolor{currentfill}%
\pgfsetlinewidth{0.803000pt}%
\definecolor{currentstroke}{rgb}{0.180392,0.180392,0.180392}%
\pgfsetstrokecolor{currentstroke}%
\pgfsetdash{}{0pt}%
\pgfsys@defobject{currentmarker}{\pgfqpoint{-0.048611in}{0.000000in}}{\pgfqpoint{-0.000000in}{0.000000in}}{%
\pgfpathmoveto{\pgfqpoint{-0.000000in}{0.000000in}}%
\pgfpathlineto{\pgfqpoint{-0.048611in}{0.000000in}}%
\pgfusepath{stroke,fill}%
}%
\begin{pgfscope}%
\pgfsys@transformshift{0.470474in}{1.856125in}%
\pgfsys@useobject{currentmarker}{}%
\end{pgfscope}%
\end{pgfscope}%
\begin{pgfscope}%
\definecolor{textcolor}{rgb}{0.180392,0.180392,0.180392}%
\pgfsetstrokecolor{textcolor}%
\pgfsetfillcolor{textcolor}%
\pgftext[x=0.310752in, y=1.810906in, left, base]{\color{textcolor}\rmfamily\fontsize{9.000000}{10.800000}\selectfont \(\displaystyle {2}\)}%
\end{pgfscope}%
\begin{pgfscope}%
\definecolor{textcolor}{rgb}{0.180392,0.180392,0.180392}%
\pgfsetstrokecolor{textcolor}%
\pgfsetfillcolor{textcolor}%
\pgftext[x=0.150823in,y=1.176678in,,bottom,rotate=90.000000]{\color{textcolor}\rmfamily\fontsize{10.800000}{12.960000}\selectfont Intensity (Unitless)}%
\end{pgfscope}%
\begin{pgfscope}%
\pgfpathrectangle{\pgfqpoint{0.470474in}{0.429287in}}{\pgfqpoint{2.717814in}{1.494783in}}%
\pgfusepath{clip}%
\pgfsetrectcap%
\pgfsetroundjoin%
\pgfsetlinewidth{2.007500pt}%
\definecolor{currentstroke}{rgb}{0.000000,0.000000,0.000000}%
\pgfsetstrokecolor{currentstroke}%
\pgfsetdash{}{0pt}%
\pgfpathmoveto{\pgfqpoint{0.460474in}{1.805254in}}%
\pgfpathlineto{\pgfqpoint{0.462645in}{1.827830in}}%
\pgfpathlineto{\pgfqpoint{0.470134in}{1.856071in}}%
\pgfpathlineto{\pgfqpoint{0.477623in}{1.832509in}}%
\pgfpathlineto{\pgfqpoint{0.485112in}{1.758942in}}%
\pgfpathlineto{\pgfqpoint{0.492601in}{1.640978in}}%
\pgfpathlineto{\pgfqpoint{0.500090in}{1.487611in}}%
\pgfpathlineto{\pgfqpoint{0.530046in}{0.774880in}}%
\pgfpathlineto{\pgfqpoint{0.537535in}{0.640349in}}%
\pgfpathlineto{\pgfqpoint{0.545024in}{0.546711in}}%
\pgfpathlineto{\pgfqpoint{0.552513in}{0.501106in}}%
\pgfpathlineto{\pgfqpoint{0.560001in}{0.507011in}}%
\pgfpathlineto{\pgfqpoint{0.567490in}{0.563975in}}%
\pgfpathlineto{\pgfqpoint{0.574979in}{0.667657in}}%
\pgfpathlineto{\pgfqpoint{0.582468in}{0.810149in}}%
\pgfpathlineto{\pgfqpoint{0.597446in}{1.165977in}}%
\pgfpathlineto{\pgfqpoint{0.612424in}{1.525005in}}%
\pgfpathlineto{\pgfqpoint{0.619913in}{1.671268in}}%
\pgfpathlineto{\pgfqpoint{0.627402in}{1.779819in}}%
\pgfpathlineto{\pgfqpoint{0.634891in}{1.842382in}}%
\pgfpathlineto{\pgfqpoint{0.642380in}{1.854186in}}%
\pgfpathlineto{\pgfqpoint{0.649869in}{1.814332in}}%
\pgfpathlineto{\pgfqpoint{0.657358in}{1.725857in}}%
\pgfpathlineto{\pgfqpoint{0.664847in}{1.595509in}}%
\pgfpathlineto{\pgfqpoint{0.679825in}{1.251379in}}%
\pgfpathlineto{\pgfqpoint{0.694803in}{0.884899in}}%
\pgfpathlineto{\pgfqpoint{0.702292in}{0.728208in}}%
\pgfpathlineto{\pgfqpoint{0.709781in}{0.605710in}}%
\pgfpathlineto{\pgfqpoint{0.717270in}{0.526747in}}%
\pgfpathlineto{\pgfqpoint{0.724759in}{0.497339in}}%
\pgfpathlineto{\pgfqpoint{0.732248in}{0.519728in}}%
\pgfpathlineto{\pgfqpoint{0.739737in}{0.592208in}}%
\pgfpathlineto{\pgfqpoint{0.747226in}{0.709251in}}%
\pgfpathlineto{\pgfqpoint{0.754715in}{0.861934in}}%
\pgfpathlineto{\pgfqpoint{0.784670in}{1.575005in}}%
\pgfpathlineto{\pgfqpoint{0.792159in}{1.710357in}}%
\pgfpathlineto{\pgfqpoint{0.799648in}{1.805016in}}%
\pgfpathlineto{\pgfqpoint{0.807137in}{1.851766in}}%
\pgfpathlineto{\pgfqpoint{0.814626in}{1.847042in}}%
\pgfpathlineto{\pgfqpoint{0.822115in}{1.791204in}}%
\pgfpathlineto{\pgfqpoint{0.829604in}{1.688509in}}%
\pgfpathlineto{\pgfqpoint{0.837093in}{1.546788in}}%
\pgfpathlineto{\pgfqpoint{0.852071in}{1.191642in}}%
\pgfpathlineto{\pgfqpoint{0.867049in}{0.832017in}}%
\pgfpathlineto{\pgfqpoint{0.874538in}{0.685017in}}%
\pgfpathlineto{\pgfqpoint{0.882027in}{0.575504in}}%
\pgfpathlineto{\pgfqpoint{0.889516in}{0.511829in}}%
\pgfpathlineto{\pgfqpoint{0.897005in}{0.498846in}}%
\pgfpathlineto{\pgfqpoint{0.904494in}{0.537546in}}%
\pgfpathlineto{\pgfqpoint{0.911983in}{0.624977in}}%
\pgfpathlineto{\pgfqpoint{0.919472in}{0.754474in}}%
\pgfpathlineto{\pgfqpoint{0.934450in}{1.097713in}}%
\pgfpathlineto{\pgfqpoint{0.949428in}{1.464574in}}%
\pgfpathlineto{\pgfqpoint{0.956917in}{1.621912in}}%
\pgfpathlineto{\pgfqpoint{0.964406in}{1.745302in}}%
\pgfpathlineto{\pgfqpoint{0.971895in}{1.825335in}}%
\pgfpathlineto{\pgfqpoint{0.979384in}{1.855909in}}%
\pgfpathlineto{\pgfqpoint{0.986872in}{1.834693in}}%
\pgfpathlineto{\pgfqpoint{0.994361in}{1.763305in}}%
\pgfpathlineto{\pgfqpoint{1.001850in}{1.647187in}}%
\pgfpathlineto{\pgfqpoint{1.009339in}{1.495193in}}%
\pgfpathlineto{\pgfqpoint{1.039295in}{0.781811in}}%
\pgfpathlineto{\pgfqpoint{1.046784in}{0.645644in}}%
\pgfpathlineto{\pgfqpoint{1.054273in}{0.549966in}}%
\pgfpathlineto{\pgfqpoint{1.061762in}{0.502073in}}%
\pgfpathlineto{\pgfqpoint{1.069251in}{0.505616in}}%
\pgfpathlineto{\pgfqpoint{1.076740in}{0.560326in}}%
\pgfpathlineto{\pgfqpoint{1.084229in}{0.662030in}}%
\pgfpathlineto{\pgfqpoint{1.091718in}{0.802975in}}%
\pgfpathlineto{\pgfqpoint{1.106696in}{1.157424in}}%
\pgfpathlineto{\pgfqpoint{1.121674in}{1.517632in}}%
\pgfpathlineto{\pgfqpoint{1.129163in}{1.665363in}}%
\pgfpathlineto{\pgfqpoint{1.136652in}{1.775832in}}%
\pgfpathlineto{\pgfqpoint{1.144141in}{1.840618in}}%
\pgfpathlineto{\pgfqpoint{1.151630in}{1.854778in}}%
\pgfpathlineto{\pgfqpoint{1.159119in}{1.817235in}}%
\pgfpathlineto{\pgfqpoint{1.166608in}{1.730851in}}%
\pgfpathlineto{\pgfqpoint{1.174097in}{1.602212in}}%
\pgfpathlineto{\pgfqpoint{1.189075in}{1.259877in}}%
\pgfpathlineto{\pgfqpoint{1.204052in}{0.892648in}}%
\pgfpathlineto{\pgfqpoint{1.211541in}{0.734670in}}%
\pgfpathlineto{\pgfqpoint{1.219030in}{0.610393in}}%
\pgfpathlineto{\pgfqpoint{1.226519in}{0.529293in}}%
\pgfpathlineto{\pgfqpoint{1.234008in}{0.497554in}}%
\pgfpathlineto{\pgfqpoint{1.241497in}{0.517596in}}%
\pgfpathlineto{\pgfqpoint{1.248986in}{0.587891in}}%
\pgfpathlineto{\pgfqpoint{1.256475in}{0.703079in}}%
\pgfpathlineto{\pgfqpoint{1.263964in}{0.854377in}}%
\pgfpathlineto{\pgfqpoint{1.293920in}{1.568043in}}%
\pgfpathlineto{\pgfqpoint{1.301409in}{1.705020in}}%
\pgfpathlineto{\pgfqpoint{1.308898in}{1.801711in}}%
\pgfpathlineto{\pgfqpoint{1.316387in}{1.850745in}}%
\pgfpathlineto{\pgfqpoint{1.323876in}{1.848383in}}%
\pgfpathlineto{\pgfqpoint{1.331365in}{1.794804in}}%
\pgfpathlineto{\pgfqpoint{1.338854in}{1.694095in}}%
\pgfpathlineto{\pgfqpoint{1.346343in}{1.553933in}}%
\pgfpathlineto{\pgfqpoint{1.361321in}{1.200194in}}%
\pgfpathlineto{\pgfqpoint{1.376299in}{0.839417in}}%
\pgfpathlineto{\pgfqpoint{1.383788in}{0.690961in}}%
\pgfpathlineto{\pgfqpoint{1.391277in}{0.579539in}}%
\pgfpathlineto{\pgfqpoint{1.398766in}{0.513647in}}%
\pgfpathlineto{\pgfqpoint{1.406255in}{0.498308in}}%
\pgfpathlineto{\pgfqpoint{1.413744in}{0.534693in}}%
\pgfpathlineto{\pgfqpoint{1.421232in}{0.620027in}}%
\pgfpathlineto{\pgfqpoint{1.428721in}{0.747804in}}%
\pgfpathlineto{\pgfqpoint{1.443699in}{1.089222in}}%
\pgfpathlineto{\pgfqpoint{1.458677in}{1.456802in}}%
\pgfpathlineto{\pgfqpoint{1.466166in}{1.615415in}}%
\pgfpathlineto{\pgfqpoint{1.473655in}{1.740574in}}%
\pgfpathlineto{\pgfqpoint{1.481144in}{1.822737in}}%
\pgfpathlineto{\pgfqpoint{1.488633in}{1.855640in}}%
\pgfpathlineto{\pgfqpoint{1.496122in}{1.836773in}}%
\pgfpathlineto{\pgfqpoint{1.503611in}{1.767574in}}%
\pgfpathlineto{\pgfqpoint{1.511100in}{1.653321in}}%
\pgfpathlineto{\pgfqpoint{1.518589in}{1.502725in}}%
\pgfpathlineto{\pgfqpoint{1.556034in}{0.651023in}}%
\pgfpathlineto{\pgfqpoint{1.563523in}{0.553321in}}%
\pgfpathlineto{\pgfqpoint{1.571012in}{0.503147in}}%
\pgfpathlineto{\pgfqpoint{1.578501in}{0.504329in}}%
\pgfpathlineto{\pgfqpoint{1.585990in}{0.556774in}}%
\pgfpathlineto{\pgfqpoint{1.593479in}{0.656485in}}%
\pgfpathlineto{\pgfqpoint{1.600968in}{0.795859in}}%
\pgfpathlineto{\pgfqpoint{1.615946in}{1.148874in}}%
\pgfpathlineto{\pgfqpoint{1.630924in}{1.510205in}}%
\pgfpathlineto{\pgfqpoint{1.638412in}{1.659380in}}%
\pgfpathlineto{\pgfqpoint{1.645901in}{1.771751in}}%
\pgfpathlineto{\pgfqpoint{1.653390in}{1.838748in}}%
\pgfpathlineto{\pgfqpoint{1.660879in}{1.855263in}}%
\pgfpathlineto{\pgfqpoint{1.668368in}{1.820037in}}%
\pgfpathlineto{\pgfqpoint{1.675857in}{1.735757in}}%
\pgfpathlineto{\pgfqpoint{1.683346in}{1.608847in}}%
\pgfpathlineto{\pgfqpoint{1.698324in}{1.268361in}}%
\pgfpathlineto{\pgfqpoint{1.713302in}{0.900443in}}%
\pgfpathlineto{\pgfqpoint{1.720791in}{0.741203in}}%
\pgfpathlineto{\pgfqpoint{1.728280in}{0.615166in}}%
\pgfpathlineto{\pgfqpoint{1.735769in}{0.531942in}}%
\pgfpathlineto{\pgfqpoint{1.743258in}{0.497877in}}%
\pgfpathlineto{\pgfqpoint{1.750747in}{0.515569in}}%
\pgfpathlineto{\pgfqpoint{1.758236in}{0.583668in}}%
\pgfpathlineto{\pgfqpoint{1.765725in}{0.696982in}}%
\pgfpathlineto{\pgfqpoint{1.773214in}{0.846871in}}%
\pgfpathlineto{\pgfqpoint{1.810659in}{1.699599in}}%
\pgfpathlineto{\pgfqpoint{1.818148in}{1.798307in}}%
\pgfpathlineto{\pgfqpoint{1.825637in}{1.849618in}}%
\pgfpathlineto{\pgfqpoint{1.833126in}{1.849618in}}%
\pgfpathlineto{\pgfqpoint{1.840615in}{1.798307in}}%
\pgfpathlineto{\pgfqpoint{1.848103in}{1.699599in}}%
\pgfpathlineto{\pgfqpoint{1.855592in}{1.561018in}}%
\pgfpathlineto{\pgfqpoint{1.870570in}{1.208742in}}%
\pgfpathlineto{\pgfqpoint{1.885548in}{0.846871in}}%
\pgfpathlineto{\pgfqpoint{1.893037in}{0.696982in}}%
\pgfpathlineto{\pgfqpoint{1.900526in}{0.583668in}}%
\pgfpathlineto{\pgfqpoint{1.908015in}{0.515569in}}%
\pgfpathlineto{\pgfqpoint{1.915504in}{0.497877in}}%
\pgfpathlineto{\pgfqpoint{1.922993in}{0.531942in}}%
\pgfpathlineto{\pgfqpoint{1.930482in}{0.615166in}}%
\pgfpathlineto{\pgfqpoint{1.937971in}{0.741203in}}%
\pgfpathlineto{\pgfqpoint{1.952949in}{1.080745in}}%
\pgfpathlineto{\pgfqpoint{1.967927in}{1.448986in}}%
\pgfpathlineto{\pgfqpoint{1.975416in}{1.608847in}}%
\pgfpathlineto{\pgfqpoint{1.982905in}{1.735757in}}%
\pgfpathlineto{\pgfqpoint{1.990394in}{1.820037in}}%
\pgfpathlineto{\pgfqpoint{1.997883in}{1.855263in}}%
\pgfpathlineto{\pgfqpoint{2.005372in}{1.838748in}}%
\pgfpathlineto{\pgfqpoint{2.012861in}{1.771751in}}%
\pgfpathlineto{\pgfqpoint{2.020350in}{1.659380in}}%
\pgfpathlineto{\pgfqpoint{2.027839in}{1.510205in}}%
\pgfpathlineto{\pgfqpoint{2.065283in}{0.656485in}}%
\pgfpathlineto{\pgfqpoint{2.072772in}{0.556774in}}%
\pgfpathlineto{\pgfqpoint{2.080261in}{0.504329in}}%
\pgfpathlineto{\pgfqpoint{2.087750in}{0.503147in}}%
\pgfpathlineto{\pgfqpoint{2.095239in}{0.553321in}}%
\pgfpathlineto{\pgfqpoint{2.102728in}{0.651023in}}%
\pgfpathlineto{\pgfqpoint{2.110217in}{0.788804in}}%
\pgfpathlineto{\pgfqpoint{2.125195in}{1.140328in}}%
\pgfpathlineto{\pgfqpoint{2.140173in}{1.502725in}}%
\pgfpathlineto{\pgfqpoint{2.147662in}{1.653321in}}%
\pgfpathlineto{\pgfqpoint{2.155151in}{1.767574in}}%
\pgfpathlineto{\pgfqpoint{2.162640in}{1.836773in}}%
\pgfpathlineto{\pgfqpoint{2.170129in}{1.855640in}}%
\pgfpathlineto{\pgfqpoint{2.177618in}{1.822737in}}%
\pgfpathlineto{\pgfqpoint{2.185107in}{1.740574in}}%
\pgfpathlineto{\pgfqpoint{2.192596in}{1.615415in}}%
\pgfpathlineto{\pgfqpoint{2.207574in}{1.276830in}}%
\pgfpathlineto{\pgfqpoint{2.222552in}{0.908281in}}%
\pgfpathlineto{\pgfqpoint{2.230041in}{0.747804in}}%
\pgfpathlineto{\pgfqpoint{2.237530in}{0.620027in}}%
\pgfpathlineto{\pgfqpoint{2.245019in}{0.534693in}}%
\pgfpathlineto{\pgfqpoint{2.252508in}{0.498308in}}%
\pgfpathlineto{\pgfqpoint{2.259997in}{0.513647in}}%
\pgfpathlineto{\pgfqpoint{2.267486in}{0.579539in}}%
\pgfpathlineto{\pgfqpoint{2.274975in}{0.690961in}}%
\pgfpathlineto{\pgfqpoint{2.282463in}{0.839417in}}%
\pgfpathlineto{\pgfqpoint{2.319908in}{1.694095in}}%
\pgfpathlineto{\pgfqpoint{2.327397in}{1.794804in}}%
\pgfpathlineto{\pgfqpoint{2.334886in}{1.848383in}}%
\pgfpathlineto{\pgfqpoint{2.342375in}{1.850745in}}%
\pgfpathlineto{\pgfqpoint{2.349864in}{1.801711in}}%
\pgfpathlineto{\pgfqpoint{2.357353in}{1.705020in}}%
\pgfpathlineto{\pgfqpoint{2.364842in}{1.568043in}}%
\pgfpathlineto{\pgfqpoint{2.379820in}{1.217285in}}%
\pgfpathlineto{\pgfqpoint{2.394798in}{0.854377in}}%
\pgfpathlineto{\pgfqpoint{2.402287in}{0.703079in}}%
\pgfpathlineto{\pgfqpoint{2.409776in}{0.587891in}}%
\pgfpathlineto{\pgfqpoint{2.417265in}{0.517596in}}%
\pgfpathlineto{\pgfqpoint{2.424754in}{0.497554in}}%
\pgfpathlineto{\pgfqpoint{2.432243in}{0.529293in}}%
\pgfpathlineto{\pgfqpoint{2.439732in}{0.610393in}}%
\pgfpathlineto{\pgfqpoint{2.447221in}{0.734670in}}%
\pgfpathlineto{\pgfqpoint{2.462199in}{1.072283in}}%
\pgfpathlineto{\pgfqpoint{2.477177in}{1.441126in}}%
\pgfpathlineto{\pgfqpoint{2.484666in}{1.602212in}}%
\pgfpathlineto{\pgfqpoint{2.492155in}{1.730851in}}%
\pgfpathlineto{\pgfqpoint{2.499643in}{1.817235in}}%
\pgfpathlineto{\pgfqpoint{2.507132in}{1.854778in}}%
\pgfpathlineto{\pgfqpoint{2.514621in}{1.840618in}}%
\pgfpathlineto{\pgfqpoint{2.522110in}{1.775832in}}%
\pgfpathlineto{\pgfqpoint{2.529599in}{1.665363in}}%
\pgfpathlineto{\pgfqpoint{2.537088in}{1.517632in}}%
\pgfpathlineto{\pgfqpoint{2.574533in}{0.662030in}}%
\pgfpathlineto{\pgfqpoint{2.582022in}{0.560326in}}%
\pgfpathlineto{\pgfqpoint{2.589511in}{0.505616in}}%
\pgfpathlineto{\pgfqpoint{2.597000in}{0.502073in}}%
\pgfpathlineto{\pgfqpoint{2.604489in}{0.549966in}}%
\pgfpathlineto{\pgfqpoint{2.611978in}{0.645644in}}%
\pgfpathlineto{\pgfqpoint{2.619467in}{0.781811in}}%
\pgfpathlineto{\pgfqpoint{2.634445in}{1.131788in}}%
\pgfpathlineto{\pgfqpoint{2.649423in}{1.495193in}}%
\pgfpathlineto{\pgfqpoint{2.656912in}{1.647187in}}%
\pgfpathlineto{\pgfqpoint{2.664401in}{1.763305in}}%
\pgfpathlineto{\pgfqpoint{2.671890in}{1.834693in}}%
\pgfpathlineto{\pgfqpoint{2.679379in}{1.855909in}}%
\pgfpathlineto{\pgfqpoint{2.686868in}{1.825335in}}%
\pgfpathlineto{\pgfqpoint{2.694357in}{1.745302in}}%
\pgfpathlineto{\pgfqpoint{2.701846in}{1.621912in}}%
\pgfpathlineto{\pgfqpoint{2.716823in}{1.285284in}}%
\pgfpathlineto{\pgfqpoint{2.731801in}{0.916162in}}%
\pgfpathlineto{\pgfqpoint{2.739290in}{0.754474in}}%
\pgfpathlineto{\pgfqpoint{2.746779in}{0.624977in}}%
\pgfpathlineto{\pgfqpoint{2.754268in}{0.537546in}}%
\pgfpathlineto{\pgfqpoint{2.761757in}{0.498846in}}%
\pgfpathlineto{\pgfqpoint{2.769246in}{0.511829in}}%
\pgfpathlineto{\pgfqpoint{2.776735in}{0.575504in}}%
\pgfpathlineto{\pgfqpoint{2.784224in}{0.685017in}}%
\pgfpathlineto{\pgfqpoint{2.791713in}{0.832017in}}%
\pgfpathlineto{\pgfqpoint{2.829158in}{1.688509in}}%
\pgfpathlineto{\pgfqpoint{2.836647in}{1.791204in}}%
\pgfpathlineto{\pgfqpoint{2.844136in}{1.847042in}}%
\pgfpathlineto{\pgfqpoint{2.851625in}{1.851766in}}%
\pgfpathlineto{\pgfqpoint{2.859114in}{1.805016in}}%
\pgfpathlineto{\pgfqpoint{2.866603in}{1.710357in}}%
\pgfpathlineto{\pgfqpoint{2.874092in}{1.575005in}}%
\pgfpathlineto{\pgfqpoint{2.889070in}{1.225822in}}%
\pgfpathlineto{\pgfqpoint{2.904048in}{0.861934in}}%
\pgfpathlineto{\pgfqpoint{2.911537in}{0.709251in}}%
\pgfpathlineto{\pgfqpoint{2.919026in}{0.592208in}}%
\pgfpathlineto{\pgfqpoint{2.926514in}{0.519728in}}%
\pgfpathlineto{\pgfqpoint{2.934003in}{0.497339in}}%
\pgfpathlineto{\pgfqpoint{2.941492in}{0.526747in}}%
\pgfpathlineto{\pgfqpoint{2.948981in}{0.605710in}}%
\pgfpathlineto{\pgfqpoint{2.956470in}{0.728208in}}%
\pgfpathlineto{\pgfqpoint{2.971448in}{1.063838in}}%
\pgfpathlineto{\pgfqpoint{2.986426in}{1.433225in}}%
\pgfpathlineto{\pgfqpoint{2.993915in}{1.595509in}}%
\pgfpathlineto{\pgfqpoint{3.001404in}{1.725857in}}%
\pgfpathlineto{\pgfqpoint{3.008893in}{1.814332in}}%
\pgfpathlineto{\pgfqpoint{3.016382in}{1.854186in}}%
\pgfpathlineto{\pgfqpoint{3.023871in}{1.842382in}}%
\pgfpathlineto{\pgfqpoint{3.031360in}{1.779819in}}%
\pgfpathlineto{\pgfqpoint{3.038849in}{1.671268in}}%
\pgfpathlineto{\pgfqpoint{3.046338in}{1.525005in}}%
\pgfpathlineto{\pgfqpoint{3.061316in}{1.165977in}}%
\pgfpathlineto{\pgfqpoint{3.076294in}{0.810149in}}%
\pgfpathlineto{\pgfqpoint{3.083783in}{0.667657in}}%
\pgfpathlineto{\pgfqpoint{3.091272in}{0.563975in}}%
\pgfpathlineto{\pgfqpoint{3.098761in}{0.507011in}}%
\pgfpathlineto{\pgfqpoint{3.106250in}{0.501106in}}%
\pgfpathlineto{\pgfqpoint{3.113739in}{0.546711in}}%
\pgfpathlineto{\pgfqpoint{3.121228in}{0.640349in}}%
\pgfpathlineto{\pgfqpoint{3.128717in}{0.774880in}}%
\pgfpathlineto{\pgfqpoint{3.143694in}{1.123255in}}%
\pgfpathlineto{\pgfqpoint{3.158672in}{1.487611in}}%
\pgfpathlineto{\pgfqpoint{3.166161in}{1.640978in}}%
\pgfpathlineto{\pgfqpoint{3.173650in}{1.758942in}}%
\pgfpathlineto{\pgfqpoint{3.181139in}{1.832509in}}%
\pgfpathlineto{\pgfqpoint{3.188628in}{1.856071in}}%
\pgfpathlineto{\pgfqpoint{3.196117in}{1.827830in}}%
\pgfpathlineto{\pgfqpoint{3.198288in}{1.805254in}}%
\pgfpathlineto{\pgfqpoint{3.198288in}{1.805254in}}%
\pgfusepath{stroke}%
\end{pgfscope}%
\begin{pgfscope}%
\pgfsetrectcap%
\pgfsetmiterjoin%
\pgfsetlinewidth{0.803000pt}%
\definecolor{currentstroke}{rgb}{0.737255,0.737255,0.737255}%
\pgfsetstrokecolor{currentstroke}%
\pgfsetdash{}{0pt}%
\pgfpathmoveto{\pgfqpoint{0.470474in}{0.429287in}}%
\pgfpathlineto{\pgfqpoint{0.470474in}{1.924069in}}%
\pgfusepath{stroke}%
\end{pgfscope}%
\begin{pgfscope}%
\pgfsetrectcap%
\pgfsetmiterjoin%
\pgfsetlinewidth{0.803000pt}%
\definecolor{currentstroke}{rgb}{0.737255,0.737255,0.737255}%
\pgfsetstrokecolor{currentstroke}%
\pgfsetdash{}{0pt}%
\pgfpathmoveto{\pgfqpoint{3.188288in}{0.429287in}}%
\pgfpathlineto{\pgfqpoint{3.188288in}{1.924069in}}%
\pgfusepath{stroke}%
\end{pgfscope}%
\begin{pgfscope}%
\pgfsetrectcap%
\pgfsetmiterjoin%
\pgfsetlinewidth{0.803000pt}%
\definecolor{currentstroke}{rgb}{0.737255,0.737255,0.737255}%
\pgfsetstrokecolor{currentstroke}%
\pgfsetdash{}{0pt}%
\pgfpathmoveto{\pgfqpoint{0.470474in}{0.429287in}}%
\pgfpathlineto{\pgfqpoint{3.188288in}{0.429287in}}%
\pgfusepath{stroke}%
\end{pgfscope}%
\begin{pgfscope}%
\pgfsetrectcap%
\pgfsetmiterjoin%
\pgfsetlinewidth{0.803000pt}%
\definecolor{currentstroke}{rgb}{0.737255,0.737255,0.737255}%
\pgfsetstrokecolor{currentstroke}%
\pgfsetdash{}{0pt}%
\pgfpathmoveto{\pgfqpoint{0.470474in}{1.924069in}}%
\pgfpathlineto{\pgfqpoint{3.188288in}{1.924069in}}%
\pgfusepath{stroke}%
\end{pgfscope}%
\begin{pgfscope}%
\definecolor{textcolor}{rgb}{0.180392,0.180392,0.180392}%
\pgfsetstrokecolor{textcolor}%
\pgfsetfillcolor{textcolor}%
\pgftext[x=1.829381in,y=2.007403in,,base]{\color{textcolor}\rmfamily\fontsize{12.960000}{15.552000}\selectfont Carrier Signal}%
\end{pgfscope}%
\begin{pgfscope}%
\pgfsetbuttcap%
\pgfsetmiterjoin%
\definecolor{currentfill}{rgb}{0.933333,0.933333,0.933333}%
\pgfsetfillcolor{currentfill}%
\pgfsetlinewidth{0.000000pt}%
\definecolor{currentstroke}{rgb}{0.000000,0.000000,0.000000}%
\pgfsetstrokecolor{currentstroke}%
\pgfsetstrokeopacity{0.000000}%
\pgfsetdash{}{0pt}%
\pgfpathmoveto{\pgfqpoint{3.386899in}{0.429287in}}%
\pgfpathlineto{\pgfqpoint{6.104713in}{0.429287in}}%
\pgfpathlineto{\pgfqpoint{6.104713in}{1.924069in}}%
\pgfpathlineto{\pgfqpoint{3.386899in}{1.924069in}}%
\pgfpathlineto{\pgfqpoint{3.386899in}{0.429287in}}%
\pgfpathclose%
\pgfusepath{fill}%
\end{pgfscope}%
\begin{pgfscope}%
\pgfpathrectangle{\pgfqpoint{3.386899in}{0.429287in}}{\pgfqpoint{2.717814in}{1.494783in}}%
\pgfusepath{clip}%
\pgfsetbuttcap%
\pgfsetroundjoin%
\pgfsetlinewidth{0.501875pt}%
\definecolor{currentstroke}{rgb}{0.698039,0.698039,0.698039}%
\pgfsetstrokecolor{currentstroke}%
\pgfsetdash{{1.850000pt}{0.800000pt}}{0.000000pt}%
\pgfpathmoveto{\pgfqpoint{3.896489in}{0.429287in}}%
\pgfpathlineto{\pgfqpoint{3.896489in}{1.924069in}}%
\pgfusepath{stroke}%
\end{pgfscope}%
\begin{pgfscope}%
\pgfsetbuttcap%
\pgfsetroundjoin%
\definecolor{currentfill}{rgb}{0.180392,0.180392,0.180392}%
\pgfsetfillcolor{currentfill}%
\pgfsetlinewidth{0.803000pt}%
\definecolor{currentstroke}{rgb}{0.180392,0.180392,0.180392}%
\pgfsetstrokecolor{currentstroke}%
\pgfsetdash{}{0pt}%
\pgfsys@defobject{currentmarker}{\pgfqpoint{0.000000in}{-0.048611in}}{\pgfqpoint{0.000000in}{0.000000in}}{%
\pgfpathmoveto{\pgfqpoint{0.000000in}{0.000000in}}%
\pgfpathlineto{\pgfqpoint{0.000000in}{-0.048611in}}%
\pgfusepath{stroke,fill}%
}%
\begin{pgfscope}%
\pgfsys@transformshift{3.896489in}{0.429287in}%
\pgfsys@useobject{currentmarker}{}%
\end{pgfscope}%
\end{pgfscope}%
\begin{pgfscope}%
\definecolor{textcolor}{rgb}{0.180392,0.180392,0.180392}%
\pgfsetstrokecolor{textcolor}%
\pgfsetfillcolor{textcolor}%
\pgftext[x=3.896489in,y=0.332064in,,top]{\color{textcolor}\rmfamily\fontsize{9.000000}{10.800000}\selectfont \(\displaystyle {0}\)}%
\end{pgfscope}%
\begin{pgfscope}%
\pgfpathrectangle{\pgfqpoint{3.386899in}{0.429287in}}{\pgfqpoint{2.717814in}{1.494783in}}%
\pgfusepath{clip}%
\pgfsetbuttcap%
\pgfsetroundjoin%
\pgfsetlinewidth{0.501875pt}%
\definecolor{currentstroke}{rgb}{0.698039,0.698039,0.698039}%
\pgfsetstrokecolor{currentstroke}%
\pgfsetdash{{1.850000pt}{0.800000pt}}{0.000000pt}%
\pgfpathmoveto{\pgfqpoint{4.575942in}{0.429287in}}%
\pgfpathlineto{\pgfqpoint{4.575942in}{1.924069in}}%
\pgfusepath{stroke}%
\end{pgfscope}%
\begin{pgfscope}%
\pgfsetbuttcap%
\pgfsetroundjoin%
\definecolor{currentfill}{rgb}{0.180392,0.180392,0.180392}%
\pgfsetfillcolor{currentfill}%
\pgfsetlinewidth{0.803000pt}%
\definecolor{currentstroke}{rgb}{0.180392,0.180392,0.180392}%
\pgfsetstrokecolor{currentstroke}%
\pgfsetdash{}{0pt}%
\pgfsys@defobject{currentmarker}{\pgfqpoint{0.000000in}{-0.048611in}}{\pgfqpoint{0.000000in}{0.000000in}}{%
\pgfpathmoveto{\pgfqpoint{0.000000in}{0.000000in}}%
\pgfpathlineto{\pgfqpoint{0.000000in}{-0.048611in}}%
\pgfusepath{stroke,fill}%
}%
\begin{pgfscope}%
\pgfsys@transformshift{4.575942in}{0.429287in}%
\pgfsys@useobject{currentmarker}{}%
\end{pgfscope}%
\end{pgfscope}%
\begin{pgfscope}%
\definecolor{textcolor}{rgb}{0.180392,0.180392,0.180392}%
\pgfsetstrokecolor{textcolor}%
\pgfsetfillcolor{textcolor}%
\pgftext[x=4.575942in,y=0.332064in,,top]{\color{textcolor}\rmfamily\fontsize{9.000000}{10.800000}\selectfont \(\displaystyle {2}\)}%
\end{pgfscope}%
\begin{pgfscope}%
\pgfpathrectangle{\pgfqpoint{3.386899in}{0.429287in}}{\pgfqpoint{2.717814in}{1.494783in}}%
\pgfusepath{clip}%
\pgfsetbuttcap%
\pgfsetroundjoin%
\pgfsetlinewidth{0.501875pt}%
\definecolor{currentstroke}{rgb}{0.698039,0.698039,0.698039}%
\pgfsetstrokecolor{currentstroke}%
\pgfsetdash{{1.850000pt}{0.800000pt}}{0.000000pt}%
\pgfpathmoveto{\pgfqpoint{5.255396in}{0.429287in}}%
\pgfpathlineto{\pgfqpoint{5.255396in}{1.924069in}}%
\pgfusepath{stroke}%
\end{pgfscope}%
\begin{pgfscope}%
\pgfsetbuttcap%
\pgfsetroundjoin%
\definecolor{currentfill}{rgb}{0.180392,0.180392,0.180392}%
\pgfsetfillcolor{currentfill}%
\pgfsetlinewidth{0.803000pt}%
\definecolor{currentstroke}{rgb}{0.180392,0.180392,0.180392}%
\pgfsetstrokecolor{currentstroke}%
\pgfsetdash{}{0pt}%
\pgfsys@defobject{currentmarker}{\pgfqpoint{0.000000in}{-0.048611in}}{\pgfqpoint{0.000000in}{0.000000in}}{%
\pgfpathmoveto{\pgfqpoint{0.000000in}{0.000000in}}%
\pgfpathlineto{\pgfqpoint{0.000000in}{-0.048611in}}%
\pgfusepath{stroke,fill}%
}%
\begin{pgfscope}%
\pgfsys@transformshift{5.255396in}{0.429287in}%
\pgfsys@useobject{currentmarker}{}%
\end{pgfscope}%
\end{pgfscope}%
\begin{pgfscope}%
\definecolor{textcolor}{rgb}{0.180392,0.180392,0.180392}%
\pgfsetstrokecolor{textcolor}%
\pgfsetfillcolor{textcolor}%
\pgftext[x=5.255396in,y=0.332064in,,top]{\color{textcolor}\rmfamily\fontsize{9.000000}{10.800000}\selectfont \(\displaystyle {4}\)}%
\end{pgfscope}%
\begin{pgfscope}%
\pgfpathrectangle{\pgfqpoint{3.386899in}{0.429287in}}{\pgfqpoint{2.717814in}{1.494783in}}%
\pgfusepath{clip}%
\pgfsetbuttcap%
\pgfsetroundjoin%
\pgfsetlinewidth{0.501875pt}%
\definecolor{currentstroke}{rgb}{0.698039,0.698039,0.698039}%
\pgfsetstrokecolor{currentstroke}%
\pgfsetdash{{1.850000pt}{0.800000pt}}{0.000000pt}%
\pgfpathmoveto{\pgfqpoint{5.934849in}{0.429287in}}%
\pgfpathlineto{\pgfqpoint{5.934849in}{1.924069in}}%
\pgfusepath{stroke}%
\end{pgfscope}%
\begin{pgfscope}%
\pgfsetbuttcap%
\pgfsetroundjoin%
\definecolor{currentfill}{rgb}{0.180392,0.180392,0.180392}%
\pgfsetfillcolor{currentfill}%
\pgfsetlinewidth{0.803000pt}%
\definecolor{currentstroke}{rgb}{0.180392,0.180392,0.180392}%
\pgfsetstrokecolor{currentstroke}%
\pgfsetdash{}{0pt}%
\pgfsys@defobject{currentmarker}{\pgfqpoint{0.000000in}{-0.048611in}}{\pgfqpoint{0.000000in}{0.000000in}}{%
\pgfpathmoveto{\pgfqpoint{0.000000in}{0.000000in}}%
\pgfpathlineto{\pgfqpoint{0.000000in}{-0.048611in}}%
\pgfusepath{stroke,fill}%
}%
\begin{pgfscope}%
\pgfsys@transformshift{5.934849in}{0.429287in}%
\pgfsys@useobject{currentmarker}{}%
\end{pgfscope}%
\end{pgfscope}%
\begin{pgfscope}%
\definecolor{textcolor}{rgb}{0.180392,0.180392,0.180392}%
\pgfsetstrokecolor{textcolor}%
\pgfsetfillcolor{textcolor}%
\pgftext[x=5.934849in,y=0.332064in,,top]{\color{textcolor}\rmfamily\fontsize{9.000000}{10.800000}\selectfont \(\displaystyle {6}\)}%
\end{pgfscope}%
\begin{pgfscope}%
\definecolor{textcolor}{rgb}{0.180392,0.180392,0.180392}%
\pgfsetstrokecolor{textcolor}%
\pgfsetfillcolor{textcolor}%
\pgftext[x=4.745806in,y=0.150823in,,top]{\color{textcolor}\rmfamily\fontsize{10.800000}{12.960000}\selectfont Time (Seconds)}%
\end{pgfscope}%
\begin{pgfscope}%
\pgfpathrectangle{\pgfqpoint{3.386899in}{0.429287in}}{\pgfqpoint{2.717814in}{1.494783in}}%
\pgfusepath{clip}%
\pgfsetbuttcap%
\pgfsetroundjoin%
\pgfsetlinewidth{0.501875pt}%
\definecolor{currentstroke}{rgb}{0.698039,0.698039,0.698039}%
\pgfsetstrokecolor{currentstroke}%
\pgfsetdash{{1.850000pt}{0.800000pt}}{0.000000pt}%
\pgfpathmoveto{\pgfqpoint{3.386899in}{0.497218in}}%
\pgfpathlineto{\pgfqpoint{6.104713in}{0.497218in}}%
\pgfusepath{stroke}%
\end{pgfscope}%
\begin{pgfscope}%
\pgfsetbuttcap%
\pgfsetroundjoin%
\definecolor{currentfill}{rgb}{0.180392,0.180392,0.180392}%
\pgfsetfillcolor{currentfill}%
\pgfsetlinewidth{0.803000pt}%
\definecolor{currentstroke}{rgb}{0.180392,0.180392,0.180392}%
\pgfsetstrokecolor{currentstroke}%
\pgfsetdash{}{0pt}%
\pgfsys@defobject{currentmarker}{\pgfqpoint{-0.048611in}{0.000000in}}{\pgfqpoint{-0.000000in}{0.000000in}}{%
\pgfpathmoveto{\pgfqpoint{-0.000000in}{0.000000in}}%
\pgfpathlineto{\pgfqpoint{-0.048611in}{0.000000in}}%
\pgfusepath{stroke,fill}%
}%
\begin{pgfscope}%
\pgfsys@transformshift{3.386899in}{0.497218in}%
\pgfsys@useobject{currentmarker}{}%
\end{pgfscope}%
\end{pgfscope}%
\begin{pgfscope}%
\pgfpathrectangle{\pgfqpoint{3.386899in}{0.429287in}}{\pgfqpoint{2.717814in}{1.494783in}}%
\pgfusepath{clip}%
\pgfsetbuttcap%
\pgfsetroundjoin%
\pgfsetlinewidth{0.501875pt}%
\definecolor{currentstroke}{rgb}{0.698039,0.698039,0.698039}%
\pgfsetstrokecolor{currentstroke}%
\pgfsetdash{{1.850000pt}{0.800000pt}}{0.000000pt}%
\pgfpathmoveto{\pgfqpoint{3.386899in}{0.836944in}}%
\pgfpathlineto{\pgfqpoint{6.104713in}{0.836944in}}%
\pgfusepath{stroke}%
\end{pgfscope}%
\begin{pgfscope}%
\pgfsetbuttcap%
\pgfsetroundjoin%
\definecolor{currentfill}{rgb}{0.180392,0.180392,0.180392}%
\pgfsetfillcolor{currentfill}%
\pgfsetlinewidth{0.803000pt}%
\definecolor{currentstroke}{rgb}{0.180392,0.180392,0.180392}%
\pgfsetstrokecolor{currentstroke}%
\pgfsetdash{}{0pt}%
\pgfsys@defobject{currentmarker}{\pgfqpoint{-0.048611in}{0.000000in}}{\pgfqpoint{-0.000000in}{0.000000in}}{%
\pgfpathmoveto{\pgfqpoint{-0.000000in}{0.000000in}}%
\pgfpathlineto{\pgfqpoint{-0.048611in}{0.000000in}}%
\pgfusepath{stroke,fill}%
}%
\begin{pgfscope}%
\pgfsys@transformshift{3.386899in}{0.836944in}%
\pgfsys@useobject{currentmarker}{}%
\end{pgfscope}%
\end{pgfscope}%
\begin{pgfscope}%
\pgfpathrectangle{\pgfqpoint{3.386899in}{0.429287in}}{\pgfqpoint{2.717814in}{1.494783in}}%
\pgfusepath{clip}%
\pgfsetbuttcap%
\pgfsetroundjoin%
\pgfsetlinewidth{0.501875pt}%
\definecolor{currentstroke}{rgb}{0.698039,0.698039,0.698039}%
\pgfsetstrokecolor{currentstroke}%
\pgfsetdash{{1.850000pt}{0.800000pt}}{0.000000pt}%
\pgfpathmoveto{\pgfqpoint{3.386899in}{1.176671in}}%
\pgfpathlineto{\pgfqpoint{6.104713in}{1.176671in}}%
\pgfusepath{stroke}%
\end{pgfscope}%
\begin{pgfscope}%
\pgfsetbuttcap%
\pgfsetroundjoin%
\definecolor{currentfill}{rgb}{0.180392,0.180392,0.180392}%
\pgfsetfillcolor{currentfill}%
\pgfsetlinewidth{0.803000pt}%
\definecolor{currentstroke}{rgb}{0.180392,0.180392,0.180392}%
\pgfsetstrokecolor{currentstroke}%
\pgfsetdash{}{0pt}%
\pgfsys@defobject{currentmarker}{\pgfqpoint{-0.048611in}{0.000000in}}{\pgfqpoint{-0.000000in}{0.000000in}}{%
\pgfpathmoveto{\pgfqpoint{-0.000000in}{0.000000in}}%
\pgfpathlineto{\pgfqpoint{-0.048611in}{0.000000in}}%
\pgfusepath{stroke,fill}%
}%
\begin{pgfscope}%
\pgfsys@transformshift{3.386899in}{1.176671in}%
\pgfsys@useobject{currentmarker}{}%
\end{pgfscope}%
\end{pgfscope}%
\begin{pgfscope}%
\pgfpathrectangle{\pgfqpoint{3.386899in}{0.429287in}}{\pgfqpoint{2.717814in}{1.494783in}}%
\pgfusepath{clip}%
\pgfsetbuttcap%
\pgfsetroundjoin%
\pgfsetlinewidth{0.501875pt}%
\definecolor{currentstroke}{rgb}{0.698039,0.698039,0.698039}%
\pgfsetstrokecolor{currentstroke}%
\pgfsetdash{{1.850000pt}{0.800000pt}}{0.000000pt}%
\pgfpathmoveto{\pgfqpoint{3.386899in}{1.516398in}}%
\pgfpathlineto{\pgfqpoint{6.104713in}{1.516398in}}%
\pgfusepath{stroke}%
\end{pgfscope}%
\begin{pgfscope}%
\pgfsetbuttcap%
\pgfsetroundjoin%
\definecolor{currentfill}{rgb}{0.180392,0.180392,0.180392}%
\pgfsetfillcolor{currentfill}%
\pgfsetlinewidth{0.803000pt}%
\definecolor{currentstroke}{rgb}{0.180392,0.180392,0.180392}%
\pgfsetstrokecolor{currentstroke}%
\pgfsetdash{}{0pt}%
\pgfsys@defobject{currentmarker}{\pgfqpoint{-0.048611in}{0.000000in}}{\pgfqpoint{-0.000000in}{0.000000in}}{%
\pgfpathmoveto{\pgfqpoint{-0.000000in}{0.000000in}}%
\pgfpathlineto{\pgfqpoint{-0.048611in}{0.000000in}}%
\pgfusepath{stroke,fill}%
}%
\begin{pgfscope}%
\pgfsys@transformshift{3.386899in}{1.516398in}%
\pgfsys@useobject{currentmarker}{}%
\end{pgfscope}%
\end{pgfscope}%
\begin{pgfscope}%
\pgfpathrectangle{\pgfqpoint{3.386899in}{0.429287in}}{\pgfqpoint{2.717814in}{1.494783in}}%
\pgfusepath{clip}%
\pgfsetbuttcap%
\pgfsetroundjoin%
\pgfsetlinewidth{0.501875pt}%
\definecolor{currentstroke}{rgb}{0.698039,0.698039,0.698039}%
\pgfsetstrokecolor{currentstroke}%
\pgfsetdash{{1.850000pt}{0.800000pt}}{0.000000pt}%
\pgfpathmoveto{\pgfqpoint{3.386899in}{1.856125in}}%
\pgfpathlineto{\pgfqpoint{6.104713in}{1.856125in}}%
\pgfusepath{stroke}%
\end{pgfscope}%
\begin{pgfscope}%
\pgfsetbuttcap%
\pgfsetroundjoin%
\definecolor{currentfill}{rgb}{0.180392,0.180392,0.180392}%
\pgfsetfillcolor{currentfill}%
\pgfsetlinewidth{0.803000pt}%
\definecolor{currentstroke}{rgb}{0.180392,0.180392,0.180392}%
\pgfsetstrokecolor{currentstroke}%
\pgfsetdash{}{0pt}%
\pgfsys@defobject{currentmarker}{\pgfqpoint{-0.048611in}{0.000000in}}{\pgfqpoint{-0.000000in}{0.000000in}}{%
\pgfpathmoveto{\pgfqpoint{-0.000000in}{0.000000in}}%
\pgfpathlineto{\pgfqpoint{-0.048611in}{0.000000in}}%
\pgfusepath{stroke,fill}%
}%
\begin{pgfscope}%
\pgfsys@transformshift{3.386899in}{1.856125in}%
\pgfsys@useobject{currentmarker}{}%
\end{pgfscope}%
\end{pgfscope}%
\begin{pgfscope}%
\pgfpathrectangle{\pgfqpoint{3.386899in}{0.429287in}}{\pgfqpoint{2.717814in}{1.494783in}}%
\pgfusepath{clip}%
\pgfsetrectcap%
\pgfsetroundjoin%
\pgfsetlinewidth{2.007500pt}%
\definecolor{currentstroke}{rgb}{0.203922,0.541176,0.741176}%
\pgfsetstrokecolor{currentstroke}%
\pgfsetdash{}{0pt}%
\pgfpathmoveto{\pgfqpoint{3.376899in}{1.176695in}}%
\pgfpathlineto{\pgfqpoint{3.708584in}{1.177904in}}%
\pgfpathlineto{\pgfqpoint{3.820919in}{1.180449in}}%
\pgfpathlineto{\pgfqpoint{3.895808in}{1.184174in}}%
\pgfpathlineto{\pgfqpoint{3.955720in}{1.189214in}}%
\pgfpathlineto{\pgfqpoint{4.008143in}{1.195839in}}%
\pgfpathlineto{\pgfqpoint{4.053076in}{1.203725in}}%
\pgfpathlineto{\pgfqpoint{4.090521in}{1.212245in}}%
\pgfpathlineto{\pgfqpoint{4.127966in}{1.222884in}}%
\pgfpathlineto{\pgfqpoint{4.165411in}{1.235979in}}%
\pgfpathlineto{\pgfqpoint{4.202856in}{1.251865in}}%
\pgfpathlineto{\pgfqpoint{4.232812in}{1.266796in}}%
\pgfpathlineto{\pgfqpoint{4.262767in}{1.283855in}}%
\pgfpathlineto{\pgfqpoint{4.292723in}{1.303156in}}%
\pgfpathlineto{\pgfqpoint{4.322679in}{1.324776in}}%
\pgfpathlineto{\pgfqpoint{4.352635in}{1.348748in}}%
\pgfpathlineto{\pgfqpoint{4.390080in}{1.381985in}}%
\pgfpathlineto{\pgfqpoint{4.427525in}{1.418683in}}%
\pgfpathlineto{\pgfqpoint{4.464970in}{1.458496in}}%
\pgfpathlineto{\pgfqpoint{4.509903in}{1.509626in}}%
\pgfpathlineto{\pgfqpoint{4.577304in}{1.590432in}}%
\pgfpathlineto{\pgfqpoint{4.652194in}{1.679650in}}%
\pgfpathlineto{\pgfqpoint{4.689639in}{1.721219in}}%
\pgfpathlineto{\pgfqpoint{4.719594in}{1.751881in}}%
\pgfpathlineto{\pgfqpoint{4.749550in}{1.779564in}}%
\pgfpathlineto{\pgfqpoint{4.779506in}{1.803684in}}%
\pgfpathlineto{\pgfqpoint{4.801973in}{1.819120in}}%
\pgfpathlineto{\pgfqpoint{4.824440in}{1.832062in}}%
\pgfpathlineto{\pgfqpoint{4.846907in}{1.842348in}}%
\pgfpathlineto{\pgfqpoint{4.869374in}{1.849845in}}%
\pgfpathlineto{\pgfqpoint{4.891841in}{1.854455in}}%
\pgfpathlineto{\pgfqpoint{4.914307in}{1.856119in}}%
\pgfpathlineto{\pgfqpoint{4.936774in}{1.854815in}}%
\pgfpathlineto{\pgfqpoint{4.959241in}{1.850559in}}%
\pgfpathlineto{\pgfqpoint{4.981708in}{1.843408in}}%
\pgfpathlineto{\pgfqpoint{5.004175in}{1.833454in}}%
\pgfpathlineto{\pgfqpoint{5.026642in}{1.820825in}}%
\pgfpathlineto{\pgfqpoint{5.049109in}{1.805682in}}%
\pgfpathlineto{\pgfqpoint{5.071576in}{1.788214in}}%
\pgfpathlineto{\pgfqpoint{5.101532in}{1.761683in}}%
\pgfpathlineto{\pgfqpoint{5.131487in}{1.731969in}}%
\pgfpathlineto{\pgfqpoint{5.168932in}{1.691282in}}%
\pgfpathlineto{\pgfqpoint{5.213866in}{1.638900in}}%
\pgfpathlineto{\pgfqpoint{5.356156in}{1.469829in}}%
\pgfpathlineto{\pgfqpoint{5.401090in}{1.421478in}}%
\pgfpathlineto{\pgfqpoint{5.438535in}{1.384540in}}%
\pgfpathlineto{\pgfqpoint{5.475980in}{1.351044in}}%
\pgfpathlineto{\pgfqpoint{5.505936in}{1.326858in}}%
\pgfpathlineto{\pgfqpoint{5.535892in}{1.305025in}}%
\pgfpathlineto{\pgfqpoint{5.565847in}{1.285516in}}%
\pgfpathlineto{\pgfqpoint{5.595803in}{1.268257in}}%
\pgfpathlineto{\pgfqpoint{5.625759in}{1.253138in}}%
\pgfpathlineto{\pgfqpoint{5.663204in}{1.237036in}}%
\pgfpathlineto{\pgfqpoint{5.700649in}{1.223749in}}%
\pgfpathlineto{\pgfqpoint{5.738094in}{1.212943in}}%
\pgfpathlineto{\pgfqpoint{5.775538in}{1.204280in}}%
\pgfpathlineto{\pgfqpoint{5.820472in}{1.196254in}}%
\pgfpathlineto{\pgfqpoint{5.872895in}{1.189501in}}%
\pgfpathlineto{\pgfqpoint{5.932807in}{1.184356in}}%
\pgfpathlineto{\pgfqpoint{6.000207in}{1.180832in}}%
\pgfpathlineto{\pgfqpoint{6.090075in}{1.178398in}}%
\pgfpathlineto{\pgfqpoint{6.114713in}{1.178012in}}%
\pgfpathlineto{\pgfqpoint{6.114713in}{1.178012in}}%
\pgfusepath{stroke}%
\end{pgfscope}%
\begin{pgfscope}%
\pgfsetrectcap%
\pgfsetmiterjoin%
\pgfsetlinewidth{0.803000pt}%
\definecolor{currentstroke}{rgb}{0.737255,0.737255,0.737255}%
\pgfsetstrokecolor{currentstroke}%
\pgfsetdash{}{0pt}%
\pgfpathmoveto{\pgfqpoint{3.386899in}{0.429287in}}%
\pgfpathlineto{\pgfqpoint{3.386899in}{1.924069in}}%
\pgfusepath{stroke}%
\end{pgfscope}%
\begin{pgfscope}%
\pgfsetrectcap%
\pgfsetmiterjoin%
\pgfsetlinewidth{0.803000pt}%
\definecolor{currentstroke}{rgb}{0.737255,0.737255,0.737255}%
\pgfsetstrokecolor{currentstroke}%
\pgfsetdash{}{0pt}%
\pgfpathmoveto{\pgfqpoint{6.104713in}{0.429287in}}%
\pgfpathlineto{\pgfqpoint{6.104713in}{1.924069in}}%
\pgfusepath{stroke}%
\end{pgfscope}%
\begin{pgfscope}%
\pgfsetrectcap%
\pgfsetmiterjoin%
\pgfsetlinewidth{0.803000pt}%
\definecolor{currentstroke}{rgb}{0.737255,0.737255,0.737255}%
\pgfsetstrokecolor{currentstroke}%
\pgfsetdash{}{0pt}%
\pgfpathmoveto{\pgfqpoint{3.386899in}{0.429287in}}%
\pgfpathlineto{\pgfqpoint{6.104713in}{0.429287in}}%
\pgfusepath{stroke}%
\end{pgfscope}%
\begin{pgfscope}%
\pgfsetrectcap%
\pgfsetmiterjoin%
\pgfsetlinewidth{0.803000pt}%
\definecolor{currentstroke}{rgb}{0.737255,0.737255,0.737255}%
\pgfsetstrokecolor{currentstroke}%
\pgfsetdash{}{0pt}%
\pgfpathmoveto{\pgfqpoint{3.386899in}{1.924069in}}%
\pgfpathlineto{\pgfqpoint{6.104713in}{1.924069in}}%
\pgfusepath{stroke}%
\end{pgfscope}%
\begin{pgfscope}%
\definecolor{textcolor}{rgb}{0.180392,0.180392,0.180392}%
\pgfsetstrokecolor{textcolor}%
\pgfsetfillcolor{textcolor}%
\pgftext[x=4.745806in,y=2.007403in,,base]{\color{textcolor}\rmfamily\fontsize{12.960000}{15.552000}\selectfont |Envelope Signal|}%
\end{pgfscope}%
\end{pgfpicture}%
\makeatother%
\endgroup%
}
		\caption{\centering{An example graph demonstrating the de-constructed elements of a Gaussian pulse, the carrier signal is on the left and the Gaussian on the right}}
		\label{fig:gaussian_decon}
	\end{figure}	
	
	\vspace{5mm}
	
	\begin{figure}[h]
		\centering
		\scalebox{0.85}{%% Creator: Matplotlib, PGF backend
%%
%% To include the figure in your LaTeX document, write
%%   \input{<filename>.pgf}
%%
%% Make sure the required packages are loaded in your preamble
%%   \usepackage{pgf}
%%
%% Also ensure that all the required font packages are loaded; for instance,
%% the lmodern package is sometimes necessary when using math font.
%%   \usepackage{lmodern}
%%
%% Figures using additional raster images can only be included by \input if
%% they are in the same directory as the main LaTeX file. For loading figures
%% from other directories you can use the `import` package
%%   \usepackage{import}
%%
%% and then include the figures with
%%   \import{<path to file>}{<filename>.pgf}
%%
%% Matplotlib used the following preamble
%%   \usepackage[T1]{fontenc} \usepackage{mathpazo}
%%
\begingroup%
\makeatletter%
\begin{pgfpicture}%
\pgfpathrectangle{\pgfpointorigin}{\pgfqpoint{6.104713in}{2.657089in}}%
\pgfusepath{use as bounding box, clip}%
\begin{pgfscope}%
\pgfsetbuttcap%
\pgfsetmiterjoin%
\definecolor{currentfill}{rgb}{1.000000,1.000000,1.000000}%
\pgfsetfillcolor{currentfill}%
\pgfsetlinewidth{0.000000pt}%
\definecolor{currentstroke}{rgb}{1.000000,1.000000,1.000000}%
\pgfsetstrokecolor{currentstroke}%
\pgfsetdash{}{0pt}%
\pgfpathmoveto{\pgfqpoint{0.000000in}{0.000000in}}%
\pgfpathlineto{\pgfqpoint{6.104713in}{0.000000in}}%
\pgfpathlineto{\pgfqpoint{6.104713in}{2.657089in}}%
\pgfpathlineto{\pgfqpoint{0.000000in}{2.657089in}}%
\pgfpathlineto{\pgfqpoint{0.000000in}{0.000000in}}%
\pgfpathclose%
\pgfusepath{fill}%
\end{pgfscope}%
\begin{pgfscope}%
\pgfsetbuttcap%
\pgfsetmiterjoin%
\definecolor{currentfill}{rgb}{0.933333,0.933333,0.933333}%
\pgfsetfillcolor{currentfill}%
\pgfsetlinewidth{0.000000pt}%
\definecolor{currentstroke}{rgb}{0.000000,0.000000,0.000000}%
\pgfsetstrokecolor{currentstroke}%
\pgfsetstrokeopacity{0.000000}%
\pgfsetdash{}{0pt}%
\pgfpathmoveto{\pgfqpoint{0.470474in}{0.429287in}}%
\pgfpathlineto{\pgfqpoint{6.104713in}{0.429287in}}%
\pgfpathlineto{\pgfqpoint{6.104713in}{2.443527in}}%
\pgfpathlineto{\pgfqpoint{0.470474in}{2.443527in}}%
\pgfpathlineto{\pgfqpoint{0.470474in}{0.429287in}}%
\pgfpathclose%
\pgfusepath{fill}%
\end{pgfscope}%
\begin{pgfscope}%
\pgfpathrectangle{\pgfqpoint{0.470474in}{0.429287in}}{\pgfqpoint{5.634238in}{2.014240in}}%
\pgfusepath{clip}%
\pgfsetbuttcap%
\pgfsetroundjoin%
\pgfsetlinewidth{0.501875pt}%
\definecolor{currentstroke}{rgb}{0.698039,0.698039,0.698039}%
\pgfsetstrokecolor{currentstroke}%
\pgfsetdash{{1.850000pt}{0.800000pt}}{0.000000pt}%
\pgfpathmoveto{\pgfqpoint{0.822614in}{0.429287in}}%
\pgfpathlineto{\pgfqpoint{0.822614in}{2.443527in}}%
\pgfusepath{stroke}%
\end{pgfscope}%
\begin{pgfscope}%
\pgfsetbuttcap%
\pgfsetroundjoin%
\definecolor{currentfill}{rgb}{0.180392,0.180392,0.180392}%
\pgfsetfillcolor{currentfill}%
\pgfsetlinewidth{0.803000pt}%
\definecolor{currentstroke}{rgb}{0.180392,0.180392,0.180392}%
\pgfsetstrokecolor{currentstroke}%
\pgfsetdash{}{0pt}%
\pgfsys@defobject{currentmarker}{\pgfqpoint{0.000000in}{-0.048611in}}{\pgfqpoint{0.000000in}{0.000000in}}{%
\pgfpathmoveto{\pgfqpoint{0.000000in}{0.000000in}}%
\pgfpathlineto{\pgfqpoint{0.000000in}{-0.048611in}}%
\pgfusepath{stroke,fill}%
}%
\begin{pgfscope}%
\pgfsys@transformshift{0.822614in}{0.429287in}%
\pgfsys@useobject{currentmarker}{}%
\end{pgfscope}%
\end{pgfscope}%
\begin{pgfscope}%
\definecolor{textcolor}{rgb}{0.180392,0.180392,0.180392}%
\pgfsetstrokecolor{textcolor}%
\pgfsetfillcolor{textcolor}%
\pgftext[x=0.822614in,y=0.332064in,,top]{\color{textcolor}\rmfamily\fontsize{9.000000}{10.800000}\selectfont \(\displaystyle {\ensuremath{-}1}\)}%
\end{pgfscope}%
\begin{pgfscope}%
\pgfpathrectangle{\pgfqpoint{0.470474in}{0.429287in}}{\pgfqpoint{5.634238in}{2.014240in}}%
\pgfusepath{clip}%
\pgfsetbuttcap%
\pgfsetroundjoin%
\pgfsetlinewidth{0.501875pt}%
\definecolor{currentstroke}{rgb}{0.698039,0.698039,0.698039}%
\pgfsetstrokecolor{currentstroke}%
\pgfsetdash{{1.850000pt}{0.800000pt}}{0.000000pt}%
\pgfpathmoveto{\pgfqpoint{1.526894in}{0.429287in}}%
\pgfpathlineto{\pgfqpoint{1.526894in}{2.443527in}}%
\pgfusepath{stroke}%
\end{pgfscope}%
\begin{pgfscope}%
\pgfsetbuttcap%
\pgfsetroundjoin%
\definecolor{currentfill}{rgb}{0.180392,0.180392,0.180392}%
\pgfsetfillcolor{currentfill}%
\pgfsetlinewidth{0.803000pt}%
\definecolor{currentstroke}{rgb}{0.180392,0.180392,0.180392}%
\pgfsetstrokecolor{currentstroke}%
\pgfsetdash{}{0pt}%
\pgfsys@defobject{currentmarker}{\pgfqpoint{0.000000in}{-0.048611in}}{\pgfqpoint{0.000000in}{0.000000in}}{%
\pgfpathmoveto{\pgfqpoint{0.000000in}{0.000000in}}%
\pgfpathlineto{\pgfqpoint{0.000000in}{-0.048611in}}%
\pgfusepath{stroke,fill}%
}%
\begin{pgfscope}%
\pgfsys@transformshift{1.526894in}{0.429287in}%
\pgfsys@useobject{currentmarker}{}%
\end{pgfscope}%
\end{pgfscope}%
\begin{pgfscope}%
\definecolor{textcolor}{rgb}{0.180392,0.180392,0.180392}%
\pgfsetstrokecolor{textcolor}%
\pgfsetfillcolor{textcolor}%
\pgftext[x=1.526894in,y=0.332064in,,top]{\color{textcolor}\rmfamily\fontsize{9.000000}{10.800000}\selectfont \(\displaystyle {0}\)}%
\end{pgfscope}%
\begin{pgfscope}%
\pgfpathrectangle{\pgfqpoint{0.470474in}{0.429287in}}{\pgfqpoint{5.634238in}{2.014240in}}%
\pgfusepath{clip}%
\pgfsetbuttcap%
\pgfsetroundjoin%
\pgfsetlinewidth{0.501875pt}%
\definecolor{currentstroke}{rgb}{0.698039,0.698039,0.698039}%
\pgfsetstrokecolor{currentstroke}%
\pgfsetdash{{1.850000pt}{0.800000pt}}{0.000000pt}%
\pgfpathmoveto{\pgfqpoint{2.231174in}{0.429287in}}%
\pgfpathlineto{\pgfqpoint{2.231174in}{2.443527in}}%
\pgfusepath{stroke}%
\end{pgfscope}%
\begin{pgfscope}%
\pgfsetbuttcap%
\pgfsetroundjoin%
\definecolor{currentfill}{rgb}{0.180392,0.180392,0.180392}%
\pgfsetfillcolor{currentfill}%
\pgfsetlinewidth{0.803000pt}%
\definecolor{currentstroke}{rgb}{0.180392,0.180392,0.180392}%
\pgfsetstrokecolor{currentstroke}%
\pgfsetdash{}{0pt}%
\pgfsys@defobject{currentmarker}{\pgfqpoint{0.000000in}{-0.048611in}}{\pgfqpoint{0.000000in}{0.000000in}}{%
\pgfpathmoveto{\pgfqpoint{0.000000in}{0.000000in}}%
\pgfpathlineto{\pgfqpoint{0.000000in}{-0.048611in}}%
\pgfusepath{stroke,fill}%
}%
\begin{pgfscope}%
\pgfsys@transformshift{2.231174in}{0.429287in}%
\pgfsys@useobject{currentmarker}{}%
\end{pgfscope}%
\end{pgfscope}%
\begin{pgfscope}%
\definecolor{textcolor}{rgb}{0.180392,0.180392,0.180392}%
\pgfsetstrokecolor{textcolor}%
\pgfsetfillcolor{textcolor}%
\pgftext[x=2.231174in,y=0.332064in,,top]{\color{textcolor}\rmfamily\fontsize{9.000000}{10.800000}\selectfont \(\displaystyle {1}\)}%
\end{pgfscope}%
\begin{pgfscope}%
\pgfpathrectangle{\pgfqpoint{0.470474in}{0.429287in}}{\pgfqpoint{5.634238in}{2.014240in}}%
\pgfusepath{clip}%
\pgfsetbuttcap%
\pgfsetroundjoin%
\pgfsetlinewidth{0.501875pt}%
\definecolor{currentstroke}{rgb}{0.698039,0.698039,0.698039}%
\pgfsetstrokecolor{currentstroke}%
\pgfsetdash{{1.850000pt}{0.800000pt}}{0.000000pt}%
\pgfpathmoveto{\pgfqpoint{2.935454in}{0.429287in}}%
\pgfpathlineto{\pgfqpoint{2.935454in}{2.443527in}}%
\pgfusepath{stroke}%
\end{pgfscope}%
\begin{pgfscope}%
\pgfsetbuttcap%
\pgfsetroundjoin%
\definecolor{currentfill}{rgb}{0.180392,0.180392,0.180392}%
\pgfsetfillcolor{currentfill}%
\pgfsetlinewidth{0.803000pt}%
\definecolor{currentstroke}{rgb}{0.180392,0.180392,0.180392}%
\pgfsetstrokecolor{currentstroke}%
\pgfsetdash{}{0pt}%
\pgfsys@defobject{currentmarker}{\pgfqpoint{0.000000in}{-0.048611in}}{\pgfqpoint{0.000000in}{0.000000in}}{%
\pgfpathmoveto{\pgfqpoint{0.000000in}{0.000000in}}%
\pgfpathlineto{\pgfqpoint{0.000000in}{-0.048611in}}%
\pgfusepath{stroke,fill}%
}%
\begin{pgfscope}%
\pgfsys@transformshift{2.935454in}{0.429287in}%
\pgfsys@useobject{currentmarker}{}%
\end{pgfscope}%
\end{pgfscope}%
\begin{pgfscope}%
\definecolor{textcolor}{rgb}{0.180392,0.180392,0.180392}%
\pgfsetstrokecolor{textcolor}%
\pgfsetfillcolor{textcolor}%
\pgftext[x=2.935454in,y=0.332064in,,top]{\color{textcolor}\rmfamily\fontsize{9.000000}{10.800000}\selectfont \(\displaystyle {2}\)}%
\end{pgfscope}%
\begin{pgfscope}%
\pgfpathrectangle{\pgfqpoint{0.470474in}{0.429287in}}{\pgfqpoint{5.634238in}{2.014240in}}%
\pgfusepath{clip}%
\pgfsetbuttcap%
\pgfsetroundjoin%
\pgfsetlinewidth{0.501875pt}%
\definecolor{currentstroke}{rgb}{0.698039,0.698039,0.698039}%
\pgfsetstrokecolor{currentstroke}%
\pgfsetdash{{1.850000pt}{0.800000pt}}{0.000000pt}%
\pgfpathmoveto{\pgfqpoint{3.639733in}{0.429287in}}%
\pgfpathlineto{\pgfqpoint{3.639733in}{2.443527in}}%
\pgfusepath{stroke}%
\end{pgfscope}%
\begin{pgfscope}%
\pgfsetbuttcap%
\pgfsetroundjoin%
\definecolor{currentfill}{rgb}{0.180392,0.180392,0.180392}%
\pgfsetfillcolor{currentfill}%
\pgfsetlinewidth{0.803000pt}%
\definecolor{currentstroke}{rgb}{0.180392,0.180392,0.180392}%
\pgfsetstrokecolor{currentstroke}%
\pgfsetdash{}{0pt}%
\pgfsys@defobject{currentmarker}{\pgfqpoint{0.000000in}{-0.048611in}}{\pgfqpoint{0.000000in}{0.000000in}}{%
\pgfpathmoveto{\pgfqpoint{0.000000in}{0.000000in}}%
\pgfpathlineto{\pgfqpoint{0.000000in}{-0.048611in}}%
\pgfusepath{stroke,fill}%
}%
\begin{pgfscope}%
\pgfsys@transformshift{3.639733in}{0.429287in}%
\pgfsys@useobject{currentmarker}{}%
\end{pgfscope}%
\end{pgfscope}%
\begin{pgfscope}%
\definecolor{textcolor}{rgb}{0.180392,0.180392,0.180392}%
\pgfsetstrokecolor{textcolor}%
\pgfsetfillcolor{textcolor}%
\pgftext[x=3.639733in,y=0.332064in,,top]{\color{textcolor}\rmfamily\fontsize{9.000000}{10.800000}\selectfont \(\displaystyle {3}\)}%
\end{pgfscope}%
\begin{pgfscope}%
\pgfpathrectangle{\pgfqpoint{0.470474in}{0.429287in}}{\pgfqpoint{5.634238in}{2.014240in}}%
\pgfusepath{clip}%
\pgfsetbuttcap%
\pgfsetroundjoin%
\pgfsetlinewidth{0.501875pt}%
\definecolor{currentstroke}{rgb}{0.698039,0.698039,0.698039}%
\pgfsetstrokecolor{currentstroke}%
\pgfsetdash{{1.850000pt}{0.800000pt}}{0.000000pt}%
\pgfpathmoveto{\pgfqpoint{4.344013in}{0.429287in}}%
\pgfpathlineto{\pgfqpoint{4.344013in}{2.443527in}}%
\pgfusepath{stroke}%
\end{pgfscope}%
\begin{pgfscope}%
\pgfsetbuttcap%
\pgfsetroundjoin%
\definecolor{currentfill}{rgb}{0.180392,0.180392,0.180392}%
\pgfsetfillcolor{currentfill}%
\pgfsetlinewidth{0.803000pt}%
\definecolor{currentstroke}{rgb}{0.180392,0.180392,0.180392}%
\pgfsetstrokecolor{currentstroke}%
\pgfsetdash{}{0pt}%
\pgfsys@defobject{currentmarker}{\pgfqpoint{0.000000in}{-0.048611in}}{\pgfqpoint{0.000000in}{0.000000in}}{%
\pgfpathmoveto{\pgfqpoint{0.000000in}{0.000000in}}%
\pgfpathlineto{\pgfqpoint{0.000000in}{-0.048611in}}%
\pgfusepath{stroke,fill}%
}%
\begin{pgfscope}%
\pgfsys@transformshift{4.344013in}{0.429287in}%
\pgfsys@useobject{currentmarker}{}%
\end{pgfscope}%
\end{pgfscope}%
\begin{pgfscope}%
\definecolor{textcolor}{rgb}{0.180392,0.180392,0.180392}%
\pgfsetstrokecolor{textcolor}%
\pgfsetfillcolor{textcolor}%
\pgftext[x=4.344013in,y=0.332064in,,top]{\color{textcolor}\rmfamily\fontsize{9.000000}{10.800000}\selectfont \(\displaystyle {4}\)}%
\end{pgfscope}%
\begin{pgfscope}%
\pgfpathrectangle{\pgfqpoint{0.470474in}{0.429287in}}{\pgfqpoint{5.634238in}{2.014240in}}%
\pgfusepath{clip}%
\pgfsetbuttcap%
\pgfsetroundjoin%
\pgfsetlinewidth{0.501875pt}%
\definecolor{currentstroke}{rgb}{0.698039,0.698039,0.698039}%
\pgfsetstrokecolor{currentstroke}%
\pgfsetdash{{1.850000pt}{0.800000pt}}{0.000000pt}%
\pgfpathmoveto{\pgfqpoint{5.048293in}{0.429287in}}%
\pgfpathlineto{\pgfqpoint{5.048293in}{2.443527in}}%
\pgfusepath{stroke}%
\end{pgfscope}%
\begin{pgfscope}%
\pgfsetbuttcap%
\pgfsetroundjoin%
\definecolor{currentfill}{rgb}{0.180392,0.180392,0.180392}%
\pgfsetfillcolor{currentfill}%
\pgfsetlinewidth{0.803000pt}%
\definecolor{currentstroke}{rgb}{0.180392,0.180392,0.180392}%
\pgfsetstrokecolor{currentstroke}%
\pgfsetdash{}{0pt}%
\pgfsys@defobject{currentmarker}{\pgfqpoint{0.000000in}{-0.048611in}}{\pgfqpoint{0.000000in}{0.000000in}}{%
\pgfpathmoveto{\pgfqpoint{0.000000in}{0.000000in}}%
\pgfpathlineto{\pgfqpoint{0.000000in}{-0.048611in}}%
\pgfusepath{stroke,fill}%
}%
\begin{pgfscope}%
\pgfsys@transformshift{5.048293in}{0.429287in}%
\pgfsys@useobject{currentmarker}{}%
\end{pgfscope}%
\end{pgfscope}%
\begin{pgfscope}%
\definecolor{textcolor}{rgb}{0.180392,0.180392,0.180392}%
\pgfsetstrokecolor{textcolor}%
\pgfsetfillcolor{textcolor}%
\pgftext[x=5.048293in,y=0.332064in,,top]{\color{textcolor}\rmfamily\fontsize{9.000000}{10.800000}\selectfont \(\displaystyle {5}\)}%
\end{pgfscope}%
\begin{pgfscope}%
\pgfpathrectangle{\pgfqpoint{0.470474in}{0.429287in}}{\pgfqpoint{5.634238in}{2.014240in}}%
\pgfusepath{clip}%
\pgfsetbuttcap%
\pgfsetroundjoin%
\pgfsetlinewidth{0.501875pt}%
\definecolor{currentstroke}{rgb}{0.698039,0.698039,0.698039}%
\pgfsetstrokecolor{currentstroke}%
\pgfsetdash{{1.850000pt}{0.800000pt}}{0.000000pt}%
\pgfpathmoveto{\pgfqpoint{5.752573in}{0.429287in}}%
\pgfpathlineto{\pgfqpoint{5.752573in}{2.443527in}}%
\pgfusepath{stroke}%
\end{pgfscope}%
\begin{pgfscope}%
\pgfsetbuttcap%
\pgfsetroundjoin%
\definecolor{currentfill}{rgb}{0.180392,0.180392,0.180392}%
\pgfsetfillcolor{currentfill}%
\pgfsetlinewidth{0.803000pt}%
\definecolor{currentstroke}{rgb}{0.180392,0.180392,0.180392}%
\pgfsetstrokecolor{currentstroke}%
\pgfsetdash{}{0pt}%
\pgfsys@defobject{currentmarker}{\pgfqpoint{0.000000in}{-0.048611in}}{\pgfqpoint{0.000000in}{0.000000in}}{%
\pgfpathmoveto{\pgfqpoint{0.000000in}{0.000000in}}%
\pgfpathlineto{\pgfqpoint{0.000000in}{-0.048611in}}%
\pgfusepath{stroke,fill}%
}%
\begin{pgfscope}%
\pgfsys@transformshift{5.752573in}{0.429287in}%
\pgfsys@useobject{currentmarker}{}%
\end{pgfscope}%
\end{pgfscope}%
\begin{pgfscope}%
\definecolor{textcolor}{rgb}{0.180392,0.180392,0.180392}%
\pgfsetstrokecolor{textcolor}%
\pgfsetfillcolor{textcolor}%
\pgftext[x=5.752573in,y=0.332064in,,top]{\color{textcolor}\rmfamily\fontsize{9.000000}{10.800000}\selectfont \(\displaystyle {6}\)}%
\end{pgfscope}%
\begin{pgfscope}%
\definecolor{textcolor}{rgb}{0.180392,0.180392,0.180392}%
\pgfsetstrokecolor{textcolor}%
\pgfsetfillcolor{textcolor}%
\pgftext[x=3.287593in,y=0.150823in,,top]{\color{textcolor}\rmfamily\fontsize{10.800000}{12.960000}\selectfont Time (Seconds)}%
\end{pgfscope}%
\begin{pgfscope}%
\pgfpathrectangle{\pgfqpoint{0.470474in}{0.429287in}}{\pgfqpoint{5.634238in}{2.014240in}}%
\pgfusepath{clip}%
\pgfsetbuttcap%
\pgfsetroundjoin%
\pgfsetlinewidth{0.501875pt}%
\definecolor{currentstroke}{rgb}{0.698039,0.698039,0.698039}%
\pgfsetstrokecolor{currentstroke}%
\pgfsetdash{{1.850000pt}{0.800000pt}}{0.000000pt}%
\pgfpathmoveto{\pgfqpoint{0.470474in}{0.520843in}}%
\pgfpathlineto{\pgfqpoint{6.104713in}{0.520843in}}%
\pgfusepath{stroke}%
\end{pgfscope}%
\begin{pgfscope}%
\pgfsetbuttcap%
\pgfsetroundjoin%
\definecolor{currentfill}{rgb}{0.180392,0.180392,0.180392}%
\pgfsetfillcolor{currentfill}%
\pgfsetlinewidth{0.803000pt}%
\definecolor{currentstroke}{rgb}{0.180392,0.180392,0.180392}%
\pgfsetstrokecolor{currentstroke}%
\pgfsetdash{}{0pt}%
\pgfsys@defobject{currentmarker}{\pgfqpoint{-0.048611in}{0.000000in}}{\pgfqpoint{-0.000000in}{0.000000in}}{%
\pgfpathmoveto{\pgfqpoint{-0.000000in}{0.000000in}}%
\pgfpathlineto{\pgfqpoint{-0.048611in}{0.000000in}}%
\pgfusepath{stroke,fill}%
}%
\begin{pgfscope}%
\pgfsys@transformshift{0.470474in}{0.520843in}%
\pgfsys@useobject{currentmarker}{}%
\end{pgfscope}%
\end{pgfscope}%
\begin{pgfscope}%
\definecolor{textcolor}{rgb}{0.180392,0.180392,0.180392}%
\pgfsetstrokecolor{textcolor}%
\pgfsetfillcolor{textcolor}%
\pgftext[x=0.206378in, y=0.475625in, left, base]{\color{textcolor}\rmfamily\fontsize{9.000000}{10.800000}\selectfont \(\displaystyle {\ensuremath{-}2}\)}%
\end{pgfscope}%
\begin{pgfscope}%
\pgfpathrectangle{\pgfqpoint{0.470474in}{0.429287in}}{\pgfqpoint{5.634238in}{2.014240in}}%
\pgfusepath{clip}%
\pgfsetbuttcap%
\pgfsetroundjoin%
\pgfsetlinewidth{0.501875pt}%
\definecolor{currentstroke}{rgb}{0.698039,0.698039,0.698039}%
\pgfsetstrokecolor{currentstroke}%
\pgfsetdash{{1.850000pt}{0.800000pt}}{0.000000pt}%
\pgfpathmoveto{\pgfqpoint{0.470474in}{0.978625in}}%
\pgfpathlineto{\pgfqpoint{6.104713in}{0.978625in}}%
\pgfusepath{stroke}%
\end{pgfscope}%
\begin{pgfscope}%
\pgfsetbuttcap%
\pgfsetroundjoin%
\definecolor{currentfill}{rgb}{0.180392,0.180392,0.180392}%
\pgfsetfillcolor{currentfill}%
\pgfsetlinewidth{0.803000pt}%
\definecolor{currentstroke}{rgb}{0.180392,0.180392,0.180392}%
\pgfsetstrokecolor{currentstroke}%
\pgfsetdash{}{0pt}%
\pgfsys@defobject{currentmarker}{\pgfqpoint{-0.048611in}{0.000000in}}{\pgfqpoint{-0.000000in}{0.000000in}}{%
\pgfpathmoveto{\pgfqpoint{-0.000000in}{0.000000in}}%
\pgfpathlineto{\pgfqpoint{-0.048611in}{0.000000in}}%
\pgfusepath{stroke,fill}%
}%
\begin{pgfscope}%
\pgfsys@transformshift{0.470474in}{0.978625in}%
\pgfsys@useobject{currentmarker}{}%
\end{pgfscope}%
\end{pgfscope}%
\begin{pgfscope}%
\definecolor{textcolor}{rgb}{0.180392,0.180392,0.180392}%
\pgfsetstrokecolor{textcolor}%
\pgfsetfillcolor{textcolor}%
\pgftext[x=0.206378in, y=0.933406in, left, base]{\color{textcolor}\rmfamily\fontsize{9.000000}{10.800000}\selectfont \(\displaystyle {\ensuremath{-}1}\)}%
\end{pgfscope}%
\begin{pgfscope}%
\pgfpathrectangle{\pgfqpoint{0.470474in}{0.429287in}}{\pgfqpoint{5.634238in}{2.014240in}}%
\pgfusepath{clip}%
\pgfsetbuttcap%
\pgfsetroundjoin%
\pgfsetlinewidth{0.501875pt}%
\definecolor{currentstroke}{rgb}{0.698039,0.698039,0.698039}%
\pgfsetstrokecolor{currentstroke}%
\pgfsetdash{{1.850000pt}{0.800000pt}}{0.000000pt}%
\pgfpathmoveto{\pgfqpoint{0.470474in}{1.436407in}}%
\pgfpathlineto{\pgfqpoint{6.104713in}{1.436407in}}%
\pgfusepath{stroke}%
\end{pgfscope}%
\begin{pgfscope}%
\pgfsetbuttcap%
\pgfsetroundjoin%
\definecolor{currentfill}{rgb}{0.180392,0.180392,0.180392}%
\pgfsetfillcolor{currentfill}%
\pgfsetlinewidth{0.803000pt}%
\definecolor{currentstroke}{rgb}{0.180392,0.180392,0.180392}%
\pgfsetstrokecolor{currentstroke}%
\pgfsetdash{}{0pt}%
\pgfsys@defobject{currentmarker}{\pgfqpoint{-0.048611in}{0.000000in}}{\pgfqpoint{-0.000000in}{0.000000in}}{%
\pgfpathmoveto{\pgfqpoint{-0.000000in}{0.000000in}}%
\pgfpathlineto{\pgfqpoint{-0.048611in}{0.000000in}}%
\pgfusepath{stroke,fill}%
}%
\begin{pgfscope}%
\pgfsys@transformshift{0.470474in}{1.436407in}%
\pgfsys@useobject{currentmarker}{}%
\end{pgfscope}%
\end{pgfscope}%
\begin{pgfscope}%
\definecolor{textcolor}{rgb}{0.180392,0.180392,0.180392}%
\pgfsetstrokecolor{textcolor}%
\pgfsetfillcolor{textcolor}%
\pgftext[x=0.310752in, y=1.391188in, left, base]{\color{textcolor}\rmfamily\fontsize{9.000000}{10.800000}\selectfont \(\displaystyle {0}\)}%
\end{pgfscope}%
\begin{pgfscope}%
\pgfpathrectangle{\pgfqpoint{0.470474in}{0.429287in}}{\pgfqpoint{5.634238in}{2.014240in}}%
\pgfusepath{clip}%
\pgfsetbuttcap%
\pgfsetroundjoin%
\pgfsetlinewidth{0.501875pt}%
\definecolor{currentstroke}{rgb}{0.698039,0.698039,0.698039}%
\pgfsetstrokecolor{currentstroke}%
\pgfsetdash{{1.850000pt}{0.800000pt}}{0.000000pt}%
\pgfpathmoveto{\pgfqpoint{0.470474in}{1.894188in}}%
\pgfpathlineto{\pgfqpoint{6.104713in}{1.894188in}}%
\pgfusepath{stroke}%
\end{pgfscope}%
\begin{pgfscope}%
\pgfsetbuttcap%
\pgfsetroundjoin%
\definecolor{currentfill}{rgb}{0.180392,0.180392,0.180392}%
\pgfsetfillcolor{currentfill}%
\pgfsetlinewidth{0.803000pt}%
\definecolor{currentstroke}{rgb}{0.180392,0.180392,0.180392}%
\pgfsetstrokecolor{currentstroke}%
\pgfsetdash{}{0pt}%
\pgfsys@defobject{currentmarker}{\pgfqpoint{-0.048611in}{0.000000in}}{\pgfqpoint{-0.000000in}{0.000000in}}{%
\pgfpathmoveto{\pgfqpoint{-0.000000in}{0.000000in}}%
\pgfpathlineto{\pgfqpoint{-0.048611in}{0.000000in}}%
\pgfusepath{stroke,fill}%
}%
\begin{pgfscope}%
\pgfsys@transformshift{0.470474in}{1.894188in}%
\pgfsys@useobject{currentmarker}{}%
\end{pgfscope}%
\end{pgfscope}%
\begin{pgfscope}%
\definecolor{textcolor}{rgb}{0.180392,0.180392,0.180392}%
\pgfsetstrokecolor{textcolor}%
\pgfsetfillcolor{textcolor}%
\pgftext[x=0.310752in, y=1.848970in, left, base]{\color{textcolor}\rmfamily\fontsize{9.000000}{10.800000}\selectfont \(\displaystyle {1}\)}%
\end{pgfscope}%
\begin{pgfscope}%
\pgfpathrectangle{\pgfqpoint{0.470474in}{0.429287in}}{\pgfqpoint{5.634238in}{2.014240in}}%
\pgfusepath{clip}%
\pgfsetbuttcap%
\pgfsetroundjoin%
\pgfsetlinewidth{0.501875pt}%
\definecolor{currentstroke}{rgb}{0.698039,0.698039,0.698039}%
\pgfsetstrokecolor{currentstroke}%
\pgfsetdash{{1.850000pt}{0.800000pt}}{0.000000pt}%
\pgfpathmoveto{\pgfqpoint{0.470474in}{2.351970in}}%
\pgfpathlineto{\pgfqpoint{6.104713in}{2.351970in}}%
\pgfusepath{stroke}%
\end{pgfscope}%
\begin{pgfscope}%
\pgfsetbuttcap%
\pgfsetroundjoin%
\definecolor{currentfill}{rgb}{0.180392,0.180392,0.180392}%
\pgfsetfillcolor{currentfill}%
\pgfsetlinewidth{0.803000pt}%
\definecolor{currentstroke}{rgb}{0.180392,0.180392,0.180392}%
\pgfsetstrokecolor{currentstroke}%
\pgfsetdash{}{0pt}%
\pgfsys@defobject{currentmarker}{\pgfqpoint{-0.048611in}{0.000000in}}{\pgfqpoint{-0.000000in}{0.000000in}}{%
\pgfpathmoveto{\pgfqpoint{-0.000000in}{0.000000in}}%
\pgfpathlineto{\pgfqpoint{-0.048611in}{0.000000in}}%
\pgfusepath{stroke,fill}%
}%
\begin{pgfscope}%
\pgfsys@transformshift{0.470474in}{2.351970in}%
\pgfsys@useobject{currentmarker}{}%
\end{pgfscope}%
\end{pgfscope}%
\begin{pgfscope}%
\definecolor{textcolor}{rgb}{0.180392,0.180392,0.180392}%
\pgfsetstrokecolor{textcolor}%
\pgfsetfillcolor{textcolor}%
\pgftext[x=0.310752in, y=2.306752in, left, base]{\color{textcolor}\rmfamily\fontsize{9.000000}{10.800000}\selectfont \(\displaystyle {2}\)}%
\end{pgfscope}%
\begin{pgfscope}%
\definecolor{textcolor}{rgb}{0.180392,0.180392,0.180392}%
\pgfsetstrokecolor{textcolor}%
\pgfsetfillcolor{textcolor}%
\pgftext[x=0.150823in,y=1.436407in,,bottom,rotate=90.000000]{\color{textcolor}\rmfamily\fontsize{10.800000}{12.960000}\selectfont Intensity (Unitless)}%
\end{pgfscope}%
\begin{pgfscope}%
\pgfpathrectangle{\pgfqpoint{0.470474in}{0.429287in}}{\pgfqpoint{5.634238in}{2.014240in}}%
\pgfusepath{clip}%
\pgfsetrectcap%
\pgfsetroundjoin%
\pgfsetlinewidth{2.007500pt}%
\definecolor{currentstroke}{rgb}{0.000000,0.501961,0.000000}%
\pgfsetstrokecolor{currentstroke}%
\pgfsetdash{}{0pt}%
\pgfpathmoveto{\pgfqpoint{0.470471in}{1.436407in}}%
\pgfpathlineto{\pgfqpoint{1.279336in}{1.435629in}}%
\pgfpathlineto{\pgfqpoint{1.320290in}{1.434883in}}%
\pgfpathlineto{\pgfqpoint{1.355060in}{1.436715in}}%
\pgfpathlineto{\pgfqpoint{1.400029in}{1.438914in}}%
\pgfpathlineto{\pgfqpoint{1.426038in}{1.437685in}}%
\pgfpathlineto{\pgfqpoint{1.495557in}{1.432786in}}%
\pgfpathlineto{\pgfqpoint{1.517143in}{1.434889in}}%
\pgfpathlineto{\pgfqpoint{1.575965in}{1.442138in}}%
\pgfpathlineto{\pgfqpoint{1.593072in}{1.440832in}}%
\pgfpathlineto{\pgfqpoint{1.612573in}{1.436983in}}%
\pgfpathlineto{\pgfqpoint{1.652802in}{1.428409in}}%
\pgfpathlineto{\pgfqpoint{1.667514in}{1.428077in}}%
\pgfpathlineto{\pgfqpoint{1.681649in}{1.430006in}}%
\pgfpathlineto{\pgfqpoint{1.697728in}{1.434559in}}%
\pgfpathlineto{\pgfqpoint{1.740703in}{1.448036in}}%
\pgfpathlineto{\pgfqpoint{1.752992in}{1.448674in}}%
\pgfpathlineto{\pgfqpoint{1.764676in}{1.447044in}}%
\pgfpathlineto{\pgfqpoint{1.777142in}{1.442991in}}%
\pgfpathlineto{\pgfqpoint{1.793341in}{1.435169in}}%
\pgfpathlineto{\pgfqpoint{1.820181in}{1.422142in}}%
\pgfpathlineto{\pgfqpoint{1.831745in}{1.419213in}}%
\pgfpathlineto{\pgfqpoint{1.841837in}{1.418915in}}%
\pgfpathlineto{\pgfqpoint{1.851627in}{1.420866in}}%
\pgfpathlineto{\pgfqpoint{1.862036in}{1.425277in}}%
\pgfpathlineto{\pgfqpoint{1.874600in}{1.433175in}}%
\pgfpathlineto{\pgfqpoint{1.911737in}{1.458189in}}%
\pgfpathlineto{\pgfqpoint{1.921182in}{1.460911in}}%
\pgfpathlineto{\pgfqpoint{1.929640in}{1.461072in}}%
\pgfpathlineto{\pgfqpoint{1.937908in}{1.458985in}}%
\pgfpathlineto{\pgfqpoint{1.946669in}{1.454406in}}%
\pgfpathlineto{\pgfqpoint{1.956846in}{1.446434in}}%
\pgfpathlineto{\pgfqpoint{1.971333in}{1.431909in}}%
\pgfpathlineto{\pgfqpoint{1.992117in}{1.411433in}}%
\pgfpathlineto{\pgfqpoint{2.001864in}{1.404984in}}%
\pgfpathlineto{\pgfqpoint{2.009836in}{1.402256in}}%
\pgfpathlineto{\pgfqpoint{2.017020in}{1.402101in}}%
\pgfpathlineto{\pgfqpoint{2.024056in}{1.404199in}}%
\pgfpathlineto{\pgfqpoint{2.031521in}{1.408834in}}%
\pgfpathlineto{\pgfqpoint{2.040050in}{1.416896in}}%
\pgfpathlineto{\pgfqpoint{2.050861in}{1.430429in}}%
\pgfpathlineto{\pgfqpoint{2.085004in}{1.475546in}}%
\pgfpathlineto{\pgfqpoint{2.092934in}{1.481234in}}%
\pgfpathlineto{\pgfqpoint{2.099604in}{1.483417in}}%
\pgfpathlineto{\pgfqpoint{2.105689in}{1.483109in}}%
\pgfpathlineto{\pgfqpoint{2.111767in}{1.480529in}}%
\pgfpathlineto{\pgfqpoint{2.118359in}{1.475197in}}%
\pgfpathlineto{\pgfqpoint{2.125986in}{1.465983in}}%
\pgfpathlineto{\pgfqpoint{2.135487in}{1.450744in}}%
\pgfpathlineto{\pgfqpoint{2.151044in}{1.420809in}}%
\pgfpathlineto{\pgfqpoint{2.166623in}{1.392535in}}%
\pgfpathlineto{\pgfqpoint{2.175462in}{1.380848in}}%
\pgfpathlineto{\pgfqpoint{2.182350in}{1.375147in}}%
\pgfpathlineto{\pgfqpoint{2.188125in}{1.373081in}}%
\pgfpathlineto{\pgfqpoint{2.193365in}{1.373521in}}%
\pgfpathlineto{\pgfqpoint{2.198654in}{1.376266in}}%
\pgfpathlineto{\pgfqpoint{2.204492in}{1.381953in}}%
\pgfpathlineto{\pgfqpoint{2.211303in}{1.391910in}}%
\pgfpathlineto{\pgfqpoint{2.219669in}{1.408388in}}%
\pgfpathlineto{\pgfqpoint{2.231332in}{1.436819in}}%
\pgfpathlineto{\pgfqpoint{2.255229in}{1.495903in}}%
\pgfpathlineto{\pgfqpoint{2.263588in}{1.510291in}}%
\pgfpathlineto{\pgfqpoint{2.270054in}{1.517376in}}%
\pgfpathlineto{\pgfqpoint{2.275329in}{1.520118in}}%
\pgfpathlineto{\pgfqpoint{2.279949in}{1.520106in}}%
\pgfpathlineto{\pgfqpoint{2.284527in}{1.517802in}}%
\pgfpathlineto{\pgfqpoint{2.289590in}{1.512595in}}%
\pgfpathlineto{\pgfqpoint{2.295527in}{1.503052in}}%
\pgfpathlineto{\pgfqpoint{2.302725in}{1.486945in}}%
\pgfpathlineto{\pgfqpoint{2.311979in}{1.460307in}}%
\pgfpathlineto{\pgfqpoint{2.328255in}{1.405192in}}%
\pgfpathlineto{\pgfqpoint{2.342249in}{1.361245in}}%
\pgfpathlineto{\pgfqpoint{2.350651in}{1.341647in}}%
\pgfpathlineto{\pgfqpoint{2.357060in}{1.331813in}}%
\pgfpathlineto{\pgfqpoint{2.362131in}{1.327704in}}%
\pgfpathlineto{\pgfqpoint{2.366343in}{1.326952in}}%
\pgfpathlineto{\pgfqpoint{2.370315in}{1.328534in}}%
\pgfpathlineto{\pgfqpoint{2.374667in}{1.332843in}}%
\pgfpathlineto{\pgfqpoint{2.379795in}{1.341319in}}%
\pgfpathlineto{\pgfqpoint{2.385999in}{1.356237in}}%
\pgfpathlineto{\pgfqpoint{2.393711in}{1.381047in}}%
\pgfpathlineto{\pgfqpoint{2.404163in}{1.422929in}}%
\pgfpathlineto{\pgfqpoint{2.432813in}{1.541731in}}%
\pgfpathlineto{\pgfqpoint{2.440419in}{1.562352in}}%
\pgfpathlineto{\pgfqpoint{2.446264in}{1.572431in}}%
\pgfpathlineto{\pgfqpoint{2.450821in}{1.576352in}}%
\pgfpathlineto{\pgfqpoint{2.454526in}{1.576843in}}%
\pgfpathlineto{\pgfqpoint{2.458054in}{1.575001in}}%
\pgfpathlineto{\pgfqpoint{2.462054in}{1.570170in}}%
\pgfpathlineto{\pgfqpoint{2.466879in}{1.560524in}}%
\pgfpathlineto{\pgfqpoint{2.472781in}{1.543308in}}%
\pgfpathlineto{\pgfqpoint{2.480119in}{1.514468in}}%
\pgfpathlineto{\pgfqpoint{2.489866in}{1.466272in}}%
\pgfpathlineto{\pgfqpoint{2.522946in}{1.294698in}}%
\pgfpathlineto{\pgfqpoint{2.529933in}{1.273058in}}%
\pgfpathlineto{\pgfqpoint{2.535300in}{1.262858in}}%
\pgfpathlineto{\pgfqpoint{2.539413in}{1.259211in}}%
\pgfpathlineto{\pgfqpoint{2.542680in}{1.259017in}}%
\pgfpathlineto{\pgfqpoint{2.545864in}{1.261176in}}%
\pgfpathlineto{\pgfqpoint{2.549603in}{1.266686in}}%
\pgfpathlineto{\pgfqpoint{2.554209in}{1.277826in}}%
\pgfpathlineto{\pgfqpoint{2.559900in}{1.297932in}}%
\pgfpathlineto{\pgfqpoint{2.566999in}{1.331886in}}%
\pgfpathlineto{\pgfqpoint{2.576324in}{1.388372in}}%
\pgfpathlineto{\pgfqpoint{2.596466in}{1.528927in}}%
\pgfpathlineto{\pgfqpoint{2.607763in}{1.597467in}}%
\pgfpathlineto{\pgfqpoint{2.615454in}{1.631691in}}%
\pgfpathlineto{\pgfqpoint{2.621306in}{1.648695in}}%
\pgfpathlineto{\pgfqpoint{2.625757in}{1.655689in}}%
\pgfpathlineto{\pgfqpoint{2.629089in}{1.657360in}}%
\pgfpathlineto{\pgfqpoint{2.631885in}{1.656344in}}%
\pgfpathlineto{\pgfqpoint{2.635012in}{1.652566in}}%
\pgfpathlineto{\pgfqpoint{2.638906in}{1.643974in}}%
\pgfpathlineto{\pgfqpoint{2.643773in}{1.627320in}}%
\pgfpathlineto{\pgfqpoint{2.649830in}{1.597946in}}%
\pgfpathlineto{\pgfqpoint{2.657471in}{1.548920in}}%
\pgfpathlineto{\pgfqpoint{2.667972in}{1.465570in}}%
\pgfpathlineto{\pgfqpoint{2.695530in}{1.240100in}}%
\pgfpathlineto{\pgfqpoint{2.703221in}{1.197821in}}%
\pgfpathlineto{\pgfqpoint{2.709067in}{1.176756in}}%
\pgfpathlineto{\pgfqpoint{2.713489in}{1.168021in}}%
\pgfpathlineto{\pgfqpoint{2.716708in}{1.165800in}}%
\pgfpathlineto{\pgfqpoint{2.719250in}{1.166574in}}%
\pgfpathlineto{\pgfqpoint{2.722068in}{1.170063in}}%
\pgfpathlineto{\pgfqpoint{2.725645in}{1.178476in}}%
\pgfpathlineto{\pgfqpoint{2.730181in}{1.195431in}}%
\pgfpathlineto{\pgfqpoint{2.735850in}{1.226044in}}%
\pgfpathlineto{\pgfqpoint{2.742964in}{1.277725in}}%
\pgfpathlineto{\pgfqpoint{2.752394in}{1.364119in}}%
\pgfpathlineto{\pgfqpoint{2.787270in}{1.700792in}}%
\pgfpathlineto{\pgfqpoint{2.793996in}{1.738167in}}%
\pgfpathlineto{\pgfqpoint{2.799123in}{1.755628in}}%
\pgfpathlineto{\pgfqpoint{2.802912in}{1.761849in}}%
\pgfpathlineto{\pgfqpoint{2.805574in}{1.762687in}}%
\pgfpathlineto{\pgfqpoint{2.807912in}{1.760979in}}%
\pgfpathlineto{\pgfqpoint{2.810800in}{1.755701in}}%
\pgfpathlineto{\pgfqpoint{2.814547in}{1.743664in}}%
\pgfpathlineto{\pgfqpoint{2.819307in}{1.720121in}}%
\pgfpathlineto{\pgfqpoint{2.825273in}{1.678371in}}%
\pgfpathlineto{\pgfqpoint{2.832823in}{1.608429in}}%
\pgfpathlineto{\pgfqpoint{2.843183in}{1.489536in}}%
\pgfpathlineto{\pgfqpoint{2.871375in}{1.155302in}}%
\pgfpathlineto{\pgfqpoint{2.879002in}{1.095013in}}%
\pgfpathlineto{\pgfqpoint{2.884806in}{1.064905in}}%
\pgfpathlineto{\pgfqpoint{2.889165in}{1.052390in}}%
\pgfpathlineto{\pgfqpoint{2.892243in}{1.049068in}}%
\pgfpathlineto{\pgfqpoint{2.894412in}{1.049532in}}%
\pgfpathlineto{\pgfqpoint{2.896778in}{1.052709in}}%
\pgfpathlineto{\pgfqpoint{2.899898in}{1.061152in}}%
\pgfpathlineto{\pgfqpoint{2.903955in}{1.079281in}}%
\pgfpathlineto{\pgfqpoint{2.909096in}{1.113458in}}%
\pgfpathlineto{\pgfqpoint{2.915540in}{1.172578in}}%
\pgfpathlineto{\pgfqpoint{2.923837in}{1.271038in}}%
\pgfpathlineto{\pgfqpoint{2.936302in}{1.449123in}}%
\pgfpathlineto{\pgfqpoint{2.955797in}{1.724792in}}%
\pgfpathlineto{\pgfqpoint{2.964312in}{1.814294in}}%
\pgfpathlineto{\pgfqpoint{2.970707in}{1.860467in}}%
\pgfpathlineto{\pgfqpoint{2.975566in}{1.881428in}}%
\pgfpathlineto{\pgfqpoint{2.979080in}{1.888417in}}%
\pgfpathlineto{\pgfqpoint{2.981405in}{1.889155in}}%
\pgfpathlineto{\pgfqpoint{2.983419in}{1.887262in}}%
\pgfpathlineto{\pgfqpoint{2.986060in}{1.881207in}}%
\pgfpathlineto{\pgfqpoint{2.989588in}{1.866832in}}%
\pgfpathlineto{\pgfqpoint{2.994131in}{1.837977in}}%
\pgfpathlineto{\pgfqpoint{2.999850in}{1.785990in}}%
\pgfpathlineto{\pgfqpoint{3.007076in}{1.698175in}}%
\pgfpathlineto{\pgfqpoint{3.016788in}{1.550346in}}%
\pgfpathlineto{\pgfqpoint{3.049135in}{1.035104in}}%
\pgfpathlineto{\pgfqpoint{3.056248in}{0.965335in}}%
\pgfpathlineto{\pgfqpoint{3.061678in}{0.931447in}}%
\pgfpathlineto{\pgfqpoint{3.065707in}{0.918235in}}%
\pgfpathlineto{\pgfqpoint{3.068432in}{0.915309in}}%
\pgfpathlineto{\pgfqpoint{3.070271in}{0.916122in}}%
\pgfpathlineto{\pgfqpoint{3.072468in}{0.920055in}}%
\pgfpathlineto{\pgfqpoint{3.075482in}{0.930685in}}%
\pgfpathlineto{\pgfqpoint{3.079454in}{0.953816in}}%
\pgfpathlineto{\pgfqpoint{3.084511in}{0.997727in}}%
\pgfpathlineto{\pgfqpoint{3.090871in}{1.074133in}}%
\pgfpathlineto{\pgfqpoint{3.099062in}{1.201737in}}%
\pgfpathlineto{\pgfqpoint{3.111246in}{1.430902in}}%
\pgfpathlineto{\pgfqpoint{3.131487in}{1.809677in}}%
\pgfpathlineto{\pgfqpoint{3.139987in}{1.927422in}}%
\pgfpathlineto{\pgfqpoint{3.146389in}{1.988388in}}%
\pgfpathlineto{\pgfqpoint{3.151256in}{2.016173in}}%
\pgfpathlineto{\pgfqpoint{3.154763in}{2.025547in}}%
\pgfpathlineto{\pgfqpoint{3.157003in}{2.026729in}}%
\pgfpathlineto{\pgfqpoint{3.158756in}{2.025014in}}%
\pgfpathlineto{\pgfqpoint{3.161137in}{2.018966in}}%
\pgfpathlineto{\pgfqpoint{3.164398in}{2.003768in}}%
\pgfpathlineto{\pgfqpoint{3.168658in}{1.972099in}}%
\pgfpathlineto{\pgfqpoint{3.174060in}{1.913628in}}%
\pgfpathlineto{\pgfqpoint{3.180878in}{1.813486in}}%
\pgfpathlineto{\pgfqpoint{3.189850in}{1.645849in}}%
\pgfpathlineto{\pgfqpoint{3.205999in}{1.293020in}}%
\pgfpathlineto{\pgfqpoint{3.219803in}{1.013222in}}%
\pgfpathlineto{\pgfqpoint{3.228170in}{0.885239in}}%
\pgfpathlineto{\pgfqpoint{3.234495in}{0.819052in}}%
\pgfpathlineto{\pgfqpoint{3.239305in}{0.789077in}}%
\pgfpathlineto{\pgfqpoint{3.242756in}{0.779142in}}%
\pgfpathlineto{\pgfqpoint{3.244911in}{0.777972in}}%
\pgfpathlineto{\pgfqpoint{3.246587in}{0.779763in}}%
\pgfpathlineto{\pgfqpoint{3.248904in}{0.786133in}}%
\pgfpathlineto{\pgfqpoint{3.252109in}{0.802350in}}%
\pgfpathlineto{\pgfqpoint{3.256313in}{0.836410in}}%
\pgfpathlineto{\pgfqpoint{3.261652in}{0.899576in}}%
\pgfpathlineto{\pgfqpoint{3.268399in}{1.008180in}}%
\pgfpathlineto{\pgfqpoint{3.277258in}{1.190013in}}%
\pgfpathlineto{\pgfqpoint{3.292633in}{1.560686in}}%
\pgfpathlineto{\pgfqpoint{3.307197in}{1.890619in}}%
\pgfpathlineto{\pgfqpoint{3.315684in}{2.036066in}}%
\pgfpathlineto{\pgfqpoint{3.322086in}{2.111537in}}%
\pgfpathlineto{\pgfqpoint{3.326967in}{2.146118in}}%
\pgfpathlineto{\pgfqpoint{3.330481in}{2.157879in}}%
\pgfpathlineto{\pgfqpoint{3.332685in}{2.159490in}}%
\pgfpathlineto{\pgfqpoint{3.334270in}{2.157877in}}%
\pgfpathlineto{\pgfqpoint{3.336474in}{2.151776in}}%
\pgfpathlineto{\pgfqpoint{3.339559in}{2.135742in}}%
\pgfpathlineto{\pgfqpoint{3.343630in}{2.101451in}}%
\pgfpathlineto{\pgfqpoint{3.348827in}{2.036906in}}%
\pgfpathlineto{\pgfqpoint{3.355391in}{1.925172in}}%
\pgfpathlineto{\pgfqpoint{3.363955in}{1.737952in}}%
\pgfpathlineto{\pgfqpoint{3.377696in}{1.380792in}}%
\pgfpathlineto{\pgfqpoint{3.394444in}{0.960163in}}%
\pgfpathlineto{\pgfqpoint{3.403085in}{0.796080in}}%
\pgfpathlineto{\pgfqpoint{3.409593in}{0.710499in}}%
\pgfpathlineto{\pgfqpoint{3.414565in}{0.670758in}}%
\pgfpathlineto{\pgfqpoint{3.418171in}{0.656815in}}%
\pgfpathlineto{\pgfqpoint{3.420446in}{0.654631in}}%
\pgfpathlineto{\pgfqpoint{3.421946in}{0.656014in}}%
\pgfpathlineto{\pgfqpoint{3.424030in}{0.661666in}}%
\pgfpathlineto{\pgfqpoint{3.426981in}{0.677043in}}%
\pgfpathlineto{\pgfqpoint{3.430911in}{0.710716in}}%
\pgfpathlineto{\pgfqpoint{3.435947in}{0.774985in}}%
\pgfpathlineto{\pgfqpoint{3.442313in}{0.887310in}}%
\pgfpathlineto{\pgfqpoint{3.450568in}{1.075838in}}%
\pgfpathlineto{\pgfqpoint{3.463090in}{1.419876in}}%
\pgfpathlineto{\pgfqpoint{3.482352in}{1.942691in}}%
\pgfpathlineto{\pgfqpoint{3.490965in}{2.117421in}}%
\pgfpathlineto{\pgfqpoint{3.497466in}{2.208740in}}%
\pgfpathlineto{\pgfqpoint{3.502438in}{2.251209in}}%
\pgfpathlineto{\pgfqpoint{3.506044in}{2.266144in}}%
\pgfpathlineto{\pgfqpoint{3.508319in}{2.268524in}}%
\pgfpathlineto{\pgfqpoint{3.509776in}{2.267162in}}%
\pgfpathlineto{\pgfqpoint{3.511812in}{2.261487in}}%
\pgfpathlineto{\pgfqpoint{3.514713in}{2.245835in}}%
\pgfpathlineto{\pgfqpoint{3.518594in}{2.211252in}}%
\pgfpathlineto{\pgfqpoint{3.523580in}{2.144826in}}%
\pgfpathlineto{\pgfqpoint{3.529891in}{2.028304in}}%
\pgfpathlineto{\pgfqpoint{3.538060in}{1.832575in}}%
\pgfpathlineto{\pgfqpoint{3.550336in}{1.477850in}}%
\pgfpathlineto{\pgfqpoint{3.570239in}{0.907054in}}%
\pgfpathlineto{\pgfqpoint{3.578838in}{0.723465in}}%
\pgfpathlineto{\pgfqpoint{3.585339in}{0.627402in}}%
\pgfpathlineto{\pgfqpoint{3.590311in}{0.582718in}}%
\pgfpathlineto{\pgfqpoint{3.593924in}{0.566951in}}%
\pgfpathlineto{\pgfqpoint{3.596192in}{0.564416in}}%
\pgfpathlineto{\pgfqpoint{3.597614in}{0.565745in}}%
\pgfpathlineto{\pgfqpoint{3.599614in}{0.571417in}}%
\pgfpathlineto{\pgfqpoint{3.602481in}{0.587254in}}%
\pgfpathlineto{\pgfqpoint{3.606326in}{0.622504in}}%
\pgfpathlineto{\pgfqpoint{3.611270in}{0.690434in}}%
\pgfpathlineto{\pgfqpoint{3.617531in}{0.809882in}}%
\pgfpathlineto{\pgfqpoint{3.625645in}{1.011033in}}%
\pgfpathlineto{\pgfqpoint{3.637765in}{1.374179in}}%
\pgfpathlineto{\pgfqpoint{3.658105in}{1.980750in}}%
\pgfpathlineto{\pgfqpoint{3.666690in}{2.170816in}}%
\pgfpathlineto{\pgfqpoint{3.673191in}{2.270448in}}%
\pgfpathlineto{\pgfqpoint{3.678170in}{2.316897in}}%
\pgfpathlineto{\pgfqpoint{3.681790in}{2.333343in}}%
\pgfpathlineto{\pgfqpoint{3.684072in}{2.336035in}}%
\pgfpathlineto{\pgfqpoint{3.685466in}{2.334740in}}%
\pgfpathlineto{\pgfqpoint{3.687438in}{2.329104in}}%
\pgfpathlineto{\pgfqpoint{3.690276in}{2.313219in}}%
\pgfpathlineto{\pgfqpoint{3.694094in}{2.277657in}}%
\pgfpathlineto{\pgfqpoint{3.699009in}{2.208865in}}%
\pgfpathlineto{\pgfqpoint{3.705242in}{2.087573in}}%
\pgfpathlineto{\pgfqpoint{3.713320in}{1.883064in}}%
\pgfpathlineto{\pgfqpoint{3.725357in}{1.514460in}}%
\pgfpathlineto{\pgfqpoint{3.745929in}{0.886446in}}%
\pgfpathlineto{\pgfqpoint{3.754514in}{0.692144in}}%
\pgfpathlineto{\pgfqpoint{3.761021in}{0.590132in}}%
\pgfpathlineto{\pgfqpoint{3.766015in}{0.542427in}}%
\pgfpathlineto{\pgfqpoint{3.769649in}{0.525462in}}%
\pgfpathlineto{\pgfqpoint{3.771945in}{0.522631in}}%
\pgfpathlineto{\pgfqpoint{3.773332in}{0.523897in}}%
\pgfpathlineto{\pgfqpoint{3.775283in}{0.529466in}}%
\pgfpathlineto{\pgfqpoint{3.778100in}{0.545266in}}%
\pgfpathlineto{\pgfqpoint{3.781896in}{0.580783in}}%
\pgfpathlineto{\pgfqpoint{3.786791in}{0.649663in}}%
\pgfpathlineto{\pgfqpoint{3.793003in}{0.771320in}}%
\pgfpathlineto{\pgfqpoint{3.801060in}{0.976714in}}%
\pgfpathlineto{\pgfqpoint{3.813061in}{1.346939in}}%
\pgfpathlineto{\pgfqpoint{3.833717in}{1.982463in}}%
\pgfpathlineto{\pgfqpoint{3.842309in}{2.178418in}}%
\pgfpathlineto{\pgfqpoint{3.848831in}{2.281529in}}%
\pgfpathlineto{\pgfqpoint{3.853838in}{2.329882in}}%
\pgfpathlineto{\pgfqpoint{3.857494in}{2.347218in}}%
\pgfpathlineto{\pgfqpoint{3.859811in}{2.350185in}}%
\pgfpathlineto{\pgfqpoint{3.861191in}{2.348979in}}%
\pgfpathlineto{\pgfqpoint{3.863128in}{2.343552in}}%
\pgfpathlineto{\pgfqpoint{3.865924in}{2.328081in}}%
\pgfpathlineto{\pgfqpoint{3.869699in}{2.293161in}}%
\pgfpathlineto{\pgfqpoint{3.874579in}{2.225145in}}%
\pgfpathlineto{\pgfqpoint{3.880777in}{2.104809in}}%
\pgfpathlineto{\pgfqpoint{3.888820in}{1.901432in}}%
\pgfpathlineto{\pgfqpoint{3.900807in}{1.534541in}}%
\pgfpathlineto{\pgfqpoint{3.921484in}{0.903252in}}%
\pgfpathlineto{\pgfqpoint{3.930091in}{0.708395in}}%
\pgfpathlineto{\pgfqpoint{3.936633in}{0.605603in}}%
\pgfpathlineto{\pgfqpoint{3.941662in}{0.557266in}}%
\pgfpathlineto{\pgfqpoint{3.945338in}{0.539829in}}%
\pgfpathlineto{\pgfqpoint{3.947677in}{0.536772in}}%
\pgfpathlineto{\pgfqpoint{3.949071in}{0.537934in}}%
\pgfpathlineto{\pgfqpoint{3.951001in}{0.543213in}}%
\pgfpathlineto{\pgfqpoint{3.953790in}{0.558311in}}%
\pgfpathlineto{\pgfqpoint{3.957558in}{0.592440in}}%
\pgfpathlineto{\pgfqpoint{3.962431in}{0.658960in}}%
\pgfpathlineto{\pgfqpoint{3.968629in}{0.776813in}}%
\pgfpathlineto{\pgfqpoint{3.976679in}{0.976106in}}%
\pgfpathlineto{\pgfqpoint{3.988729in}{1.336940in}}%
\pgfpathlineto{\pgfqpoint{4.009195in}{1.947375in}}%
\pgfpathlineto{\pgfqpoint{4.017837in}{2.138931in}}%
\pgfpathlineto{\pgfqpoint{4.024401in}{2.240046in}}%
\pgfpathlineto{\pgfqpoint{4.029458in}{2.287855in}}%
\pgfpathlineto{\pgfqpoint{4.033162in}{2.305254in}}%
\pgfpathlineto{\pgfqpoint{4.035536in}{2.308406in}}%
\pgfpathlineto{\pgfqpoint{4.036951in}{2.307304in}}%
\pgfpathlineto{\pgfqpoint{4.038881in}{2.302222in}}%
\pgfpathlineto{\pgfqpoint{4.041670in}{2.287648in}}%
\pgfpathlineto{\pgfqpoint{4.045438in}{2.254674in}}%
\pgfpathlineto{\pgfqpoint{4.050311in}{2.190411in}}%
\pgfpathlineto{\pgfqpoint{4.056516in}{2.076468in}}%
\pgfpathlineto{\pgfqpoint{4.064594in}{1.883439in}}%
\pgfpathlineto{\pgfqpoint{4.076764in}{1.532116in}}%
\pgfpathlineto{\pgfqpoint{4.096878in}{0.955120in}}%
\pgfpathlineto{\pgfqpoint{4.105555in}{0.769534in}}%
\pgfpathlineto{\pgfqpoint{4.112154in}{0.671270in}}%
\pgfpathlineto{\pgfqpoint{4.117239in}{0.624649in}}%
\pgfpathlineto{\pgfqpoint{4.120979in}{0.607491in}}%
\pgfpathlineto{\pgfqpoint{4.123394in}{0.604278in}}%
\pgfpathlineto{\pgfqpoint{4.124845in}{0.605327in}}%
\pgfpathlineto{\pgfqpoint{4.126789in}{0.610221in}}%
\pgfpathlineto{\pgfqpoint{4.129592in}{0.624245in}}%
\pgfpathlineto{\pgfqpoint{4.133374in}{0.655916in}}%
\pgfpathlineto{\pgfqpoint{4.138269in}{0.717615in}}%
\pgfpathlineto{\pgfqpoint{4.144509in}{0.827040in}}%
\pgfpathlineto{\pgfqpoint{4.152643in}{1.012368in}}%
\pgfpathlineto{\pgfqpoint{4.165046in}{1.352837in}}%
\pgfpathlineto{\pgfqpoint{4.184505in}{1.881232in}}%
\pgfpathlineto{\pgfqpoint{4.193231in}{2.058845in}}%
\pgfpathlineto{\pgfqpoint{4.199872in}{2.153181in}}%
\pgfpathlineto{\pgfqpoint{4.204999in}{2.198195in}}%
\pgfpathlineto{\pgfqpoint{4.208781in}{2.214943in}}%
\pgfpathlineto{\pgfqpoint{4.211246in}{2.218193in}}%
\pgfpathlineto{\pgfqpoint{4.212739in}{2.217214in}}%
\pgfpathlineto{\pgfqpoint{4.214704in}{2.212550in}}%
\pgfpathlineto{\pgfqpoint{4.217521in}{2.199243in}}%
\pgfpathlineto{\pgfqpoint{4.221325in}{2.169203in}}%
\pgfpathlineto{\pgfqpoint{4.226240in}{2.110821in}}%
\pgfpathlineto{\pgfqpoint{4.232509in}{2.007430in}}%
\pgfpathlineto{\pgfqpoint{4.240706in}{1.832063in}}%
\pgfpathlineto{\pgfqpoint{4.253376in}{1.506392in}}%
\pgfpathlineto{\pgfqpoint{4.272167in}{1.031104in}}%
\pgfpathlineto{\pgfqpoint{4.280935in}{0.864078in}}%
\pgfpathlineto{\pgfqpoint{4.287611in}{0.775154in}}%
\pgfpathlineto{\pgfqpoint{4.292774in}{0.732514in}}%
\pgfpathlineto{\pgfqpoint{4.296598in}{0.716481in}}%
\pgfpathlineto{\pgfqpoint{4.299112in}{0.713286in}}%
\pgfpathlineto{\pgfqpoint{4.300669in}{0.714233in}}%
\pgfpathlineto{\pgfqpoint{4.302676in}{0.718739in}}%
\pgfpathlineto{\pgfqpoint{4.305535in}{0.731473in}}%
\pgfpathlineto{\pgfqpoint{4.309381in}{0.759998in}}%
\pgfpathlineto{\pgfqpoint{4.314353in}{0.815266in}}%
\pgfpathlineto{\pgfqpoint{4.320698in}{0.912931in}}%
\pgfpathlineto{\pgfqpoint{4.329030in}{1.078661in}}%
\pgfpathlineto{\pgfqpoint{4.342271in}{1.393379in}}%
\pgfpathlineto{\pgfqpoint{4.359849in}{1.800229in}}%
\pgfpathlineto{\pgfqpoint{4.368653in}{1.954650in}}%
\pgfpathlineto{\pgfqpoint{4.375358in}{2.037031in}}%
\pgfpathlineto{\pgfqpoint{4.380555in}{2.076759in}}%
\pgfpathlineto{\pgfqpoint{4.384415in}{2.091824in}}%
\pgfpathlineto{\pgfqpoint{4.386978in}{2.094910in}}%
\pgfpathlineto{\pgfqpoint{4.388612in}{2.093992in}}%
\pgfpathlineto{\pgfqpoint{4.390676in}{2.089645in}}%
\pgfpathlineto{\pgfqpoint{4.393584in}{2.077550in}}%
\pgfpathlineto{\pgfqpoint{4.397486in}{2.050668in}}%
\pgfpathlineto{\pgfqpoint{4.402529in}{1.998827in}}%
\pgfpathlineto{\pgfqpoint{4.408966in}{1.907553in}}%
\pgfpathlineto{\pgfqpoint{4.417467in}{1.752437in}}%
\pgfpathlineto{\pgfqpoint{4.431538in}{1.447777in}}%
\pgfpathlineto{\pgfqpoint{4.447722in}{1.111569in}}%
\pgfpathlineto{\pgfqpoint{4.456512in}{0.972584in}}%
\pgfpathlineto{\pgfqpoint{4.463217in}{0.898332in}}%
\pgfpathlineto{\pgfqpoint{4.468421in}{0.862467in}}%
\pgfpathlineto{\pgfqpoint{4.472295in}{0.848828in}}%
\pgfpathlineto{\pgfqpoint{4.474886in}{0.846025in}}%
\pgfpathlineto{\pgfqpoint{4.476612in}{0.846969in}}%
\pgfpathlineto{\pgfqpoint{4.478767in}{0.851279in}}%
\pgfpathlineto{\pgfqpoint{4.481767in}{0.862986in}}%
\pgfpathlineto{\pgfqpoint{4.485768in}{0.888610in}}%
\pgfpathlineto{\pgfqpoint{4.490923in}{0.937506in}}%
\pgfpathlineto{\pgfqpoint{4.497522in}{1.023256in}}%
\pgfpathlineto{\pgfqpoint{4.506318in}{1.169354in}}%
\pgfpathlineto{\pgfqpoint{4.522376in}{1.481649in}}%
\pgfpathlineto{\pgfqpoint{4.536328in}{1.733583in}}%
\pgfpathlineto{\pgfqpoint{4.544913in}{1.851394in}}%
\pgfpathlineto{\pgfqpoint{4.551505in}{1.914266in}}%
\pgfpathlineto{\pgfqpoint{4.556618in}{1.944283in}}%
\pgfpathlineto{\pgfqpoint{4.560414in}{1.955434in}}%
\pgfpathlineto{\pgfqpoint{4.562957in}{1.957525in}}%
\pgfpathlineto{\pgfqpoint{4.564788in}{1.956365in}}%
\pgfpathlineto{\pgfqpoint{4.567112in}{1.951707in}}%
\pgfpathlineto{\pgfqpoint{4.570302in}{1.939612in}}%
\pgfpathlineto{\pgfqpoint{4.574514in}{1.913910in}}%
\pgfpathlineto{\pgfqpoint{4.579923in}{1.865766in}}%
\pgfpathlineto{\pgfqpoint{4.586867in}{1.782207in}}%
\pgfpathlineto{\pgfqpoint{4.596325in}{1.638797in}}%
\pgfpathlineto{\pgfqpoint{4.629370in}{1.113290in}}%
\pgfpathlineto{\pgfqpoint{4.636716in}{1.040765in}}%
\pgfpathlineto{\pgfqpoint{4.642442in}{1.003879in}}%
\pgfpathlineto{\pgfqpoint{4.646815in}{0.988192in}}%
\pgfpathlineto{\pgfqpoint{4.649921in}{0.983781in}}%
\pgfpathlineto{\pgfqpoint{4.651999in}{0.983955in}}%
\pgfpathlineto{\pgfqpoint{4.654168in}{0.986789in}}%
\pgfpathlineto{\pgfqpoint{4.657098in}{0.994849in}}%
\pgfpathlineto{\pgfqpoint{4.660992in}{1.012868in}}%
\pgfpathlineto{\pgfqpoint{4.666021in}{1.047745in}}%
\pgfpathlineto{\pgfqpoint{4.672458in}{1.109429in}}%
\pgfpathlineto{\pgfqpoint{4.680994in}{1.214713in}}%
\pgfpathlineto{\pgfqpoint{4.695389in}{1.424972in}}%
\pgfpathlineto{\pgfqpoint{4.711123in}{1.644195in}}%
\pgfpathlineto{\pgfqpoint{4.719934in}{1.737774in}}%
\pgfpathlineto{\pgfqpoint{4.726681in}{1.788017in}}%
\pgfpathlineto{\pgfqpoint{4.731949in}{1.812483in}}%
\pgfpathlineto{\pgfqpoint{4.735914in}{1.821908in}}%
\pgfpathlineto{\pgfqpoint{4.738682in}{1.823851in}}%
\pgfpathlineto{\pgfqpoint{4.740801in}{1.822771in}}%
\pgfpathlineto{\pgfqpoint{4.743365in}{1.818527in}}%
\pgfpathlineto{\pgfqpoint{4.746781in}{1.807987in}}%
\pgfpathlineto{\pgfqpoint{4.751232in}{1.786240in}}%
\pgfpathlineto{\pgfqpoint{4.756929in}{1.746247in}}%
\pgfpathlineto{\pgfqpoint{4.764303in}{1.677267in}}%
\pgfpathlineto{\pgfqpoint{4.774713in}{1.556612in}}%
\pgfpathlineto{\pgfqpoint{4.801567in}{1.236611in}}%
\pgfpathlineto{\pgfqpoint{4.809729in}{1.169075in}}%
\pgfpathlineto{\pgfqpoint{4.816068in}{1.133305in}}%
\pgfpathlineto{\pgfqpoint{4.820998in}{1.116654in}}%
\pgfpathlineto{\pgfqpoint{4.824674in}{1.110839in}}%
\pgfpathlineto{\pgfqpoint{4.827287in}{1.110154in}}%
\pgfpathlineto{\pgfqpoint{4.829646in}{1.111976in}}%
\pgfpathlineto{\pgfqpoint{4.832611in}{1.117495in}}%
\pgfpathlineto{\pgfqpoint{4.836492in}{1.129980in}}%
\pgfpathlineto{\pgfqpoint{4.841485in}{1.154299in}}%
\pgfpathlineto{\pgfqpoint{4.847880in}{1.197535in}}%
\pgfpathlineto{\pgfqpoint{4.856353in}{1.271482in}}%
\pgfpathlineto{\pgfqpoint{4.870488in}{1.417943in}}%
\pgfpathlineto{\pgfqpoint{4.886594in}{1.577827in}}%
\pgfpathlineto{\pgfqpoint{4.895461in}{1.644906in}}%
\pgfpathlineto{\pgfqpoint{4.902272in}{1.681068in}}%
\pgfpathlineto{\pgfqpoint{4.907610in}{1.698766in}}%
\pgfpathlineto{\pgfqpoint{4.911688in}{1.705662in}}%
\pgfpathlineto{\pgfqpoint{4.914674in}{1.707035in}}%
\pgfpathlineto{\pgfqpoint{4.917217in}{1.705771in}}%
\pgfpathlineto{\pgfqpoint{4.920217in}{1.701451in}}%
\pgfpathlineto{\pgfqpoint{4.924083in}{1.691515in}}%
\pgfpathlineto{\pgfqpoint{4.929034in}{1.672028in}}%
\pgfpathlineto{\pgfqpoint{4.935366in}{1.637231in}}%
\pgfpathlineto{\pgfqpoint{4.943726in}{1.577682in}}%
\pgfpathlineto{\pgfqpoint{4.957339in}{1.461997in}}%
\pgfpathlineto{\pgfqpoint{4.974263in}{1.322979in}}%
\pgfpathlineto{\pgfqpoint{4.983179in}{1.267259in}}%
\pgfpathlineto{\pgfqpoint{4.990039in}{1.237148in}}%
\pgfpathlineto{\pgfqpoint{4.995434in}{1.222360in}}%
\pgfpathlineto{\pgfqpoint{4.999596in}{1.216568in}}%
\pgfpathlineto{\pgfqpoint{5.002751in}{1.215469in}}%
\pgfpathlineto{\pgfqpoint{5.005568in}{1.216864in}}%
\pgfpathlineto{\pgfqpoint{5.008864in}{1.221280in}}%
\pgfpathlineto{\pgfqpoint{5.013034in}{1.231001in}}%
\pgfpathlineto{\pgfqpoint{5.018337in}{1.249567in}}%
\pgfpathlineto{\pgfqpoint{5.025126in}{1.282184in}}%
\pgfpathlineto{\pgfqpoint{5.034303in}{1.338277in}}%
\pgfpathlineto{\pgfqpoint{5.069503in}{1.565332in}}%
\pgfpathlineto{\pgfqpoint{5.076715in}{1.592939in}}%
\pgfpathlineto{\pgfqpoint{5.082419in}{1.606899in}}%
\pgfpathlineto{\pgfqpoint{5.086892in}{1.612685in}}%
\pgfpathlineto{\pgfqpoint{5.090392in}{1.614009in}}%
\pgfpathlineto{\pgfqpoint{5.093526in}{1.612831in}}%
\pgfpathlineto{\pgfqpoint{5.097061in}{1.608885in}}%
\pgfpathlineto{\pgfqpoint{5.101442in}{1.600316in}}%
\pgfpathlineto{\pgfqpoint{5.106964in}{1.584167in}}%
\pgfpathlineto{\pgfqpoint{5.114042in}{1.555963in}}%
\pgfpathlineto{\pgfqpoint{5.123782in}{1.507056in}}%
\pgfpathlineto{\pgfqpoint{5.154918in}{1.343921in}}%
\pgfpathlineto{\pgfqpoint{5.162700in}{1.317553in}}%
\pgfpathlineto{\pgfqpoint{5.168863in}{1.303768in}}%
\pgfpathlineto{\pgfqpoint{5.173779in}{1.297626in}}%
\pgfpathlineto{\pgfqpoint{5.177730in}{1.295878in}}%
\pgfpathlineto{\pgfqpoint{5.181251in}{1.296696in}}%
\pgfpathlineto{\pgfqpoint{5.185068in}{1.300046in}}%
\pgfpathlineto{\pgfqpoint{5.189674in}{1.307335in}}%
\pgfpathlineto{\pgfqpoint{5.195414in}{1.320965in}}%
\pgfpathlineto{\pgfqpoint{5.202767in}{1.344674in}}%
\pgfpathlineto{\pgfqpoint{5.213070in}{1.386238in}}%
\pgfpathlineto{\pgfqpoint{5.240974in}{1.502422in}}%
\pgfpathlineto{\pgfqpoint{5.249221in}{1.525862in}}%
\pgfpathlineto{\pgfqpoint{5.255750in}{1.538328in}}%
\pgfpathlineto{\pgfqpoint{5.261039in}{1.544103in}}%
\pgfpathlineto{\pgfqpoint{5.265420in}{1.545877in}}%
\pgfpathlineto{\pgfqpoint{5.269413in}{1.545138in}}%
\pgfpathlineto{\pgfqpoint{5.273674in}{1.541949in}}%
\pgfpathlineto{\pgfqpoint{5.278716in}{1.535156in}}%
\pgfpathlineto{\pgfqpoint{5.284942in}{1.522715in}}%
\pgfpathlineto{\pgfqpoint{5.292978in}{1.501243in}}%
\pgfpathlineto{\pgfqpoint{5.305049in}{1.461783in}}%
\pgfpathlineto{\pgfqpoint{5.325931in}{1.393728in}}%
\pgfpathlineto{\pgfqpoint{5.334960in}{1.371954in}}%
\pgfpathlineto{\pgfqpoint{5.342052in}{1.360172in}}%
\pgfpathlineto{\pgfqpoint{5.347870in}{1.354450in}}%
\pgfpathlineto{\pgfqpoint{5.352807in}{1.352488in}}%
\pgfpathlineto{\pgfqpoint{5.357370in}{1.353026in}}%
\pgfpathlineto{\pgfqpoint{5.362159in}{1.355933in}}%
\pgfpathlineto{\pgfqpoint{5.367709in}{1.362097in}}%
\pgfpathlineto{\pgfqpoint{5.374505in}{1.373242in}}%
\pgfpathlineto{\pgfqpoint{5.383393in}{1.392479in}}%
\pgfpathlineto{\pgfqpoint{5.399036in}{1.432760in}}%
\pgfpathlineto{\pgfqpoint{5.414044in}{1.468959in}}%
\pgfpathlineto{\pgfqpoint{5.423178in}{1.485520in}}%
\pgfpathlineto{\pgfqpoint{5.430446in}{1.494442in}}%
\pgfpathlineto{\pgfqpoint{5.436553in}{1.498667in}}%
\pgfpathlineto{\pgfqpoint{5.441968in}{1.499836in}}%
\pgfpathlineto{\pgfqpoint{5.447229in}{1.498687in}}%
\pgfpathlineto{\pgfqpoint{5.452899in}{1.495066in}}%
\pgfpathlineto{\pgfqpoint{5.459526in}{1.488015in}}%
\pgfpathlineto{\pgfqpoint{5.467837in}{1.475664in}}%
\pgfpathlineto{\pgfqpoint{5.480042in}{1.453029in}}%
\pgfpathlineto{\pgfqpoint{5.502107in}{1.412041in}}%
\pgfpathlineto{\pgfqpoint{5.511481in}{1.399474in}}%
\pgfpathlineto{\pgfqpoint{5.519073in}{1.392771in}}%
\pgfpathlineto{\pgfqpoint{5.525658in}{1.389767in}}%
\pgfpathlineto{\pgfqpoint{5.531771in}{1.389367in}}%
\pgfpathlineto{\pgfqpoint{5.537969in}{1.391231in}}%
\pgfpathlineto{\pgfqpoint{5.544829in}{1.395730in}}%
\pgfpathlineto{\pgfqpoint{5.553062in}{1.403973in}}%
\pgfpathlineto{\pgfqpoint{5.564309in}{1.418702in}}%
\pgfpathlineto{\pgfqpoint{5.592755in}{1.457210in}}%
\pgfpathlineto{\pgfqpoint{5.602044in}{1.465427in}}%
\pgfpathlineto{\pgfqpoint{5.609918in}{1.469570in}}%
\pgfpathlineto{\pgfqpoint{5.617130in}{1.470919in}}%
\pgfpathlineto{\pgfqpoint{5.624257in}{1.469982in}}%
\pgfpathlineto{\pgfqpoint{5.631927in}{1.466656in}}%
\pgfpathlineto{\pgfqpoint{5.640949in}{1.460175in}}%
\pgfpathlineto{\pgfqpoint{5.653344in}{1.448246in}}%
\pgfpathlineto{\pgfqpoint{5.680170in}{1.421827in}}%
\pgfpathlineto{\pgfqpoint{5.690361in}{1.415295in}}%
\pgfpathlineto{\pgfqpoint{5.699178in}{1.412195in}}%
\pgfpathlineto{\pgfqpoint{5.707517in}{1.411591in}}%
\pgfpathlineto{\pgfqpoint{5.716074in}{1.413233in}}%
\pgfpathlineto{\pgfqpoint{5.725681in}{1.417436in}}%
\pgfpathlineto{\pgfqpoint{5.738083in}{1.425492in}}%
\pgfpathlineto{\pgfqpoint{5.771917in}{1.448724in}}%
\pgfpathlineto{\pgfqpoint{5.782727in}{1.452719in}}%
\pgfpathlineto{\pgfqpoint{5.792651in}{1.454048in}}%
\pgfpathlineto{\pgfqpoint{5.802665in}{1.453124in}}%
\pgfpathlineto{\pgfqpoint{5.813814in}{1.449780in}}%
\pgfpathlineto{\pgfqpoint{5.828632in}{1.442796in}}%
\pgfpathlineto{\pgfqpoint{5.858782in}{1.428174in}}%
\pgfpathlineto{\pgfqpoint{5.871431in}{1.424879in}}%
\pgfpathlineto{\pgfqpoint{5.883186in}{1.424105in}}%
\pgfpathlineto{\pgfqpoint{5.895461in}{1.425569in}}%
\pgfpathlineto{\pgfqpoint{5.910300in}{1.429714in}}%
\pgfpathlineto{\pgfqpoint{5.955156in}{1.443771in}}%
\pgfpathlineto{\pgfqpoint{5.969354in}{1.444885in}}%
\pgfpathlineto{\pgfqpoint{5.984067in}{1.443760in}}%
\pgfpathlineto{\pgfqpoint{6.002765in}{1.439920in}}%
\pgfpathlineto{\pgfqpoint{6.040705in}{1.431695in}}%
\pgfpathlineto{\pgfqpoint{6.057960in}{1.430669in}}%
\pgfpathlineto{\pgfqpoint{6.076165in}{1.431887in}}%
\pgfpathlineto{\pgfqpoint{6.104717in}{1.436407in}}%
\pgfpathlineto{\pgfqpoint{6.104717in}{1.436407in}}%
\pgfusepath{stroke}%
\end{pgfscope}%
\begin{pgfscope}%
\pgfpathrectangle{\pgfqpoint{0.470474in}{0.429287in}}{\pgfqpoint{5.634238in}{2.014240in}}%
\pgfusepath{clip}%
\pgfsetbuttcap%
\pgfsetroundjoin%
\pgfsetlinewidth{2.007500pt}%
\definecolor{currentstroke}{rgb}{1.000000,0.647059,0.000000}%
\pgfsetstrokecolor{currentstroke}%
\pgfsetdash{{2.000000pt}{3.300000pt}}{0.000000pt}%
\pgfpathmoveto{\pgfqpoint{0.470471in}{1.436418in}}%
\pgfpathlineto{\pgfqpoint{1.247080in}{1.437590in}}%
\pgfpathlineto{\pgfqpoint{1.478168in}{1.440117in}}%
\pgfpathlineto{\pgfqpoint{1.634018in}{1.443953in}}%
\pgfpathlineto{\pgfqpoint{1.755704in}{1.449104in}}%
\pgfpathlineto{\pgfqpoint{1.857536in}{1.455587in}}%
\pgfpathlineto{\pgfqpoint{1.946353in}{1.463427in}}%
\pgfpathlineto{\pgfqpoint{2.026014in}{1.472656in}}%
\pgfpathlineto{\pgfqpoint{2.098935in}{1.483315in}}%
\pgfpathlineto{\pgfqpoint{2.166764in}{1.495454in}}%
\pgfpathlineto{\pgfqpoint{2.230670in}{1.509131in}}%
\pgfpathlineto{\pgfqpoint{2.291548in}{1.524421in}}%
\pgfpathlineto{\pgfqpoint{2.350109in}{1.541411in}}%
\pgfpathlineto{\pgfqpoint{2.406938in}{1.560207in}}%
\pgfpathlineto{\pgfqpoint{2.462547in}{1.580940in}}%
\pgfpathlineto{\pgfqpoint{2.517411in}{1.603768in}}%
\pgfpathlineto{\pgfqpoint{2.571978in}{1.628887in}}%
\pgfpathlineto{\pgfqpoint{2.626729in}{1.656548in}}%
\pgfpathlineto{\pgfqpoint{2.682184in}{1.687070in}}%
\pgfpathlineto{\pgfqpoint{2.738998in}{1.720898in}}%
\pgfpathlineto{\pgfqpoint{2.798045in}{1.758672in}}%
\pgfpathlineto{\pgfqpoint{2.860663in}{1.801411in}}%
\pgfpathlineto{\pgfqpoint{2.929260in}{1.850983in}}%
\pgfpathlineto{\pgfqpoint{3.009611in}{1.911906in}}%
\pgfpathlineto{\pgfqpoint{3.134311in}{2.009685in}}%
\pgfpathlineto{\pgfqpoint{3.251566in}{2.100628in}}%
\pgfpathlineto{\pgfqpoint{3.321832in}{2.152328in}}%
\pgfpathlineto{\pgfqpoint{3.380133in}{2.192530in}}%
\pgfpathlineto{\pgfqpoint{3.431763in}{2.225486in}}%
\pgfpathlineto{\pgfqpoint{3.478957in}{2.253017in}}%
\pgfpathlineto{\pgfqpoint{3.522925in}{2.276130in}}%
\pgfpathlineto{\pgfqpoint{3.564450in}{2.295479in}}%
\pgfpathlineto{\pgfqpoint{3.604073in}{2.311516in}}%
\pgfpathlineto{\pgfqpoint{3.642202in}{2.324574in}}%
\pgfpathlineto{\pgfqpoint{3.679163in}{2.334900in}}%
\pgfpathlineto{\pgfqpoint{3.715229in}{2.342682in}}%
\pgfpathlineto{\pgfqpoint{3.750640in}{2.348060in}}%
\pgfpathlineto{\pgfqpoint{3.785615in}{2.351130in}}%
\pgfpathlineto{\pgfqpoint{3.820364in}{2.351951in}}%
\pgfpathlineto{\pgfqpoint{3.855078in}{2.350548in}}%
\pgfpathlineto{\pgfqpoint{3.889961in}{2.346909in}}%
\pgfpathlineto{\pgfqpoint{3.925217in}{2.340988in}}%
\pgfpathlineto{\pgfqpoint{3.961058in}{2.332703in}}%
\pgfpathlineto{\pgfqpoint{3.997716in}{2.321932in}}%
\pgfpathlineto{\pgfqpoint{4.035451in}{2.308509in}}%
\pgfpathlineto{\pgfqpoint{4.074567in}{2.292212in}}%
\pgfpathlineto{\pgfqpoint{4.115443in}{2.272745in}}%
\pgfpathlineto{\pgfqpoint{4.158566in}{2.249713in}}%
\pgfpathlineto{\pgfqpoint{4.204612in}{2.222560in}}%
\pgfpathlineto{\pgfqpoint{4.254581in}{2.190462in}}%
\pgfpathlineto{\pgfqpoint{4.310169in}{2.152046in}}%
\pgfpathlineto{\pgfqpoint{4.374766in}{2.104606in}}%
\pgfpathlineto{\pgfqpoint{4.458230in}{2.040366in}}%
\pgfpathlineto{\pgfqpoint{4.698249in}{1.854024in}}%
\pgfpathlineto{\pgfqpoint{4.770621in}{1.801638in}}%
\pgfpathlineto{\pgfqpoint{4.835259in}{1.757550in}}%
\pgfpathlineto{\pgfqpoint{4.895595in}{1.719057in}}%
\pgfpathlineto{\pgfqpoint{4.953304in}{1.684852in}}%
\pgfpathlineto{\pgfqpoint{5.009407in}{1.654166in}}%
\pgfpathlineto{\pgfqpoint{5.064615in}{1.626490in}}%
\pgfpathlineto{\pgfqpoint{5.119493in}{1.601459in}}%
\pgfpathlineto{\pgfqpoint{5.174532in}{1.578793in}}%
\pgfpathlineto{\pgfqpoint{5.230184in}{1.558279in}}%
\pgfpathlineto{\pgfqpoint{5.286914in}{1.539740in}}%
\pgfpathlineto{\pgfqpoint{5.345214in}{1.523035in}}%
\pgfpathlineto{\pgfqpoint{5.405635in}{1.508046in}}%
\pgfpathlineto{\pgfqpoint{5.468823in}{1.494677in}}%
\pgfpathlineto{\pgfqpoint{5.535581in}{1.482844in}}%
\pgfpathlineto{\pgfqpoint{5.606939in}{1.472480in}}%
\pgfpathlineto{\pgfqpoint{5.684276in}{1.463526in}}%
\pgfpathlineto{\pgfqpoint{5.769543in}{1.455932in}}%
\pgfpathlineto{\pgfqpoint{5.865663in}{1.449654in}}%
\pgfpathlineto{\pgfqpoint{5.977362in}{1.444652in}}%
\pgfpathlineto{\pgfqpoint{6.104717in}{1.441063in}}%
\pgfpathlineto{\pgfqpoint{6.104717in}{1.441063in}}%
\pgfusepath{stroke}%
\end{pgfscope}%
\begin{pgfscope}%
\pgfpathrectangle{\pgfqpoint{0.470474in}{0.429287in}}{\pgfqpoint{5.634238in}{2.014240in}}%
\pgfusepath{clip}%
\pgfsetbuttcap%
\pgfsetroundjoin%
\pgfsetlinewidth{2.007500pt}%
\definecolor{currentstroke}{rgb}{1.000000,0.647059,0.000000}%
\pgfsetstrokecolor{currentstroke}%
\pgfsetdash{{2.000000pt}{3.300000pt}}{0.000000pt}%
\pgfpathmoveto{\pgfqpoint{0.470471in}{1.436395in}}%
\pgfpathlineto{\pgfqpoint{1.247080in}{1.435224in}}%
\pgfpathlineto{\pgfqpoint{1.478168in}{1.432697in}}%
\pgfpathlineto{\pgfqpoint{1.634018in}{1.428860in}}%
\pgfpathlineto{\pgfqpoint{1.755704in}{1.423709in}}%
\pgfpathlineto{\pgfqpoint{1.857536in}{1.417226in}}%
\pgfpathlineto{\pgfqpoint{1.946353in}{1.409387in}}%
\pgfpathlineto{\pgfqpoint{2.026014in}{1.400157in}}%
\pgfpathlineto{\pgfqpoint{2.098935in}{1.389498in}}%
\pgfpathlineto{\pgfqpoint{2.166764in}{1.377360in}}%
\pgfpathlineto{\pgfqpoint{2.230670in}{1.363682in}}%
\pgfpathlineto{\pgfqpoint{2.291548in}{1.348392in}}%
\pgfpathlineto{\pgfqpoint{2.350109in}{1.331402in}}%
\pgfpathlineto{\pgfqpoint{2.406938in}{1.312606in}}%
\pgfpathlineto{\pgfqpoint{2.462547in}{1.291874in}}%
\pgfpathlineto{\pgfqpoint{2.517411in}{1.269045in}}%
\pgfpathlineto{\pgfqpoint{2.571978in}{1.243926in}}%
\pgfpathlineto{\pgfqpoint{2.626729in}{1.216265in}}%
\pgfpathlineto{\pgfqpoint{2.682184in}{1.185743in}}%
\pgfpathlineto{\pgfqpoint{2.738998in}{1.151915in}}%
\pgfpathlineto{\pgfqpoint{2.798045in}{1.114141in}}%
\pgfpathlineto{\pgfqpoint{2.860663in}{1.071403in}}%
\pgfpathlineto{\pgfqpoint{2.929260in}{1.021830in}}%
\pgfpathlineto{\pgfqpoint{3.009611in}{0.960907in}}%
\pgfpathlineto{\pgfqpoint{3.134311in}{0.863128in}}%
\pgfpathlineto{\pgfqpoint{3.251566in}{0.772185in}}%
\pgfpathlineto{\pgfqpoint{3.321832in}{0.720486in}}%
\pgfpathlineto{\pgfqpoint{3.380133in}{0.680284in}}%
\pgfpathlineto{\pgfqpoint{3.431763in}{0.647327in}}%
\pgfpathlineto{\pgfqpoint{3.478957in}{0.619796in}}%
\pgfpathlineto{\pgfqpoint{3.522925in}{0.596683in}}%
\pgfpathlineto{\pgfqpoint{3.564450in}{0.577334in}}%
\pgfpathlineto{\pgfqpoint{3.604073in}{0.561297in}}%
\pgfpathlineto{\pgfqpoint{3.642202in}{0.548239in}}%
\pgfpathlineto{\pgfqpoint{3.679163in}{0.537913in}}%
\pgfpathlineto{\pgfqpoint{3.715229in}{0.530131in}}%
\pgfpathlineto{\pgfqpoint{3.750640in}{0.524753in}}%
\pgfpathlineto{\pgfqpoint{3.785615in}{0.521684in}}%
\pgfpathlineto{\pgfqpoint{3.820364in}{0.520862in}}%
\pgfpathlineto{\pgfqpoint{3.855078in}{0.522265in}}%
\pgfpathlineto{\pgfqpoint{3.889961in}{0.525904in}}%
\pgfpathlineto{\pgfqpoint{3.925217in}{0.531825in}}%
\pgfpathlineto{\pgfqpoint{3.961058in}{0.540110in}}%
\pgfpathlineto{\pgfqpoint{3.997716in}{0.550881in}}%
\pgfpathlineto{\pgfqpoint{4.035451in}{0.564304in}}%
\pgfpathlineto{\pgfqpoint{4.074567in}{0.580601in}}%
\pgfpathlineto{\pgfqpoint{4.115443in}{0.600068in}}%
\pgfpathlineto{\pgfqpoint{4.158566in}{0.623100in}}%
\pgfpathlineto{\pgfqpoint{4.204612in}{0.650253in}}%
\pgfpathlineto{\pgfqpoint{4.254581in}{0.682351in}}%
\pgfpathlineto{\pgfqpoint{4.310169in}{0.720767in}}%
\pgfpathlineto{\pgfqpoint{4.374766in}{0.768207in}}%
\pgfpathlineto{\pgfqpoint{4.458230in}{0.832447in}}%
\pgfpathlineto{\pgfqpoint{4.698249in}{1.018789in}}%
\pgfpathlineto{\pgfqpoint{4.770621in}{1.071176in}}%
\pgfpathlineto{\pgfqpoint{4.835259in}{1.115263in}}%
\pgfpathlineto{\pgfqpoint{4.895595in}{1.153756in}}%
\pgfpathlineto{\pgfqpoint{4.953304in}{1.187961in}}%
\pgfpathlineto{\pgfqpoint{5.009407in}{1.218647in}}%
\pgfpathlineto{\pgfqpoint{5.064615in}{1.246323in}}%
\pgfpathlineto{\pgfqpoint{5.119493in}{1.271354in}}%
\pgfpathlineto{\pgfqpoint{5.174532in}{1.294020in}}%
\pgfpathlineto{\pgfqpoint{5.230184in}{1.314534in}}%
\pgfpathlineto{\pgfqpoint{5.286914in}{1.333073in}}%
\pgfpathlineto{\pgfqpoint{5.345214in}{1.349778in}}%
\pgfpathlineto{\pgfqpoint{5.405635in}{1.364767in}}%
\pgfpathlineto{\pgfqpoint{5.468823in}{1.378137in}}%
\pgfpathlineto{\pgfqpoint{5.535581in}{1.389969in}}%
\pgfpathlineto{\pgfqpoint{5.606939in}{1.400333in}}%
\pgfpathlineto{\pgfqpoint{5.684276in}{1.409287in}}%
\pgfpathlineto{\pgfqpoint{5.769543in}{1.416881in}}%
\pgfpathlineto{\pgfqpoint{5.865663in}{1.423159in}}%
\pgfpathlineto{\pgfqpoint{5.977362in}{1.428162in}}%
\pgfpathlineto{\pgfqpoint{6.104717in}{1.431750in}}%
\pgfpathlineto{\pgfqpoint{6.104717in}{1.431750in}}%
\pgfusepath{stroke}%
\end{pgfscope}%
\begin{pgfscope}%
\pgfsetrectcap%
\pgfsetmiterjoin%
\pgfsetlinewidth{0.803000pt}%
\definecolor{currentstroke}{rgb}{0.737255,0.737255,0.737255}%
\pgfsetstrokecolor{currentstroke}%
\pgfsetdash{}{0pt}%
\pgfpathmoveto{\pgfqpoint{0.470474in}{0.429287in}}%
\pgfpathlineto{\pgfqpoint{0.470474in}{2.443527in}}%
\pgfusepath{stroke}%
\end{pgfscope}%
\begin{pgfscope}%
\pgfsetrectcap%
\pgfsetmiterjoin%
\pgfsetlinewidth{0.803000pt}%
\definecolor{currentstroke}{rgb}{0.737255,0.737255,0.737255}%
\pgfsetstrokecolor{currentstroke}%
\pgfsetdash{}{0pt}%
\pgfpathmoveto{\pgfqpoint{6.104713in}{0.429287in}}%
\pgfpathlineto{\pgfqpoint{6.104713in}{2.443527in}}%
\pgfusepath{stroke}%
\end{pgfscope}%
\begin{pgfscope}%
\pgfsetrectcap%
\pgfsetmiterjoin%
\pgfsetlinewidth{0.803000pt}%
\definecolor{currentstroke}{rgb}{0.737255,0.737255,0.737255}%
\pgfsetstrokecolor{currentstroke}%
\pgfsetdash{}{0pt}%
\pgfpathmoveto{\pgfqpoint{0.470474in}{0.429287in}}%
\pgfpathlineto{\pgfqpoint{6.104713in}{0.429287in}}%
\pgfusepath{stroke}%
\end{pgfscope}%
\begin{pgfscope}%
\pgfsetrectcap%
\pgfsetmiterjoin%
\pgfsetlinewidth{0.803000pt}%
\definecolor{currentstroke}{rgb}{0.737255,0.737255,0.737255}%
\pgfsetstrokecolor{currentstroke}%
\pgfsetdash{}{0pt}%
\pgfpathmoveto{\pgfqpoint{0.470474in}{2.443527in}}%
\pgfpathlineto{\pgfqpoint{6.104713in}{2.443527in}}%
\pgfusepath{stroke}%
\end{pgfscope}%
\begin{pgfscope}%
\pgfsetroundcap%
\pgfsetroundjoin%
\pgfsetlinewidth{0.501875pt}%
\definecolor{currentstroke}{rgb}{0.000000,0.000000,0.000000}%
\pgfsetstrokecolor{currentstroke}%
\pgfsetdash{}{0pt}%
\pgfpathmoveto{\pgfqpoint{1.476151in}{1.682233in}}%
\pgfpathquadraticcurveto{\pgfqpoint{1.953789in}{1.562984in}}{\pgfqpoint{2.442787in}{1.658335in}}%
\pgfusepath{stroke}%
\end{pgfscope}%
\begin{pgfscope}%
\pgfsetroundcap%
\pgfsetroundjoin%
\pgfsetlinewidth{0.501875pt}%
\definecolor{currentstroke}{rgb}{0.000000,0.000000,0.000000}%
\pgfsetstrokecolor{currentstroke}%
\pgfsetdash{}{0pt}%
\pgfpathmoveto{\pgfqpoint{2.388926in}{1.673304in}}%
\pgfpathlineto{\pgfqpoint{2.442787in}{1.658335in}}%
\pgfpathlineto{\pgfqpoint{2.398496in}{1.624228in}}%
\pgfusepath{stroke}%
\end{pgfscope}%
\begin{pgfscope}%
\definecolor{textcolor}{rgb}{0.180392,0.180392,0.180392}%
\pgfsetstrokecolor{textcolor}%
\pgfsetfillcolor{textcolor}%
\pgftext[x=1.174754in,y=1.779743in,,]{\color{textcolor}\rmfamily\fontsize{9.000000}{10.800000}\selectfont Envelope signal, A(r,t)}%
\end{pgfscope}%
\begin{pgfscope}%
\pgfsetroundcap%
\pgfsetroundjoin%
\pgfsetlinewidth{0.501875pt}%
\definecolor{currentstroke}{rgb}{0.000000,0.000000,0.000000}%
\pgfsetstrokecolor{currentstroke}%
\pgfsetdash{}{0pt}%
\pgfpathmoveto{\pgfqpoint{1.224417in}{1.663893in}}%
\pgfpathquadraticcurveto{\pgfqpoint{1.595163in}{0.877654in}}{\pgfqpoint{2.444470in}{1.194903in}}%
\pgfusepath{stroke}%
\end{pgfscope}%
\begin{pgfscope}%
\pgfsetroundcap%
\pgfsetroundjoin%
\pgfsetlinewidth{0.501875pt}%
\definecolor{currentstroke}{rgb}{0.000000,0.000000,0.000000}%
\pgfsetstrokecolor{currentstroke}%
\pgfsetdash{}{0pt}%
\pgfpathmoveto{\pgfqpoint{2.388883in}{1.200827in}}%
\pgfpathlineto{\pgfqpoint{2.444470in}{1.194903in}}%
\pgfpathlineto{\pgfqpoint{2.406379in}{1.153988in}}%
\pgfusepath{stroke}%
\end{pgfscope}%
\begin{pgfscope}%
\definecolor{textcolor}{rgb}{0.180392,0.180392,0.180392}%
\pgfsetstrokecolor{textcolor}%
\pgfsetfillcolor{textcolor}%
\pgftext[x=1.174754in,y=1.779743in,,]{\color{textcolor}\rmfamily\fontsize{9.000000}{10.800000}\selectfont Envelope signal, A(r,t)}%
\end{pgfscope}%
\begin{pgfscope}%
\pgfsetroundcap%
\pgfsetroundjoin%
\pgfsetlinewidth{0.501875pt}%
\definecolor{currentstroke}{rgb}{0.000000,0.000000,0.000000}%
\pgfsetstrokecolor{currentstroke}%
\pgfsetdash{}{0pt}%
\pgfpathmoveto{\pgfqpoint{2.092385in}{0.732644in}}%
\pgfpathquadraticcurveto{\pgfqpoint{2.570880in}{0.852267in}}{\pgfqpoint{3.041843in}{0.970008in}}%
\pgfusepath{stroke}%
\end{pgfscope}%
\begin{pgfscope}%
\pgfsetroundcap%
\pgfsetroundjoin%
\pgfsetlinewidth{0.501875pt}%
\definecolor{currentstroke}{rgb}{0.000000,0.000000,0.000000}%
\pgfsetstrokecolor{currentstroke}%
\pgfsetdash{}{0pt}%
\pgfpathmoveto{\pgfqpoint{2.987272in}{0.982135in}}%
\pgfpathlineto{\pgfqpoint{3.041843in}{0.970008in}}%
\pgfpathlineto{\pgfqpoint{2.999399in}{0.933628in}}%
\pgfusepath{stroke}%
\end{pgfscope}%
\begin{pgfscope}%
\definecolor{textcolor}{rgb}{0.180392,0.180392,0.180392}%
\pgfsetstrokecolor{textcolor}%
\pgfsetfillcolor{textcolor}%
\pgftext[x=1.702964in,y=0.635288in,,]{\color{textcolor}\rmfamily\fontsize{9.000000}{10.800000}\selectfont Signal Propagating in Fibre, E(r,t)}%
\end{pgfscope}%
\begin{pgfscope}%
\definecolor{textcolor}{rgb}{0.180392,0.180392,0.180392}%
\pgfsetstrokecolor{textcolor}%
\pgfsetfillcolor{textcolor}%
\pgftext[x=3.287593in,y=2.526860in,,base]{\color{textcolor}\rmfamily\fontsize{12.960000}{15.552000}\selectfont Pulsed Gaussian Signal}%
\end{pgfscope}%
\end{pgfpicture}%
\makeatother%
\endgroup%
}
		\caption{\centering{An example graph of the Gaussian Pulse, the envelope signal is shown as a dotted line, the signal propagated in the medium is the solid line.}}
		\label{fig:gaussian_demo}
	\end{figure}	
	
	\pagebreak
	
	It is also important to note here that the envelope signal is itself comprised of two components, a positive and a negative portion, this is easily displayed in the frequency domain, observe the spectrum (Figure \ref{fig:gaussian_spectrum}) of the propagating signal (displayed before in Figure \ref{fig:gaussian_demo}), there are a positive and negative frequency components: \linebreak
	
	\begin{figure}[h]
		\centering
		\scalebox{0.85}{%% Creator: Matplotlib, PGF backend
%%
%% To include the figure in your LaTeX document, write
%%   \input{<filename>.pgf}
%%
%% Make sure the required packages are loaded in your preamble
%%   \usepackage{pgf}
%%
%% Also ensure that all the required font packages are loaded; for instance,
%% the lmodern package is sometimes necessary when using math font.
%%   \usepackage{lmodern}
%%
%% Figures using additional raster images can only be included by \input if
%% they are in the same directory as the main LaTeX file. For loading figures
%% from other directories you can use the `import` package
%%   \usepackage{import}
%%
%% and then include the figures with
%%   \import{<path to file>}{<filename>.pgf}
%%
%% Matplotlib used the following preamble
%%   \usepackage[T1]{fontenc} \usepackage{mathpazo}
%%
\begingroup%
\makeatletter%
\begin{pgfpicture}%
\pgfpathrectangle{\pgfpointorigin}{\pgfqpoint{6.020575in}{2.250534in}}%
\pgfusepath{use as bounding box, clip}%
\begin{pgfscope}%
\pgfsetbuttcap%
\pgfsetmiterjoin%
\definecolor{currentfill}{rgb}{1.000000,1.000000,1.000000}%
\pgfsetfillcolor{currentfill}%
\pgfsetlinewidth{0.000000pt}%
\definecolor{currentstroke}{rgb}{1.000000,1.000000,1.000000}%
\pgfsetstrokecolor{currentstroke}%
\pgfsetdash{}{0pt}%
\pgfpathmoveto{\pgfqpoint{0.000000in}{0.000000in}}%
\pgfpathlineto{\pgfqpoint{6.020575in}{0.000000in}}%
\pgfpathlineto{\pgfqpoint{6.020575in}{2.250534in}}%
\pgfpathlineto{\pgfqpoint{0.000000in}{2.250534in}}%
\pgfpathlineto{\pgfqpoint{0.000000in}{0.000000in}}%
\pgfpathclose%
\pgfusepath{fill}%
\end{pgfscope}%
\begin{pgfscope}%
\pgfsetbuttcap%
\pgfsetmiterjoin%
\definecolor{currentfill}{rgb}{0.933333,0.933333,0.933333}%
\pgfsetfillcolor{currentfill}%
\pgfsetlinewidth{0.000000pt}%
\definecolor{currentstroke}{rgb}{0.000000,0.000000,0.000000}%
\pgfsetstrokecolor{currentstroke}%
\pgfsetstrokeopacity{0.000000}%
\pgfsetdash{}{0pt}%
\pgfpathmoveto{\pgfqpoint{0.477328in}{0.429287in}}%
\pgfpathlineto{\pgfqpoint{5.958075in}{0.429287in}}%
\pgfpathlineto{\pgfqpoint{5.958075in}{2.036972in}}%
\pgfpathlineto{\pgfqpoint{0.477328in}{2.036972in}}%
\pgfpathlineto{\pgfqpoint{0.477328in}{0.429287in}}%
\pgfpathclose%
\pgfusepath{fill}%
\end{pgfscope}%
\begin{pgfscope}%
\pgfpathrectangle{\pgfqpoint{0.477328in}{0.429287in}}{\pgfqpoint{5.480747in}{1.607685in}}%
\pgfusepath{clip}%
\pgfsetbuttcap%
\pgfsetroundjoin%
\pgfsetlinewidth{0.501875pt}%
\definecolor{currentstroke}{rgb}{0.698039,0.698039,0.698039}%
\pgfsetstrokecolor{currentstroke}%
\pgfsetdash{{1.850000pt}{0.800000pt}}{0.000000pt}%
\pgfpathmoveto{\pgfqpoint{0.477328in}{0.429287in}}%
\pgfpathlineto{\pgfqpoint{0.477328in}{2.036972in}}%
\pgfusepath{stroke}%
\end{pgfscope}%
\begin{pgfscope}%
\pgfsetbuttcap%
\pgfsetroundjoin%
\definecolor{currentfill}{rgb}{0.180392,0.180392,0.180392}%
\pgfsetfillcolor{currentfill}%
\pgfsetlinewidth{0.803000pt}%
\definecolor{currentstroke}{rgb}{0.180392,0.180392,0.180392}%
\pgfsetstrokecolor{currentstroke}%
\pgfsetdash{}{0pt}%
\pgfsys@defobject{currentmarker}{\pgfqpoint{0.000000in}{-0.048611in}}{\pgfqpoint{0.000000in}{0.000000in}}{%
\pgfpathmoveto{\pgfqpoint{0.000000in}{0.000000in}}%
\pgfpathlineto{\pgfqpoint{0.000000in}{-0.048611in}}%
\pgfusepath{stroke,fill}%
}%
\begin{pgfscope}%
\pgfsys@transformshift{0.477328in}{0.429287in}%
\pgfsys@useobject{currentmarker}{}%
\end{pgfscope}%
\end{pgfscope}%
\begin{pgfscope}%
\definecolor{textcolor}{rgb}{0.180392,0.180392,0.180392}%
\pgfsetstrokecolor{textcolor}%
\pgfsetfillcolor{textcolor}%
\pgftext[x=0.477328in,y=0.332064in,,top]{\color{textcolor}\rmfamily\fontsize{9.000000}{10.800000}\selectfont \(\displaystyle {\ensuremath{-}30}\)}%
\end{pgfscope}%
\begin{pgfscope}%
\pgfpathrectangle{\pgfqpoint{0.477328in}{0.429287in}}{\pgfqpoint{5.480747in}{1.607685in}}%
\pgfusepath{clip}%
\pgfsetbuttcap%
\pgfsetroundjoin%
\pgfsetlinewidth{0.501875pt}%
\definecolor{currentstroke}{rgb}{0.698039,0.698039,0.698039}%
\pgfsetstrokecolor{currentstroke}%
\pgfsetdash{{1.850000pt}{0.800000pt}}{0.000000pt}%
\pgfpathmoveto{\pgfqpoint{1.390785in}{0.429287in}}%
\pgfpathlineto{\pgfqpoint{1.390785in}{2.036972in}}%
\pgfusepath{stroke}%
\end{pgfscope}%
\begin{pgfscope}%
\pgfsetbuttcap%
\pgfsetroundjoin%
\definecolor{currentfill}{rgb}{0.180392,0.180392,0.180392}%
\pgfsetfillcolor{currentfill}%
\pgfsetlinewidth{0.803000pt}%
\definecolor{currentstroke}{rgb}{0.180392,0.180392,0.180392}%
\pgfsetstrokecolor{currentstroke}%
\pgfsetdash{}{0pt}%
\pgfsys@defobject{currentmarker}{\pgfqpoint{0.000000in}{-0.048611in}}{\pgfqpoint{0.000000in}{0.000000in}}{%
\pgfpathmoveto{\pgfqpoint{0.000000in}{0.000000in}}%
\pgfpathlineto{\pgfqpoint{0.000000in}{-0.048611in}}%
\pgfusepath{stroke,fill}%
}%
\begin{pgfscope}%
\pgfsys@transformshift{1.390785in}{0.429287in}%
\pgfsys@useobject{currentmarker}{}%
\end{pgfscope}%
\end{pgfscope}%
\begin{pgfscope}%
\definecolor{textcolor}{rgb}{0.180392,0.180392,0.180392}%
\pgfsetstrokecolor{textcolor}%
\pgfsetfillcolor{textcolor}%
\pgftext[x=1.390785in,y=0.332064in,,top]{\color{textcolor}\rmfamily\fontsize{9.000000}{10.800000}\selectfont \(\displaystyle {\ensuremath{-}20}\)}%
\end{pgfscope}%
\begin{pgfscope}%
\pgfpathrectangle{\pgfqpoint{0.477328in}{0.429287in}}{\pgfqpoint{5.480747in}{1.607685in}}%
\pgfusepath{clip}%
\pgfsetbuttcap%
\pgfsetroundjoin%
\pgfsetlinewidth{0.501875pt}%
\definecolor{currentstroke}{rgb}{0.698039,0.698039,0.698039}%
\pgfsetstrokecolor{currentstroke}%
\pgfsetdash{{1.850000pt}{0.800000pt}}{0.000000pt}%
\pgfpathmoveto{\pgfqpoint{2.304243in}{0.429287in}}%
\pgfpathlineto{\pgfqpoint{2.304243in}{2.036972in}}%
\pgfusepath{stroke}%
\end{pgfscope}%
\begin{pgfscope}%
\pgfsetbuttcap%
\pgfsetroundjoin%
\definecolor{currentfill}{rgb}{0.180392,0.180392,0.180392}%
\pgfsetfillcolor{currentfill}%
\pgfsetlinewidth{0.803000pt}%
\definecolor{currentstroke}{rgb}{0.180392,0.180392,0.180392}%
\pgfsetstrokecolor{currentstroke}%
\pgfsetdash{}{0pt}%
\pgfsys@defobject{currentmarker}{\pgfqpoint{0.000000in}{-0.048611in}}{\pgfqpoint{0.000000in}{0.000000in}}{%
\pgfpathmoveto{\pgfqpoint{0.000000in}{0.000000in}}%
\pgfpathlineto{\pgfqpoint{0.000000in}{-0.048611in}}%
\pgfusepath{stroke,fill}%
}%
\begin{pgfscope}%
\pgfsys@transformshift{2.304243in}{0.429287in}%
\pgfsys@useobject{currentmarker}{}%
\end{pgfscope}%
\end{pgfscope}%
\begin{pgfscope}%
\definecolor{textcolor}{rgb}{0.180392,0.180392,0.180392}%
\pgfsetstrokecolor{textcolor}%
\pgfsetfillcolor{textcolor}%
\pgftext[x=2.304243in,y=0.332064in,,top]{\color{textcolor}\rmfamily\fontsize{9.000000}{10.800000}\selectfont \(\displaystyle {\ensuremath{-}10}\)}%
\end{pgfscope}%
\begin{pgfscope}%
\pgfpathrectangle{\pgfqpoint{0.477328in}{0.429287in}}{\pgfqpoint{5.480747in}{1.607685in}}%
\pgfusepath{clip}%
\pgfsetbuttcap%
\pgfsetroundjoin%
\pgfsetlinewidth{0.501875pt}%
\definecolor{currentstroke}{rgb}{0.698039,0.698039,0.698039}%
\pgfsetstrokecolor{currentstroke}%
\pgfsetdash{{1.850000pt}{0.800000pt}}{0.000000pt}%
\pgfpathmoveto{\pgfqpoint{3.217701in}{0.429287in}}%
\pgfpathlineto{\pgfqpoint{3.217701in}{2.036972in}}%
\pgfusepath{stroke}%
\end{pgfscope}%
\begin{pgfscope}%
\pgfsetbuttcap%
\pgfsetroundjoin%
\definecolor{currentfill}{rgb}{0.180392,0.180392,0.180392}%
\pgfsetfillcolor{currentfill}%
\pgfsetlinewidth{0.803000pt}%
\definecolor{currentstroke}{rgb}{0.180392,0.180392,0.180392}%
\pgfsetstrokecolor{currentstroke}%
\pgfsetdash{}{0pt}%
\pgfsys@defobject{currentmarker}{\pgfqpoint{0.000000in}{-0.048611in}}{\pgfqpoint{0.000000in}{0.000000in}}{%
\pgfpathmoveto{\pgfqpoint{0.000000in}{0.000000in}}%
\pgfpathlineto{\pgfqpoint{0.000000in}{-0.048611in}}%
\pgfusepath{stroke,fill}%
}%
\begin{pgfscope}%
\pgfsys@transformshift{3.217701in}{0.429287in}%
\pgfsys@useobject{currentmarker}{}%
\end{pgfscope}%
\end{pgfscope}%
\begin{pgfscope}%
\definecolor{textcolor}{rgb}{0.180392,0.180392,0.180392}%
\pgfsetstrokecolor{textcolor}%
\pgfsetfillcolor{textcolor}%
\pgftext[x=3.217701in,y=0.332064in,,top]{\color{textcolor}\rmfamily\fontsize{9.000000}{10.800000}\selectfont \(\displaystyle {0}\)}%
\end{pgfscope}%
\begin{pgfscope}%
\pgfpathrectangle{\pgfqpoint{0.477328in}{0.429287in}}{\pgfqpoint{5.480747in}{1.607685in}}%
\pgfusepath{clip}%
\pgfsetbuttcap%
\pgfsetroundjoin%
\pgfsetlinewidth{0.501875pt}%
\definecolor{currentstroke}{rgb}{0.698039,0.698039,0.698039}%
\pgfsetstrokecolor{currentstroke}%
\pgfsetdash{{1.850000pt}{0.800000pt}}{0.000000pt}%
\pgfpathmoveto{\pgfqpoint{4.131159in}{0.429287in}}%
\pgfpathlineto{\pgfqpoint{4.131159in}{2.036972in}}%
\pgfusepath{stroke}%
\end{pgfscope}%
\begin{pgfscope}%
\pgfsetbuttcap%
\pgfsetroundjoin%
\definecolor{currentfill}{rgb}{0.180392,0.180392,0.180392}%
\pgfsetfillcolor{currentfill}%
\pgfsetlinewidth{0.803000pt}%
\definecolor{currentstroke}{rgb}{0.180392,0.180392,0.180392}%
\pgfsetstrokecolor{currentstroke}%
\pgfsetdash{}{0pt}%
\pgfsys@defobject{currentmarker}{\pgfqpoint{0.000000in}{-0.048611in}}{\pgfqpoint{0.000000in}{0.000000in}}{%
\pgfpathmoveto{\pgfqpoint{0.000000in}{0.000000in}}%
\pgfpathlineto{\pgfqpoint{0.000000in}{-0.048611in}}%
\pgfusepath{stroke,fill}%
}%
\begin{pgfscope}%
\pgfsys@transformshift{4.131159in}{0.429287in}%
\pgfsys@useobject{currentmarker}{}%
\end{pgfscope}%
\end{pgfscope}%
\begin{pgfscope}%
\definecolor{textcolor}{rgb}{0.180392,0.180392,0.180392}%
\pgfsetstrokecolor{textcolor}%
\pgfsetfillcolor{textcolor}%
\pgftext[x=4.131159in,y=0.332064in,,top]{\color{textcolor}\rmfamily\fontsize{9.000000}{10.800000}\selectfont \(\displaystyle {10}\)}%
\end{pgfscope}%
\begin{pgfscope}%
\pgfpathrectangle{\pgfqpoint{0.477328in}{0.429287in}}{\pgfqpoint{5.480747in}{1.607685in}}%
\pgfusepath{clip}%
\pgfsetbuttcap%
\pgfsetroundjoin%
\pgfsetlinewidth{0.501875pt}%
\definecolor{currentstroke}{rgb}{0.698039,0.698039,0.698039}%
\pgfsetstrokecolor{currentstroke}%
\pgfsetdash{{1.850000pt}{0.800000pt}}{0.000000pt}%
\pgfpathmoveto{\pgfqpoint{5.044617in}{0.429287in}}%
\pgfpathlineto{\pgfqpoint{5.044617in}{2.036972in}}%
\pgfusepath{stroke}%
\end{pgfscope}%
\begin{pgfscope}%
\pgfsetbuttcap%
\pgfsetroundjoin%
\definecolor{currentfill}{rgb}{0.180392,0.180392,0.180392}%
\pgfsetfillcolor{currentfill}%
\pgfsetlinewidth{0.803000pt}%
\definecolor{currentstroke}{rgb}{0.180392,0.180392,0.180392}%
\pgfsetstrokecolor{currentstroke}%
\pgfsetdash{}{0pt}%
\pgfsys@defobject{currentmarker}{\pgfqpoint{0.000000in}{-0.048611in}}{\pgfqpoint{0.000000in}{0.000000in}}{%
\pgfpathmoveto{\pgfqpoint{0.000000in}{0.000000in}}%
\pgfpathlineto{\pgfqpoint{0.000000in}{-0.048611in}}%
\pgfusepath{stroke,fill}%
}%
\begin{pgfscope}%
\pgfsys@transformshift{5.044617in}{0.429287in}%
\pgfsys@useobject{currentmarker}{}%
\end{pgfscope}%
\end{pgfscope}%
\begin{pgfscope}%
\definecolor{textcolor}{rgb}{0.180392,0.180392,0.180392}%
\pgfsetstrokecolor{textcolor}%
\pgfsetfillcolor{textcolor}%
\pgftext[x=5.044617in,y=0.332064in,,top]{\color{textcolor}\rmfamily\fontsize{9.000000}{10.800000}\selectfont \(\displaystyle {20}\)}%
\end{pgfscope}%
\begin{pgfscope}%
\pgfpathrectangle{\pgfqpoint{0.477328in}{0.429287in}}{\pgfqpoint{5.480747in}{1.607685in}}%
\pgfusepath{clip}%
\pgfsetbuttcap%
\pgfsetroundjoin%
\pgfsetlinewidth{0.501875pt}%
\definecolor{currentstroke}{rgb}{0.698039,0.698039,0.698039}%
\pgfsetstrokecolor{currentstroke}%
\pgfsetdash{{1.850000pt}{0.800000pt}}{0.000000pt}%
\pgfpathmoveto{\pgfqpoint{5.958075in}{0.429287in}}%
\pgfpathlineto{\pgfqpoint{5.958075in}{2.036972in}}%
\pgfusepath{stroke}%
\end{pgfscope}%
\begin{pgfscope}%
\pgfsetbuttcap%
\pgfsetroundjoin%
\definecolor{currentfill}{rgb}{0.180392,0.180392,0.180392}%
\pgfsetfillcolor{currentfill}%
\pgfsetlinewidth{0.803000pt}%
\definecolor{currentstroke}{rgb}{0.180392,0.180392,0.180392}%
\pgfsetstrokecolor{currentstroke}%
\pgfsetdash{}{0pt}%
\pgfsys@defobject{currentmarker}{\pgfqpoint{0.000000in}{-0.048611in}}{\pgfqpoint{0.000000in}{0.000000in}}{%
\pgfpathmoveto{\pgfqpoint{0.000000in}{0.000000in}}%
\pgfpathlineto{\pgfqpoint{0.000000in}{-0.048611in}}%
\pgfusepath{stroke,fill}%
}%
\begin{pgfscope}%
\pgfsys@transformshift{5.958075in}{0.429287in}%
\pgfsys@useobject{currentmarker}{}%
\end{pgfscope}%
\end{pgfscope}%
\begin{pgfscope}%
\definecolor{textcolor}{rgb}{0.180392,0.180392,0.180392}%
\pgfsetstrokecolor{textcolor}%
\pgfsetfillcolor{textcolor}%
\pgftext[x=5.958075in,y=0.332064in,,top]{\color{textcolor}\rmfamily\fontsize{9.000000}{10.800000}\selectfont \(\displaystyle {30}\)}%
\end{pgfscope}%
\begin{pgfscope}%
\definecolor{textcolor}{rgb}{0.180392,0.180392,0.180392}%
\pgfsetstrokecolor{textcolor}%
\pgfsetfillcolor{textcolor}%
\pgftext[x=3.217701in,y=0.150823in,,top]{\color{textcolor}\rmfamily\fontsize{10.800000}{12.960000}\selectfont Angular Frequency, \(\displaystyle \omega\)}%
\end{pgfscope}%
\begin{pgfscope}%
\pgfpathrectangle{\pgfqpoint{0.477328in}{0.429287in}}{\pgfqpoint{5.480747in}{1.607685in}}%
\pgfusepath{clip}%
\pgfsetbuttcap%
\pgfsetroundjoin%
\pgfsetlinewidth{0.501875pt}%
\definecolor{currentstroke}{rgb}{0.698039,0.698039,0.698039}%
\pgfsetstrokecolor{currentstroke}%
\pgfsetdash{{1.850000pt}{0.800000pt}}{0.000000pt}%
\pgfpathmoveto{\pgfqpoint{0.477328in}{0.502363in}}%
\pgfpathlineto{\pgfqpoint{5.958075in}{0.502363in}}%
\pgfusepath{stroke}%
\end{pgfscope}%
\begin{pgfscope}%
\pgfsetbuttcap%
\pgfsetroundjoin%
\definecolor{currentfill}{rgb}{0.180392,0.180392,0.180392}%
\pgfsetfillcolor{currentfill}%
\pgfsetlinewidth{0.803000pt}%
\definecolor{currentstroke}{rgb}{0.180392,0.180392,0.180392}%
\pgfsetstrokecolor{currentstroke}%
\pgfsetdash{}{0pt}%
\pgfsys@defobject{currentmarker}{\pgfqpoint{-0.048611in}{0.000000in}}{\pgfqpoint{-0.000000in}{0.000000in}}{%
\pgfpathmoveto{\pgfqpoint{-0.000000in}{0.000000in}}%
\pgfpathlineto{\pgfqpoint{-0.048611in}{0.000000in}}%
\pgfusepath{stroke,fill}%
}%
\begin{pgfscope}%
\pgfsys@transformshift{0.477328in}{0.502363in}%
\pgfsys@useobject{currentmarker}{}%
\end{pgfscope}%
\end{pgfscope}%
\begin{pgfscope}%
\definecolor{textcolor}{rgb}{0.180392,0.180392,0.180392}%
\pgfsetstrokecolor{textcolor}%
\pgfsetfillcolor{textcolor}%
\pgftext[x=0.223855in, y=0.457145in, left, base]{\color{textcolor}\rmfamily\fontsize{9.000000}{10.800000}\selectfont \(\displaystyle {0.0}\)}%
\end{pgfscope}%
\begin{pgfscope}%
\pgfpathrectangle{\pgfqpoint{0.477328in}{0.429287in}}{\pgfqpoint{5.480747in}{1.607685in}}%
\pgfusepath{clip}%
\pgfsetbuttcap%
\pgfsetroundjoin%
\pgfsetlinewidth{0.501875pt}%
\definecolor{currentstroke}{rgb}{0.698039,0.698039,0.698039}%
\pgfsetstrokecolor{currentstroke}%
\pgfsetdash{{1.850000pt}{0.800000pt}}{0.000000pt}%
\pgfpathmoveto{\pgfqpoint{0.477328in}{0.794670in}}%
\pgfpathlineto{\pgfqpoint{5.958075in}{0.794670in}}%
\pgfusepath{stroke}%
\end{pgfscope}%
\begin{pgfscope}%
\pgfsetbuttcap%
\pgfsetroundjoin%
\definecolor{currentfill}{rgb}{0.180392,0.180392,0.180392}%
\pgfsetfillcolor{currentfill}%
\pgfsetlinewidth{0.803000pt}%
\definecolor{currentstroke}{rgb}{0.180392,0.180392,0.180392}%
\pgfsetstrokecolor{currentstroke}%
\pgfsetdash{}{0pt}%
\pgfsys@defobject{currentmarker}{\pgfqpoint{-0.048611in}{0.000000in}}{\pgfqpoint{-0.000000in}{0.000000in}}{%
\pgfpathmoveto{\pgfqpoint{-0.000000in}{0.000000in}}%
\pgfpathlineto{\pgfqpoint{-0.048611in}{0.000000in}}%
\pgfusepath{stroke,fill}%
}%
\begin{pgfscope}%
\pgfsys@transformshift{0.477328in}{0.794670in}%
\pgfsys@useobject{currentmarker}{}%
\end{pgfscope}%
\end{pgfscope}%
\begin{pgfscope}%
\definecolor{textcolor}{rgb}{0.180392,0.180392,0.180392}%
\pgfsetstrokecolor{textcolor}%
\pgfsetfillcolor{textcolor}%
\pgftext[x=0.223855in, y=0.749451in, left, base]{\color{textcolor}\rmfamily\fontsize{9.000000}{10.800000}\selectfont \(\displaystyle {0.2}\)}%
\end{pgfscope}%
\begin{pgfscope}%
\pgfpathrectangle{\pgfqpoint{0.477328in}{0.429287in}}{\pgfqpoint{5.480747in}{1.607685in}}%
\pgfusepath{clip}%
\pgfsetbuttcap%
\pgfsetroundjoin%
\pgfsetlinewidth{0.501875pt}%
\definecolor{currentstroke}{rgb}{0.698039,0.698039,0.698039}%
\pgfsetstrokecolor{currentstroke}%
\pgfsetdash{{1.850000pt}{0.800000pt}}{0.000000pt}%
\pgfpathmoveto{\pgfqpoint{0.477328in}{1.086976in}}%
\pgfpathlineto{\pgfqpoint{5.958075in}{1.086976in}}%
\pgfusepath{stroke}%
\end{pgfscope}%
\begin{pgfscope}%
\pgfsetbuttcap%
\pgfsetroundjoin%
\definecolor{currentfill}{rgb}{0.180392,0.180392,0.180392}%
\pgfsetfillcolor{currentfill}%
\pgfsetlinewidth{0.803000pt}%
\definecolor{currentstroke}{rgb}{0.180392,0.180392,0.180392}%
\pgfsetstrokecolor{currentstroke}%
\pgfsetdash{}{0pt}%
\pgfsys@defobject{currentmarker}{\pgfqpoint{-0.048611in}{0.000000in}}{\pgfqpoint{-0.000000in}{0.000000in}}{%
\pgfpathmoveto{\pgfqpoint{-0.000000in}{0.000000in}}%
\pgfpathlineto{\pgfqpoint{-0.048611in}{0.000000in}}%
\pgfusepath{stroke,fill}%
}%
\begin{pgfscope}%
\pgfsys@transformshift{0.477328in}{1.086976in}%
\pgfsys@useobject{currentmarker}{}%
\end{pgfscope}%
\end{pgfscope}%
\begin{pgfscope}%
\definecolor{textcolor}{rgb}{0.180392,0.180392,0.180392}%
\pgfsetstrokecolor{textcolor}%
\pgfsetfillcolor{textcolor}%
\pgftext[x=0.223855in, y=1.041758in, left, base]{\color{textcolor}\rmfamily\fontsize{9.000000}{10.800000}\selectfont \(\displaystyle {0.4}\)}%
\end{pgfscope}%
\begin{pgfscope}%
\pgfpathrectangle{\pgfqpoint{0.477328in}{0.429287in}}{\pgfqpoint{5.480747in}{1.607685in}}%
\pgfusepath{clip}%
\pgfsetbuttcap%
\pgfsetroundjoin%
\pgfsetlinewidth{0.501875pt}%
\definecolor{currentstroke}{rgb}{0.698039,0.698039,0.698039}%
\pgfsetstrokecolor{currentstroke}%
\pgfsetdash{{1.850000pt}{0.800000pt}}{0.000000pt}%
\pgfpathmoveto{\pgfqpoint{0.477328in}{1.379283in}}%
\pgfpathlineto{\pgfqpoint{5.958075in}{1.379283in}}%
\pgfusepath{stroke}%
\end{pgfscope}%
\begin{pgfscope}%
\pgfsetbuttcap%
\pgfsetroundjoin%
\definecolor{currentfill}{rgb}{0.180392,0.180392,0.180392}%
\pgfsetfillcolor{currentfill}%
\pgfsetlinewidth{0.803000pt}%
\definecolor{currentstroke}{rgb}{0.180392,0.180392,0.180392}%
\pgfsetstrokecolor{currentstroke}%
\pgfsetdash{}{0pt}%
\pgfsys@defobject{currentmarker}{\pgfqpoint{-0.048611in}{0.000000in}}{\pgfqpoint{-0.000000in}{0.000000in}}{%
\pgfpathmoveto{\pgfqpoint{-0.000000in}{0.000000in}}%
\pgfpathlineto{\pgfqpoint{-0.048611in}{0.000000in}}%
\pgfusepath{stroke,fill}%
}%
\begin{pgfscope}%
\pgfsys@transformshift{0.477328in}{1.379283in}%
\pgfsys@useobject{currentmarker}{}%
\end{pgfscope}%
\end{pgfscope}%
\begin{pgfscope}%
\definecolor{textcolor}{rgb}{0.180392,0.180392,0.180392}%
\pgfsetstrokecolor{textcolor}%
\pgfsetfillcolor{textcolor}%
\pgftext[x=0.223855in, y=1.334064in, left, base]{\color{textcolor}\rmfamily\fontsize{9.000000}{10.800000}\selectfont \(\displaystyle {0.6}\)}%
\end{pgfscope}%
\begin{pgfscope}%
\pgfpathrectangle{\pgfqpoint{0.477328in}{0.429287in}}{\pgfqpoint{5.480747in}{1.607685in}}%
\pgfusepath{clip}%
\pgfsetbuttcap%
\pgfsetroundjoin%
\pgfsetlinewidth{0.501875pt}%
\definecolor{currentstroke}{rgb}{0.698039,0.698039,0.698039}%
\pgfsetstrokecolor{currentstroke}%
\pgfsetdash{{1.850000pt}{0.800000pt}}{0.000000pt}%
\pgfpathmoveto{\pgfqpoint{0.477328in}{1.671589in}}%
\pgfpathlineto{\pgfqpoint{5.958075in}{1.671589in}}%
\pgfusepath{stroke}%
\end{pgfscope}%
\begin{pgfscope}%
\pgfsetbuttcap%
\pgfsetroundjoin%
\definecolor{currentfill}{rgb}{0.180392,0.180392,0.180392}%
\pgfsetfillcolor{currentfill}%
\pgfsetlinewidth{0.803000pt}%
\definecolor{currentstroke}{rgb}{0.180392,0.180392,0.180392}%
\pgfsetstrokecolor{currentstroke}%
\pgfsetdash{}{0pt}%
\pgfsys@defobject{currentmarker}{\pgfqpoint{-0.048611in}{0.000000in}}{\pgfqpoint{-0.000000in}{0.000000in}}{%
\pgfpathmoveto{\pgfqpoint{-0.000000in}{0.000000in}}%
\pgfpathlineto{\pgfqpoint{-0.048611in}{0.000000in}}%
\pgfusepath{stroke,fill}%
}%
\begin{pgfscope}%
\pgfsys@transformshift{0.477328in}{1.671589in}%
\pgfsys@useobject{currentmarker}{}%
\end{pgfscope}%
\end{pgfscope}%
\begin{pgfscope}%
\definecolor{textcolor}{rgb}{0.180392,0.180392,0.180392}%
\pgfsetstrokecolor{textcolor}%
\pgfsetfillcolor{textcolor}%
\pgftext[x=0.223855in, y=1.626371in, left, base]{\color{textcolor}\rmfamily\fontsize{9.000000}{10.800000}\selectfont \(\displaystyle {0.8}\)}%
\end{pgfscope}%
\begin{pgfscope}%
\pgfpathrectangle{\pgfqpoint{0.477328in}{0.429287in}}{\pgfqpoint{5.480747in}{1.607685in}}%
\pgfusepath{clip}%
\pgfsetbuttcap%
\pgfsetroundjoin%
\pgfsetlinewidth{0.501875pt}%
\definecolor{currentstroke}{rgb}{0.698039,0.698039,0.698039}%
\pgfsetstrokecolor{currentstroke}%
\pgfsetdash{{1.850000pt}{0.800000pt}}{0.000000pt}%
\pgfpathmoveto{\pgfqpoint{0.477328in}{1.963896in}}%
\pgfpathlineto{\pgfqpoint{5.958075in}{1.963896in}}%
\pgfusepath{stroke}%
\end{pgfscope}%
\begin{pgfscope}%
\pgfsetbuttcap%
\pgfsetroundjoin%
\definecolor{currentfill}{rgb}{0.180392,0.180392,0.180392}%
\pgfsetfillcolor{currentfill}%
\pgfsetlinewidth{0.803000pt}%
\definecolor{currentstroke}{rgb}{0.180392,0.180392,0.180392}%
\pgfsetstrokecolor{currentstroke}%
\pgfsetdash{}{0pt}%
\pgfsys@defobject{currentmarker}{\pgfqpoint{-0.048611in}{0.000000in}}{\pgfqpoint{-0.000000in}{0.000000in}}{%
\pgfpathmoveto{\pgfqpoint{-0.000000in}{0.000000in}}%
\pgfpathlineto{\pgfqpoint{-0.048611in}{0.000000in}}%
\pgfusepath{stroke,fill}%
}%
\begin{pgfscope}%
\pgfsys@transformshift{0.477328in}{1.963896in}%
\pgfsys@useobject{currentmarker}{}%
\end{pgfscope}%
\end{pgfscope}%
\begin{pgfscope}%
\definecolor{textcolor}{rgb}{0.180392,0.180392,0.180392}%
\pgfsetstrokecolor{textcolor}%
\pgfsetfillcolor{textcolor}%
\pgftext[x=0.223855in, y=1.918677in, left, base]{\color{textcolor}\rmfamily\fontsize{9.000000}{10.800000}\selectfont \(\displaystyle {1.0}\)}%
\end{pgfscope}%
\begin{pgfscope}%
\definecolor{textcolor}{rgb}{0.180392,0.180392,0.180392}%
\pgfsetstrokecolor{textcolor}%
\pgfsetfillcolor{textcolor}%
\pgftext[x=0.168300in,y=1.233129in,,bottom,rotate=90.000000]{\color{textcolor}\rmfamily\fontsize{10.800000}{12.960000}\selectfont \(\displaystyle \tilde{E}(\omega)\)}%
\end{pgfscope}%
\begin{pgfscope}%
\pgfpathrectangle{\pgfqpoint{0.477328in}{0.429287in}}{\pgfqpoint{5.480747in}{1.607685in}}%
\pgfusepath{clip}%
\pgfsetrectcap%
\pgfsetroundjoin%
\pgfsetlinewidth{2.007500pt}%
\definecolor{currentstroke}{rgb}{0.000000,0.501961,0.000000}%
\pgfsetstrokecolor{currentstroke}%
\pgfsetdash{}{0pt}%
\pgfpathmoveto{\pgfqpoint{0.474256in}{0.502372in}}%
\pgfpathlineto{\pgfqpoint{0.566087in}{0.503104in}}%
\pgfpathlineto{\pgfqpoint{0.600523in}{0.505358in}}%
\pgfpathlineto{\pgfqpoint{0.623481in}{0.509390in}}%
\pgfpathlineto{\pgfqpoint{0.634960in}{0.512874in}}%
\pgfpathlineto{\pgfqpoint{0.646439in}{0.517840in}}%
\pgfpathlineto{\pgfqpoint{0.657918in}{0.524794in}}%
\pgfpathlineto{\pgfqpoint{0.669397in}{0.534363in}}%
\pgfpathlineto{\pgfqpoint{0.680875in}{0.547300in}}%
\pgfpathlineto{\pgfqpoint{0.692354in}{0.564478in}}%
\pgfpathlineto{\pgfqpoint{0.703833in}{0.586877in}}%
\pgfpathlineto{\pgfqpoint{0.715312in}{0.615551in}}%
\pgfpathlineto{\pgfqpoint{0.726791in}{0.651580in}}%
\pgfpathlineto{\pgfqpoint{0.738270in}{0.695995in}}%
\pgfpathlineto{\pgfqpoint{0.749748in}{0.749693in}}%
\pgfpathlineto{\pgfqpoint{0.761227in}{0.813333in}}%
\pgfpathlineto{\pgfqpoint{0.772706in}{0.887223in}}%
\pgfpathlineto{\pgfqpoint{0.784185in}{0.971207in}}%
\pgfpathlineto{\pgfqpoint{0.807143in}{1.165963in}}%
\pgfpathlineto{\pgfqpoint{0.864537in}{1.702093in}}%
\pgfpathlineto{\pgfqpoint{0.876016in}{1.790449in}}%
\pgfpathlineto{\pgfqpoint{0.887495in}{1.863644in}}%
\pgfpathlineto{\pgfqpoint{0.898973in}{1.918459in}}%
\pgfpathlineto{\pgfqpoint{0.910452in}{1.952402in}}%
\pgfpathlineto{\pgfqpoint{0.921931in}{1.963895in}}%
\pgfpathlineto{\pgfqpoint{0.933410in}{1.952400in}}%
\pgfpathlineto{\pgfqpoint{0.944889in}{1.918456in}}%
\pgfpathlineto{\pgfqpoint{0.956368in}{1.863639in}}%
\pgfpathlineto{\pgfqpoint{0.967847in}{1.790442in}}%
\pgfpathlineto{\pgfqpoint{0.979325in}{1.702086in}}%
\pgfpathlineto{\pgfqpoint{1.002283in}{1.494983in}}%
\pgfpathlineto{\pgfqpoint{1.036720in}{1.165954in}}%
\pgfpathlineto{\pgfqpoint{1.059677in}{0.971200in}}%
\pgfpathlineto{\pgfqpoint{1.071156in}{0.887217in}}%
\pgfpathlineto{\pgfqpoint{1.082635in}{0.813328in}}%
\pgfpathlineto{\pgfqpoint{1.094114in}{0.749688in}}%
\pgfpathlineto{\pgfqpoint{1.105593in}{0.695991in}}%
\pgfpathlineto{\pgfqpoint{1.117072in}{0.651577in}}%
\pgfpathlineto{\pgfqpoint{1.128550in}{0.615549in}}%
\pgfpathlineto{\pgfqpoint{1.140029in}{0.586875in}}%
\pgfpathlineto{\pgfqpoint{1.151508in}{0.564476in}}%
\pgfpathlineto{\pgfqpoint{1.162987in}{0.547299in}}%
\pgfpathlineto{\pgfqpoint{1.174466in}{0.534363in}}%
\pgfpathlineto{\pgfqpoint{1.185945in}{0.524793in}}%
\pgfpathlineto{\pgfqpoint{1.197424in}{0.517839in}}%
\pgfpathlineto{\pgfqpoint{1.208902in}{0.512874in}}%
\pgfpathlineto{\pgfqpoint{1.231860in}{0.506987in}}%
\pgfpathlineto{\pgfqpoint{1.254818in}{0.504273in}}%
\pgfpathlineto{\pgfqpoint{1.289254in}{0.502813in}}%
\pgfpathlineto{\pgfqpoint{1.381085in}{0.502368in}}%
\pgfpathlineto{\pgfqpoint{5.134669in}{0.502633in}}%
\pgfpathlineto{\pgfqpoint{5.180584in}{0.504273in}}%
\pgfpathlineto{\pgfqpoint{5.203542in}{0.506987in}}%
\pgfpathlineto{\pgfqpoint{5.226500in}{0.512874in}}%
\pgfpathlineto{\pgfqpoint{5.237979in}{0.517839in}}%
\pgfpathlineto{\pgfqpoint{5.249457in}{0.524793in}}%
\pgfpathlineto{\pgfqpoint{5.260936in}{0.534363in}}%
\pgfpathlineto{\pgfqpoint{5.272415in}{0.547299in}}%
\pgfpathlineto{\pgfqpoint{5.283894in}{0.564476in}}%
\pgfpathlineto{\pgfqpoint{5.295373in}{0.586875in}}%
\pgfpathlineto{\pgfqpoint{5.306852in}{0.615549in}}%
\pgfpathlineto{\pgfqpoint{5.318331in}{0.651577in}}%
\pgfpathlineto{\pgfqpoint{5.329809in}{0.695991in}}%
\pgfpathlineto{\pgfqpoint{5.341288in}{0.749688in}}%
\pgfpathlineto{\pgfqpoint{5.352767in}{0.813328in}}%
\pgfpathlineto{\pgfqpoint{5.364246in}{0.887217in}}%
\pgfpathlineto{\pgfqpoint{5.375725in}{0.971200in}}%
\pgfpathlineto{\pgfqpoint{5.398682in}{1.165954in}}%
\pgfpathlineto{\pgfqpoint{5.456077in}{1.702086in}}%
\pgfpathlineto{\pgfqpoint{5.467556in}{1.790442in}}%
\pgfpathlineto{\pgfqpoint{5.479034in}{1.863639in}}%
\pgfpathlineto{\pgfqpoint{5.490513in}{1.918456in}}%
\pgfpathlineto{\pgfqpoint{5.501992in}{1.952400in}}%
\pgfpathlineto{\pgfqpoint{5.513471in}{1.963895in}}%
\pgfpathlineto{\pgfqpoint{5.524950in}{1.952402in}}%
\pgfpathlineto{\pgfqpoint{5.536429in}{1.918459in}}%
\pgfpathlineto{\pgfqpoint{5.547908in}{1.863644in}}%
\pgfpathlineto{\pgfqpoint{5.559386in}{1.790449in}}%
\pgfpathlineto{\pgfqpoint{5.570865in}{1.702093in}}%
\pgfpathlineto{\pgfqpoint{5.593823in}{1.494992in}}%
\pgfpathlineto{\pgfqpoint{5.628259in}{1.165963in}}%
\pgfpathlineto{\pgfqpoint{5.651217in}{0.971207in}}%
\pgfpathlineto{\pgfqpoint{5.662696in}{0.887223in}}%
\pgfpathlineto{\pgfqpoint{5.674175in}{0.813333in}}%
\pgfpathlineto{\pgfqpoint{5.685654in}{0.749693in}}%
\pgfpathlineto{\pgfqpoint{5.697133in}{0.695995in}}%
\pgfpathlineto{\pgfqpoint{5.708611in}{0.651580in}}%
\pgfpathlineto{\pgfqpoint{5.720090in}{0.615551in}}%
\pgfpathlineto{\pgfqpoint{5.731569in}{0.586877in}}%
\pgfpathlineto{\pgfqpoint{5.743048in}{0.564478in}}%
\pgfpathlineto{\pgfqpoint{5.754527in}{0.547300in}}%
\pgfpathlineto{\pgfqpoint{5.766006in}{0.534363in}}%
\pgfpathlineto{\pgfqpoint{5.777484in}{0.524794in}}%
\pgfpathlineto{\pgfqpoint{5.788963in}{0.517840in}}%
\pgfpathlineto{\pgfqpoint{5.800442in}{0.512874in}}%
\pgfpathlineto{\pgfqpoint{5.823400in}{0.506987in}}%
\pgfpathlineto{\pgfqpoint{5.846358in}{0.504273in}}%
\pgfpathlineto{\pgfqpoint{5.880794in}{0.502813in}}%
\pgfpathlineto{\pgfqpoint{5.961146in}{0.502372in}}%
\pgfpathlineto{\pgfqpoint{5.961146in}{0.502372in}}%
\pgfusepath{stroke}%
\end{pgfscope}%
\begin{pgfscope}%
\pgfsetrectcap%
\pgfsetmiterjoin%
\pgfsetlinewidth{0.803000pt}%
\definecolor{currentstroke}{rgb}{0.737255,0.737255,0.737255}%
\pgfsetstrokecolor{currentstroke}%
\pgfsetdash{}{0pt}%
\pgfpathmoveto{\pgfqpoint{0.477328in}{0.429287in}}%
\pgfpathlineto{\pgfqpoint{0.477328in}{2.036972in}}%
\pgfusepath{stroke}%
\end{pgfscope}%
\begin{pgfscope}%
\pgfsetrectcap%
\pgfsetmiterjoin%
\pgfsetlinewidth{0.803000pt}%
\definecolor{currentstroke}{rgb}{0.737255,0.737255,0.737255}%
\pgfsetstrokecolor{currentstroke}%
\pgfsetdash{}{0pt}%
\pgfpathmoveto{\pgfqpoint{5.958075in}{0.429287in}}%
\pgfpathlineto{\pgfqpoint{5.958075in}{2.036972in}}%
\pgfusepath{stroke}%
\end{pgfscope}%
\begin{pgfscope}%
\pgfsetrectcap%
\pgfsetmiterjoin%
\pgfsetlinewidth{0.803000pt}%
\definecolor{currentstroke}{rgb}{0.737255,0.737255,0.737255}%
\pgfsetstrokecolor{currentstroke}%
\pgfsetdash{}{0pt}%
\pgfpathmoveto{\pgfqpoint{0.477328in}{0.429287in}}%
\pgfpathlineto{\pgfqpoint{5.958075in}{0.429287in}}%
\pgfusepath{stroke}%
\end{pgfscope}%
\begin{pgfscope}%
\pgfsetrectcap%
\pgfsetmiterjoin%
\pgfsetlinewidth{0.803000pt}%
\definecolor{currentstroke}{rgb}{0.737255,0.737255,0.737255}%
\pgfsetstrokecolor{currentstroke}%
\pgfsetdash{}{0pt}%
\pgfpathmoveto{\pgfqpoint{0.477328in}{2.036972in}}%
\pgfpathlineto{\pgfqpoint{5.958075in}{2.036972in}}%
\pgfusepath{stroke}%
\end{pgfscope}%
\begin{pgfscope}%
\definecolor{textcolor}{rgb}{0.180392,0.180392,0.180392}%
\pgfsetstrokecolor{textcolor}%
\pgfsetfillcolor{textcolor}%
\pgftext[x=3.217701in,y=2.120305in,,base]{\color{textcolor}\rmfamily\fontsize{12.960000}{15.552000}\selectfont Pulsed Gaussian Spectrum}%
\end{pgfscope}%
\end{pgfpicture}%
\makeatother%
\endgroup%
}
		\caption{\centering{The Gaussian Pulse spectrum}}
		\label{fig:gaussian_spectrum}
	\end{figure}
	
	\vspace{5mm}
	
	Note that the two Gaussians are centred at the carrier frequency and that the waist frequency interval at 1/e intensity ($\Delta\omega$) must be significantly smaller than the carrier frequency of the signal ($\omega_0$), i.e. ${\Delta\omega}\ll{\omega_0}$. This is shown in the normalised frequency spectrum graph, Figure \ref{fig:gaussian_spectrum_normalised}: \linebreak
	
	\begin{figure}[h]
		\centering
		\scalebox{0.85}{%% Creator: Matplotlib, PGF backend
%%
%% To include the figure in your LaTeX document, write
%%   \input{<filename>.pgf}
%%
%% Make sure the required packages are loaded in your preamble
%%   \usepackage{pgf}
%%
%% Also ensure that all the required font packages are loaded; for instance,
%% the lmodern package is sometimes necessary when using math font.
%%   \usepackage{lmodern}
%%
%% Figures using additional raster images can only be included by \input if
%% they are in the same directory as the main LaTeX file. For loading figures
%% from other directories you can use the `import` package
%%   \usepackage{import}
%%
%% and then include the figures with
%%   \import{<path to file>}{<filename>.pgf}
%%
%% Matplotlib used the following preamble
%%   \usepackage[T1]{fontenc} \usepackage{mathpazo}
%%
\begingroup%
\makeatletter%
\begin{pgfpicture}%
\pgfpathrectangle{\pgfpointorigin}{\pgfqpoint{6.086066in}{4.655390in}}%
\pgfusepath{use as bounding box, clip}%
\begin{pgfscope}%
\pgfsetbuttcap%
\pgfsetmiterjoin%
\definecolor{currentfill}{rgb}{1.000000,1.000000,1.000000}%
\pgfsetfillcolor{currentfill}%
\pgfsetlinewidth{0.000000pt}%
\definecolor{currentstroke}{rgb}{1.000000,1.000000,1.000000}%
\pgfsetstrokecolor{currentstroke}%
\pgfsetdash{}{0pt}%
\pgfpathmoveto{\pgfqpoint{0.000000in}{0.000000in}}%
\pgfpathlineto{\pgfqpoint{6.086066in}{0.000000in}}%
\pgfpathlineto{\pgfqpoint{6.086066in}{4.655390in}}%
\pgfpathlineto{\pgfqpoint{0.000000in}{4.655390in}}%
\pgfpathlineto{\pgfqpoint{0.000000in}{0.000000in}}%
\pgfpathclose%
\pgfusepath{fill}%
\end{pgfscope}%
\begin{pgfscope}%
\pgfsetbuttcap%
\pgfsetmiterjoin%
\definecolor{currentfill}{rgb}{0.933333,0.933333,0.933333}%
\pgfsetfillcolor{currentfill}%
\pgfsetlinewidth{0.000000pt}%
\definecolor{currentstroke}{rgb}{0.000000,0.000000,0.000000}%
\pgfsetstrokecolor{currentstroke}%
\pgfsetstrokeopacity{0.000000}%
\pgfsetdash{}{0pt}%
\pgfpathmoveto{\pgfqpoint{0.477328in}{0.583522in}}%
\pgfpathlineto{\pgfqpoint{6.086066in}{0.583522in}}%
\pgfpathlineto{\pgfqpoint{6.086066in}{4.441828in}}%
\pgfpathlineto{\pgfqpoint{0.477328in}{4.441828in}}%
\pgfpathlineto{\pgfqpoint{0.477328in}{0.583522in}}%
\pgfpathclose%
\pgfusepath{fill}%
\end{pgfscope}%
\begin{pgfscope}%
\pgfpathrectangle{\pgfqpoint{0.477328in}{0.583522in}}{\pgfqpoint{5.608738in}{3.858305in}}%
\pgfusepath{clip}%
\pgfsetbuttcap%
\pgfsetroundjoin%
\pgfsetlinewidth{0.501875pt}%
\definecolor{currentstroke}{rgb}{0.698039,0.698039,0.698039}%
\pgfsetstrokecolor{currentstroke}%
\pgfsetdash{{1.850000pt}{0.800000pt}}{0.000000pt}%
\pgfpathmoveto{\pgfqpoint{1.124490in}{0.583522in}}%
\pgfpathlineto{\pgfqpoint{1.124490in}{4.441828in}}%
\pgfusepath{stroke}%
\end{pgfscope}%
\begin{pgfscope}%
\pgfsetbuttcap%
\pgfsetroundjoin%
\definecolor{currentfill}{rgb}{0.180392,0.180392,0.180392}%
\pgfsetfillcolor{currentfill}%
\pgfsetlinewidth{0.803000pt}%
\definecolor{currentstroke}{rgb}{0.180392,0.180392,0.180392}%
\pgfsetstrokecolor{currentstroke}%
\pgfsetdash{}{0pt}%
\pgfsys@defobject{currentmarker}{\pgfqpoint{0.000000in}{-0.048611in}}{\pgfqpoint{0.000000in}{0.000000in}}{%
\pgfpathmoveto{\pgfqpoint{0.000000in}{0.000000in}}%
\pgfpathlineto{\pgfqpoint{0.000000in}{-0.048611in}}%
\pgfusepath{stroke,fill}%
}%
\begin{pgfscope}%
\pgfsys@transformshift{1.124490in}{0.583522in}%
\pgfsys@useobject{currentmarker}{}%
\end{pgfscope}%
\end{pgfscope}%
\begin{pgfscope}%
\definecolor{textcolor}{rgb}{0.180392,0.180392,0.180392}%
\pgfsetstrokecolor{textcolor}%
\pgfsetfillcolor{textcolor}%
\pgftext[x=1.124490in,y=0.486300in,,top]{\color{textcolor}\rmfamily\fontsize{9.000000}{10.800000}\selectfont \(\displaystyle {\ensuremath{-}1.0}\)}%
\end{pgfscope}%
\begin{pgfscope}%
\pgfpathrectangle{\pgfqpoint{0.477328in}{0.583522in}}{\pgfqpoint{5.608738in}{3.858305in}}%
\pgfusepath{clip}%
\pgfsetbuttcap%
\pgfsetroundjoin%
\pgfsetlinewidth{0.501875pt}%
\definecolor{currentstroke}{rgb}{0.698039,0.698039,0.698039}%
\pgfsetstrokecolor{currentstroke}%
\pgfsetdash{{1.850000pt}{0.800000pt}}{0.000000pt}%
\pgfpathmoveto{\pgfqpoint{2.203093in}{0.583522in}}%
\pgfpathlineto{\pgfqpoint{2.203093in}{4.441828in}}%
\pgfusepath{stroke}%
\end{pgfscope}%
\begin{pgfscope}%
\pgfsetbuttcap%
\pgfsetroundjoin%
\definecolor{currentfill}{rgb}{0.180392,0.180392,0.180392}%
\pgfsetfillcolor{currentfill}%
\pgfsetlinewidth{0.803000pt}%
\definecolor{currentstroke}{rgb}{0.180392,0.180392,0.180392}%
\pgfsetstrokecolor{currentstroke}%
\pgfsetdash{}{0pt}%
\pgfsys@defobject{currentmarker}{\pgfqpoint{0.000000in}{-0.048611in}}{\pgfqpoint{0.000000in}{0.000000in}}{%
\pgfpathmoveto{\pgfqpoint{0.000000in}{0.000000in}}%
\pgfpathlineto{\pgfqpoint{0.000000in}{-0.048611in}}%
\pgfusepath{stroke,fill}%
}%
\begin{pgfscope}%
\pgfsys@transformshift{2.203093in}{0.583522in}%
\pgfsys@useobject{currentmarker}{}%
\end{pgfscope}%
\end{pgfscope}%
\begin{pgfscope}%
\definecolor{textcolor}{rgb}{0.180392,0.180392,0.180392}%
\pgfsetstrokecolor{textcolor}%
\pgfsetfillcolor{textcolor}%
\pgftext[x=2.203093in,y=0.486300in,,top]{\color{textcolor}\rmfamily\fontsize{9.000000}{10.800000}\selectfont \(\displaystyle {\ensuremath{-}0.5}\)}%
\end{pgfscope}%
\begin{pgfscope}%
\pgfpathrectangle{\pgfqpoint{0.477328in}{0.583522in}}{\pgfqpoint{5.608738in}{3.858305in}}%
\pgfusepath{clip}%
\pgfsetbuttcap%
\pgfsetroundjoin%
\pgfsetlinewidth{0.501875pt}%
\definecolor{currentstroke}{rgb}{0.698039,0.698039,0.698039}%
\pgfsetstrokecolor{currentstroke}%
\pgfsetdash{{1.850000pt}{0.800000pt}}{0.000000pt}%
\pgfpathmoveto{\pgfqpoint{3.281697in}{0.583522in}}%
\pgfpathlineto{\pgfqpoint{3.281697in}{4.441828in}}%
\pgfusepath{stroke}%
\end{pgfscope}%
\begin{pgfscope}%
\pgfsetbuttcap%
\pgfsetroundjoin%
\definecolor{currentfill}{rgb}{0.180392,0.180392,0.180392}%
\pgfsetfillcolor{currentfill}%
\pgfsetlinewidth{0.803000pt}%
\definecolor{currentstroke}{rgb}{0.180392,0.180392,0.180392}%
\pgfsetstrokecolor{currentstroke}%
\pgfsetdash{}{0pt}%
\pgfsys@defobject{currentmarker}{\pgfqpoint{0.000000in}{-0.048611in}}{\pgfqpoint{0.000000in}{0.000000in}}{%
\pgfpathmoveto{\pgfqpoint{0.000000in}{0.000000in}}%
\pgfpathlineto{\pgfqpoint{0.000000in}{-0.048611in}}%
\pgfusepath{stroke,fill}%
}%
\begin{pgfscope}%
\pgfsys@transformshift{3.281697in}{0.583522in}%
\pgfsys@useobject{currentmarker}{}%
\end{pgfscope}%
\end{pgfscope}%
\begin{pgfscope}%
\definecolor{textcolor}{rgb}{0.180392,0.180392,0.180392}%
\pgfsetstrokecolor{textcolor}%
\pgfsetfillcolor{textcolor}%
\pgftext[x=3.281697in,y=0.486300in,,top]{\color{textcolor}\rmfamily\fontsize{9.000000}{10.800000}\selectfont \(\displaystyle {0.0}\)}%
\end{pgfscope}%
\begin{pgfscope}%
\pgfpathrectangle{\pgfqpoint{0.477328in}{0.583522in}}{\pgfqpoint{5.608738in}{3.858305in}}%
\pgfusepath{clip}%
\pgfsetbuttcap%
\pgfsetroundjoin%
\pgfsetlinewidth{0.501875pt}%
\definecolor{currentstroke}{rgb}{0.698039,0.698039,0.698039}%
\pgfsetstrokecolor{currentstroke}%
\pgfsetdash{{1.850000pt}{0.800000pt}}{0.000000pt}%
\pgfpathmoveto{\pgfqpoint{4.360300in}{0.583522in}}%
\pgfpathlineto{\pgfqpoint{4.360300in}{4.441828in}}%
\pgfusepath{stroke}%
\end{pgfscope}%
\begin{pgfscope}%
\pgfsetbuttcap%
\pgfsetroundjoin%
\definecolor{currentfill}{rgb}{0.180392,0.180392,0.180392}%
\pgfsetfillcolor{currentfill}%
\pgfsetlinewidth{0.803000pt}%
\definecolor{currentstroke}{rgb}{0.180392,0.180392,0.180392}%
\pgfsetstrokecolor{currentstroke}%
\pgfsetdash{}{0pt}%
\pgfsys@defobject{currentmarker}{\pgfqpoint{0.000000in}{-0.048611in}}{\pgfqpoint{0.000000in}{0.000000in}}{%
\pgfpathmoveto{\pgfqpoint{0.000000in}{0.000000in}}%
\pgfpathlineto{\pgfqpoint{0.000000in}{-0.048611in}}%
\pgfusepath{stroke,fill}%
}%
\begin{pgfscope}%
\pgfsys@transformshift{4.360300in}{0.583522in}%
\pgfsys@useobject{currentmarker}{}%
\end{pgfscope}%
\end{pgfscope}%
\begin{pgfscope}%
\definecolor{textcolor}{rgb}{0.180392,0.180392,0.180392}%
\pgfsetstrokecolor{textcolor}%
\pgfsetfillcolor{textcolor}%
\pgftext[x=4.360300in,y=0.486300in,,top]{\color{textcolor}\rmfamily\fontsize{9.000000}{10.800000}\selectfont \(\displaystyle {0.5}\)}%
\end{pgfscope}%
\begin{pgfscope}%
\pgfpathrectangle{\pgfqpoint{0.477328in}{0.583522in}}{\pgfqpoint{5.608738in}{3.858305in}}%
\pgfusepath{clip}%
\pgfsetbuttcap%
\pgfsetroundjoin%
\pgfsetlinewidth{0.501875pt}%
\definecolor{currentstroke}{rgb}{0.698039,0.698039,0.698039}%
\pgfsetstrokecolor{currentstroke}%
\pgfsetdash{{1.850000pt}{0.800000pt}}{0.000000pt}%
\pgfpathmoveto{\pgfqpoint{5.438904in}{0.583522in}}%
\pgfpathlineto{\pgfqpoint{5.438904in}{4.441828in}}%
\pgfusepath{stroke}%
\end{pgfscope}%
\begin{pgfscope}%
\pgfsetbuttcap%
\pgfsetroundjoin%
\definecolor{currentfill}{rgb}{0.180392,0.180392,0.180392}%
\pgfsetfillcolor{currentfill}%
\pgfsetlinewidth{0.803000pt}%
\definecolor{currentstroke}{rgb}{0.180392,0.180392,0.180392}%
\pgfsetstrokecolor{currentstroke}%
\pgfsetdash{}{0pt}%
\pgfsys@defobject{currentmarker}{\pgfqpoint{0.000000in}{-0.048611in}}{\pgfqpoint{0.000000in}{0.000000in}}{%
\pgfpathmoveto{\pgfqpoint{0.000000in}{0.000000in}}%
\pgfpathlineto{\pgfqpoint{0.000000in}{-0.048611in}}%
\pgfusepath{stroke,fill}%
}%
\begin{pgfscope}%
\pgfsys@transformshift{5.438904in}{0.583522in}%
\pgfsys@useobject{currentmarker}{}%
\end{pgfscope}%
\end{pgfscope}%
\begin{pgfscope}%
\definecolor{textcolor}{rgb}{0.180392,0.180392,0.180392}%
\pgfsetstrokecolor{textcolor}%
\pgfsetfillcolor{textcolor}%
\pgftext[x=5.438904in,y=0.486300in,,top]{\color{textcolor}\rmfamily\fontsize{9.000000}{10.800000}\selectfont \(\displaystyle {1.0}\)}%
\end{pgfscope}%
\begin{pgfscope}%
\definecolor{textcolor}{rgb}{0.180392,0.180392,0.180392}%
\pgfsetstrokecolor{textcolor}%
\pgfsetfillcolor{textcolor}%
\pgftext[x=3.281697in,y=0.305059in,,top]{\color{textcolor}\rmfamily\fontsize{10.800000}{12.960000}\selectfont Normalised Angular Frequency, \(\displaystyle \frac{\omega}{\omega_0}\)}%
\end{pgfscope}%
\begin{pgfscope}%
\pgfpathrectangle{\pgfqpoint{0.477328in}{0.583522in}}{\pgfqpoint{5.608738in}{3.858305in}}%
\pgfusepath{clip}%
\pgfsetbuttcap%
\pgfsetroundjoin%
\pgfsetlinewidth{0.501875pt}%
\definecolor{currentstroke}{rgb}{0.698039,0.698039,0.698039}%
\pgfsetstrokecolor{currentstroke}%
\pgfsetdash{{1.850000pt}{0.800000pt}}{0.000000pt}%
\pgfpathmoveto{\pgfqpoint{0.477328in}{0.758900in}}%
\pgfpathlineto{\pgfqpoint{6.086066in}{0.758900in}}%
\pgfusepath{stroke}%
\end{pgfscope}%
\begin{pgfscope}%
\pgfsetbuttcap%
\pgfsetroundjoin%
\definecolor{currentfill}{rgb}{0.180392,0.180392,0.180392}%
\pgfsetfillcolor{currentfill}%
\pgfsetlinewidth{0.803000pt}%
\definecolor{currentstroke}{rgb}{0.180392,0.180392,0.180392}%
\pgfsetstrokecolor{currentstroke}%
\pgfsetdash{}{0pt}%
\pgfsys@defobject{currentmarker}{\pgfqpoint{-0.048611in}{0.000000in}}{\pgfqpoint{-0.000000in}{0.000000in}}{%
\pgfpathmoveto{\pgfqpoint{-0.000000in}{0.000000in}}%
\pgfpathlineto{\pgfqpoint{-0.048611in}{0.000000in}}%
\pgfusepath{stroke,fill}%
}%
\begin{pgfscope}%
\pgfsys@transformshift{0.477328in}{0.758900in}%
\pgfsys@useobject{currentmarker}{}%
\end{pgfscope}%
\end{pgfscope}%
\begin{pgfscope}%
\definecolor{textcolor}{rgb}{0.180392,0.180392,0.180392}%
\pgfsetstrokecolor{textcolor}%
\pgfsetfillcolor{textcolor}%
\pgftext[x=0.223855in, y=0.713681in, left, base]{\color{textcolor}\rmfamily\fontsize{9.000000}{10.800000}\selectfont \(\displaystyle {0.0}\)}%
\end{pgfscope}%
\begin{pgfscope}%
\pgfpathrectangle{\pgfqpoint{0.477328in}{0.583522in}}{\pgfqpoint{5.608738in}{3.858305in}}%
\pgfusepath{clip}%
\pgfsetbuttcap%
\pgfsetroundjoin%
\pgfsetlinewidth{0.501875pt}%
\definecolor{currentstroke}{rgb}{0.698039,0.698039,0.698039}%
\pgfsetstrokecolor{currentstroke}%
\pgfsetdash{{1.850000pt}{0.800000pt}}{0.000000pt}%
\pgfpathmoveto{\pgfqpoint{0.477328in}{1.460410in}}%
\pgfpathlineto{\pgfqpoint{6.086066in}{1.460410in}}%
\pgfusepath{stroke}%
\end{pgfscope}%
\begin{pgfscope}%
\pgfsetbuttcap%
\pgfsetroundjoin%
\definecolor{currentfill}{rgb}{0.180392,0.180392,0.180392}%
\pgfsetfillcolor{currentfill}%
\pgfsetlinewidth{0.803000pt}%
\definecolor{currentstroke}{rgb}{0.180392,0.180392,0.180392}%
\pgfsetstrokecolor{currentstroke}%
\pgfsetdash{}{0pt}%
\pgfsys@defobject{currentmarker}{\pgfqpoint{-0.048611in}{0.000000in}}{\pgfqpoint{-0.000000in}{0.000000in}}{%
\pgfpathmoveto{\pgfqpoint{-0.000000in}{0.000000in}}%
\pgfpathlineto{\pgfqpoint{-0.048611in}{0.000000in}}%
\pgfusepath{stroke,fill}%
}%
\begin{pgfscope}%
\pgfsys@transformshift{0.477328in}{1.460410in}%
\pgfsys@useobject{currentmarker}{}%
\end{pgfscope}%
\end{pgfscope}%
\begin{pgfscope}%
\definecolor{textcolor}{rgb}{0.180392,0.180392,0.180392}%
\pgfsetstrokecolor{textcolor}%
\pgfsetfillcolor{textcolor}%
\pgftext[x=0.223855in, y=1.415192in, left, base]{\color{textcolor}\rmfamily\fontsize{9.000000}{10.800000}\selectfont \(\displaystyle {0.2}\)}%
\end{pgfscope}%
\begin{pgfscope}%
\pgfpathrectangle{\pgfqpoint{0.477328in}{0.583522in}}{\pgfqpoint{5.608738in}{3.858305in}}%
\pgfusepath{clip}%
\pgfsetbuttcap%
\pgfsetroundjoin%
\pgfsetlinewidth{0.501875pt}%
\definecolor{currentstroke}{rgb}{0.698039,0.698039,0.698039}%
\pgfsetstrokecolor{currentstroke}%
\pgfsetdash{{1.850000pt}{0.800000pt}}{0.000000pt}%
\pgfpathmoveto{\pgfqpoint{0.477328in}{2.161920in}}%
\pgfpathlineto{\pgfqpoint{6.086066in}{2.161920in}}%
\pgfusepath{stroke}%
\end{pgfscope}%
\begin{pgfscope}%
\pgfsetbuttcap%
\pgfsetroundjoin%
\definecolor{currentfill}{rgb}{0.180392,0.180392,0.180392}%
\pgfsetfillcolor{currentfill}%
\pgfsetlinewidth{0.803000pt}%
\definecolor{currentstroke}{rgb}{0.180392,0.180392,0.180392}%
\pgfsetstrokecolor{currentstroke}%
\pgfsetdash{}{0pt}%
\pgfsys@defobject{currentmarker}{\pgfqpoint{-0.048611in}{0.000000in}}{\pgfqpoint{-0.000000in}{0.000000in}}{%
\pgfpathmoveto{\pgfqpoint{-0.000000in}{0.000000in}}%
\pgfpathlineto{\pgfqpoint{-0.048611in}{0.000000in}}%
\pgfusepath{stroke,fill}%
}%
\begin{pgfscope}%
\pgfsys@transformshift{0.477328in}{2.161920in}%
\pgfsys@useobject{currentmarker}{}%
\end{pgfscope}%
\end{pgfscope}%
\begin{pgfscope}%
\definecolor{textcolor}{rgb}{0.180392,0.180392,0.180392}%
\pgfsetstrokecolor{textcolor}%
\pgfsetfillcolor{textcolor}%
\pgftext[x=0.223855in, y=2.116702in, left, base]{\color{textcolor}\rmfamily\fontsize{9.000000}{10.800000}\selectfont \(\displaystyle {0.4}\)}%
\end{pgfscope}%
\begin{pgfscope}%
\pgfpathrectangle{\pgfqpoint{0.477328in}{0.583522in}}{\pgfqpoint{5.608738in}{3.858305in}}%
\pgfusepath{clip}%
\pgfsetbuttcap%
\pgfsetroundjoin%
\pgfsetlinewidth{0.501875pt}%
\definecolor{currentstroke}{rgb}{0.698039,0.698039,0.698039}%
\pgfsetstrokecolor{currentstroke}%
\pgfsetdash{{1.850000pt}{0.800000pt}}{0.000000pt}%
\pgfpathmoveto{\pgfqpoint{0.477328in}{2.863430in}}%
\pgfpathlineto{\pgfqpoint{6.086066in}{2.863430in}}%
\pgfusepath{stroke}%
\end{pgfscope}%
\begin{pgfscope}%
\pgfsetbuttcap%
\pgfsetroundjoin%
\definecolor{currentfill}{rgb}{0.180392,0.180392,0.180392}%
\pgfsetfillcolor{currentfill}%
\pgfsetlinewidth{0.803000pt}%
\definecolor{currentstroke}{rgb}{0.180392,0.180392,0.180392}%
\pgfsetstrokecolor{currentstroke}%
\pgfsetdash{}{0pt}%
\pgfsys@defobject{currentmarker}{\pgfqpoint{-0.048611in}{0.000000in}}{\pgfqpoint{-0.000000in}{0.000000in}}{%
\pgfpathmoveto{\pgfqpoint{-0.000000in}{0.000000in}}%
\pgfpathlineto{\pgfqpoint{-0.048611in}{0.000000in}}%
\pgfusepath{stroke,fill}%
}%
\begin{pgfscope}%
\pgfsys@transformshift{0.477328in}{2.863430in}%
\pgfsys@useobject{currentmarker}{}%
\end{pgfscope}%
\end{pgfscope}%
\begin{pgfscope}%
\definecolor{textcolor}{rgb}{0.180392,0.180392,0.180392}%
\pgfsetstrokecolor{textcolor}%
\pgfsetfillcolor{textcolor}%
\pgftext[x=0.223855in, y=2.818212in, left, base]{\color{textcolor}\rmfamily\fontsize{9.000000}{10.800000}\selectfont \(\displaystyle {0.6}\)}%
\end{pgfscope}%
\begin{pgfscope}%
\pgfpathrectangle{\pgfqpoint{0.477328in}{0.583522in}}{\pgfqpoint{5.608738in}{3.858305in}}%
\pgfusepath{clip}%
\pgfsetbuttcap%
\pgfsetroundjoin%
\pgfsetlinewidth{0.501875pt}%
\definecolor{currentstroke}{rgb}{0.698039,0.698039,0.698039}%
\pgfsetstrokecolor{currentstroke}%
\pgfsetdash{{1.850000pt}{0.800000pt}}{0.000000pt}%
\pgfpathmoveto{\pgfqpoint{0.477328in}{3.564941in}}%
\pgfpathlineto{\pgfqpoint{6.086066in}{3.564941in}}%
\pgfusepath{stroke}%
\end{pgfscope}%
\begin{pgfscope}%
\pgfsetbuttcap%
\pgfsetroundjoin%
\definecolor{currentfill}{rgb}{0.180392,0.180392,0.180392}%
\pgfsetfillcolor{currentfill}%
\pgfsetlinewidth{0.803000pt}%
\definecolor{currentstroke}{rgb}{0.180392,0.180392,0.180392}%
\pgfsetstrokecolor{currentstroke}%
\pgfsetdash{}{0pt}%
\pgfsys@defobject{currentmarker}{\pgfqpoint{-0.048611in}{0.000000in}}{\pgfqpoint{-0.000000in}{0.000000in}}{%
\pgfpathmoveto{\pgfqpoint{-0.000000in}{0.000000in}}%
\pgfpathlineto{\pgfqpoint{-0.048611in}{0.000000in}}%
\pgfusepath{stroke,fill}%
}%
\begin{pgfscope}%
\pgfsys@transformshift{0.477328in}{3.564941in}%
\pgfsys@useobject{currentmarker}{}%
\end{pgfscope}%
\end{pgfscope}%
\begin{pgfscope}%
\definecolor{textcolor}{rgb}{0.180392,0.180392,0.180392}%
\pgfsetstrokecolor{textcolor}%
\pgfsetfillcolor{textcolor}%
\pgftext[x=0.223855in, y=3.519722in, left, base]{\color{textcolor}\rmfamily\fontsize{9.000000}{10.800000}\selectfont \(\displaystyle {0.8}\)}%
\end{pgfscope}%
\begin{pgfscope}%
\pgfpathrectangle{\pgfqpoint{0.477328in}{0.583522in}}{\pgfqpoint{5.608738in}{3.858305in}}%
\pgfusepath{clip}%
\pgfsetbuttcap%
\pgfsetroundjoin%
\pgfsetlinewidth{0.501875pt}%
\definecolor{currentstroke}{rgb}{0.698039,0.698039,0.698039}%
\pgfsetstrokecolor{currentstroke}%
\pgfsetdash{{1.850000pt}{0.800000pt}}{0.000000pt}%
\pgfpathmoveto{\pgfqpoint{0.477328in}{4.266451in}}%
\pgfpathlineto{\pgfqpoint{6.086066in}{4.266451in}}%
\pgfusepath{stroke}%
\end{pgfscope}%
\begin{pgfscope}%
\pgfsetbuttcap%
\pgfsetroundjoin%
\definecolor{currentfill}{rgb}{0.180392,0.180392,0.180392}%
\pgfsetfillcolor{currentfill}%
\pgfsetlinewidth{0.803000pt}%
\definecolor{currentstroke}{rgb}{0.180392,0.180392,0.180392}%
\pgfsetstrokecolor{currentstroke}%
\pgfsetdash{}{0pt}%
\pgfsys@defobject{currentmarker}{\pgfqpoint{-0.048611in}{0.000000in}}{\pgfqpoint{-0.000000in}{0.000000in}}{%
\pgfpathmoveto{\pgfqpoint{-0.000000in}{0.000000in}}%
\pgfpathlineto{\pgfqpoint{-0.048611in}{0.000000in}}%
\pgfusepath{stroke,fill}%
}%
\begin{pgfscope}%
\pgfsys@transformshift{0.477328in}{4.266451in}%
\pgfsys@useobject{currentmarker}{}%
\end{pgfscope}%
\end{pgfscope}%
\begin{pgfscope}%
\definecolor{textcolor}{rgb}{0.180392,0.180392,0.180392}%
\pgfsetstrokecolor{textcolor}%
\pgfsetfillcolor{textcolor}%
\pgftext[x=0.223855in, y=4.221232in, left, base]{\color{textcolor}\rmfamily\fontsize{9.000000}{10.800000}\selectfont \(\displaystyle {1.0}\)}%
\end{pgfscope}%
\begin{pgfscope}%
\definecolor{textcolor}{rgb}{0.180392,0.180392,0.180392}%
\pgfsetstrokecolor{textcolor}%
\pgfsetfillcolor{textcolor}%
\pgftext[x=0.168300in,y=2.512675in,,bottom,rotate=90.000000]{\color{textcolor}\rmfamily\fontsize{10.800000}{12.960000}\selectfont \(\displaystyle \tilde{E}(\omega)\)}%
\end{pgfscope}%
\begin{pgfscope}%
\pgfpathrectangle{\pgfqpoint{0.477328in}{0.583522in}}{\pgfqpoint{5.608738in}{3.858305in}}%
\pgfusepath{clip}%
\pgfsetrectcap%
\pgfsetroundjoin%
\pgfsetlinewidth{2.007500pt}%
\definecolor{currentstroke}{rgb}{0.000000,0.501961,0.000000}%
\pgfsetstrokecolor{currentstroke}%
\pgfsetdash{}{0pt}%
\pgfpathmoveto{\pgfqpoint{0.467328in}{0.758900in}}%
\pgfpathlineto{\pgfqpoint{0.779337in}{0.759980in}}%
\pgfpathlineto{\pgfqpoint{0.800909in}{0.761776in}}%
\pgfpathlineto{\pgfqpoint{0.822481in}{0.766088in}}%
\pgfpathlineto{\pgfqpoint{0.833267in}{0.769997in}}%
\pgfpathlineto{\pgfqpoint{0.844053in}{0.775764in}}%
\pgfpathlineto{\pgfqpoint{0.854839in}{0.784126in}}%
\pgfpathlineto{\pgfqpoint{0.865625in}{0.796043in}}%
\pgfpathlineto{\pgfqpoint{0.876411in}{0.812732in}}%
\pgfpathlineto{\pgfqpoint{0.887197in}{0.835698in}}%
\pgfpathlineto{\pgfqpoint{0.897983in}{0.866744in}}%
\pgfpathlineto{\pgfqpoint{0.908769in}{0.907969in}}%
\pgfpathlineto{\pgfqpoint{0.919555in}{0.961725in}}%
\pgfpathlineto{\pgfqpoint{0.930341in}{1.030541in}}%
\pgfpathlineto{\pgfqpoint{0.941127in}{1.117007in}}%
\pgfpathlineto{\pgfqpoint{0.951913in}{1.223598in}}%
\pgfpathlineto{\pgfqpoint{0.962699in}{1.352469in}}%
\pgfpathlineto{\pgfqpoint{0.973485in}{1.505200in}}%
\pgfpathlineto{\pgfqpoint{0.984271in}{1.682530in}}%
\pgfpathlineto{\pgfqpoint{0.995057in}{1.884085in}}%
\pgfpathlineto{\pgfqpoint{1.016629in}{2.351481in}}%
\pgfpathlineto{\pgfqpoint{1.070560in}{3.638147in}}%
\pgfpathlineto{\pgfqpoint{1.081346in}{3.850193in}}%
\pgfpathlineto{\pgfqpoint{1.092132in}{4.025856in}}%
\pgfpathlineto{\pgfqpoint{1.102918in}{4.157407in}}%
\pgfpathlineto{\pgfqpoint{1.113704in}{4.238867in}}%
\pgfpathlineto{\pgfqpoint{1.124490in}{4.266450in}}%
\pgfpathlineto{\pgfqpoint{1.135276in}{4.238862in}}%
\pgfpathlineto{\pgfqpoint{1.146062in}{4.157399in}}%
\pgfpathlineto{\pgfqpoint{1.156848in}{4.025843in}}%
\pgfpathlineto{\pgfqpoint{1.167634in}{3.850177in}}%
\pgfpathlineto{\pgfqpoint{1.178420in}{3.638129in}}%
\pgfpathlineto{\pgfqpoint{1.199992in}{3.141101in}}%
\pgfpathlineto{\pgfqpoint{1.232350in}{2.351461in}}%
\pgfpathlineto{\pgfqpoint{1.253922in}{1.884068in}}%
\pgfpathlineto{\pgfqpoint{1.264708in}{1.682514in}}%
\pgfpathlineto{\pgfqpoint{1.275494in}{1.505187in}}%
\pgfpathlineto{\pgfqpoint{1.286280in}{1.352458in}}%
\pgfpathlineto{\pgfqpoint{1.297066in}{1.223589in}}%
\pgfpathlineto{\pgfqpoint{1.307852in}{1.116999in}}%
\pgfpathlineto{\pgfqpoint{1.318638in}{1.030535in}}%
\pgfpathlineto{\pgfqpoint{1.329424in}{0.961720in}}%
\pgfpathlineto{\pgfqpoint{1.340210in}{0.907966in}}%
\pgfpathlineto{\pgfqpoint{1.350996in}{0.866741in}}%
\pgfpathlineto{\pgfqpoint{1.361783in}{0.835696in}}%
\pgfpathlineto{\pgfqpoint{1.372569in}{0.812730in}}%
\pgfpathlineto{\pgfqpoint{1.383355in}{0.796042in}}%
\pgfpathlineto{\pgfqpoint{1.394141in}{0.784125in}}%
\pgfpathlineto{\pgfqpoint{1.404927in}{0.775764in}}%
\pgfpathlineto{\pgfqpoint{1.415713in}{0.769997in}}%
\pgfpathlineto{\pgfqpoint{1.426499in}{0.766088in}}%
\pgfpathlineto{\pgfqpoint{1.448071in}{0.761776in}}%
\pgfpathlineto{\pgfqpoint{1.480429in}{0.759547in}}%
\pgfpathlineto{\pgfqpoint{1.545145in}{0.758921in}}%
\pgfpathlineto{\pgfqpoint{2.774753in}{0.758900in}}%
\pgfpathlineto{\pgfqpoint{5.093751in}{0.759980in}}%
\pgfpathlineto{\pgfqpoint{5.115323in}{0.761776in}}%
\pgfpathlineto{\pgfqpoint{5.136895in}{0.766088in}}%
\pgfpathlineto{\pgfqpoint{5.147681in}{0.769997in}}%
\pgfpathlineto{\pgfqpoint{5.158467in}{0.775764in}}%
\pgfpathlineto{\pgfqpoint{5.169253in}{0.784125in}}%
\pgfpathlineto{\pgfqpoint{5.180039in}{0.796042in}}%
\pgfpathlineto{\pgfqpoint{5.190825in}{0.812730in}}%
\pgfpathlineto{\pgfqpoint{5.201611in}{0.835696in}}%
\pgfpathlineto{\pgfqpoint{5.212397in}{0.866741in}}%
\pgfpathlineto{\pgfqpoint{5.223183in}{0.907966in}}%
\pgfpathlineto{\pgfqpoint{5.233969in}{0.961720in}}%
\pgfpathlineto{\pgfqpoint{5.244755in}{1.030535in}}%
\pgfpathlineto{\pgfqpoint{5.255541in}{1.116999in}}%
\pgfpathlineto{\pgfqpoint{5.266327in}{1.223589in}}%
\pgfpathlineto{\pgfqpoint{5.277113in}{1.352458in}}%
\pgfpathlineto{\pgfqpoint{5.287899in}{1.505187in}}%
\pgfpathlineto{\pgfqpoint{5.298685in}{1.682514in}}%
\pgfpathlineto{\pgfqpoint{5.309471in}{1.884068in}}%
\pgfpathlineto{\pgfqpoint{5.331043in}{2.351461in}}%
\pgfpathlineto{\pgfqpoint{5.384974in}{3.638129in}}%
\pgfpathlineto{\pgfqpoint{5.395760in}{3.850177in}}%
\pgfpathlineto{\pgfqpoint{5.406546in}{4.025843in}}%
\pgfpathlineto{\pgfqpoint{5.417332in}{4.157399in}}%
\pgfpathlineto{\pgfqpoint{5.428118in}{4.238862in}}%
\pgfpathlineto{\pgfqpoint{5.438904in}{4.266450in}}%
\pgfpathlineto{\pgfqpoint{5.449690in}{4.238867in}}%
\pgfpathlineto{\pgfqpoint{5.460476in}{4.157407in}}%
\pgfpathlineto{\pgfqpoint{5.471262in}{4.025856in}}%
\pgfpathlineto{\pgfqpoint{5.482048in}{3.850193in}}%
\pgfpathlineto{\pgfqpoint{5.492834in}{3.638147in}}%
\pgfpathlineto{\pgfqpoint{5.514406in}{3.141122in}}%
\pgfpathlineto{\pgfqpoint{5.546764in}{2.351481in}}%
\pgfpathlineto{\pgfqpoint{5.568336in}{1.884085in}}%
\pgfpathlineto{\pgfqpoint{5.579122in}{1.682530in}}%
\pgfpathlineto{\pgfqpoint{5.589908in}{1.505200in}}%
\pgfpathlineto{\pgfqpoint{5.600694in}{1.352469in}}%
\pgfpathlineto{\pgfqpoint{5.611480in}{1.223598in}}%
\pgfpathlineto{\pgfqpoint{5.622266in}{1.117007in}}%
\pgfpathlineto{\pgfqpoint{5.633052in}{1.030541in}}%
\pgfpathlineto{\pgfqpoint{5.643838in}{0.961725in}}%
\pgfpathlineto{\pgfqpoint{5.654625in}{0.907969in}}%
\pgfpathlineto{\pgfqpoint{5.665411in}{0.866744in}}%
\pgfpathlineto{\pgfqpoint{5.676197in}{0.835698in}}%
\pgfpathlineto{\pgfqpoint{5.686983in}{0.812732in}}%
\pgfpathlineto{\pgfqpoint{5.697769in}{0.796043in}}%
\pgfpathlineto{\pgfqpoint{5.708555in}{0.784126in}}%
\pgfpathlineto{\pgfqpoint{5.719341in}{0.775764in}}%
\pgfpathlineto{\pgfqpoint{5.730127in}{0.769997in}}%
\pgfpathlineto{\pgfqpoint{5.740913in}{0.766088in}}%
\pgfpathlineto{\pgfqpoint{5.762485in}{0.761776in}}%
\pgfpathlineto{\pgfqpoint{5.794843in}{0.759547in}}%
\pgfpathlineto{\pgfqpoint{5.859559in}{0.758921in}}%
\pgfpathlineto{\pgfqpoint{6.096066in}{0.758900in}}%
\pgfpathlineto{\pgfqpoint{6.096066in}{0.758900in}}%
\pgfusepath{stroke}%
\end{pgfscope}%
\begin{pgfscope}%
\pgfsetrectcap%
\pgfsetmiterjoin%
\pgfsetlinewidth{0.803000pt}%
\definecolor{currentstroke}{rgb}{0.737255,0.737255,0.737255}%
\pgfsetstrokecolor{currentstroke}%
\pgfsetdash{}{0pt}%
\pgfpathmoveto{\pgfqpoint{0.477328in}{0.583522in}}%
\pgfpathlineto{\pgfqpoint{0.477328in}{4.441828in}}%
\pgfusepath{stroke}%
\end{pgfscope}%
\begin{pgfscope}%
\pgfsetrectcap%
\pgfsetmiterjoin%
\pgfsetlinewidth{0.803000pt}%
\definecolor{currentstroke}{rgb}{0.737255,0.737255,0.737255}%
\pgfsetstrokecolor{currentstroke}%
\pgfsetdash{}{0pt}%
\pgfpathmoveto{\pgfqpoint{6.086066in}{0.583522in}}%
\pgfpathlineto{\pgfqpoint{6.086066in}{4.441828in}}%
\pgfusepath{stroke}%
\end{pgfscope}%
\begin{pgfscope}%
\pgfsetrectcap%
\pgfsetmiterjoin%
\pgfsetlinewidth{0.803000pt}%
\definecolor{currentstroke}{rgb}{0.737255,0.737255,0.737255}%
\pgfsetstrokecolor{currentstroke}%
\pgfsetdash{}{0pt}%
\pgfpathmoveto{\pgfqpoint{0.477328in}{0.583522in}}%
\pgfpathlineto{\pgfqpoint{6.086066in}{0.583522in}}%
\pgfusepath{stroke}%
\end{pgfscope}%
\begin{pgfscope}%
\pgfsetrectcap%
\pgfsetmiterjoin%
\pgfsetlinewidth{0.803000pt}%
\definecolor{currentstroke}{rgb}{0.737255,0.737255,0.737255}%
\pgfsetstrokecolor{currentstroke}%
\pgfsetdash{}{0pt}%
\pgfpathmoveto{\pgfqpoint{0.477328in}{4.441828in}}%
\pgfpathlineto{\pgfqpoint{6.086066in}{4.441828in}}%
\pgfusepath{stroke}%
\end{pgfscope}%
\begin{pgfscope}%
\pgfsetroundcap%
\pgfsetroundjoin%
\pgfsetlinewidth{0.501875pt}%
\definecolor{currentstroke}{rgb}{0.000000,0.000000,0.000000}%
\pgfsetstrokecolor{currentstroke}%
\pgfsetdash{}{0pt}%
\pgfpathmoveto{\pgfqpoint{1.338215in}{2.046171in}}%
\pgfpathquadraticcurveto{\pgfqpoint{1.169719in}{2.046171in}}{\pgfqpoint{1.008987in}{2.046171in}}%
\pgfusepath{stroke}%
\end{pgfscope}%
\begin{pgfscope}%
\pgfsetroundcap%
\pgfsetroundjoin%
\pgfsetlinewidth{0.501875pt}%
\definecolor{currentstroke}{rgb}{0.000000,0.000000,0.000000}%
\pgfsetstrokecolor{currentstroke}%
\pgfsetdash{}{0pt}%
\pgfpathmoveto{\pgfqpoint{1.058987in}{2.021171in}}%
\pgfpathlineto{\pgfqpoint{1.008987in}{2.046171in}}%
\pgfpathlineto{\pgfqpoint{1.058987in}{2.071171in}}%
\pgfusepath{stroke}%
\end{pgfscope}%
\begin{pgfscope}%
\definecolor{textcolor}{rgb}{0.180392,0.180392,0.180392}%
\pgfsetstrokecolor{textcolor}%
\pgfsetfillcolor{textcolor}%
\pgftext[x=1.663792in,y=2.046171in,,]{\color{textcolor}\rmfamily\fontsize{9.000000}{10.800000}\selectfont \textbf{\(\displaystyle {\Delta\omega}\ll{\omega_0}\)}}%
\end{pgfscope}%
\begin{pgfscope}%
\pgfsetroundcap%
\pgfsetroundjoin%
\pgfsetlinewidth{0.501875pt}%
\definecolor{currentstroke}{rgb}{0.000000,0.000000,0.000000}%
\pgfsetstrokecolor{currentstroke}%
\pgfsetdash{}{0pt}%
\pgfpathmoveto{\pgfqpoint{1.152250in}{2.046171in}}%
\pgfpathquadraticcurveto{\pgfqpoint{1.199986in}{2.046171in}}{\pgfqpoint{1.239958in}{2.046171in}}%
\pgfusepath{stroke}%
\end{pgfscope}%
\begin{pgfscope}%
\pgfsetroundcap%
\pgfsetroundjoin%
\pgfsetlinewidth{0.501875pt}%
\definecolor{currentstroke}{rgb}{0.000000,0.000000,0.000000}%
\pgfsetstrokecolor{currentstroke}%
\pgfsetdash{}{0pt}%
\pgfpathmoveto{\pgfqpoint{1.189958in}{2.071171in}}%
\pgfpathlineto{\pgfqpoint{1.239958in}{2.046171in}}%
\pgfpathlineto{\pgfqpoint{1.189958in}{2.021171in}}%
\pgfusepath{stroke}%
\end{pgfscope}%
\begin{pgfscope}%
\definecolor{textcolor}{rgb}{0.180392,0.180392,0.180392}%
\pgfsetstrokecolor{textcolor}%
\pgfsetfillcolor{textcolor}%
\pgftext[x=3.281697in,y=4.525161in,,base]{\color{textcolor}\rmfamily\fontsize{12.960000}{15.552000}\selectfont Pulsed Gaussian Normalised Spectrum}%
\end{pgfscope}%
\end{pgfpicture}%
\makeatother%
\endgroup%
}
		\caption{\centering{The Gaussian Pulse on the normalised frequency spectrum}}
		\label{fig:gaussian_spectrum_normalised}
	\end{figure}
	
	\pagebreak
	
	




\newpage
\setstretch{1}  % Reduce bibliography line spacing
\bibliographystyle{IEEETran}
\bibliography{references.bib}
\end{document}
