\documentclass[colorlinks,11pt,a4paper,normalphoto,withhyper,ragged2e]{altareport}


%%%%%%%%%%%%%%%%%%%%%%%%%%%%%%%%%%%%%%%%%%%
%%%%%%%%%% DEFAULT PACKAGES & SETTINGS %%%%%%%%%%
\usepackage[utf8]{inputenc}
\usepackage{setspace} %1.5 line spacing
\usepackage{notoccite} %% Citation numbering
\usepackage{lscape} %% Landscape table
\usepackage{caption} %% Adds a newline in the table caption

%% The paracol package lets you typeset columns of text in parallel
\usepackage{paracol}
\usepackage[none]{hyphenat}

%% Document and Theme Fonts
\usepackage[T1]{fontenc}
\usepackage{paratype}
\usepackage[defaultsans]{lato}
%\usepackage[sfdefault,light,condensed]{roboto}
%\usepackage[rm]{roboto}
%\usepackage[defaultsans]{lato}
%\usepackage{sourcesanspro}
%\usepackage[rm]{merriweather}

\setlength{\intextsep}{4pt} % Set defualt spacing around floats

\captionsetup{font=footnotesize} % Make Captions a sensible size

%%%%%%%%%%%%%%%%%%%%%%%%%%%%%%%%%%%%%%%%%%%


%%%%%%%%%%%%%%%%%%%%%%%%%%%%%%%%%%%%%%%%%%%
%%%%%%%%%% THEMES %%%%%%%%%%

%% Standard theme options are below, leave blank for B&W / no colours (BoringDefault). Note the theme will be set to default if you enter a non-exsistant theme name.
\SetTheme{UNIBS}
%% UNIBS
%% UNILIM
%% PastelBlue
%% GreenAndGold
%% Purple
%% PastelRed
%% BoringDefault (Leave blank / enter anything not found above)

%%%%%%%%%%%%%%%%%%%%%%%%%%%%%%%%%%%%%%%%%%%






%%%%%%%%%%%%%%%%%%%%%%%%%%%%%%%%%%%%%%%%%%%
%%%%%%%%%% DOCUMENT SPECIFIC PACKAGES %%%%%%%%%%

\usepackage{amssymb}
\usepackage{amsfonts}
\usepackage{mathtools}
\usepackage{relsize}

\usepackage{pythontex} % Run python code in this latex doc

%%%%%%%%% Karnaugh Map Package & Settings %%%%%%%%%
\usepackage[export]{adjustbox}

\usetikzlibrary{matrix,calc}
\usepackage{karnaugh-map}

\colorlet{LightRed}{red!60!}
\colorlet{LightBlue}{blue!60!}
\colorlet{LightYellow}{yellow!60!}
\colorlet{LightGreen}{green!60!}
\colorlet{LightOrange}{orange!60!}

%%%%%%%%% MATLAB Language Settings %%%%%%%%%
\usepackage[numbered,framed]{matlab-prettifier} % To add code listings from matlab
\lstMakeShortInline[style=Matlab-editor]" %% This makes " an escape character to write in matlab editor font


%%%%% Settings for python pgf graphs %%%%%
\usepackage{pgfplots}
\usetikzlibrary{arrows.meta}

\pgfplotsset{compat=newest,
    width=6cm,
    height=3cm,
    scale only axis=true,
    max space between ticks=25pt,
    try min ticks=5,
    every axis/.style={
        axis y line=left,
        axis x line=bottom,
        axis line style={thick,->,>=latex, shorten >=-.4cm}
    },
    every axis plot/.append style={thick},
    tick style={black, thick}
}
\tikzset{
    semithick/.style={line width=0.8pt},
}

\usepgfplotslibrary{groupplots}
\usepgfplotslibrary{dateplot}


% Reduce space around captions
% \captionsetup{aboveskip=5pt, belowskip=5pt}
%%%%%%%%%%%%%%%%%%%%%%%%%%%%%%%%%%%%%%%%%%%




%%%%%%%%%%%%%%%%%%%%%%%%%%%%%%%%%%%%%%%%%%%
%%%%%%%%%% USEFUL SETTINGS %%%%%%%%%%
%% Change some font sizes, this will override the defaults
\renewcommand{\ReportTitleFont}{\Huge\rmfamily\bfseries} %% Title Page - Main Title
\renewcommand{\ReportSubTitleFont}{\huge\bfseries} %% Title Page - Sub-Title
\renewcommand{\ReportSectionFont}{\LARGE\rmfamily\bfseries} %% Section Title
\renewcommand{\ReportSubSectionFont}{\large\bfseries} %% SubSection Title
\renewcommand{\FootNoteFont}{\footnotesize} %% Footnotes and Header/Footer

%% Change the bullets for itemize and rating marker
\renewcommand{\itemmarker}{{\small\textbullet}}
\renewcommand{\ratingmarker}{\faCircle}

%% Change the page layout
\geometry{left=1.5cm,right=1.5cm,top=3cm,bottom=3cm,columnsep=8mm}
\onehalfspace   % 1.5 line spacing

\definecolor{CommentGreen}{HTML}{228B22}
%%%%%%%%%%%%%%%%%%%%%%%%%%%%%%%%%%%%%%%%%%%




%%%%%%%%%%%%%%%%%%%%%%%%%%%%%%%%%%%%%%%%%%%
%\include{references.bib}

%%%%%%%%%% TITLE PAGE INFO %%%%%%%%%%
\ReportTitle{Microwave Engineering}
\SubTitle{Notes \& Tutorials}
\Author{Andrew Simon Wilson}
\ReportDate{\today}
\FacultyOrLocation{EMIMEO Programme}
\ModCoord{Prof. Maria Antonietta Vincenti}

%%%%%%%%%%%%%%%%%%%%%%%%%%%%%%%%%%%%%%%%%%%


\newcommand*\circled[1]{\tikz[baseline=(char.base)]{
            \node[shape=circle,draw,inner sep=0.5pt] (char) {#1};}}


\begin{document}

\MakeReportTitlePage


%%%%% CONTENTS %%%%%
\pagenumbering{roman} % Start roman numbering
\setcounter{page}{1}


%%%%%%%%%% YOUR NAME, PROFESSION, PORTRAIT, CONTACT INFO, SOCIAL MEDIA ETC. %%%%%%%%%%
\name{Andrew Simon Wilson, BEng}
\tagline{Post-graduate Master's Student - EMIMEO Programme}

\personalinfo{
  \email{andrew.s.wilson@protonmail.com}
  \linkedin{andrew-simon-wilson}
  \github{AS-Wilson}
  \phone{+44 7930 560 383}
}

%% You can add multiple photos on the left or right
% \photoR{3cm}{Images/a-wilson-potrait.jpg}
% \photoL{3cm}{Yacht_High,Suitcase_High}


\section*{Author Details}
\makeauthordetails

%% Table of contents print level -1: part, 0: chapter, 1: section, 2:sub-section, 3:sub-sub-section, etc.
\setcounter{tocdepth}{2} 
\tableofcontents %% Prints a list of all sections based on the above command
%\listoffigures %% Prints a list of all figures in the report
%\listoftables %% Prints a list of all tables in the report




%%%%%%%%%% DOCUMENT CONTENT BEGINS HERE %%%%%%%%%%

%%%%% INTRO %%%%%
\section*{Introduction}
\textbf{TODO - PUT SOMETHING RELEVANT HERE}
I wrote this document for the students studying Optical Communication Networks to have a nice set of notes, and correct reference code and graphs for the module. I hope that it is sufficient for this task and it helps all of your studies. \linebreak
I spent have spent a lot of time developing the template used to make this {\LaTeX} document, I want others to benefit from this work so the source code for this template is available on GitHub \cite{latex_template_github}.
\newpage
\pagenumbering{arabic} % Start document numbering - roman numbering





\textbf{TODO - PREVIOUS LESSONS} \linebreak



\section{Introduction}



\pagebreak



\section{Maxwell's Equations and Boundary Conditions}



\pagebreak




\section{Wave Equations, Energy and Power, Polarization}



\pagebreak




\section{Transmission, Reflection at Normal and Oblique Incidence Reciprocity theorem, Image theory}



\pagebreak





\section{Problems on the basics of EM theory}



\pagebreak





\section{Transmission Line Theory 1}



\pagebreak



\section{Transmission Line Theory 2}



\pagebreak



\section{Problems for Transmission Line Theory}



\pagebreak



\section{Transmission Lines and Waveguides}



\pagebreak



\section{Waveguides}



\pagebreak



\section{Rectangular and Circular Waveguides}



\pagebreak



\section{Grounded Slab and Other Waveguides}



\pagebreak



\section{Problems for Waveguides}



\pagebreak



\section{Impedance Matching}


\subsection{Part One}
\textbf{TODO - PART ONE}



\pagebreak



\subsection{Part Two - Quarter-Wave, Binomial Multi-section, and Chebshev Multi-section Transformers}

\subsubsection{The Quarter-Wave Transformer}
\textbf{TODO - CIRCUIT DIAGRAMS OF TRANSFORMER CIRCUIT} \linebreak
\textbf{TODO - PREAMBLE ABOUT THE TRANSFORMER} \linebreak

Pros: \linebreak

\begin{itemize}[leftmargin=1cm]
	\item Simple Design
	\item Can be modified to be a multi-section transformer
\end{itemize}


Cons: \linebreak

\begin{itemize}[leftmargin=1cm]
	\item It can only match real loads
	\item Single, central frequency design and operation range
\end{itemize}



\textbf{TODO - GTRAPH OF REFLECTION COEFF. VS THETA = BETA times L}

\paragraph{For a TEM Transmission Line}
\textbf{TODO}




\subsubsection{Theory of Small Reflections}
The quarter-wave shown, is useful as a transformer but it is limited in it's frequency range, it is however possible to cascade transformer sections to obtain more desirable frequency domain properties, this is accomplished using the theory of small reflections. We shall first study a single section. \linebreak

\paragraph{Single Section Transformer}

\textbf{TODO - CIRCUIT DIAGRAM FOR SINGLE SECTION} \linebreak

We can imagine that we have a single section of a transmission line which has an incident wave $T_{12}$ and a reflected wave $T_{21}$, we can state that: \linebreak

\begin{align*}
	T_{21} = 1 + \Gamma_1 \\
	T_{12} = 1+ \Gamma_2 \\
\end{align*}

These imply: \linebreak

\begin{gather*}
	\Gamma_1 = \frac{Z_2 - Z_1}{Z_2 + Z_1} \\
	\Gamma_3 = \frac{Z_L - Z_2}{Z_L + Z_2} \\
	\therefore \Gamma_2 = -\Gamma_1 \\
	\implies T_{21} = \frac{2Z_2}{Z_2 + Z_1} \\
	\& T_{12} = \frac{2Z_1}{Z_2 + Z_1} \\
\end{gather*}

We can see and imagine that these reflections will repeat an infinite number of times, the following diagram illustrates this: \linebreak

\textbf{TODO - Diagram of reflections} \linebreak

So what is gamma in this model?: \linebreak

\begin{gather*}
	\Gamma = \Gamma_1 + T_{12} T_{21} \Gamma_3 e^{-2j\theta} + T_{12} T_{21} \Gamma_3^2 \Gamma_2 e^{-4j\theta} \dots \\
	\text{This is more concisely stated as:} \\
	\Gamma = \Gamma_1 + T_{12} T_{21} \Gamma_3 e^{-2j\theta} \displaystyle\sum_{n=0}^{\infty} \Gamma_2^n \Gamma_3^n e ^{-j2n\theta }
\end{gather*}


If the discontinuities of the above equations are small, i.e. $|Z_1 - Z_2|$ and $|Z_2 - Z_L|$ are small, this implies: \linebreak

\begin{gather*}
	| \Gamma_1 \Gamma_2 | \ll 1 \\
	\Gamma \approx \Gamma_1 + \Gamma_3 e^{-2j\theta} \\
\end{gather*}




\paragraph{Multi-Section Transformer}
So, now we can apply the previous formulas in a cascade, one after the other, to obtain more desireable frequency charateristics. However, there are some assumptions  required in order to do this, all the lines involved must have an identical electrical length, $\theta$. And, the load impedance, $Z_L$, must be bigger than the input impedance, $Z_0$, i.e. $Z_L > Z_0$. This assumptions are necessary so that the individual section impedances, $Z_n$, will decrease and the individual section reflection coefficients will be less than 0, i.e. $\Gamma_n < 0$. The diagram of the scheme of a Multi Section Transformer is shown below: \linebreak


\textbf{TODO - DIAGRAM OF MULTI SECTION TRANSFORMER} \linebreak

So given these conditions we can state: \linebreak 

\begin{gather*}
	\Gamma(\theta) = \Gamma_0 + \Gamma_1 e^{-j2\theta} + \Gamma_4 e^{-j4\theta} + \dots + \Gamma_n e^{-j2n\theta}
\end{gather*}


If he transformer is symmetrical, i.e.: \linebreak

\begin{gather*}
	\Gamma_0 = \Gamma_n \\
	\Gamma_1 = \Gamma_{n-1} \\
	\Gamma_2 = \Gamma_{n-2} \\
	\dots \\
\end{gather*}



This simplifies further to:

\begin{gather*}
	\textbf{If n is even:} \\
	\Gamma(\theta) = \Gamma_0 + \Gamma_1 e^{-j2\theta} + \Gamma_4 e^{-j4\theta} + \dots + \Gamma_n e^{-j2n\theta} \\
	\textbf{If n is odd:} \\
	\Gamma(\theta) = \Gamma_0 + \Gamma_1 e^{-j2\theta} + \Gamma_4 e^{-j4\theta} + \dots + \Gamma_n e^{-j2n\theta} \\
\end{gather*}



\paragraph{Binomial, Multi-Section Transformer} \vspace{5mm}

This type of transformer will give us the flattest response possible, at the expense of the overall bandwidth of the transformer: \linebreak

\begin{gather*}
	\Gamma(theta) = A (1 + e^{-2j\theta})^n
\end{gather*}











\newpage
\setstretch{1}  % Reduce bibliography line spacing
\bibliographystyle{IEEETran}
\bibliography{references.bib}
\end{document}
