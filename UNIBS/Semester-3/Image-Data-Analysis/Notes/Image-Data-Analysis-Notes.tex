\documentclass[colorlinks,11pt,a4paper,normalphoto,withhyper,ragged2e]{altareport}


%%%%%%%%%%%%%%%%%%%%%%%%%%%%%%%%%%%%%%%%%%%
%%%%%%%%%% DEFAULT PACKAGES & SETTINGS %%%%%%%%%%
\usepackage[utf8]{inputenc}
\usepackage{setspace} %1.5 line spacing
\usepackage{notoccite} %% Citation numbering
\usepackage{lscape} %% Landscape table
\usepackage{caption} %% Adds a newline in the table caption

%% The paracol package lets you typeset columns of text in parallel
\usepackage{paracol}
\usepackage[none]{hyphenat}

%% Document and Theme Fonts
\usepackage[T1]{fontenc}
\usepackage{paratype}
\usepackage[defaultsans]{lato}
%\usepackage[sfdefault,light,condensed]{roboto}
%\usepackage[rm]{roboto}
%\usepackage[defaultsans]{lato}
%\usepackage{sourcesanspro}
%\usepackage[rm]{merriweather}

\setlength{\intextsep}{4pt} % Set defualt spacing around floats

\captionsetup{font=footnotesize} % Make Captions a sensible size

%%%%%%%%%%%%%%%%%%%%%%%%%%%%%%%%%%%%%%%%%%%


%%%%%%%%%%%%%%%%%%%%%%%%%%%%%%%%%%%%%%%%%%%
%%%%%%%%%% THEMES %%%%%%%%%%

%% Standard theme options are below, leave blank for B&W / no colours (BoringDefault). Note the theme will be set to default if you enter a non-exsistant theme name.
\SetTheme{UNIBS}
%% UNIBS
%% UNILIM
%% PastelBlue
%% GreenAndGold
%% Purple
%% PastelRed
%% BoringDefault (Leave blank / enter anything not found above)

%%%%%%%%%%%%%%%%%%%%%%%%%%%%%%%%%%%%%%%%%%%






%%%%%%%%%%%%%%%%%%%%%%%%%%%%%%%%%%%%%%%%%%%
%%%%%%%%%% DOCUMENT SPECIFIC PACKAGES %%%%%%%%%%

\usepackage{amssymb}
\usepackage{amsfonts}
\usepackage{mathtools}
\usepackage{relsize}

\usepackage{pythontex} % Run python code in this latex doc

%%%%%%%%% Karnaugh Map Package & Settings %%%%%%%%%
\usepackage[export]{adjustbox}

\usetikzlibrary{matrix,calc}
\usepackage{karnaugh-map}

\colorlet{LightRed}{red!60!}
\colorlet{LightBlue}{blue!60!}
\colorlet{LightYellow}{yellow!60!}
\colorlet{LightGreen}{green!60!}
\colorlet{LightOrange}{orange!60!}

%%%%%%%%% MATLAB Language Settings %%%%%%%%%
\usepackage[numbered,framed]{matlab-prettifier} % To add code listings from matlab
\lstMakeShortInline[style=Matlab-editor]" %% This makes " an escape character to write in matlab editor font


%%%%% Settings for python pgf graphs %%%%%
\usepackage{pgfplots}
\usetikzlibrary{arrows.meta}

\pgfplotsset{compat=newest,
    width=6cm,
    height=3cm,
    scale only axis=true,
    max space between ticks=25pt,
    try min ticks=5,
    every axis/.style={
        axis y line=left,
        axis x line=bottom,
        axis line style={thick,->,>=latex, shorten >=-.4cm}
    },
    every axis plot/.append style={thick},
    tick style={black, thick}
}
\tikzset{
    semithick/.style={line width=0.8pt},
}

\usepgfplotslibrary{groupplots}
\usepgfplotslibrary{dateplot}


% Reduce space around captions
% \captionsetup{aboveskip=5pt, belowskip=5pt}


%%%%% Python Code Listings %%%%%

\usepackage{color} 
\usepackage{listings} 
\usepackage{setspace} 
 
\definecolor{Code}{HTML}{181818} 
\definecolor{Decorators}{HTML}{3C57AE} 
\definecolor{Numbers}{HTML}{77787E} 
\definecolor{MatchingBrackets}{HTML}{181818} 
\definecolor{Keywords}{HTML}{3C57AE}
\definecolor{self}{HTML}{3C57AE}
\definecolor{Strings}{HTML}{628D3D} 
\definecolor{Comments}{HTML}{9876AA}
\definecolor{Backquotes}{HTML}{181818} 
\definecolor{Classname}{HTML}{BF8F00} 
\definecolor{FunctionName}{HTML}{BF8F00}
\definecolor{Operators}{HTML}{181818}  
\definecolor{Background}{HTML}{FFFFFF} 


\lstset{
	language=Python,
	showstringspaces=false,
	formfeed=\newpage,
	numbers=left,
	numberstyle=\tiny\color{Numbers}, 
	numbersep=1em, 
	xleftmargin=0.5cm, 
	framextopmargin=2em, 
	framexbottommargin=2em, 
	showspaces=false, 
	showtabs=false, 
	showstringspaces=false, 
	frame=l, 
	tabsize=4, 
	% Basic 
	basicstyle=\ttfamily\scriptsize\setstretch{1}, 
	backgroundcolor=\color{Background}, 
	% Comments 
	commentstyle=\color{Comments}, 
	% Strings 
	stringstyle=\color{Strings}, 
	morecomment=[s][\color{Strings}]{"""}{"""}, 
	morecomment=[s][\color{Strings}]{'''}{'''}, 
	% keywords 
	morekeywords = {import,from,class,def,for,while,if,is,in,elif,else,not,and,or,print,break,continue,return,True,False,None,access,as,,del,except,exec,finally,global,import,lambda,pass,print,raise,try,assert}, 
	keywordstyle={\color{Keywords}\bfseries}, 
	% additional keywords 
	morekeywords={[2]@invariant,pylab,numpy,np,scipy}, 
	keywordstyle={[2]\color{Decorators}\slshape}, 
	emph={self}, 
	emphstyle={\scriptsize\color{self}\slshape},
} 

%%%%%%%%%%%%%%%%%%%%%%%%%%%%%%%%%%%%%%%%%%%




%%%%%%%%%%%%%%%%%%%%%%%%%%%%%%%%%%%%%%%%%%%
%%%%%%%%%% USEFUL SETTINGS %%%%%%%%%%
%% Change some font sizes, this will override the defaults
\renewcommand{\ReportTitleFont}{\Huge\rmfamily\bfseries} %% Title Page - Main Title
\renewcommand{\ReportSubTitleFont}{\huge\bfseries} %% Title Page - Sub-Title
\renewcommand{\ReportSectionFont}{\LARGE\rmfamily\bfseries} %% Section Title
\renewcommand{\ReportSubSectionFont}{\large\bfseries} %% SubSection Title
\renewcommand{\FootNoteFont}{\footnotesize} %% Footnotes and Header/Footer

%% Change the bullets for itemize and rating marker
\renewcommand{\itemmarker}{{\small\textbullet}}
\renewcommand{\ratingmarker}{\faCircle}

%% Change the page layout
\geometry{left=1.5cm,right=1.5cm,top=3cm,bottom=3cm,columnsep=8mm}
\onehalfspace   % 1.5 line spacing

\definecolor{CommentGreen}{HTML}{228B22}
%%%%%%%%%%%%%%%%%%%%%%%%%%%%%%%%%%%%%%%%%%%




%%%%%%%%%%%%%%%%%%%%%%%%%%%%%%%%%%%%%%%%%%%
%\include{references.bib}

%%%%%%%%%% TITLE PAGE INFO %%%%%%%%%%
\ReportTitle{Optical Communication Networks}
\SubTitle{Notes \& Tutorials}
\Author{Andrew Simon Wilson}
\ReportDate{\today}
\FacultyOrLocation{EMIMEO Programme}
\ModCoord{Prof. Fabio Baronio}

%%%%%%%%%%%%%%%%%%%%%%%%%%%%%%%%%%%%%%%%%%%


\newcommand*\circled[1]{\tikz[baseline=(char.base)]{
            \node[shape=circle,draw,inner sep=0.5pt] (char) {#1};}}


\begin{document}

\MakeReportTitlePage


%%%%% CONTENTS %%%%%
\pagenumbering{roman} % Start roman numbering
\setcounter{page}{1}


%%%%%%%%%% YOUR NAME, PROFESSION, PORTRAIT, CONTACT INFO, SOCIAL MEDIA ETC. %%%%%%%%%%
\name{Andrew Simon Wilson, BEng}
\tagline{Post-graduate Master's Student - EMIMEO Programme}

\personalinfo{
  \email{andrew.s.wilson@protonmail.com}
  \linkedin{andrew-simon-wilson} 
  \github{AS-Wilson}
  \phone{+44 7930 560 383}
}

%% You can add multiple photos on the left or right
% \photoR{3cm}{Images/a-wilson-potrait.jpg}
% \photoL{3cm}{Yacht_High,Suitcase_High}


\section*{Author Details}
\makeauthordetails

%% Table of contents print level -1: part, 0: chapter, 1: section, 2:sub-section, 3:sub-sub-section, etc.
\setcounter{tocdepth}{2} 
\tableofcontents %% Prints a list of all sections based on the above command
%\listoffigures %% Prints a list of all figures in the report
%\listoftables %% Prints a list of all tables in the report




%%%%%%%%%% DOCUMENT CONTENT BEGINS HERE %%%%%%%%%%

%%%%% INTRO %%%%%
\section*{Introduction}
I wrote this document for the students studying Optical Communication Networks to have a nice set of notes, and correct reference code and graphs for the module. I hope that it is sufficient for this task and it helps all of your studies. \linebreak
I spent have spent a lot of time developing the template used to make this {\LaTeX} document, I want others to benefit from this work so the source code for this template is available on GitHub \cite{latex_template_github}.
\newpage
\pagenumbering{arabic} % Start document numbering - roman numbering




\section{Introduction, The NLSE, and Optical Fibre Propagation}

	\subsection{Introduction}
	The main goal of this class is the investigation of the evolution of the optical pulses which propagate in an optical fibre. \linebreak
	
	We will consider the propagation of pulses from two perspectives:
	\begin{enumerate}[leftmargin=1cm]
		\item Theoretically
		\item Via numerical simulation, using software such as MATLAB, Python, or C/C++
	\end{enumerate}
	
	\vspace{5mm}
	
	We will analyse and understand a number of different optical effects and regimes / models. These include:
	\begin{itemize}[leftmargin=1cm]
		\item Group Velocity Dispersion (GVD, this is a linear effect)
		\item Self-Phase Modulation (this is a non-linear effect)
		\item Optical, Self-Trapped Waves (Solitons)
		\item Abnormal, Extreme Waves
		\item Optical Shocks
	\end{itemize}
	
	\vspace{5mm}
	
	The assessment for the course will consist of a piece of coursework on the numerical dynamics of optical pulses propagating under different linear and non-linear regimes.
	After this coursework is handed in, a type of oral examination will be performed with each student about their coursework. \linebreak
	Each of you is invited to work in groups of 2-3 persons, each with different jobs. \linebreak
	
	Finally, should the lectures leave you with any confusion, or you wish to study further, the suggested textbook for this course to aid in study (if this is needed) is Non-linear Fibre Optics by Govind P. Agrawal \cite{nl_fibre_optics}.


	\pagebreak


	\subsection{The Non Linear Schr\"{o}dinger Equation (3+3D)}
	The first step to considering the propagation of optical pules in a fibre is to know that an optical fibre is a non-linear and dispersive medium and that any propagation with this waveguide is governed by the fundamental and universal modal of optical wave dynamics; the Non Linear Schr\"{o}dinger Equation (NLSE). \linebreak
	
	In Optical Communication Components Prof. Constantino De Angelis will analytically study the properties of the NLSE. This course, however, will consider different regimes from theoretical viewpoints, as well as numerical simulation of each of these regimes to understand the linear and non-liner effects and their uses. \linebreak
	
	So, without further ado, the Non Linear Schr\"{o}dinger Equation (NLSE) (3+3D) is given by Equation \ref{eq:nlse}:
	\begin{equation} \label{eq:nlse}
       \mathlarger{ j \frac{\partial A(r,t)}{\partial z} + \frac{1}{2 \beta} \frac{\partial^2 A(r,t)}{\partial x^2} + \frac{1}{2 \beta} \frac{\partial^2 A(r,t)}{\partial y^2} - \frac{ \beta ''}{2} \frac{\partial^2 A(r,t)}{\partial t^2} + \chi^{(3)} |A(r,t)|^2 A(r,t) = 0 }
	\end{equation}
	
	\begin{align*}
		\text{Where:}& \\
		r & = (x, y, z) \text{, this is the 3-dimensional spatial coordinates} \\
		t & \text{, is the time coordinate} \\
		A & (r,t) \text{, is the slowly varying (compared to the carrier signal) envelope of the signal} \\
		E & (r,t) = Re[A(r,t) e^{i(\omega_0t + \beta_0z)}] \text{, is the electrical field of the pulse and the carrier signal} \\
		e&^{{i(\omega_0t + \beta_0z)}} \text{, is the optical carrier signal at the angular frequency } \omega_0 \text{ and `wavenumber' } \beta_0 
	\end{align*}
	
	
	\pagebreak
	
	
	\subsubsection{A Quick Detour - The Gaussian Pulse} \label{sec:detour_gaussian}
	The type of communication signal model that we will deal with most often initially in the course is that of the Pulsed Gaussian or Gaussian Pulse. This signal is the combination of a lower frequency Gaussian (this is where the information is really communicated) and a higher frequency carrier sine/cosine component, due to how high the frequency of the carrier typically is (10THz) it cannot be easily sampled, so we instead send information via the slowly varying envelope. To give an idea of what the this model looks like these individual signals are shown separately in Figure \ref{fig:gaussian_decon}. \linebreak
	
	Please note that we will give the Gaussian as a function most simply described by Equation \ref{eq:gaussian_time_def}:
	
	\begin{equation} \label{eq:gaussian_time_def}
		\mathlarger{ \mathlarger{ f(t) = I e^{-\frac{t^2}{2t^2_0}} } }
	\end{equation}
	
	{\footnotesize Also note $t_0$ is expressed here as the half-waist duration at an intensity of $\frac{1}{e}$, there are other definitions which are useful for other purposes} \linebreak
	
	If one has a linear system (or one that can be approximated as such a type of system) these two signals can be super-positioned to create a Gaussian pulse, which is shown in Figure \ref{fig:gaussian_demo} \linebreak
	
	\begin{figure}[h]
		\centering
		\scalebox{0.85}{%% Creator: Matplotlib, PGF backend
%%
%% To include the figure in your LaTeX document, write
%%   \input{<filename>.pgf}
%%
%% Make sure the required packages are loaded in your preamble
%%   \usepackage{pgf}
%%
%% Also ensure that all the required font packages are loaded; for instance,
%% the lmodern package is sometimes necessary when using math font.
%%   \usepackage{lmodern}
%%
%% Figures using additional raster images can only be included by \input if
%% they are in the same directory as the main LaTeX file. For loading figures
%% from other directories you can use the `import` package
%%   \usepackage{import}
%%
%% and then include the figures with
%%   \import{<path to file>}{<filename>.pgf}
%%
%% Matplotlib used the following preamble
%%   \usepackage[T1]{fontenc} \usepackage{mathpazo}
%%
\begingroup%
\makeatletter%
\begin{pgfpicture}%
\pgfpathrectangle{\pgfpointorigin}{\pgfqpoint{6.104713in}{2.137646in}}%
\pgfusepath{use as bounding box, clip}%
\begin{pgfscope}%
\pgfsetbuttcap%
\pgfsetmiterjoin%
\definecolor{currentfill}{rgb}{1.000000,1.000000,1.000000}%
\pgfsetfillcolor{currentfill}%
\pgfsetlinewidth{0.000000pt}%
\definecolor{currentstroke}{rgb}{1.000000,1.000000,1.000000}%
\pgfsetstrokecolor{currentstroke}%
\pgfsetdash{}{0pt}%
\pgfpathmoveto{\pgfqpoint{0.000000in}{0.000000in}}%
\pgfpathlineto{\pgfqpoint{6.104713in}{0.000000in}}%
\pgfpathlineto{\pgfqpoint{6.104713in}{2.137646in}}%
\pgfpathlineto{\pgfqpoint{0.000000in}{2.137646in}}%
\pgfpathlineto{\pgfqpoint{0.000000in}{0.000000in}}%
\pgfpathclose%
\pgfusepath{fill}%
\end{pgfscope}%
\begin{pgfscope}%
\pgfsetbuttcap%
\pgfsetmiterjoin%
\definecolor{currentfill}{rgb}{0.933333,0.933333,0.933333}%
\pgfsetfillcolor{currentfill}%
\pgfsetlinewidth{0.000000pt}%
\definecolor{currentstroke}{rgb}{0.000000,0.000000,0.000000}%
\pgfsetstrokecolor{currentstroke}%
\pgfsetstrokeopacity{0.000000}%
\pgfsetdash{}{0pt}%
\pgfpathmoveto{\pgfqpoint{0.470474in}{0.429287in}}%
\pgfpathlineto{\pgfqpoint{3.188288in}{0.429287in}}%
\pgfpathlineto{\pgfqpoint{3.188288in}{1.924084in}}%
\pgfpathlineto{\pgfqpoint{0.470474in}{1.924084in}}%
\pgfpathlineto{\pgfqpoint{0.470474in}{0.429287in}}%
\pgfpathclose%
\pgfusepath{fill}%
\end{pgfscope}%
\begin{pgfscope}%
\pgfpathrectangle{\pgfqpoint{0.470474in}{0.429287in}}{\pgfqpoint{2.717814in}{1.494797in}}%
\pgfusepath{clip}%
\pgfsetbuttcap%
\pgfsetroundjoin%
\pgfsetlinewidth{0.501875pt}%
\definecolor{currentstroke}{rgb}{0.698039,0.698039,0.698039}%
\pgfsetstrokecolor{currentstroke}%
\pgfsetdash{{1.850000pt}{0.800000pt}}{0.000000pt}%
\pgfpathmoveto{\pgfqpoint{0.980064in}{0.429287in}}%
\pgfpathlineto{\pgfqpoint{0.980064in}{1.924084in}}%
\pgfusepath{stroke}%
\end{pgfscope}%
\begin{pgfscope}%
\pgfsetbuttcap%
\pgfsetroundjoin%
\definecolor{currentfill}{rgb}{0.180392,0.180392,0.180392}%
\pgfsetfillcolor{currentfill}%
\pgfsetlinewidth{0.803000pt}%
\definecolor{currentstroke}{rgb}{0.180392,0.180392,0.180392}%
\pgfsetstrokecolor{currentstroke}%
\pgfsetdash{}{0pt}%
\pgfsys@defobject{currentmarker}{\pgfqpoint{0.000000in}{-0.048611in}}{\pgfqpoint{0.000000in}{0.000000in}}{%
\pgfpathmoveto{\pgfqpoint{0.000000in}{0.000000in}}%
\pgfpathlineto{\pgfqpoint{0.000000in}{-0.048611in}}%
\pgfusepath{stroke,fill}%
}%
\begin{pgfscope}%
\pgfsys@transformshift{0.980064in}{0.429287in}%
\pgfsys@useobject{currentmarker}{}%
\end{pgfscope}%
\end{pgfscope}%
\begin{pgfscope}%
\definecolor{textcolor}{rgb}{0.180392,0.180392,0.180392}%
\pgfsetstrokecolor{textcolor}%
\pgfsetfillcolor{textcolor}%
\pgftext[x=0.980064in,y=0.332064in,,top]{\color{textcolor}\rmfamily\fontsize{9.000000}{10.800000}\selectfont \(\displaystyle {0}\)}%
\end{pgfscope}%
\begin{pgfscope}%
\pgfpathrectangle{\pgfqpoint{0.470474in}{0.429287in}}{\pgfqpoint{2.717814in}{1.494797in}}%
\pgfusepath{clip}%
\pgfsetbuttcap%
\pgfsetroundjoin%
\pgfsetlinewidth{0.501875pt}%
\definecolor{currentstroke}{rgb}{0.698039,0.698039,0.698039}%
\pgfsetstrokecolor{currentstroke}%
\pgfsetdash{{1.850000pt}{0.800000pt}}{0.000000pt}%
\pgfpathmoveto{\pgfqpoint{1.659518in}{0.429287in}}%
\pgfpathlineto{\pgfqpoint{1.659518in}{1.924084in}}%
\pgfusepath{stroke}%
\end{pgfscope}%
\begin{pgfscope}%
\pgfsetbuttcap%
\pgfsetroundjoin%
\definecolor{currentfill}{rgb}{0.180392,0.180392,0.180392}%
\pgfsetfillcolor{currentfill}%
\pgfsetlinewidth{0.803000pt}%
\definecolor{currentstroke}{rgb}{0.180392,0.180392,0.180392}%
\pgfsetstrokecolor{currentstroke}%
\pgfsetdash{}{0pt}%
\pgfsys@defobject{currentmarker}{\pgfqpoint{0.000000in}{-0.048611in}}{\pgfqpoint{0.000000in}{0.000000in}}{%
\pgfpathmoveto{\pgfqpoint{0.000000in}{0.000000in}}%
\pgfpathlineto{\pgfqpoint{0.000000in}{-0.048611in}}%
\pgfusepath{stroke,fill}%
}%
\begin{pgfscope}%
\pgfsys@transformshift{1.659518in}{0.429287in}%
\pgfsys@useobject{currentmarker}{}%
\end{pgfscope}%
\end{pgfscope}%
\begin{pgfscope}%
\definecolor{textcolor}{rgb}{0.180392,0.180392,0.180392}%
\pgfsetstrokecolor{textcolor}%
\pgfsetfillcolor{textcolor}%
\pgftext[x=1.659518in,y=0.332064in,,top]{\color{textcolor}\rmfamily\fontsize{9.000000}{10.800000}\selectfont \(\displaystyle {2}\)}%
\end{pgfscope}%
\begin{pgfscope}%
\pgfpathrectangle{\pgfqpoint{0.470474in}{0.429287in}}{\pgfqpoint{2.717814in}{1.494797in}}%
\pgfusepath{clip}%
\pgfsetbuttcap%
\pgfsetroundjoin%
\pgfsetlinewidth{0.501875pt}%
\definecolor{currentstroke}{rgb}{0.698039,0.698039,0.698039}%
\pgfsetstrokecolor{currentstroke}%
\pgfsetdash{{1.850000pt}{0.800000pt}}{0.000000pt}%
\pgfpathmoveto{\pgfqpoint{2.338971in}{0.429287in}}%
\pgfpathlineto{\pgfqpoint{2.338971in}{1.924084in}}%
\pgfusepath{stroke}%
\end{pgfscope}%
\begin{pgfscope}%
\pgfsetbuttcap%
\pgfsetroundjoin%
\definecolor{currentfill}{rgb}{0.180392,0.180392,0.180392}%
\pgfsetfillcolor{currentfill}%
\pgfsetlinewidth{0.803000pt}%
\definecolor{currentstroke}{rgb}{0.180392,0.180392,0.180392}%
\pgfsetstrokecolor{currentstroke}%
\pgfsetdash{}{0pt}%
\pgfsys@defobject{currentmarker}{\pgfqpoint{0.000000in}{-0.048611in}}{\pgfqpoint{0.000000in}{0.000000in}}{%
\pgfpathmoveto{\pgfqpoint{0.000000in}{0.000000in}}%
\pgfpathlineto{\pgfqpoint{0.000000in}{-0.048611in}}%
\pgfusepath{stroke,fill}%
}%
\begin{pgfscope}%
\pgfsys@transformshift{2.338971in}{0.429287in}%
\pgfsys@useobject{currentmarker}{}%
\end{pgfscope}%
\end{pgfscope}%
\begin{pgfscope}%
\definecolor{textcolor}{rgb}{0.180392,0.180392,0.180392}%
\pgfsetstrokecolor{textcolor}%
\pgfsetfillcolor{textcolor}%
\pgftext[x=2.338971in,y=0.332064in,,top]{\color{textcolor}\rmfamily\fontsize{9.000000}{10.800000}\selectfont \(\displaystyle {4}\)}%
\end{pgfscope}%
\begin{pgfscope}%
\pgfpathrectangle{\pgfqpoint{0.470474in}{0.429287in}}{\pgfqpoint{2.717814in}{1.494797in}}%
\pgfusepath{clip}%
\pgfsetbuttcap%
\pgfsetroundjoin%
\pgfsetlinewidth{0.501875pt}%
\definecolor{currentstroke}{rgb}{0.698039,0.698039,0.698039}%
\pgfsetstrokecolor{currentstroke}%
\pgfsetdash{{1.850000pt}{0.800000pt}}{0.000000pt}%
\pgfpathmoveto{\pgfqpoint{3.018425in}{0.429287in}}%
\pgfpathlineto{\pgfqpoint{3.018425in}{1.924084in}}%
\pgfusepath{stroke}%
\end{pgfscope}%
\begin{pgfscope}%
\pgfsetbuttcap%
\pgfsetroundjoin%
\definecolor{currentfill}{rgb}{0.180392,0.180392,0.180392}%
\pgfsetfillcolor{currentfill}%
\pgfsetlinewidth{0.803000pt}%
\definecolor{currentstroke}{rgb}{0.180392,0.180392,0.180392}%
\pgfsetstrokecolor{currentstroke}%
\pgfsetdash{}{0pt}%
\pgfsys@defobject{currentmarker}{\pgfqpoint{0.000000in}{-0.048611in}}{\pgfqpoint{0.000000in}{0.000000in}}{%
\pgfpathmoveto{\pgfqpoint{0.000000in}{0.000000in}}%
\pgfpathlineto{\pgfqpoint{0.000000in}{-0.048611in}}%
\pgfusepath{stroke,fill}%
}%
\begin{pgfscope}%
\pgfsys@transformshift{3.018425in}{0.429287in}%
\pgfsys@useobject{currentmarker}{}%
\end{pgfscope}%
\end{pgfscope}%
\begin{pgfscope}%
\definecolor{textcolor}{rgb}{0.180392,0.180392,0.180392}%
\pgfsetstrokecolor{textcolor}%
\pgfsetfillcolor{textcolor}%
\pgftext[x=3.018425in,y=0.332064in,,top]{\color{textcolor}\rmfamily\fontsize{9.000000}{10.800000}\selectfont \(\displaystyle {6}\)}%
\end{pgfscope}%
\begin{pgfscope}%
\definecolor{textcolor}{rgb}{0.180392,0.180392,0.180392}%
\pgfsetstrokecolor{textcolor}%
\pgfsetfillcolor{textcolor}%
\pgftext[x=1.829381in,y=0.150823in,,top]{\color{textcolor}\rmfamily\fontsize{10.800000}{12.960000}\selectfont Time (Seconds)}%
\end{pgfscope}%
\begin{pgfscope}%
\pgfpathrectangle{\pgfqpoint{0.470474in}{0.429287in}}{\pgfqpoint{2.717814in}{1.494797in}}%
\pgfusepath{clip}%
\pgfsetbuttcap%
\pgfsetroundjoin%
\pgfsetlinewidth{0.501875pt}%
\definecolor{currentstroke}{rgb}{0.698039,0.698039,0.698039}%
\pgfsetstrokecolor{currentstroke}%
\pgfsetdash{{1.850000pt}{0.800000pt}}{0.000000pt}%
\pgfpathmoveto{\pgfqpoint{0.470474in}{0.497232in}}%
\pgfpathlineto{\pgfqpoint{3.188288in}{0.497232in}}%
\pgfusepath{stroke}%
\end{pgfscope}%
\begin{pgfscope}%
\pgfsetbuttcap%
\pgfsetroundjoin%
\definecolor{currentfill}{rgb}{0.180392,0.180392,0.180392}%
\pgfsetfillcolor{currentfill}%
\pgfsetlinewidth{0.803000pt}%
\definecolor{currentstroke}{rgb}{0.180392,0.180392,0.180392}%
\pgfsetstrokecolor{currentstroke}%
\pgfsetdash{}{0pt}%
\pgfsys@defobject{currentmarker}{\pgfqpoint{-0.048611in}{0.000000in}}{\pgfqpoint{-0.000000in}{0.000000in}}{%
\pgfpathmoveto{\pgfqpoint{-0.000000in}{0.000000in}}%
\pgfpathlineto{\pgfqpoint{-0.048611in}{0.000000in}}%
\pgfusepath{stroke,fill}%
}%
\begin{pgfscope}%
\pgfsys@transformshift{0.470474in}{0.497232in}%
\pgfsys@useobject{currentmarker}{}%
\end{pgfscope}%
\end{pgfscope}%
\begin{pgfscope}%
\definecolor{textcolor}{rgb}{0.180392,0.180392,0.180392}%
\pgfsetstrokecolor{textcolor}%
\pgfsetfillcolor{textcolor}%
\pgftext[x=0.206378in, y=0.452014in, left, base]{\color{textcolor}\rmfamily\fontsize{9.000000}{10.800000}\selectfont \(\displaystyle {\ensuremath{-}2}\)}%
\end{pgfscope}%
\begin{pgfscope}%
\pgfpathrectangle{\pgfqpoint{0.470474in}{0.429287in}}{\pgfqpoint{2.717814in}{1.494797in}}%
\pgfusepath{clip}%
\pgfsetbuttcap%
\pgfsetroundjoin%
\pgfsetlinewidth{0.501875pt}%
\definecolor{currentstroke}{rgb}{0.698039,0.698039,0.698039}%
\pgfsetstrokecolor{currentstroke}%
\pgfsetdash{{1.850000pt}{0.800000pt}}{0.000000pt}%
\pgfpathmoveto{\pgfqpoint{0.470474in}{0.836959in}}%
\pgfpathlineto{\pgfqpoint{3.188288in}{0.836959in}}%
\pgfusepath{stroke}%
\end{pgfscope}%
\begin{pgfscope}%
\pgfsetbuttcap%
\pgfsetroundjoin%
\definecolor{currentfill}{rgb}{0.180392,0.180392,0.180392}%
\pgfsetfillcolor{currentfill}%
\pgfsetlinewidth{0.803000pt}%
\definecolor{currentstroke}{rgb}{0.180392,0.180392,0.180392}%
\pgfsetstrokecolor{currentstroke}%
\pgfsetdash{}{0pt}%
\pgfsys@defobject{currentmarker}{\pgfqpoint{-0.048611in}{0.000000in}}{\pgfqpoint{-0.000000in}{0.000000in}}{%
\pgfpathmoveto{\pgfqpoint{-0.000000in}{0.000000in}}%
\pgfpathlineto{\pgfqpoint{-0.048611in}{0.000000in}}%
\pgfusepath{stroke,fill}%
}%
\begin{pgfscope}%
\pgfsys@transformshift{0.470474in}{0.836959in}%
\pgfsys@useobject{currentmarker}{}%
\end{pgfscope}%
\end{pgfscope}%
\begin{pgfscope}%
\definecolor{textcolor}{rgb}{0.180392,0.180392,0.180392}%
\pgfsetstrokecolor{textcolor}%
\pgfsetfillcolor{textcolor}%
\pgftext[x=0.206378in, y=0.791740in, left, base]{\color{textcolor}\rmfamily\fontsize{9.000000}{10.800000}\selectfont \(\displaystyle {\ensuremath{-}1}\)}%
\end{pgfscope}%
\begin{pgfscope}%
\pgfpathrectangle{\pgfqpoint{0.470474in}{0.429287in}}{\pgfqpoint{2.717814in}{1.494797in}}%
\pgfusepath{clip}%
\pgfsetbuttcap%
\pgfsetroundjoin%
\pgfsetlinewidth{0.501875pt}%
\definecolor{currentstroke}{rgb}{0.698039,0.698039,0.698039}%
\pgfsetstrokecolor{currentstroke}%
\pgfsetdash{{1.850000pt}{0.800000pt}}{0.000000pt}%
\pgfpathmoveto{\pgfqpoint{0.470474in}{1.176685in}}%
\pgfpathlineto{\pgfqpoint{3.188288in}{1.176685in}}%
\pgfusepath{stroke}%
\end{pgfscope}%
\begin{pgfscope}%
\pgfsetbuttcap%
\pgfsetroundjoin%
\definecolor{currentfill}{rgb}{0.180392,0.180392,0.180392}%
\pgfsetfillcolor{currentfill}%
\pgfsetlinewidth{0.803000pt}%
\definecolor{currentstroke}{rgb}{0.180392,0.180392,0.180392}%
\pgfsetstrokecolor{currentstroke}%
\pgfsetdash{}{0pt}%
\pgfsys@defobject{currentmarker}{\pgfqpoint{-0.048611in}{0.000000in}}{\pgfqpoint{-0.000000in}{0.000000in}}{%
\pgfpathmoveto{\pgfqpoint{-0.000000in}{0.000000in}}%
\pgfpathlineto{\pgfqpoint{-0.048611in}{0.000000in}}%
\pgfusepath{stroke,fill}%
}%
\begin{pgfscope}%
\pgfsys@transformshift{0.470474in}{1.176685in}%
\pgfsys@useobject{currentmarker}{}%
\end{pgfscope}%
\end{pgfscope}%
\begin{pgfscope}%
\definecolor{textcolor}{rgb}{0.180392,0.180392,0.180392}%
\pgfsetstrokecolor{textcolor}%
\pgfsetfillcolor{textcolor}%
\pgftext[x=0.310752in, y=1.131467in, left, base]{\color{textcolor}\rmfamily\fontsize{9.000000}{10.800000}\selectfont \(\displaystyle {0}\)}%
\end{pgfscope}%
\begin{pgfscope}%
\pgfpathrectangle{\pgfqpoint{0.470474in}{0.429287in}}{\pgfqpoint{2.717814in}{1.494797in}}%
\pgfusepath{clip}%
\pgfsetbuttcap%
\pgfsetroundjoin%
\pgfsetlinewidth{0.501875pt}%
\definecolor{currentstroke}{rgb}{0.698039,0.698039,0.698039}%
\pgfsetstrokecolor{currentstroke}%
\pgfsetdash{{1.850000pt}{0.800000pt}}{0.000000pt}%
\pgfpathmoveto{\pgfqpoint{0.470474in}{1.516412in}}%
\pgfpathlineto{\pgfqpoint{3.188288in}{1.516412in}}%
\pgfusepath{stroke}%
\end{pgfscope}%
\begin{pgfscope}%
\pgfsetbuttcap%
\pgfsetroundjoin%
\definecolor{currentfill}{rgb}{0.180392,0.180392,0.180392}%
\pgfsetfillcolor{currentfill}%
\pgfsetlinewidth{0.803000pt}%
\definecolor{currentstroke}{rgb}{0.180392,0.180392,0.180392}%
\pgfsetstrokecolor{currentstroke}%
\pgfsetdash{}{0pt}%
\pgfsys@defobject{currentmarker}{\pgfqpoint{-0.048611in}{0.000000in}}{\pgfqpoint{-0.000000in}{0.000000in}}{%
\pgfpathmoveto{\pgfqpoint{-0.000000in}{0.000000in}}%
\pgfpathlineto{\pgfqpoint{-0.048611in}{0.000000in}}%
\pgfusepath{stroke,fill}%
}%
\begin{pgfscope}%
\pgfsys@transformshift{0.470474in}{1.516412in}%
\pgfsys@useobject{currentmarker}{}%
\end{pgfscope}%
\end{pgfscope}%
\begin{pgfscope}%
\definecolor{textcolor}{rgb}{0.180392,0.180392,0.180392}%
\pgfsetstrokecolor{textcolor}%
\pgfsetfillcolor{textcolor}%
\pgftext[x=0.310752in, y=1.471194in, left, base]{\color{textcolor}\rmfamily\fontsize{9.000000}{10.800000}\selectfont \(\displaystyle {1}\)}%
\end{pgfscope}%
\begin{pgfscope}%
\pgfpathrectangle{\pgfqpoint{0.470474in}{0.429287in}}{\pgfqpoint{2.717814in}{1.494797in}}%
\pgfusepath{clip}%
\pgfsetbuttcap%
\pgfsetroundjoin%
\pgfsetlinewidth{0.501875pt}%
\definecolor{currentstroke}{rgb}{0.698039,0.698039,0.698039}%
\pgfsetstrokecolor{currentstroke}%
\pgfsetdash{{1.850000pt}{0.800000pt}}{0.000000pt}%
\pgfpathmoveto{\pgfqpoint{0.470474in}{1.856139in}}%
\pgfpathlineto{\pgfqpoint{3.188288in}{1.856139in}}%
\pgfusepath{stroke}%
\end{pgfscope}%
\begin{pgfscope}%
\pgfsetbuttcap%
\pgfsetroundjoin%
\definecolor{currentfill}{rgb}{0.180392,0.180392,0.180392}%
\pgfsetfillcolor{currentfill}%
\pgfsetlinewidth{0.803000pt}%
\definecolor{currentstroke}{rgb}{0.180392,0.180392,0.180392}%
\pgfsetstrokecolor{currentstroke}%
\pgfsetdash{}{0pt}%
\pgfsys@defobject{currentmarker}{\pgfqpoint{-0.048611in}{0.000000in}}{\pgfqpoint{-0.000000in}{0.000000in}}{%
\pgfpathmoveto{\pgfqpoint{-0.000000in}{0.000000in}}%
\pgfpathlineto{\pgfqpoint{-0.048611in}{0.000000in}}%
\pgfusepath{stroke,fill}%
}%
\begin{pgfscope}%
\pgfsys@transformshift{0.470474in}{1.856139in}%
\pgfsys@useobject{currentmarker}{}%
\end{pgfscope}%
\end{pgfscope}%
\begin{pgfscope}%
\definecolor{textcolor}{rgb}{0.180392,0.180392,0.180392}%
\pgfsetstrokecolor{textcolor}%
\pgfsetfillcolor{textcolor}%
\pgftext[x=0.310752in, y=1.810920in, left, base]{\color{textcolor}\rmfamily\fontsize{9.000000}{10.800000}\selectfont \(\displaystyle {2}\)}%
\end{pgfscope}%
\begin{pgfscope}%
\definecolor{textcolor}{rgb}{0.180392,0.180392,0.180392}%
\pgfsetstrokecolor{textcolor}%
\pgfsetfillcolor{textcolor}%
\pgftext[x=0.150823in,y=1.176685in,,bottom,rotate=90.000000]{\color{textcolor}\rmfamily\fontsize{10.800000}{12.960000}\selectfont Intensity (Unitless)}%
\end{pgfscope}%
\begin{pgfscope}%
\pgfpathrectangle{\pgfqpoint{0.470474in}{0.429287in}}{\pgfqpoint{2.717814in}{1.494797in}}%
\pgfusepath{clip}%
\pgfsetrectcap%
\pgfsetroundjoin%
\pgfsetlinewidth{2.007500pt}%
\definecolor{currentstroke}{rgb}{0.000000,0.000000,0.000000}%
\pgfsetstrokecolor{currentstroke}%
\pgfsetdash{}{0pt}%
\pgfpathmoveto{\pgfqpoint{0.470473in}{1.176595in}}%
\pgfpathlineto{\pgfqpoint{0.481266in}{1.663306in}}%
\pgfpathlineto{\pgfqpoint{0.486399in}{1.804416in}}%
\pgfpathlineto{\pgfqpoint{0.489878in}{1.849925in}}%
\pgfpathlineto{\pgfqpoint{0.491746in}{1.856136in}}%
\pgfpathlineto{\pgfqpoint{0.491906in}{1.856065in}}%
\pgfpathlineto{\pgfqpoint{0.492769in}{1.854044in}}%
\pgfpathlineto{\pgfqpoint{0.494335in}{1.843341in}}%
\pgfpathlineto{\pgfqpoint{0.496791in}{1.808647in}}%
\pgfpathlineto{\pgfqpoint{0.500311in}{1.723097in}}%
\pgfpathlineto{\pgfqpoint{0.505284in}{1.541263in}}%
\pgfpathlineto{\pgfqpoint{0.513876in}{1.129682in}}%
\pgfpathlineto{\pgfqpoint{0.523806in}{0.687450in}}%
\pgfpathlineto{\pgfqpoint{0.528916in}{0.547978in}}%
\pgfpathlineto{\pgfqpoint{0.532374in}{0.503240in}}%
\pgfpathlineto{\pgfqpoint{0.534212in}{0.497235in}}%
\pgfpathlineto{\pgfqpoint{0.534385in}{0.497316in}}%
\pgfpathlineto{\pgfqpoint{0.535258in}{0.499421in}}%
\pgfpathlineto{\pgfqpoint{0.536842in}{0.510428in}}%
\pgfpathlineto{\pgfqpoint{0.539318in}{0.545860in}}%
\pgfpathlineto{\pgfqpoint{0.542861in}{0.632822in}}%
\pgfpathlineto{\pgfqpoint{0.547872in}{0.817310in}}%
\pgfpathlineto{\pgfqpoint{0.556658in}{1.239518in}}%
\pgfpathlineto{\pgfqpoint{0.566326in}{1.667813in}}%
\pgfpathlineto{\pgfqpoint{0.571419in}{1.806102in}}%
\pgfpathlineto{\pgfqpoint{0.574867in}{1.850310in}}%
\pgfpathlineto{\pgfqpoint{0.576681in}{1.856135in}}%
\pgfpathlineto{\pgfqpoint{0.576861in}{1.856047in}}%
\pgfpathlineto{\pgfqpoint{0.577745in}{1.853867in}}%
\pgfpathlineto{\pgfqpoint{0.579341in}{1.842605in}}%
\pgfpathlineto{\pgfqpoint{0.581835in}{1.806555in}}%
\pgfpathlineto{\pgfqpoint{0.585399in}{1.718391in}}%
\pgfpathlineto{\pgfqpoint{0.590440in}{1.531702in}}%
\pgfpathlineto{\pgfqpoint{0.599399in}{1.100093in}}%
\pgfpathlineto{\pgfqpoint{0.608850in}{0.683556in}}%
\pgfpathlineto{\pgfqpoint{0.613926in}{0.546499in}}%
\pgfpathlineto{\pgfqpoint{0.617357in}{0.502905in}}%
\pgfpathlineto{\pgfqpoint{0.619147in}{0.497235in}}%
\pgfpathlineto{\pgfqpoint{0.619341in}{0.497335in}}%
\pgfpathlineto{\pgfqpoint{0.620234in}{0.499603in}}%
\pgfpathlineto{\pgfqpoint{0.621848in}{0.511176in}}%
\pgfpathlineto{\pgfqpoint{0.624362in}{0.547975in}}%
\pgfpathlineto{\pgfqpoint{0.627949in}{0.637567in}}%
\pgfpathlineto{\pgfqpoint{0.633032in}{0.827071in}}%
\pgfpathlineto{\pgfqpoint{0.642221in}{1.271065in}}%
\pgfpathlineto{\pgfqpoint{0.651404in}{1.672859in}}%
\pgfpathlineto{\pgfqpoint{0.656449in}{1.807951in}}%
\pgfpathlineto{\pgfqpoint{0.659860in}{1.850705in}}%
\pgfpathlineto{\pgfqpoint{0.661613in}{1.856135in}}%
\pgfpathlineto{\pgfqpoint{0.661820in}{1.856023in}}%
\pgfpathlineto{\pgfqpoint{0.662724in}{1.853667in}}%
\pgfpathlineto{\pgfqpoint{0.664355in}{1.841779in}}%
\pgfpathlineto{\pgfqpoint{0.666886in}{1.804289in}}%
\pgfpathlineto{\pgfqpoint{0.670497in}{1.713299in}}%
\pgfpathlineto{\pgfqpoint{0.675617in}{1.521160in}}%
\pgfpathlineto{\pgfqpoint{0.685054in}{1.064078in}}%
\pgfpathlineto{\pgfqpoint{0.693975in}{0.676910in}}%
\pgfpathlineto{\pgfqpoint{0.698986in}{0.544101in}}%
\pgfpathlineto{\pgfqpoint{0.702373in}{0.502368in}}%
\pgfpathlineto{\pgfqpoint{0.704079in}{0.497235in}}%
\pgfpathlineto{\pgfqpoint{0.704310in}{0.497371in}}%
\pgfpathlineto{\pgfqpoint{0.705234in}{0.499896in}}%
\pgfpathlineto{\pgfqpoint{0.706892in}{0.512335in}}%
\pgfpathlineto{\pgfqpoint{0.709457in}{0.551129in}}%
\pgfpathlineto{\pgfqpoint{0.713109in}{0.644607in}}%
\pgfpathlineto{\pgfqpoint{0.718296in}{0.841526in}}%
\pgfpathlineto{\pgfqpoint{0.728264in}{1.325984in}}%
\pgfpathlineto{\pgfqpoint{0.736699in}{1.685162in}}%
\pgfpathlineto{\pgfqpoint{0.741629in}{1.812456in}}%
\pgfpathlineto{\pgfqpoint{0.744955in}{1.851691in}}%
\pgfpathlineto{\pgfqpoint{0.746548in}{1.856135in}}%
\pgfpathlineto{\pgfqpoint{0.746823in}{1.855947in}}%
\pgfpathlineto{\pgfqpoint{0.747791in}{1.853051in}}%
\pgfpathlineto{\pgfqpoint{0.749510in}{1.839382in}}%
\pgfpathlineto{\pgfqpoint{0.752147in}{1.797756in}}%
\pgfpathlineto{\pgfqpoint{0.755891in}{1.698742in}}%
\pgfpathlineto{\pgfqpoint{0.761231in}{1.491147in}}%
\pgfpathlineto{\pgfqpoint{0.783347in}{0.555132in}}%
\pgfpathlineto{\pgfqpoint{0.786931in}{0.504930in}}%
\pgfpathlineto{\pgfqpoint{0.789004in}{0.497234in}}%
\pgfpathlineto{\pgfqpoint{0.789068in}{0.497251in}}%
\pgfpathlineto{\pgfqpoint{0.789867in}{0.498732in}}%
\pgfpathlineto{\pgfqpoint{0.791314in}{0.507437in}}%
\pgfpathlineto{\pgfqpoint{0.793631in}{0.537256in}}%
\pgfpathlineto{\pgfqpoint{0.796977in}{0.613047in}}%
\pgfpathlineto{\pgfqpoint{0.801689in}{0.776552in}}%
\pgfpathlineto{\pgfqpoint{0.809245in}{1.128653in}}%
\pgfpathlineto{\pgfqpoint{0.821081in}{1.666397in}}%
\pgfpathlineto{\pgfqpoint{0.826187in}{1.805588in}}%
\pgfpathlineto{\pgfqpoint{0.829642in}{1.850176in}}%
\pgfpathlineto{\pgfqpoint{0.831473in}{1.856136in}}%
\pgfpathlineto{\pgfqpoint{0.831650in}{1.856052in}}%
\pgfpathlineto{\pgfqpoint{0.832526in}{1.853922in}}%
\pgfpathlineto{\pgfqpoint{0.834113in}{1.842841in}}%
\pgfpathlineto{\pgfqpoint{0.836596in}{1.807192in}}%
\pgfpathlineto{\pgfqpoint{0.840146in}{1.719829in}}%
\pgfpathlineto{\pgfqpoint{0.845167in}{1.534607in}}%
\pgfpathlineto{\pgfqpoint{0.854007in}{1.109430in}}%
\pgfpathlineto{\pgfqpoint{0.863604in}{0.684966in}}%
\pgfpathlineto{\pgfqpoint{0.868690in}{0.547074in}}%
\pgfpathlineto{\pgfqpoint{0.872135in}{0.503015in}}%
\pgfpathlineto{\pgfqpoint{0.873942in}{0.497235in}}%
\pgfpathlineto{\pgfqpoint{0.874126in}{0.497327in}}%
\pgfpathlineto{\pgfqpoint{0.875012in}{0.499532in}}%
\pgfpathlineto{\pgfqpoint{0.876612in}{0.510868in}}%
\pgfpathlineto{\pgfqpoint{0.879109in}{0.547073in}}%
\pgfpathlineto{\pgfqpoint{0.882677in}{0.635497in}}%
\pgfpathlineto{\pgfqpoint{0.887728in}{0.822837in}}%
\pgfpathlineto{\pgfqpoint{0.896734in}{1.257012in}}%
\pgfpathlineto{\pgfqpoint{0.906128in}{1.670404in}}%
\pgfpathlineto{\pgfqpoint{0.911197in}{1.807064in}}%
\pgfpathlineto{\pgfqpoint{0.914624in}{1.850510in}}%
\pgfpathlineto{\pgfqpoint{0.916408in}{1.856135in}}%
\pgfpathlineto{\pgfqpoint{0.916602in}{1.856035in}}%
\pgfpathlineto{\pgfqpoint{0.917495in}{1.853768in}}%
\pgfpathlineto{\pgfqpoint{0.919109in}{1.842194in}}%
\pgfpathlineto{\pgfqpoint{0.921623in}{1.805394in}}%
\pgfpathlineto{\pgfqpoint{0.925210in}{1.715802in}}%
\pgfpathlineto{\pgfqpoint{0.930293in}{1.526297in}}%
\pgfpathlineto{\pgfqpoint{0.939482in}{1.082303in}}%
\pgfpathlineto{\pgfqpoint{0.948665in}{0.680509in}}%
\pgfpathlineto{\pgfqpoint{0.953710in}{0.545418in}}%
\pgfpathlineto{\pgfqpoint{0.957121in}{0.502665in}}%
\pgfpathlineto{\pgfqpoint{0.958874in}{0.497235in}}%
\pgfpathlineto{\pgfqpoint{0.959081in}{0.497348in}}%
\pgfpathlineto{\pgfqpoint{0.959985in}{0.499704in}}%
\pgfpathlineto{\pgfqpoint{0.961615in}{0.511592in}}%
\pgfpathlineto{\pgfqpoint{0.964146in}{0.549083in}}%
\pgfpathlineto{\pgfqpoint{0.967758in}{0.640073in}}%
\pgfpathlineto{\pgfqpoint{0.972877in}{0.832213in}}%
\pgfpathlineto{\pgfqpoint{0.982315in}{1.289295in}}%
\pgfpathlineto{\pgfqpoint{0.991236in}{1.676463in}}%
\pgfpathlineto{\pgfqpoint{0.996247in}{1.809270in}}%
\pgfpathlineto{\pgfqpoint{0.999634in}{1.851003in}}%
\pgfpathlineto{\pgfqpoint{1.001340in}{1.856135in}}%
\pgfpathlineto{\pgfqpoint{1.001571in}{1.856000in}}%
\pgfpathlineto{\pgfqpoint{1.002495in}{1.853474in}}%
\pgfpathlineto{\pgfqpoint{1.004153in}{1.841035in}}%
\pgfpathlineto{\pgfqpoint{1.006718in}{1.802240in}}%
\pgfpathlineto{\pgfqpoint{1.010370in}{1.708761in}}%
\pgfpathlineto{\pgfqpoint{1.015557in}{1.511842in}}%
\pgfpathlineto{\pgfqpoint{1.025525in}{1.027384in}}%
\pgfpathlineto{\pgfqpoint{1.033960in}{0.668207in}}%
\pgfpathlineto{\pgfqpoint{1.038890in}{0.540914in}}%
\pgfpathlineto{\pgfqpoint{1.042216in}{0.501679in}}%
\pgfpathlineto{\pgfqpoint{1.043809in}{0.497236in}}%
\pgfpathlineto{\pgfqpoint{1.044084in}{0.497424in}}%
\pgfpathlineto{\pgfqpoint{1.045052in}{0.500320in}}%
\pgfpathlineto{\pgfqpoint{1.046771in}{0.513989in}}%
\pgfpathlineto{\pgfqpoint{1.049408in}{0.555616in}}%
\pgfpathlineto{\pgfqpoint{1.053151in}{0.654630in}}%
\pgfpathlineto{\pgfqpoint{1.058492in}{0.862226in}}%
\pgfpathlineto{\pgfqpoint{1.080608in}{1.798239in}}%
\pgfpathlineto{\pgfqpoint{1.084192in}{1.848441in}}%
\pgfpathlineto{\pgfqpoint{1.086265in}{1.856136in}}%
\pgfpathlineto{\pgfqpoint{1.086329in}{1.856120in}}%
\pgfpathlineto{\pgfqpoint{1.087128in}{1.854638in}}%
\pgfpathlineto{\pgfqpoint{1.088575in}{1.845933in}}%
\pgfpathlineto{\pgfqpoint{1.090892in}{1.816114in}}%
\pgfpathlineto{\pgfqpoint{1.094238in}{1.740322in}}%
\pgfpathlineto{\pgfqpoint{1.098950in}{1.576816in}}%
\pgfpathlineto{\pgfqpoint{1.106506in}{1.224714in}}%
\pgfpathlineto{\pgfqpoint{1.118342in}{0.686972in}}%
\pgfpathlineto{\pgfqpoint{1.123448in}{0.547782in}}%
\pgfpathlineto{\pgfqpoint{1.126903in}{0.503194in}}%
\pgfpathlineto{\pgfqpoint{1.128734in}{0.497235in}}%
\pgfpathlineto{\pgfqpoint{1.128911in}{0.497318in}}%
\pgfpathlineto{\pgfqpoint{1.129787in}{0.499449in}}%
\pgfpathlineto{\pgfqpoint{1.131374in}{0.510530in}}%
\pgfpathlineto{\pgfqpoint{1.133857in}{0.546179in}}%
\pgfpathlineto{\pgfqpoint{1.137407in}{0.633543in}}%
\pgfpathlineto{\pgfqpoint{1.142428in}{0.818766in}}%
\pgfpathlineto{\pgfqpoint{1.151271in}{1.244113in}}%
\pgfpathlineto{\pgfqpoint{1.160865in}{1.668407in}}%
\pgfpathlineto{\pgfqpoint{1.165951in}{1.806297in}}%
\pgfpathlineto{\pgfqpoint{1.169396in}{1.850356in}}%
\pgfpathlineto{\pgfqpoint{1.171203in}{1.856135in}}%
\pgfpathlineto{\pgfqpoint{1.171387in}{1.856044in}}%
\pgfpathlineto{\pgfqpoint{1.172273in}{1.853838in}}%
\pgfpathlineto{\pgfqpoint{1.173873in}{1.842502in}}%
\pgfpathlineto{\pgfqpoint{1.176370in}{1.806296in}}%
\pgfpathlineto{\pgfqpoint{1.179937in}{1.717872in}}%
\pgfpathlineto{\pgfqpoint{1.184989in}{1.530531in}}%
\pgfpathlineto{\pgfqpoint{1.193995in}{1.096356in}}%
\pgfpathlineto{\pgfqpoint{1.203389in}{0.682964in}}%
\pgfpathlineto{\pgfqpoint{1.208458in}{0.546305in}}%
\pgfpathlineto{\pgfqpoint{1.211885in}{0.502860in}}%
\pgfpathlineto{\pgfqpoint{1.213669in}{0.497235in}}%
\pgfpathlineto{\pgfqpoint{1.213863in}{0.497336in}}%
\pgfpathlineto{\pgfqpoint{1.214756in}{0.499603in}}%
\pgfpathlineto{\pgfqpoint{1.216370in}{0.511177in}}%
\pgfpathlineto{\pgfqpoint{1.218884in}{0.547977in}}%
\pgfpathlineto{\pgfqpoint{1.222471in}{0.637571in}}%
\pgfpathlineto{\pgfqpoint{1.227554in}{0.827076in}}%
\pgfpathlineto{\pgfqpoint{1.236743in}{1.271071in}}%
\pgfpathlineto{\pgfqpoint{1.245926in}{1.672863in}}%
\pgfpathlineto{\pgfqpoint{1.250971in}{1.807953in}}%
\pgfpathlineto{\pgfqpoint{1.254382in}{1.850706in}}%
\pgfpathlineto{\pgfqpoint{1.256135in}{1.856135in}}%
\pgfpathlineto{\pgfqpoint{1.256342in}{1.856023in}}%
\pgfpathlineto{\pgfqpoint{1.257246in}{1.853667in}}%
\pgfpathlineto{\pgfqpoint{1.258876in}{1.841778in}}%
\pgfpathlineto{\pgfqpoint{1.261407in}{1.804286in}}%
\pgfpathlineto{\pgfqpoint{1.265019in}{1.713295in}}%
\pgfpathlineto{\pgfqpoint{1.270138in}{1.521155in}}%
\pgfpathlineto{\pgfqpoint{1.279576in}{1.064072in}}%
\pgfpathlineto{\pgfqpoint{1.288497in}{0.676906in}}%
\pgfpathlineto{\pgfqpoint{1.293508in}{0.544099in}}%
\pgfpathlineto{\pgfqpoint{1.296895in}{0.502367in}}%
\pgfpathlineto{\pgfqpoint{1.298601in}{0.497235in}}%
\pgfpathlineto{\pgfqpoint{1.298828in}{0.497368in}}%
\pgfpathlineto{\pgfqpoint{1.299749in}{0.499867in}}%
\pgfpathlineto{\pgfqpoint{1.301403in}{0.512229in}}%
\pgfpathlineto{\pgfqpoint{1.303965in}{0.550865in}}%
\pgfpathlineto{\pgfqpoint{1.307614in}{0.644080in}}%
\pgfpathlineto{\pgfqpoint{1.312794in}{0.840492in}}%
\pgfpathlineto{\pgfqpoint{1.322694in}{1.321488in}}%
\pgfpathlineto{\pgfqpoint{1.331187in}{1.684031in}}%
\pgfpathlineto{\pgfqpoint{1.336127in}{1.812035in}}%
\pgfpathlineto{\pgfqpoint{1.339460in}{1.851594in}}%
\pgfpathlineto{\pgfqpoint{1.341070in}{1.856135in}}%
\pgfpathlineto{\pgfqpoint{1.341338in}{1.855955in}}%
\pgfpathlineto{\pgfqpoint{1.342300in}{1.853115in}}%
\pgfpathlineto{\pgfqpoint{1.344009in}{1.839644in}}%
\pgfpathlineto{\pgfqpoint{1.346635in}{1.798444in}}%
\pgfpathlineto{\pgfqpoint{1.350365in}{1.700265in}}%
\pgfpathlineto{\pgfqpoint{1.355682in}{1.494317in}}%
\pgfpathlineto{\pgfqpoint{1.367477in}{0.920774in}}%
\pgfpathlineto{\pgfqpoint{1.374716in}{0.635392in}}%
\pgfpathlineto{\pgfqpoint{1.379320in}{0.529308in}}%
\pgfpathlineto{\pgfqpoint{1.382387in}{0.499490in}}%
\pgfpathlineto{\pgfqpoint{1.383556in}{0.497240in}}%
\pgfpathlineto{\pgfqpoint{1.383974in}{0.497668in}}%
\pgfpathlineto{\pgfqpoint{1.385085in}{0.501957in}}%
\pgfpathlineto{\pgfqpoint{1.386994in}{0.519938in}}%
\pgfpathlineto{\pgfqpoint{1.389855in}{0.571173in}}%
\pgfpathlineto{\pgfqpoint{1.393887in}{0.688517in}}%
\pgfpathlineto{\pgfqpoint{1.399754in}{0.932531in}}%
\pgfpathlineto{\pgfqpoint{1.418283in}{1.749590in}}%
\pgfpathlineto{\pgfqpoint{1.422536in}{1.834517in}}%
\pgfpathlineto{\pgfqpoint{1.425309in}{1.855360in}}%
\pgfpathlineto{\pgfqpoint{1.426066in}{1.856116in}}%
\pgfpathlineto{\pgfqpoint{1.426555in}{1.855470in}}%
\pgfpathlineto{\pgfqpoint{1.427768in}{1.850039in}}%
\pgfpathlineto{\pgfqpoint{1.429803in}{1.828800in}}%
\pgfpathlineto{\pgfqpoint{1.432810in}{1.770647in}}%
\pgfpathlineto{\pgfqpoint{1.437039in}{1.640241in}}%
\pgfpathlineto{\pgfqpoint{1.443314in}{1.368783in}}%
\pgfpathlineto{\pgfqpoint{1.459614in}{0.636425in}}%
\pgfpathlineto{\pgfqpoint{1.464228in}{0.529672in}}%
\pgfpathlineto{\pgfqpoint{1.467302in}{0.499560in}}%
\pgfpathlineto{\pgfqpoint{1.468488in}{0.497240in}}%
\pgfpathlineto{\pgfqpoint{1.468902in}{0.497662in}}%
\pgfpathlineto{\pgfqpoint{1.470010in}{0.501917in}}%
\pgfpathlineto{\pgfqpoint{1.471916in}{0.519807in}}%
\pgfpathlineto{\pgfqpoint{1.474769in}{0.570787in}}%
\pgfpathlineto{\pgfqpoint{1.478795in}{0.687687in}}%
\pgfpathlineto{\pgfqpoint{1.484649in}{0.930780in}}%
\pgfpathlineto{\pgfqpoint{1.503255in}{1.750690in}}%
\pgfpathlineto{\pgfqpoint{1.507499in}{1.834900in}}%
\pgfpathlineto{\pgfqpoint{1.510257in}{1.855401in}}%
\pgfpathlineto{\pgfqpoint{1.511001in}{1.856115in}}%
\pgfpathlineto{\pgfqpoint{1.511490in}{1.855462in}}%
\pgfpathlineto{\pgfqpoint{1.512703in}{1.850017in}}%
\pgfpathlineto{\pgfqpoint{1.514738in}{1.828752in}}%
\pgfpathlineto{\pgfqpoint{1.517748in}{1.770481in}}%
\pgfpathlineto{\pgfqpoint{1.521981in}{1.639866in}}%
\pgfpathlineto{\pgfqpoint{1.528263in}{1.367963in}}%
\pgfpathlineto{\pgfqpoint{1.544529in}{0.636943in}}%
\pgfpathlineto{\pgfqpoint{1.549149in}{0.529829in}}%
\pgfpathlineto{\pgfqpoint{1.552227in}{0.499588in}}%
\pgfpathlineto{\pgfqpoint{1.553416in}{0.497239in}}%
\pgfpathlineto{\pgfqpoint{1.553831in}{0.497656in}}%
\pgfpathlineto{\pgfqpoint{1.554938in}{0.501897in}}%
\pgfpathlineto{\pgfqpoint{1.556841in}{0.519720in}}%
\pgfpathlineto{\pgfqpoint{1.559694in}{0.570633in}}%
\pgfpathlineto{\pgfqpoint{1.563717in}{0.687332in}}%
\pgfpathlineto{\pgfqpoint{1.569563in}{0.929985in}}%
\pgfpathlineto{\pgfqpoint{1.588208in}{1.751238in}}%
\pgfpathlineto{\pgfqpoint{1.592444in}{1.835069in}}%
\pgfpathlineto{\pgfqpoint{1.595196in}{1.855416in}}%
\pgfpathlineto{\pgfqpoint{1.595936in}{1.856113in}}%
\pgfpathlineto{\pgfqpoint{1.596422in}{1.855462in}}%
\pgfpathlineto{\pgfqpoint{1.597635in}{1.850016in}}%
\pgfpathlineto{\pgfqpoint{1.599670in}{1.828752in}}%
\pgfpathlineto{\pgfqpoint{1.602680in}{1.770480in}}%
\pgfpathlineto{\pgfqpoint{1.606913in}{1.639865in}}%
\pgfpathlineto{\pgfqpoint{1.613194in}{1.367962in}}%
\pgfpathlineto{\pgfqpoint{1.629461in}{0.636942in}}%
\pgfpathlineto{\pgfqpoint{1.634081in}{0.529828in}}%
\pgfpathlineto{\pgfqpoint{1.637159in}{0.499588in}}%
\pgfpathlineto{\pgfqpoint{1.638348in}{0.497239in}}%
\pgfpathlineto{\pgfqpoint{1.638762in}{0.497656in}}%
\pgfpathlineto{\pgfqpoint{1.639870in}{0.501897in}}%
\pgfpathlineto{\pgfqpoint{1.641772in}{0.519720in}}%
\pgfpathlineto{\pgfqpoint{1.644626in}{0.570633in}}%
\pgfpathlineto{\pgfqpoint{1.648648in}{0.687333in}}%
\pgfpathlineto{\pgfqpoint{1.654495in}{0.929985in}}%
\pgfpathlineto{\pgfqpoint{1.673139in}{1.751238in}}%
\pgfpathlineto{\pgfqpoint{1.677376in}{1.835069in}}%
\pgfpathlineto{\pgfqpoint{1.680127in}{1.855417in}}%
\pgfpathlineto{\pgfqpoint{1.680868in}{1.856113in}}%
\pgfpathlineto{\pgfqpoint{1.681354in}{1.855462in}}%
\pgfpathlineto{\pgfqpoint{1.682567in}{1.850016in}}%
\pgfpathlineto{\pgfqpoint{1.684602in}{1.828752in}}%
\pgfpathlineto{\pgfqpoint{1.687612in}{1.770480in}}%
\pgfpathlineto{\pgfqpoint{1.691845in}{1.639865in}}%
\pgfpathlineto{\pgfqpoint{1.698126in}{1.367961in}}%
\pgfpathlineto{\pgfqpoint{1.714392in}{0.636942in}}%
\pgfpathlineto{\pgfqpoint{1.719013in}{0.529828in}}%
\pgfpathlineto{\pgfqpoint{1.722090in}{0.499588in}}%
\pgfpathlineto{\pgfqpoint{1.723279in}{0.497239in}}%
\pgfpathlineto{\pgfqpoint{1.723694in}{0.497656in}}%
\pgfpathlineto{\pgfqpoint{1.724801in}{0.501897in}}%
\pgfpathlineto{\pgfqpoint{1.726704in}{0.519720in}}%
\pgfpathlineto{\pgfqpoint{1.729558in}{0.570634in}}%
\pgfpathlineto{\pgfqpoint{1.733580in}{0.687333in}}%
\pgfpathlineto{\pgfqpoint{1.739427in}{0.929986in}}%
\pgfpathlineto{\pgfqpoint{1.758071in}{1.751239in}}%
\pgfpathlineto{\pgfqpoint{1.762307in}{1.835069in}}%
\pgfpathlineto{\pgfqpoint{1.765059in}{1.855417in}}%
\pgfpathlineto{\pgfqpoint{1.765800in}{1.856113in}}%
\pgfpathlineto{\pgfqpoint{1.766285in}{1.855462in}}%
\pgfpathlineto{\pgfqpoint{1.767498in}{1.850016in}}%
\pgfpathlineto{\pgfqpoint{1.769533in}{1.828751in}}%
\pgfpathlineto{\pgfqpoint{1.772543in}{1.770480in}}%
\pgfpathlineto{\pgfqpoint{1.776776in}{1.639864in}}%
\pgfpathlineto{\pgfqpoint{1.783058in}{1.367960in}}%
\pgfpathlineto{\pgfqpoint{1.799324in}{0.636941in}}%
\pgfpathlineto{\pgfqpoint{1.803944in}{0.529828in}}%
\pgfpathlineto{\pgfqpoint{1.807022in}{0.499588in}}%
\pgfpathlineto{\pgfqpoint{1.808211in}{0.497239in}}%
\pgfpathlineto{\pgfqpoint{1.808626in}{0.497656in}}%
\pgfpathlineto{\pgfqpoint{1.809733in}{0.501897in}}%
\pgfpathlineto{\pgfqpoint{1.811636in}{0.519720in}}%
\pgfpathlineto{\pgfqpoint{1.814489in}{0.570634in}}%
\pgfpathlineto{\pgfqpoint{1.818512in}{0.687334in}}%
\pgfpathlineto{\pgfqpoint{1.824358in}{0.929987in}}%
\pgfpathlineto{\pgfqpoint{1.843003in}{1.751239in}}%
\pgfpathlineto{\pgfqpoint{1.847239in}{1.835070in}}%
\pgfpathlineto{\pgfqpoint{1.849991in}{1.855417in}}%
\pgfpathlineto{\pgfqpoint{1.850731in}{1.856113in}}%
\pgfpathlineto{\pgfqpoint{1.851217in}{1.855462in}}%
\pgfpathlineto{\pgfqpoint{1.852433in}{1.849993in}}%
\pgfpathlineto{\pgfqpoint{1.854472in}{1.828655in}}%
\pgfpathlineto{\pgfqpoint{1.857485in}{1.770230in}}%
\pgfpathlineto{\pgfqpoint{1.861722in}{1.639364in}}%
\pgfpathlineto{\pgfqpoint{1.868010in}{1.366976in}}%
\pgfpathlineto{\pgfqpoint{1.884239in}{0.637460in}}%
\pgfpathlineto{\pgfqpoint{1.888862in}{0.530037in}}%
\pgfpathlineto{\pgfqpoint{1.891947in}{0.499616in}}%
\pgfpathlineto{\pgfqpoint{1.893143in}{0.497239in}}%
\pgfpathlineto{\pgfqpoint{1.893554in}{0.497650in}}%
\pgfpathlineto{\pgfqpoint{1.894658in}{0.501858in}}%
\pgfpathlineto{\pgfqpoint{1.896557in}{0.519590in}}%
\pgfpathlineto{\pgfqpoint{1.899404in}{0.570249in}}%
\pgfpathlineto{\pgfqpoint{1.903420in}{0.686506in}}%
\pgfpathlineto{\pgfqpoint{1.909253in}{0.928238in}}%
\pgfpathlineto{\pgfqpoint{1.927978in}{1.752421in}}%
\pgfpathlineto{\pgfqpoint{1.932201in}{1.835448in}}%
\pgfpathlineto{\pgfqpoint{1.934939in}{1.855455in}}%
\pgfpathlineto{\pgfqpoint{1.935666in}{1.856112in}}%
\pgfpathlineto{\pgfqpoint{1.936152in}{1.855455in}}%
\pgfpathlineto{\pgfqpoint{1.937368in}{1.849970in}}%
\pgfpathlineto{\pgfqpoint{1.939407in}{1.828607in}}%
\pgfpathlineto{\pgfqpoint{1.942420in}{1.770146in}}%
\pgfpathlineto{\pgfqpoint{1.946657in}{1.639238in}}%
\pgfpathlineto{\pgfqpoint{1.952948in}{1.366647in}}%
\pgfpathlineto{\pgfqpoint{1.969164in}{0.637667in}}%
\pgfpathlineto{\pgfqpoint{1.973791in}{0.530089in}}%
\pgfpathlineto{\pgfqpoint{1.976875in}{0.499631in}}%
\pgfpathlineto{\pgfqpoint{1.978075in}{0.497239in}}%
\pgfpathlineto{\pgfqpoint{1.978486in}{0.497650in}}%
\pgfpathlineto{\pgfqpoint{1.979590in}{0.501858in}}%
\pgfpathlineto{\pgfqpoint{1.981489in}{0.519590in}}%
\pgfpathlineto{\pgfqpoint{1.984336in}{0.570249in}}%
\pgfpathlineto{\pgfqpoint{1.988351in}{0.686507in}}%
\pgfpathlineto{\pgfqpoint{1.994184in}{0.928239in}}%
\pgfpathlineto{\pgfqpoint{2.012910in}{1.752422in}}%
\pgfpathlineto{\pgfqpoint{2.017133in}{1.835448in}}%
\pgfpathlineto{\pgfqpoint{2.019871in}{1.855455in}}%
\pgfpathlineto{\pgfqpoint{2.020598in}{1.856112in}}%
\pgfpathlineto{\pgfqpoint{2.021084in}{1.855455in}}%
\pgfpathlineto{\pgfqpoint{2.022300in}{1.849970in}}%
\pgfpathlineto{\pgfqpoint{2.024339in}{1.828606in}}%
\pgfpathlineto{\pgfqpoint{2.027352in}{1.770146in}}%
\pgfpathlineto{\pgfqpoint{2.031588in}{1.639237in}}%
\pgfpathlineto{\pgfqpoint{2.037880in}{1.366646in}}%
\pgfpathlineto{\pgfqpoint{2.054095in}{0.637667in}}%
\pgfpathlineto{\pgfqpoint{2.058722in}{0.530089in}}%
\pgfpathlineto{\pgfqpoint{2.061807in}{0.499631in}}%
\pgfpathlineto{\pgfqpoint{2.063006in}{0.497239in}}%
\pgfpathlineto{\pgfqpoint{2.063417in}{0.497650in}}%
\pgfpathlineto{\pgfqpoint{2.064521in}{0.501858in}}%
\pgfpathlineto{\pgfqpoint{2.066421in}{0.519590in}}%
\pgfpathlineto{\pgfqpoint{2.069267in}{0.570250in}}%
\pgfpathlineto{\pgfqpoint{2.073283in}{0.686507in}}%
\pgfpathlineto{\pgfqpoint{2.079116in}{0.928240in}}%
\pgfpathlineto{\pgfqpoint{2.097842in}{1.752422in}}%
\pgfpathlineto{\pgfqpoint{2.102065in}{1.835448in}}%
\pgfpathlineto{\pgfqpoint{2.104803in}{1.855456in}}%
\pgfpathlineto{\pgfqpoint{2.105530in}{1.856111in}}%
\pgfpathlineto{\pgfqpoint{2.106016in}{1.855455in}}%
\pgfpathlineto{\pgfqpoint{2.107232in}{1.849970in}}%
\pgfpathlineto{\pgfqpoint{2.109270in}{1.828606in}}%
\pgfpathlineto{\pgfqpoint{2.112284in}{1.770146in}}%
\pgfpathlineto{\pgfqpoint{2.116520in}{1.639237in}}%
\pgfpathlineto{\pgfqpoint{2.122812in}{1.366646in}}%
\pgfpathlineto{\pgfqpoint{2.139027in}{0.637666in}}%
\pgfpathlineto{\pgfqpoint{2.143654in}{0.530088in}}%
\pgfpathlineto{\pgfqpoint{2.146739in}{0.499631in}}%
\pgfpathlineto{\pgfqpoint{2.147938in}{0.497239in}}%
\pgfpathlineto{\pgfqpoint{2.148349in}{0.497650in}}%
\pgfpathlineto{\pgfqpoint{2.149453in}{0.501858in}}%
\pgfpathlineto{\pgfqpoint{2.151352in}{0.519591in}}%
\pgfpathlineto{\pgfqpoint{2.154199in}{0.570250in}}%
\pgfpathlineto{\pgfqpoint{2.158215in}{0.686508in}}%
\pgfpathlineto{\pgfqpoint{2.164048in}{0.928241in}}%
\pgfpathlineto{\pgfqpoint{2.182774in}{1.752423in}}%
\pgfpathlineto{\pgfqpoint{2.186996in}{1.835449in}}%
\pgfpathlineto{\pgfqpoint{2.189735in}{1.855456in}}%
\pgfpathlineto{\pgfqpoint{2.190462in}{1.856111in}}%
\pgfpathlineto{\pgfqpoint{2.190947in}{1.855454in}}%
\pgfpathlineto{\pgfqpoint{2.192164in}{1.849970in}}%
\pgfpathlineto{\pgfqpoint{2.194202in}{1.828606in}}%
\pgfpathlineto{\pgfqpoint{2.197215in}{1.770145in}}%
\pgfpathlineto{\pgfqpoint{2.201452in}{1.639236in}}%
\pgfpathlineto{\pgfqpoint{2.207743in}{1.366645in}}%
\pgfpathlineto{\pgfqpoint{2.223959in}{0.637666in}}%
\pgfpathlineto{\pgfqpoint{2.228586in}{0.530088in}}%
\pgfpathlineto{\pgfqpoint{2.231670in}{0.499630in}}%
\pgfpathlineto{\pgfqpoint{2.232870in}{0.497239in}}%
\pgfpathlineto{\pgfqpoint{2.233281in}{0.497650in}}%
\pgfpathlineto{\pgfqpoint{2.234385in}{0.501858in}}%
\pgfpathlineto{\pgfqpoint{2.236284in}{0.519591in}}%
\pgfpathlineto{\pgfqpoint{2.239131in}{0.570250in}}%
\pgfpathlineto{\pgfqpoint{2.243146in}{0.686508in}}%
\pgfpathlineto{\pgfqpoint{2.248979in}{0.928242in}}%
\pgfpathlineto{\pgfqpoint{2.267705in}{1.752423in}}%
\pgfpathlineto{\pgfqpoint{2.271928in}{1.835449in}}%
\pgfpathlineto{\pgfqpoint{2.274666in}{1.855456in}}%
\pgfpathlineto{\pgfqpoint{2.275393in}{1.856111in}}%
\pgfpathlineto{\pgfqpoint{2.275879in}{1.855454in}}%
\pgfpathlineto{\pgfqpoint{2.277095in}{1.849970in}}%
\pgfpathlineto{\pgfqpoint{2.279134in}{1.828606in}}%
\pgfpathlineto{\pgfqpoint{2.282147in}{1.770145in}}%
\pgfpathlineto{\pgfqpoint{2.286383in}{1.639235in}}%
\pgfpathlineto{\pgfqpoint{2.292675in}{1.366644in}}%
\pgfpathlineto{\pgfqpoint{2.308890in}{0.637665in}}%
\pgfpathlineto{\pgfqpoint{2.313517in}{0.530088in}}%
\pgfpathlineto{\pgfqpoint{2.316602in}{0.499630in}}%
\pgfpathlineto{\pgfqpoint{2.317801in}{0.497239in}}%
\pgfpathlineto{\pgfqpoint{2.318212in}{0.497650in}}%
\pgfpathlineto{\pgfqpoint{2.319317in}{0.501858in}}%
\pgfpathlineto{\pgfqpoint{2.321216in}{0.519591in}}%
\pgfpathlineto{\pgfqpoint{2.324062in}{0.570251in}}%
\pgfpathlineto{\pgfqpoint{2.328078in}{0.686509in}}%
\pgfpathlineto{\pgfqpoint{2.333911in}{0.928242in}}%
\pgfpathlineto{\pgfqpoint{2.352637in}{1.752424in}}%
\pgfpathlineto{\pgfqpoint{2.356860in}{1.835449in}}%
\pgfpathlineto{\pgfqpoint{2.359598in}{1.855456in}}%
\pgfpathlineto{\pgfqpoint{2.360325in}{1.856111in}}%
\pgfpathlineto{\pgfqpoint{2.360811in}{1.855454in}}%
\pgfpathlineto{\pgfqpoint{2.362027in}{1.849970in}}%
\pgfpathlineto{\pgfqpoint{2.364065in}{1.828606in}}%
\pgfpathlineto{\pgfqpoint{2.367079in}{1.770144in}}%
\pgfpathlineto{\pgfqpoint{2.371315in}{1.639235in}}%
\pgfpathlineto{\pgfqpoint{2.377607in}{1.366643in}}%
\pgfpathlineto{\pgfqpoint{2.393822in}{0.637665in}}%
\pgfpathlineto{\pgfqpoint{2.398449in}{0.530088in}}%
\pgfpathlineto{\pgfqpoint{2.401534in}{0.499630in}}%
\pgfpathlineto{\pgfqpoint{2.402733in}{0.497239in}}%
\pgfpathlineto{\pgfqpoint{2.403144in}{0.497650in}}%
\pgfpathlineto{\pgfqpoint{2.404248in}{0.501858in}}%
\pgfpathlineto{\pgfqpoint{2.406147in}{0.519591in}}%
\pgfpathlineto{\pgfqpoint{2.408994in}{0.570251in}}%
\pgfpathlineto{\pgfqpoint{2.413010in}{0.686510in}}%
\pgfpathlineto{\pgfqpoint{2.418843in}{0.928243in}}%
\pgfpathlineto{\pgfqpoint{2.437569in}{1.752424in}}%
\pgfpathlineto{\pgfqpoint{2.441791in}{1.835449in}}%
\pgfpathlineto{\pgfqpoint{2.444530in}{1.855456in}}%
\pgfpathlineto{\pgfqpoint{2.445257in}{1.856111in}}%
\pgfpathlineto{\pgfqpoint{2.445742in}{1.855454in}}%
\pgfpathlineto{\pgfqpoint{2.446959in}{1.849969in}}%
\pgfpathlineto{\pgfqpoint{2.448997in}{1.828605in}}%
\pgfpathlineto{\pgfqpoint{2.452010in}{1.770144in}}%
\pgfpathlineto{\pgfqpoint{2.456247in}{1.639234in}}%
\pgfpathlineto{\pgfqpoint{2.462539in}{1.366642in}}%
\pgfpathlineto{\pgfqpoint{2.478754in}{0.637664in}}%
\pgfpathlineto{\pgfqpoint{2.483381in}{0.530087in}}%
\pgfpathlineto{\pgfqpoint{2.486465in}{0.499630in}}%
\pgfpathlineto{\pgfqpoint{2.487665in}{0.497239in}}%
\pgfpathlineto{\pgfqpoint{2.488076in}{0.497650in}}%
\pgfpathlineto{\pgfqpoint{2.489180in}{0.501858in}}%
\pgfpathlineto{\pgfqpoint{2.491079in}{0.519591in}}%
\pgfpathlineto{\pgfqpoint{2.493926in}{0.570252in}}%
\pgfpathlineto{\pgfqpoint{2.497941in}{0.686510in}}%
\pgfpathlineto{\pgfqpoint{2.503775in}{0.928244in}}%
\pgfpathlineto{\pgfqpoint{2.522500in}{1.752425in}}%
\pgfpathlineto{\pgfqpoint{2.526723in}{1.835449in}}%
\pgfpathlineto{\pgfqpoint{2.529461in}{1.855456in}}%
\pgfpathlineto{\pgfqpoint{2.530188in}{1.856111in}}%
\pgfpathlineto{\pgfqpoint{2.530674in}{1.855454in}}%
\pgfpathlineto{\pgfqpoint{2.531890in}{1.849969in}}%
\pgfpathlineto{\pgfqpoint{2.533929in}{1.828605in}}%
\pgfpathlineto{\pgfqpoint{2.536942in}{1.770144in}}%
\pgfpathlineto{\pgfqpoint{2.541178in}{1.639234in}}%
\pgfpathlineto{\pgfqpoint{2.547470in}{1.366642in}}%
\pgfpathlineto{\pgfqpoint{2.563685in}{0.637664in}}%
\pgfpathlineto{\pgfqpoint{2.568312in}{0.530087in}}%
\pgfpathlineto{\pgfqpoint{2.571397in}{0.499630in}}%
\pgfpathlineto{\pgfqpoint{2.572596in}{0.497239in}}%
\pgfpathlineto{\pgfqpoint{2.573007in}{0.497650in}}%
\pgfpathlineto{\pgfqpoint{2.574112in}{0.501858in}}%
\pgfpathlineto{\pgfqpoint{2.576011in}{0.519592in}}%
\pgfpathlineto{\pgfqpoint{2.578858in}{0.570252in}}%
\pgfpathlineto{\pgfqpoint{2.582873in}{0.686511in}}%
\pgfpathlineto{\pgfqpoint{2.588706in}{0.928245in}}%
\pgfpathlineto{\pgfqpoint{2.607432in}{1.752425in}}%
\pgfpathlineto{\pgfqpoint{2.611655in}{1.835450in}}%
\pgfpathlineto{\pgfqpoint{2.614393in}{1.855456in}}%
\pgfpathlineto{\pgfqpoint{2.615120in}{1.856111in}}%
\pgfpathlineto{\pgfqpoint{2.615606in}{1.855454in}}%
\pgfpathlineto{\pgfqpoint{2.616822in}{1.849969in}}%
\pgfpathlineto{\pgfqpoint{2.618860in}{1.828605in}}%
\pgfpathlineto{\pgfqpoint{2.621874in}{1.770143in}}%
\pgfpathlineto{\pgfqpoint{2.626110in}{1.639233in}}%
\pgfpathlineto{\pgfqpoint{2.632402in}{1.366641in}}%
\pgfpathlineto{\pgfqpoint{2.648617in}{0.637663in}}%
\pgfpathlineto{\pgfqpoint{2.653244in}{0.530087in}}%
\pgfpathlineto{\pgfqpoint{2.656329in}{0.499630in}}%
\pgfpathlineto{\pgfqpoint{2.657528in}{0.497239in}}%
\pgfpathlineto{\pgfqpoint{2.657939in}{0.497650in}}%
\pgfpathlineto{\pgfqpoint{2.659043in}{0.501859in}}%
\pgfpathlineto{\pgfqpoint{2.660942in}{0.519592in}}%
\pgfpathlineto{\pgfqpoint{2.663789in}{0.570252in}}%
\pgfpathlineto{\pgfqpoint{2.667805in}{0.686511in}}%
\pgfpathlineto{\pgfqpoint{2.673638in}{0.928246in}}%
\pgfpathlineto{\pgfqpoint{2.692364in}{1.752425in}}%
\pgfpathlineto{\pgfqpoint{2.696586in}{1.835450in}}%
\pgfpathlineto{\pgfqpoint{2.699325in}{1.855456in}}%
\pgfpathlineto{\pgfqpoint{2.700052in}{1.856111in}}%
\pgfpathlineto{\pgfqpoint{2.700537in}{1.855454in}}%
\pgfpathlineto{\pgfqpoint{2.701754in}{1.849969in}}%
\pgfpathlineto{\pgfqpoint{2.703792in}{1.828605in}}%
\pgfpathlineto{\pgfqpoint{2.706805in}{1.770143in}}%
\pgfpathlineto{\pgfqpoint{2.711042in}{1.639232in}}%
\pgfpathlineto{\pgfqpoint{2.717334in}{1.366640in}}%
\pgfpathlineto{\pgfqpoint{2.733549in}{0.637663in}}%
\pgfpathlineto{\pgfqpoint{2.738176in}{0.530087in}}%
\pgfpathlineto{\pgfqpoint{2.741261in}{0.499630in}}%
\pgfpathlineto{\pgfqpoint{2.742460in}{0.497239in}}%
\pgfpathlineto{\pgfqpoint{2.742871in}{0.497650in}}%
\pgfpathlineto{\pgfqpoint{2.743975in}{0.501859in}}%
\pgfpathlineto{\pgfqpoint{2.745874in}{0.519592in}}%
\pgfpathlineto{\pgfqpoint{2.748721in}{0.570253in}}%
\pgfpathlineto{\pgfqpoint{2.752737in}{0.686512in}}%
\pgfpathlineto{\pgfqpoint{2.758570in}{0.928246in}}%
\pgfpathlineto{\pgfqpoint{2.777295in}{1.752426in}}%
\pgfpathlineto{\pgfqpoint{2.781518in}{1.835450in}}%
\pgfpathlineto{\pgfqpoint{2.784256in}{1.855456in}}%
\pgfpathlineto{\pgfqpoint{2.784983in}{1.856111in}}%
\pgfpathlineto{\pgfqpoint{2.785469in}{1.855454in}}%
\pgfpathlineto{\pgfqpoint{2.786685in}{1.849969in}}%
\pgfpathlineto{\pgfqpoint{2.788724in}{1.828604in}}%
\pgfpathlineto{\pgfqpoint{2.791737in}{1.770142in}}%
\pgfpathlineto{\pgfqpoint{2.795974in}{1.639232in}}%
\pgfpathlineto{\pgfqpoint{2.802265in}{1.366639in}}%
\pgfpathlineto{\pgfqpoint{2.818480in}{0.637662in}}%
\pgfpathlineto{\pgfqpoint{2.823108in}{0.530086in}}%
\pgfpathlineto{\pgfqpoint{2.826192in}{0.499630in}}%
\pgfpathlineto{\pgfqpoint{2.827391in}{0.497239in}}%
\pgfpathlineto{\pgfqpoint{2.827803in}{0.497650in}}%
\pgfpathlineto{\pgfqpoint{2.828907in}{0.501859in}}%
\pgfpathlineto{\pgfqpoint{2.830806in}{0.519592in}}%
\pgfpathlineto{\pgfqpoint{2.833653in}{0.570253in}}%
\pgfpathlineto{\pgfqpoint{2.837668in}{0.686512in}}%
\pgfpathlineto{\pgfqpoint{2.843501in}{0.928247in}}%
\pgfpathlineto{\pgfqpoint{2.862227in}{1.752426in}}%
\pgfpathlineto{\pgfqpoint{2.866450in}{1.835450in}}%
\pgfpathlineto{\pgfqpoint{2.869188in}{1.855456in}}%
\pgfpathlineto{\pgfqpoint{2.869915in}{1.856111in}}%
\pgfpathlineto{\pgfqpoint{2.870401in}{1.855454in}}%
\pgfpathlineto{\pgfqpoint{2.871617in}{1.849969in}}%
\pgfpathlineto{\pgfqpoint{2.873655in}{1.828604in}}%
\pgfpathlineto{\pgfqpoint{2.876669in}{1.770142in}}%
\pgfpathlineto{\pgfqpoint{2.880905in}{1.639231in}}%
\pgfpathlineto{\pgfqpoint{2.887197in}{1.366638in}}%
\pgfpathlineto{\pgfqpoint{2.903412in}{0.637662in}}%
\pgfpathlineto{\pgfqpoint{2.908039in}{0.530086in}}%
\pgfpathlineto{\pgfqpoint{2.911124in}{0.499630in}}%
\pgfpathlineto{\pgfqpoint{2.912323in}{0.497239in}}%
\pgfpathlineto{\pgfqpoint{2.912734in}{0.497650in}}%
\pgfpathlineto{\pgfqpoint{2.913838in}{0.501859in}}%
\pgfpathlineto{\pgfqpoint{2.915737in}{0.519593in}}%
\pgfpathlineto{\pgfqpoint{2.918584in}{0.570254in}}%
\pgfpathlineto{\pgfqpoint{2.922600in}{0.686513in}}%
\pgfpathlineto{\pgfqpoint{2.928433in}{0.928248in}}%
\pgfpathlineto{\pgfqpoint{2.947159in}{1.752427in}}%
\pgfpathlineto{\pgfqpoint{2.951382in}{1.835450in}}%
\pgfpathlineto{\pgfqpoint{2.954120in}{1.855456in}}%
\pgfpathlineto{\pgfqpoint{2.954847in}{1.856111in}}%
\pgfpathlineto{\pgfqpoint{2.955333in}{1.855454in}}%
\pgfpathlineto{\pgfqpoint{2.956549in}{1.849969in}}%
\pgfpathlineto{\pgfqpoint{2.958587in}{1.828604in}}%
\pgfpathlineto{\pgfqpoint{2.961601in}{1.770141in}}%
\pgfpathlineto{\pgfqpoint{2.965837in}{1.639230in}}%
\pgfpathlineto{\pgfqpoint{2.972129in}{1.366637in}}%
\pgfpathlineto{\pgfqpoint{2.988344in}{0.637661in}}%
\pgfpathlineto{\pgfqpoint{2.992971in}{0.530086in}}%
\pgfpathlineto{\pgfqpoint{2.996056in}{0.499630in}}%
\pgfpathlineto{\pgfqpoint{2.997255in}{0.497239in}}%
\pgfpathlineto{\pgfqpoint{2.997666in}{0.497650in}}%
\pgfpathlineto{\pgfqpoint{2.998770in}{0.501859in}}%
\pgfpathlineto{\pgfqpoint{3.000669in}{0.519593in}}%
\pgfpathlineto{\pgfqpoint{3.003516in}{0.570254in}}%
\pgfpathlineto{\pgfqpoint{3.007532in}{0.686514in}}%
\pgfpathlineto{\pgfqpoint{3.013365in}{0.928249in}}%
\pgfpathlineto{\pgfqpoint{3.032090in}{1.752427in}}%
\pgfpathlineto{\pgfqpoint{3.036313in}{1.835451in}}%
\pgfpathlineto{\pgfqpoint{3.039051in}{1.855456in}}%
\pgfpathlineto{\pgfqpoint{3.039778in}{1.856111in}}%
\pgfpathlineto{\pgfqpoint{3.040264in}{1.855454in}}%
\pgfpathlineto{\pgfqpoint{3.041480in}{1.849969in}}%
\pgfpathlineto{\pgfqpoint{3.043519in}{1.828604in}}%
\pgfpathlineto{\pgfqpoint{3.046532in}{1.770141in}}%
\pgfpathlineto{\pgfqpoint{3.050769in}{1.639230in}}%
\pgfpathlineto{\pgfqpoint{3.057060in}{1.366637in}}%
\pgfpathlineto{\pgfqpoint{3.073276in}{0.637660in}}%
\pgfpathlineto{\pgfqpoint{3.077903in}{0.530086in}}%
\pgfpathlineto{\pgfqpoint{3.080987in}{0.499630in}}%
\pgfpathlineto{\pgfqpoint{3.082187in}{0.497239in}}%
\pgfpathlineto{\pgfqpoint{3.082598in}{0.497650in}}%
\pgfpathlineto{\pgfqpoint{3.083702in}{0.501859in}}%
\pgfpathlineto{\pgfqpoint{3.085601in}{0.519593in}}%
\pgfpathlineto{\pgfqpoint{3.088448in}{0.570254in}}%
\pgfpathlineto{\pgfqpoint{3.092463in}{0.686514in}}%
\pgfpathlineto{\pgfqpoint{3.098296in}{0.928250in}}%
\pgfpathlineto{\pgfqpoint{3.117022in}{1.752428in}}%
\pgfpathlineto{\pgfqpoint{3.121245in}{1.835451in}}%
\pgfpathlineto{\pgfqpoint{3.123983in}{1.855456in}}%
\pgfpathlineto{\pgfqpoint{3.124710in}{1.856111in}}%
\pgfpathlineto{\pgfqpoint{3.125196in}{1.855454in}}%
\pgfpathlineto{\pgfqpoint{3.126412in}{1.849969in}}%
\pgfpathlineto{\pgfqpoint{3.128451in}{1.828603in}}%
\pgfpathlineto{\pgfqpoint{3.131464in}{1.770141in}}%
\pgfpathlineto{\pgfqpoint{3.135700in}{1.639229in}}%
\pgfpathlineto{\pgfqpoint{3.141992in}{1.366636in}}%
\pgfpathlineto{\pgfqpoint{3.158207in}{0.637660in}}%
\pgfpathlineto{\pgfqpoint{3.162834in}{0.530085in}}%
\pgfpathlineto{\pgfqpoint{3.165919in}{0.499630in}}%
\pgfpathlineto{\pgfqpoint{3.167118in}{0.497239in}}%
\pgfpathlineto{\pgfqpoint{3.167529in}{0.497650in}}%
\pgfpathlineto{\pgfqpoint{3.168633in}{0.501859in}}%
\pgfpathlineto{\pgfqpoint{3.170532in}{0.519593in}}%
\pgfpathlineto{\pgfqpoint{3.173379in}{0.570255in}}%
\pgfpathlineto{\pgfqpoint{3.177395in}{0.686515in}}%
\pgfpathlineto{\pgfqpoint{3.183228in}{0.928250in}}%
\pgfpathlineto{\pgfqpoint{3.188290in}{1.176793in}}%
\pgfpathlineto{\pgfqpoint{3.188290in}{1.176793in}}%
\pgfusepath{stroke}%
\end{pgfscope}%
\begin{pgfscope}%
\pgfsetrectcap%
\pgfsetmiterjoin%
\pgfsetlinewidth{0.803000pt}%
\definecolor{currentstroke}{rgb}{0.737255,0.737255,0.737255}%
\pgfsetstrokecolor{currentstroke}%
\pgfsetdash{}{0pt}%
\pgfpathmoveto{\pgfqpoint{0.470474in}{0.429287in}}%
\pgfpathlineto{\pgfqpoint{0.470474in}{1.924084in}}%
\pgfusepath{stroke}%
\end{pgfscope}%
\begin{pgfscope}%
\pgfsetrectcap%
\pgfsetmiterjoin%
\pgfsetlinewidth{0.803000pt}%
\definecolor{currentstroke}{rgb}{0.737255,0.737255,0.737255}%
\pgfsetstrokecolor{currentstroke}%
\pgfsetdash{}{0pt}%
\pgfpathmoveto{\pgfqpoint{3.188288in}{0.429287in}}%
\pgfpathlineto{\pgfqpoint{3.188288in}{1.924084in}}%
\pgfusepath{stroke}%
\end{pgfscope}%
\begin{pgfscope}%
\pgfsetrectcap%
\pgfsetmiterjoin%
\pgfsetlinewidth{0.803000pt}%
\definecolor{currentstroke}{rgb}{0.737255,0.737255,0.737255}%
\pgfsetstrokecolor{currentstroke}%
\pgfsetdash{}{0pt}%
\pgfpathmoveto{\pgfqpoint{0.470474in}{0.429287in}}%
\pgfpathlineto{\pgfqpoint{3.188288in}{0.429287in}}%
\pgfusepath{stroke}%
\end{pgfscope}%
\begin{pgfscope}%
\pgfsetrectcap%
\pgfsetmiterjoin%
\pgfsetlinewidth{0.803000pt}%
\definecolor{currentstroke}{rgb}{0.737255,0.737255,0.737255}%
\pgfsetstrokecolor{currentstroke}%
\pgfsetdash{}{0pt}%
\pgfpathmoveto{\pgfqpoint{0.470474in}{1.924084in}}%
\pgfpathlineto{\pgfqpoint{3.188288in}{1.924084in}}%
\pgfusepath{stroke}%
\end{pgfscope}%
\begin{pgfscope}%
\definecolor{textcolor}{rgb}{0.180392,0.180392,0.180392}%
\pgfsetstrokecolor{textcolor}%
\pgfsetfillcolor{textcolor}%
\pgftext[x=1.829381in,y=2.007417in,,base]{\color{textcolor}\rmfamily\fontsize{12.960000}{15.552000}\selectfont Carrier Signal}%
\end{pgfscope}%
\begin{pgfscope}%
\pgfsetbuttcap%
\pgfsetmiterjoin%
\definecolor{currentfill}{rgb}{0.933333,0.933333,0.933333}%
\pgfsetfillcolor{currentfill}%
\pgfsetlinewidth{0.000000pt}%
\definecolor{currentstroke}{rgb}{0.000000,0.000000,0.000000}%
\pgfsetstrokecolor{currentstroke}%
\pgfsetstrokeopacity{0.000000}%
\pgfsetdash{}{0pt}%
\pgfpathmoveto{\pgfqpoint{3.386899in}{0.429287in}}%
\pgfpathlineto{\pgfqpoint{6.104713in}{0.429287in}}%
\pgfpathlineto{\pgfqpoint{6.104713in}{1.924084in}}%
\pgfpathlineto{\pgfqpoint{3.386899in}{1.924084in}}%
\pgfpathlineto{\pgfqpoint{3.386899in}{0.429287in}}%
\pgfpathclose%
\pgfusepath{fill}%
\end{pgfscope}%
\begin{pgfscope}%
\pgfpathrectangle{\pgfqpoint{3.386899in}{0.429287in}}{\pgfqpoint{2.717814in}{1.494797in}}%
\pgfusepath{clip}%
\pgfsetbuttcap%
\pgfsetroundjoin%
\pgfsetlinewidth{0.501875pt}%
\definecolor{currentstroke}{rgb}{0.698039,0.698039,0.698039}%
\pgfsetstrokecolor{currentstroke}%
\pgfsetdash{{1.850000pt}{0.800000pt}}{0.000000pt}%
\pgfpathmoveto{\pgfqpoint{3.896489in}{0.429287in}}%
\pgfpathlineto{\pgfqpoint{3.896489in}{1.924084in}}%
\pgfusepath{stroke}%
\end{pgfscope}%
\begin{pgfscope}%
\pgfsetbuttcap%
\pgfsetroundjoin%
\definecolor{currentfill}{rgb}{0.180392,0.180392,0.180392}%
\pgfsetfillcolor{currentfill}%
\pgfsetlinewidth{0.803000pt}%
\definecolor{currentstroke}{rgb}{0.180392,0.180392,0.180392}%
\pgfsetstrokecolor{currentstroke}%
\pgfsetdash{}{0pt}%
\pgfsys@defobject{currentmarker}{\pgfqpoint{0.000000in}{-0.048611in}}{\pgfqpoint{0.000000in}{0.000000in}}{%
\pgfpathmoveto{\pgfqpoint{0.000000in}{0.000000in}}%
\pgfpathlineto{\pgfqpoint{0.000000in}{-0.048611in}}%
\pgfusepath{stroke,fill}%
}%
\begin{pgfscope}%
\pgfsys@transformshift{3.896489in}{0.429287in}%
\pgfsys@useobject{currentmarker}{}%
\end{pgfscope}%
\end{pgfscope}%
\begin{pgfscope}%
\definecolor{textcolor}{rgb}{0.180392,0.180392,0.180392}%
\pgfsetstrokecolor{textcolor}%
\pgfsetfillcolor{textcolor}%
\pgftext[x=3.896489in,y=0.332064in,,top]{\color{textcolor}\rmfamily\fontsize{9.000000}{10.800000}\selectfont \(\displaystyle {0}\)}%
\end{pgfscope}%
\begin{pgfscope}%
\pgfpathrectangle{\pgfqpoint{3.386899in}{0.429287in}}{\pgfqpoint{2.717814in}{1.494797in}}%
\pgfusepath{clip}%
\pgfsetbuttcap%
\pgfsetroundjoin%
\pgfsetlinewidth{0.501875pt}%
\definecolor{currentstroke}{rgb}{0.698039,0.698039,0.698039}%
\pgfsetstrokecolor{currentstroke}%
\pgfsetdash{{1.850000pt}{0.800000pt}}{0.000000pt}%
\pgfpathmoveto{\pgfqpoint{4.575942in}{0.429287in}}%
\pgfpathlineto{\pgfqpoint{4.575942in}{1.924084in}}%
\pgfusepath{stroke}%
\end{pgfscope}%
\begin{pgfscope}%
\pgfsetbuttcap%
\pgfsetroundjoin%
\definecolor{currentfill}{rgb}{0.180392,0.180392,0.180392}%
\pgfsetfillcolor{currentfill}%
\pgfsetlinewidth{0.803000pt}%
\definecolor{currentstroke}{rgb}{0.180392,0.180392,0.180392}%
\pgfsetstrokecolor{currentstroke}%
\pgfsetdash{}{0pt}%
\pgfsys@defobject{currentmarker}{\pgfqpoint{0.000000in}{-0.048611in}}{\pgfqpoint{0.000000in}{0.000000in}}{%
\pgfpathmoveto{\pgfqpoint{0.000000in}{0.000000in}}%
\pgfpathlineto{\pgfqpoint{0.000000in}{-0.048611in}}%
\pgfusepath{stroke,fill}%
}%
\begin{pgfscope}%
\pgfsys@transformshift{4.575942in}{0.429287in}%
\pgfsys@useobject{currentmarker}{}%
\end{pgfscope}%
\end{pgfscope}%
\begin{pgfscope}%
\definecolor{textcolor}{rgb}{0.180392,0.180392,0.180392}%
\pgfsetstrokecolor{textcolor}%
\pgfsetfillcolor{textcolor}%
\pgftext[x=4.575942in,y=0.332064in,,top]{\color{textcolor}\rmfamily\fontsize{9.000000}{10.800000}\selectfont \(\displaystyle {2}\)}%
\end{pgfscope}%
\begin{pgfscope}%
\pgfpathrectangle{\pgfqpoint{3.386899in}{0.429287in}}{\pgfqpoint{2.717814in}{1.494797in}}%
\pgfusepath{clip}%
\pgfsetbuttcap%
\pgfsetroundjoin%
\pgfsetlinewidth{0.501875pt}%
\definecolor{currentstroke}{rgb}{0.698039,0.698039,0.698039}%
\pgfsetstrokecolor{currentstroke}%
\pgfsetdash{{1.850000pt}{0.800000pt}}{0.000000pt}%
\pgfpathmoveto{\pgfqpoint{5.255396in}{0.429287in}}%
\pgfpathlineto{\pgfqpoint{5.255396in}{1.924084in}}%
\pgfusepath{stroke}%
\end{pgfscope}%
\begin{pgfscope}%
\pgfsetbuttcap%
\pgfsetroundjoin%
\definecolor{currentfill}{rgb}{0.180392,0.180392,0.180392}%
\pgfsetfillcolor{currentfill}%
\pgfsetlinewidth{0.803000pt}%
\definecolor{currentstroke}{rgb}{0.180392,0.180392,0.180392}%
\pgfsetstrokecolor{currentstroke}%
\pgfsetdash{}{0pt}%
\pgfsys@defobject{currentmarker}{\pgfqpoint{0.000000in}{-0.048611in}}{\pgfqpoint{0.000000in}{0.000000in}}{%
\pgfpathmoveto{\pgfqpoint{0.000000in}{0.000000in}}%
\pgfpathlineto{\pgfqpoint{0.000000in}{-0.048611in}}%
\pgfusepath{stroke,fill}%
}%
\begin{pgfscope}%
\pgfsys@transformshift{5.255396in}{0.429287in}%
\pgfsys@useobject{currentmarker}{}%
\end{pgfscope}%
\end{pgfscope}%
\begin{pgfscope}%
\definecolor{textcolor}{rgb}{0.180392,0.180392,0.180392}%
\pgfsetstrokecolor{textcolor}%
\pgfsetfillcolor{textcolor}%
\pgftext[x=5.255396in,y=0.332064in,,top]{\color{textcolor}\rmfamily\fontsize{9.000000}{10.800000}\selectfont \(\displaystyle {4}\)}%
\end{pgfscope}%
\begin{pgfscope}%
\pgfpathrectangle{\pgfqpoint{3.386899in}{0.429287in}}{\pgfqpoint{2.717814in}{1.494797in}}%
\pgfusepath{clip}%
\pgfsetbuttcap%
\pgfsetroundjoin%
\pgfsetlinewidth{0.501875pt}%
\definecolor{currentstroke}{rgb}{0.698039,0.698039,0.698039}%
\pgfsetstrokecolor{currentstroke}%
\pgfsetdash{{1.850000pt}{0.800000pt}}{0.000000pt}%
\pgfpathmoveto{\pgfqpoint{5.934849in}{0.429287in}}%
\pgfpathlineto{\pgfqpoint{5.934849in}{1.924084in}}%
\pgfusepath{stroke}%
\end{pgfscope}%
\begin{pgfscope}%
\pgfsetbuttcap%
\pgfsetroundjoin%
\definecolor{currentfill}{rgb}{0.180392,0.180392,0.180392}%
\pgfsetfillcolor{currentfill}%
\pgfsetlinewidth{0.803000pt}%
\definecolor{currentstroke}{rgb}{0.180392,0.180392,0.180392}%
\pgfsetstrokecolor{currentstroke}%
\pgfsetdash{}{0pt}%
\pgfsys@defobject{currentmarker}{\pgfqpoint{0.000000in}{-0.048611in}}{\pgfqpoint{0.000000in}{0.000000in}}{%
\pgfpathmoveto{\pgfqpoint{0.000000in}{0.000000in}}%
\pgfpathlineto{\pgfqpoint{0.000000in}{-0.048611in}}%
\pgfusepath{stroke,fill}%
}%
\begin{pgfscope}%
\pgfsys@transformshift{5.934849in}{0.429287in}%
\pgfsys@useobject{currentmarker}{}%
\end{pgfscope}%
\end{pgfscope}%
\begin{pgfscope}%
\definecolor{textcolor}{rgb}{0.180392,0.180392,0.180392}%
\pgfsetstrokecolor{textcolor}%
\pgfsetfillcolor{textcolor}%
\pgftext[x=5.934849in,y=0.332064in,,top]{\color{textcolor}\rmfamily\fontsize{9.000000}{10.800000}\selectfont \(\displaystyle {6}\)}%
\end{pgfscope}%
\begin{pgfscope}%
\definecolor{textcolor}{rgb}{0.180392,0.180392,0.180392}%
\pgfsetstrokecolor{textcolor}%
\pgfsetfillcolor{textcolor}%
\pgftext[x=4.745806in,y=0.150823in,,top]{\color{textcolor}\rmfamily\fontsize{10.800000}{12.960000}\selectfont Time (Seconds)}%
\end{pgfscope}%
\begin{pgfscope}%
\pgfpathrectangle{\pgfqpoint{3.386899in}{0.429287in}}{\pgfqpoint{2.717814in}{1.494797in}}%
\pgfusepath{clip}%
\pgfsetbuttcap%
\pgfsetroundjoin%
\pgfsetlinewidth{0.501875pt}%
\definecolor{currentstroke}{rgb}{0.698039,0.698039,0.698039}%
\pgfsetstrokecolor{currentstroke}%
\pgfsetdash{{1.850000pt}{0.800000pt}}{0.000000pt}%
\pgfpathmoveto{\pgfqpoint{3.386899in}{0.497232in}}%
\pgfpathlineto{\pgfqpoint{6.104713in}{0.497232in}}%
\pgfusepath{stroke}%
\end{pgfscope}%
\begin{pgfscope}%
\pgfsetbuttcap%
\pgfsetroundjoin%
\definecolor{currentfill}{rgb}{0.180392,0.180392,0.180392}%
\pgfsetfillcolor{currentfill}%
\pgfsetlinewidth{0.803000pt}%
\definecolor{currentstroke}{rgb}{0.180392,0.180392,0.180392}%
\pgfsetstrokecolor{currentstroke}%
\pgfsetdash{}{0pt}%
\pgfsys@defobject{currentmarker}{\pgfqpoint{-0.048611in}{0.000000in}}{\pgfqpoint{-0.000000in}{0.000000in}}{%
\pgfpathmoveto{\pgfqpoint{-0.000000in}{0.000000in}}%
\pgfpathlineto{\pgfqpoint{-0.048611in}{0.000000in}}%
\pgfusepath{stroke,fill}%
}%
\begin{pgfscope}%
\pgfsys@transformshift{3.386899in}{0.497232in}%
\pgfsys@useobject{currentmarker}{}%
\end{pgfscope}%
\end{pgfscope}%
\begin{pgfscope}%
\pgfpathrectangle{\pgfqpoint{3.386899in}{0.429287in}}{\pgfqpoint{2.717814in}{1.494797in}}%
\pgfusepath{clip}%
\pgfsetbuttcap%
\pgfsetroundjoin%
\pgfsetlinewidth{0.501875pt}%
\definecolor{currentstroke}{rgb}{0.698039,0.698039,0.698039}%
\pgfsetstrokecolor{currentstroke}%
\pgfsetdash{{1.850000pt}{0.800000pt}}{0.000000pt}%
\pgfpathmoveto{\pgfqpoint{3.386899in}{0.836959in}}%
\pgfpathlineto{\pgfqpoint{6.104713in}{0.836959in}}%
\pgfusepath{stroke}%
\end{pgfscope}%
\begin{pgfscope}%
\pgfsetbuttcap%
\pgfsetroundjoin%
\definecolor{currentfill}{rgb}{0.180392,0.180392,0.180392}%
\pgfsetfillcolor{currentfill}%
\pgfsetlinewidth{0.803000pt}%
\definecolor{currentstroke}{rgb}{0.180392,0.180392,0.180392}%
\pgfsetstrokecolor{currentstroke}%
\pgfsetdash{}{0pt}%
\pgfsys@defobject{currentmarker}{\pgfqpoint{-0.048611in}{0.000000in}}{\pgfqpoint{-0.000000in}{0.000000in}}{%
\pgfpathmoveto{\pgfqpoint{-0.000000in}{0.000000in}}%
\pgfpathlineto{\pgfqpoint{-0.048611in}{0.000000in}}%
\pgfusepath{stroke,fill}%
}%
\begin{pgfscope}%
\pgfsys@transformshift{3.386899in}{0.836959in}%
\pgfsys@useobject{currentmarker}{}%
\end{pgfscope}%
\end{pgfscope}%
\begin{pgfscope}%
\pgfpathrectangle{\pgfqpoint{3.386899in}{0.429287in}}{\pgfqpoint{2.717814in}{1.494797in}}%
\pgfusepath{clip}%
\pgfsetbuttcap%
\pgfsetroundjoin%
\pgfsetlinewidth{0.501875pt}%
\definecolor{currentstroke}{rgb}{0.698039,0.698039,0.698039}%
\pgfsetstrokecolor{currentstroke}%
\pgfsetdash{{1.850000pt}{0.800000pt}}{0.000000pt}%
\pgfpathmoveto{\pgfqpoint{3.386899in}{1.176685in}}%
\pgfpathlineto{\pgfqpoint{6.104713in}{1.176685in}}%
\pgfusepath{stroke}%
\end{pgfscope}%
\begin{pgfscope}%
\pgfsetbuttcap%
\pgfsetroundjoin%
\definecolor{currentfill}{rgb}{0.180392,0.180392,0.180392}%
\pgfsetfillcolor{currentfill}%
\pgfsetlinewidth{0.803000pt}%
\definecolor{currentstroke}{rgb}{0.180392,0.180392,0.180392}%
\pgfsetstrokecolor{currentstroke}%
\pgfsetdash{}{0pt}%
\pgfsys@defobject{currentmarker}{\pgfqpoint{-0.048611in}{0.000000in}}{\pgfqpoint{-0.000000in}{0.000000in}}{%
\pgfpathmoveto{\pgfqpoint{-0.000000in}{0.000000in}}%
\pgfpathlineto{\pgfqpoint{-0.048611in}{0.000000in}}%
\pgfusepath{stroke,fill}%
}%
\begin{pgfscope}%
\pgfsys@transformshift{3.386899in}{1.176685in}%
\pgfsys@useobject{currentmarker}{}%
\end{pgfscope}%
\end{pgfscope}%
\begin{pgfscope}%
\pgfpathrectangle{\pgfqpoint{3.386899in}{0.429287in}}{\pgfqpoint{2.717814in}{1.494797in}}%
\pgfusepath{clip}%
\pgfsetbuttcap%
\pgfsetroundjoin%
\pgfsetlinewidth{0.501875pt}%
\definecolor{currentstroke}{rgb}{0.698039,0.698039,0.698039}%
\pgfsetstrokecolor{currentstroke}%
\pgfsetdash{{1.850000pt}{0.800000pt}}{0.000000pt}%
\pgfpathmoveto{\pgfqpoint{3.386899in}{1.516412in}}%
\pgfpathlineto{\pgfqpoint{6.104713in}{1.516412in}}%
\pgfusepath{stroke}%
\end{pgfscope}%
\begin{pgfscope}%
\pgfsetbuttcap%
\pgfsetroundjoin%
\definecolor{currentfill}{rgb}{0.180392,0.180392,0.180392}%
\pgfsetfillcolor{currentfill}%
\pgfsetlinewidth{0.803000pt}%
\definecolor{currentstroke}{rgb}{0.180392,0.180392,0.180392}%
\pgfsetstrokecolor{currentstroke}%
\pgfsetdash{}{0pt}%
\pgfsys@defobject{currentmarker}{\pgfqpoint{-0.048611in}{0.000000in}}{\pgfqpoint{-0.000000in}{0.000000in}}{%
\pgfpathmoveto{\pgfqpoint{-0.000000in}{0.000000in}}%
\pgfpathlineto{\pgfqpoint{-0.048611in}{0.000000in}}%
\pgfusepath{stroke,fill}%
}%
\begin{pgfscope}%
\pgfsys@transformshift{3.386899in}{1.516412in}%
\pgfsys@useobject{currentmarker}{}%
\end{pgfscope}%
\end{pgfscope}%
\begin{pgfscope}%
\pgfpathrectangle{\pgfqpoint{3.386899in}{0.429287in}}{\pgfqpoint{2.717814in}{1.494797in}}%
\pgfusepath{clip}%
\pgfsetbuttcap%
\pgfsetroundjoin%
\pgfsetlinewidth{0.501875pt}%
\definecolor{currentstroke}{rgb}{0.698039,0.698039,0.698039}%
\pgfsetstrokecolor{currentstroke}%
\pgfsetdash{{1.850000pt}{0.800000pt}}{0.000000pt}%
\pgfpathmoveto{\pgfqpoint{3.386899in}{1.856139in}}%
\pgfpathlineto{\pgfqpoint{6.104713in}{1.856139in}}%
\pgfusepath{stroke}%
\end{pgfscope}%
\begin{pgfscope}%
\pgfsetbuttcap%
\pgfsetroundjoin%
\definecolor{currentfill}{rgb}{0.180392,0.180392,0.180392}%
\pgfsetfillcolor{currentfill}%
\pgfsetlinewidth{0.803000pt}%
\definecolor{currentstroke}{rgb}{0.180392,0.180392,0.180392}%
\pgfsetstrokecolor{currentstroke}%
\pgfsetdash{}{0pt}%
\pgfsys@defobject{currentmarker}{\pgfqpoint{-0.048611in}{0.000000in}}{\pgfqpoint{-0.000000in}{0.000000in}}{%
\pgfpathmoveto{\pgfqpoint{-0.000000in}{0.000000in}}%
\pgfpathlineto{\pgfqpoint{-0.048611in}{0.000000in}}%
\pgfusepath{stroke,fill}%
}%
\begin{pgfscope}%
\pgfsys@transformshift{3.386899in}{1.856139in}%
\pgfsys@useobject{currentmarker}{}%
\end{pgfscope}%
\end{pgfscope}%
\begin{pgfscope}%
\pgfpathrectangle{\pgfqpoint{3.386899in}{0.429287in}}{\pgfqpoint{2.717814in}{1.494797in}}%
\pgfusepath{clip}%
\pgfsetrectcap%
\pgfsetroundjoin%
\pgfsetlinewidth{2.007500pt}%
\definecolor{currentstroke}{rgb}{0.203922,0.541176,0.741176}%
\pgfsetstrokecolor{currentstroke}%
\pgfsetdash{}{0pt}%
\pgfpathmoveto{\pgfqpoint{3.386897in}{1.176694in}}%
\pgfpathlineto{\pgfqpoint{3.788369in}{1.177853in}}%
\pgfpathlineto{\pgfqpoint{3.903340in}{1.180374in}}%
\pgfpathlineto{\pgfqpoint{3.981144in}{1.184215in}}%
\pgfpathlineto{\pgfqpoint{4.042169in}{1.189387in}}%
\pgfpathlineto{\pgfqpoint{4.093522in}{1.195921in}}%
\pgfpathlineto{\pgfqpoint{4.138614in}{1.203862in}}%
\pgfpathlineto{\pgfqpoint{4.179385in}{1.213271in}}%
\pgfpathlineto{\pgfqpoint{4.217077in}{1.224231in}}%
\pgfpathlineto{\pgfqpoint{4.252552in}{1.236849in}}%
\pgfpathlineto{\pgfqpoint{4.286453in}{1.251262in}}%
\pgfpathlineto{\pgfqpoint{4.319294in}{1.267640in}}%
\pgfpathlineto{\pgfqpoint{4.351517in}{1.286203in}}%
\pgfpathlineto{\pgfqpoint{4.383523in}{1.307221in}}%
\pgfpathlineto{\pgfqpoint{4.415716in}{1.331043in}}%
\pgfpathlineto{\pgfqpoint{4.448547in}{1.358136in}}%
\pgfpathlineto{\pgfqpoint{4.482577in}{1.389143in}}%
\pgfpathlineto{\pgfqpoint{4.518629in}{1.425057in}}%
\pgfpathlineto{\pgfqpoint{4.558146in}{1.467645in}}%
\pgfpathlineto{\pgfqpoint{4.604590in}{1.521115in}}%
\pgfpathlineto{\pgfqpoint{4.679452in}{1.611308in}}%
\pgfpathlineto{\pgfqpoint{4.742461in}{1.685766in}}%
\pgfpathlineto{\pgfqpoint{4.781788in}{1.728861in}}%
\pgfpathlineto{\pgfqpoint{4.814263in}{1.761249in}}%
\pgfpathlineto{\pgfqpoint{4.842820in}{1.786673in}}%
\pgfpathlineto{\pgfqpoint{4.868717in}{1.806822in}}%
\pgfpathlineto{\pgfqpoint{4.892668in}{1.822699in}}%
\pgfpathlineto{\pgfqpoint{4.915141in}{1.834977in}}%
\pgfpathlineto{\pgfqpoint{4.936486in}{1.844146in}}%
\pgfpathlineto{\pgfqpoint{4.956989in}{1.850563in}}%
\pgfpathlineto{\pgfqpoint{4.976903in}{1.854488in}}%
\pgfpathlineto{\pgfqpoint{4.996465in}{1.856088in}}%
\pgfpathlineto{\pgfqpoint{5.015901in}{1.855450in}}%
\pgfpathlineto{\pgfqpoint{5.035431in}{1.852577in}}%
\pgfpathlineto{\pgfqpoint{5.055285in}{1.847393in}}%
\pgfpathlineto{\pgfqpoint{5.075689in}{1.839743in}}%
\pgfpathlineto{\pgfqpoint{5.096895in}{1.829385in}}%
\pgfpathlineto{\pgfqpoint{5.119184in}{1.815982in}}%
\pgfpathlineto{\pgfqpoint{5.142894in}{1.799079in}}%
\pgfpathlineto{\pgfqpoint{5.168472in}{1.778051in}}%
\pgfpathlineto{\pgfqpoint{5.196574in}{1.751994in}}%
\pgfpathlineto{\pgfqpoint{5.228315in}{1.719434in}}%
\pgfpathlineto{\pgfqpoint{5.266018in}{1.677432in}}%
\pgfpathlineto{\pgfqpoint{5.317340in}{1.616636in}}%
\pgfpathlineto{\pgfqpoint{5.430017in}{1.482333in}}%
\pgfpathlineto{\pgfqpoint{5.473631in}{1.434417in}}%
\pgfpathlineto{\pgfqpoint{5.511888in}{1.395622in}}%
\pgfpathlineto{\pgfqpoint{5.547339in}{1.362787in}}%
\pgfpathlineto{\pgfqpoint{5.581148in}{1.334461in}}%
\pgfpathlineto{\pgfqpoint{5.614024in}{1.309789in}}%
\pgfpathlineto{\pgfqpoint{5.646478in}{1.288191in}}%
\pgfpathlineto{\pgfqpoint{5.678939in}{1.269245in}}%
\pgfpathlineto{\pgfqpoint{5.711807in}{1.252629in}}%
\pgfpathlineto{\pgfqpoint{5.745505in}{1.238086in}}%
\pgfpathlineto{\pgfqpoint{5.780496in}{1.225414in}}%
\pgfpathlineto{\pgfqpoint{5.817343in}{1.214451in}}%
\pgfpathlineto{\pgfqpoint{5.856762in}{1.205066in}}%
\pgfpathlineto{\pgfqpoint{5.899727in}{1.197156in}}%
\pgfpathlineto{\pgfqpoint{5.947652in}{1.190638in}}%
\pgfpathlineto{\pgfqpoint{6.002776in}{1.185445in}}%
\pgfpathlineto{\pgfqpoint{6.069020in}{1.181521in}}%
\pgfpathlineto{\pgfqpoint{6.104715in}{1.180141in}}%
\pgfpathlineto{\pgfqpoint{6.104715in}{1.180141in}}%
\pgfusepath{stroke}%
\end{pgfscope}%
\begin{pgfscope}%
\pgfsetrectcap%
\pgfsetmiterjoin%
\pgfsetlinewidth{0.803000pt}%
\definecolor{currentstroke}{rgb}{0.737255,0.737255,0.737255}%
\pgfsetstrokecolor{currentstroke}%
\pgfsetdash{}{0pt}%
\pgfpathmoveto{\pgfqpoint{3.386899in}{0.429287in}}%
\pgfpathlineto{\pgfqpoint{3.386899in}{1.924084in}}%
\pgfusepath{stroke}%
\end{pgfscope}%
\begin{pgfscope}%
\pgfsetrectcap%
\pgfsetmiterjoin%
\pgfsetlinewidth{0.803000pt}%
\definecolor{currentstroke}{rgb}{0.737255,0.737255,0.737255}%
\pgfsetstrokecolor{currentstroke}%
\pgfsetdash{}{0pt}%
\pgfpathmoveto{\pgfqpoint{6.104713in}{0.429287in}}%
\pgfpathlineto{\pgfqpoint{6.104713in}{1.924084in}}%
\pgfusepath{stroke}%
\end{pgfscope}%
\begin{pgfscope}%
\pgfsetrectcap%
\pgfsetmiterjoin%
\pgfsetlinewidth{0.803000pt}%
\definecolor{currentstroke}{rgb}{0.737255,0.737255,0.737255}%
\pgfsetstrokecolor{currentstroke}%
\pgfsetdash{}{0pt}%
\pgfpathmoveto{\pgfqpoint{3.386899in}{0.429287in}}%
\pgfpathlineto{\pgfqpoint{6.104713in}{0.429287in}}%
\pgfusepath{stroke}%
\end{pgfscope}%
\begin{pgfscope}%
\pgfsetrectcap%
\pgfsetmiterjoin%
\pgfsetlinewidth{0.803000pt}%
\definecolor{currentstroke}{rgb}{0.737255,0.737255,0.737255}%
\pgfsetstrokecolor{currentstroke}%
\pgfsetdash{}{0pt}%
\pgfpathmoveto{\pgfqpoint{3.386899in}{1.924084in}}%
\pgfpathlineto{\pgfqpoint{6.104713in}{1.924084in}}%
\pgfusepath{stroke}%
\end{pgfscope}%
\begin{pgfscope}%
\definecolor{textcolor}{rgb}{0.180392,0.180392,0.180392}%
\pgfsetstrokecolor{textcolor}%
\pgfsetfillcolor{textcolor}%
\pgftext[x=4.745806in,y=2.007417in,,base]{\color{textcolor}\rmfamily\fontsize{12.960000}{15.552000}\selectfont |Envelope Signal|}%
\end{pgfscope}%
\end{pgfpicture}%
\makeatother%
\endgroup%
}
		\caption{\centering{An example graph demonstrating the de-constructed elements of a Gaussian pulse, the carrier signal is on the left and the positive portion of Gaussian on the right}}
		\label{fig:gaussian_decon}
	\end{figure}	
	
	\vspace{5mm}
	
	\begin{figure}[h]
		\centering
		\scalebox{0.85}{%% Creator: Matplotlib, PGF backend
%%
%% To include the figure in your LaTeX document, write
%%   \input{<filename>.pgf}
%%
%% Make sure the required packages are loaded in your preamble
%%   \usepackage{pgf}
%%
%% Also ensure that all the required font packages are loaded; for instance,
%% the lmodern package is sometimes necessary when using math font.
%%   \usepackage{lmodern}
%%
%% Figures using additional raster images can only be included by \input if
%% they are in the same directory as the main LaTeX file. For loading figures
%% from other directories you can use the `import` package
%%   \usepackage{import}
%%
%% and then include the figures with
%%   \import{<path to file>}{<filename>.pgf}
%%
%% Matplotlib used the following preamble
%%   \usepackage[T1]{fontenc} \usepackage{mathpazo}
%%
\begingroup%
\makeatletter%
\begin{pgfpicture}%
\pgfpathrectangle{\pgfpointorigin}{\pgfqpoint{6.098914in}{4.522069in}}%
\pgfusepath{use as bounding box, clip}%
\begin{pgfscope}%
\pgfsetbuttcap%
\pgfsetmiterjoin%
\definecolor{currentfill}{rgb}{1.000000,1.000000,1.000000}%
\pgfsetfillcolor{currentfill}%
\pgfsetlinewidth{0.000000pt}%
\definecolor{currentstroke}{rgb}{1.000000,1.000000,1.000000}%
\pgfsetstrokecolor{currentstroke}%
\pgfsetdash{}{0pt}%
\pgfpathmoveto{\pgfqpoint{0.000000in}{0.000000in}}%
\pgfpathlineto{\pgfqpoint{6.098914in}{0.000000in}}%
\pgfpathlineto{\pgfqpoint{6.098914in}{4.522069in}}%
\pgfpathlineto{\pgfqpoint{0.000000in}{4.522069in}}%
\pgfpathlineto{\pgfqpoint{0.000000in}{0.000000in}}%
\pgfpathclose%
\pgfusepath{fill}%
\end{pgfscope}%
\begin{pgfscope}%
\pgfsetbuttcap%
\pgfsetmiterjoin%
\definecolor{currentfill}{rgb}{0.933333,0.933333,0.933333}%
\pgfsetfillcolor{currentfill}%
\pgfsetlinewidth{0.000000pt}%
\definecolor{currentstroke}{rgb}{0.000000,0.000000,0.000000}%
\pgfsetstrokecolor{currentstroke}%
\pgfsetstrokeopacity{0.000000}%
\pgfsetdash{}{0pt}%
\pgfpathmoveto{\pgfqpoint{0.564224in}{0.429287in}}%
\pgfpathlineto{\pgfqpoint{6.098914in}{0.429287in}}%
\pgfpathlineto{\pgfqpoint{6.098914in}{4.308507in}}%
\pgfpathlineto{\pgfqpoint{0.564224in}{4.308507in}}%
\pgfpathlineto{\pgfqpoint{0.564224in}{0.429287in}}%
\pgfpathclose%
\pgfusepath{fill}%
\end{pgfscope}%
\begin{pgfscope}%
\pgfpathrectangle{\pgfqpoint{0.564224in}{0.429287in}}{\pgfqpoint{5.534689in}{3.879221in}}%
\pgfusepath{clip}%
\pgfsetbuttcap%
\pgfsetroundjoin%
\pgfsetlinewidth{0.501875pt}%
\definecolor{currentstroke}{rgb}{0.698039,0.698039,0.698039}%
\pgfsetstrokecolor{currentstroke}%
\pgfsetdash{{1.850000pt}{0.800000pt}}{0.000000pt}%
\pgfpathmoveto{\pgfqpoint{0.910142in}{0.429287in}}%
\pgfpathlineto{\pgfqpoint{0.910142in}{4.308507in}}%
\pgfusepath{stroke}%
\end{pgfscope}%
\begin{pgfscope}%
\pgfsetbuttcap%
\pgfsetroundjoin%
\definecolor{currentfill}{rgb}{0.180392,0.180392,0.180392}%
\pgfsetfillcolor{currentfill}%
\pgfsetlinewidth{0.803000pt}%
\definecolor{currentstroke}{rgb}{0.180392,0.180392,0.180392}%
\pgfsetstrokecolor{currentstroke}%
\pgfsetdash{}{0pt}%
\pgfsys@defobject{currentmarker}{\pgfqpoint{0.000000in}{-0.048611in}}{\pgfqpoint{0.000000in}{0.000000in}}{%
\pgfpathmoveto{\pgfqpoint{0.000000in}{0.000000in}}%
\pgfpathlineto{\pgfqpoint{0.000000in}{-0.048611in}}%
\pgfusepath{stroke,fill}%
}%
\begin{pgfscope}%
\pgfsys@transformshift{0.910142in}{0.429287in}%
\pgfsys@useobject{currentmarker}{}%
\end{pgfscope}%
\end{pgfscope}%
\begin{pgfscope}%
\definecolor{textcolor}{rgb}{0.180392,0.180392,0.180392}%
\pgfsetstrokecolor{textcolor}%
\pgfsetfillcolor{textcolor}%
\pgftext[x=0.910142in,y=0.332064in,,top]{\color{textcolor}\rmfamily\fontsize{9.000000}{10.800000}\selectfont \(\displaystyle {\ensuremath{-}1}\)}%
\end{pgfscope}%
\begin{pgfscope}%
\pgfpathrectangle{\pgfqpoint{0.564224in}{0.429287in}}{\pgfqpoint{5.534689in}{3.879221in}}%
\pgfusepath{clip}%
\pgfsetbuttcap%
\pgfsetroundjoin%
\pgfsetlinewidth{0.501875pt}%
\definecolor{currentstroke}{rgb}{0.698039,0.698039,0.698039}%
\pgfsetstrokecolor{currentstroke}%
\pgfsetdash{{1.850000pt}{0.800000pt}}{0.000000pt}%
\pgfpathmoveto{\pgfqpoint{1.601979in}{0.429287in}}%
\pgfpathlineto{\pgfqpoint{1.601979in}{4.308507in}}%
\pgfusepath{stroke}%
\end{pgfscope}%
\begin{pgfscope}%
\pgfsetbuttcap%
\pgfsetroundjoin%
\definecolor{currentfill}{rgb}{0.180392,0.180392,0.180392}%
\pgfsetfillcolor{currentfill}%
\pgfsetlinewidth{0.803000pt}%
\definecolor{currentstroke}{rgb}{0.180392,0.180392,0.180392}%
\pgfsetstrokecolor{currentstroke}%
\pgfsetdash{}{0pt}%
\pgfsys@defobject{currentmarker}{\pgfqpoint{0.000000in}{-0.048611in}}{\pgfqpoint{0.000000in}{0.000000in}}{%
\pgfpathmoveto{\pgfqpoint{0.000000in}{0.000000in}}%
\pgfpathlineto{\pgfqpoint{0.000000in}{-0.048611in}}%
\pgfusepath{stroke,fill}%
}%
\begin{pgfscope}%
\pgfsys@transformshift{1.601979in}{0.429287in}%
\pgfsys@useobject{currentmarker}{}%
\end{pgfscope}%
\end{pgfscope}%
\begin{pgfscope}%
\definecolor{textcolor}{rgb}{0.180392,0.180392,0.180392}%
\pgfsetstrokecolor{textcolor}%
\pgfsetfillcolor{textcolor}%
\pgftext[x=1.601979in,y=0.332064in,,top]{\color{textcolor}\rmfamily\fontsize{9.000000}{10.800000}\selectfont \(\displaystyle {0}\)}%
\end{pgfscope}%
\begin{pgfscope}%
\pgfpathrectangle{\pgfqpoint{0.564224in}{0.429287in}}{\pgfqpoint{5.534689in}{3.879221in}}%
\pgfusepath{clip}%
\pgfsetbuttcap%
\pgfsetroundjoin%
\pgfsetlinewidth{0.501875pt}%
\definecolor{currentstroke}{rgb}{0.698039,0.698039,0.698039}%
\pgfsetstrokecolor{currentstroke}%
\pgfsetdash{{1.850000pt}{0.800000pt}}{0.000000pt}%
\pgfpathmoveto{\pgfqpoint{2.293815in}{0.429287in}}%
\pgfpathlineto{\pgfqpoint{2.293815in}{4.308507in}}%
\pgfusepath{stroke}%
\end{pgfscope}%
\begin{pgfscope}%
\pgfsetbuttcap%
\pgfsetroundjoin%
\definecolor{currentfill}{rgb}{0.180392,0.180392,0.180392}%
\pgfsetfillcolor{currentfill}%
\pgfsetlinewidth{0.803000pt}%
\definecolor{currentstroke}{rgb}{0.180392,0.180392,0.180392}%
\pgfsetstrokecolor{currentstroke}%
\pgfsetdash{}{0pt}%
\pgfsys@defobject{currentmarker}{\pgfqpoint{0.000000in}{-0.048611in}}{\pgfqpoint{0.000000in}{0.000000in}}{%
\pgfpathmoveto{\pgfqpoint{0.000000in}{0.000000in}}%
\pgfpathlineto{\pgfqpoint{0.000000in}{-0.048611in}}%
\pgfusepath{stroke,fill}%
}%
\begin{pgfscope}%
\pgfsys@transformshift{2.293815in}{0.429287in}%
\pgfsys@useobject{currentmarker}{}%
\end{pgfscope}%
\end{pgfscope}%
\begin{pgfscope}%
\definecolor{textcolor}{rgb}{0.180392,0.180392,0.180392}%
\pgfsetstrokecolor{textcolor}%
\pgfsetfillcolor{textcolor}%
\pgftext[x=2.293815in,y=0.332064in,,top]{\color{textcolor}\rmfamily\fontsize{9.000000}{10.800000}\selectfont \(\displaystyle {1}\)}%
\end{pgfscope}%
\begin{pgfscope}%
\pgfpathrectangle{\pgfqpoint{0.564224in}{0.429287in}}{\pgfqpoint{5.534689in}{3.879221in}}%
\pgfusepath{clip}%
\pgfsetbuttcap%
\pgfsetroundjoin%
\pgfsetlinewidth{0.501875pt}%
\definecolor{currentstroke}{rgb}{0.698039,0.698039,0.698039}%
\pgfsetstrokecolor{currentstroke}%
\pgfsetdash{{1.850000pt}{0.800000pt}}{0.000000pt}%
\pgfpathmoveto{\pgfqpoint{2.985651in}{0.429287in}}%
\pgfpathlineto{\pgfqpoint{2.985651in}{4.308507in}}%
\pgfusepath{stroke}%
\end{pgfscope}%
\begin{pgfscope}%
\pgfsetbuttcap%
\pgfsetroundjoin%
\definecolor{currentfill}{rgb}{0.180392,0.180392,0.180392}%
\pgfsetfillcolor{currentfill}%
\pgfsetlinewidth{0.803000pt}%
\definecolor{currentstroke}{rgb}{0.180392,0.180392,0.180392}%
\pgfsetstrokecolor{currentstroke}%
\pgfsetdash{}{0pt}%
\pgfsys@defobject{currentmarker}{\pgfqpoint{0.000000in}{-0.048611in}}{\pgfqpoint{0.000000in}{0.000000in}}{%
\pgfpathmoveto{\pgfqpoint{0.000000in}{0.000000in}}%
\pgfpathlineto{\pgfqpoint{0.000000in}{-0.048611in}}%
\pgfusepath{stroke,fill}%
}%
\begin{pgfscope}%
\pgfsys@transformshift{2.985651in}{0.429287in}%
\pgfsys@useobject{currentmarker}{}%
\end{pgfscope}%
\end{pgfscope}%
\begin{pgfscope}%
\definecolor{textcolor}{rgb}{0.180392,0.180392,0.180392}%
\pgfsetstrokecolor{textcolor}%
\pgfsetfillcolor{textcolor}%
\pgftext[x=2.985651in,y=0.332064in,,top]{\color{textcolor}\rmfamily\fontsize{9.000000}{10.800000}\selectfont \(\displaystyle {2}\)}%
\end{pgfscope}%
\begin{pgfscope}%
\pgfpathrectangle{\pgfqpoint{0.564224in}{0.429287in}}{\pgfqpoint{5.534689in}{3.879221in}}%
\pgfusepath{clip}%
\pgfsetbuttcap%
\pgfsetroundjoin%
\pgfsetlinewidth{0.501875pt}%
\definecolor{currentstroke}{rgb}{0.698039,0.698039,0.698039}%
\pgfsetstrokecolor{currentstroke}%
\pgfsetdash{{1.850000pt}{0.800000pt}}{0.000000pt}%
\pgfpathmoveto{\pgfqpoint{3.677487in}{0.429287in}}%
\pgfpathlineto{\pgfqpoint{3.677487in}{4.308507in}}%
\pgfusepath{stroke}%
\end{pgfscope}%
\begin{pgfscope}%
\pgfsetbuttcap%
\pgfsetroundjoin%
\definecolor{currentfill}{rgb}{0.180392,0.180392,0.180392}%
\pgfsetfillcolor{currentfill}%
\pgfsetlinewidth{0.803000pt}%
\definecolor{currentstroke}{rgb}{0.180392,0.180392,0.180392}%
\pgfsetstrokecolor{currentstroke}%
\pgfsetdash{}{0pt}%
\pgfsys@defobject{currentmarker}{\pgfqpoint{0.000000in}{-0.048611in}}{\pgfqpoint{0.000000in}{0.000000in}}{%
\pgfpathmoveto{\pgfqpoint{0.000000in}{0.000000in}}%
\pgfpathlineto{\pgfqpoint{0.000000in}{-0.048611in}}%
\pgfusepath{stroke,fill}%
}%
\begin{pgfscope}%
\pgfsys@transformshift{3.677487in}{0.429287in}%
\pgfsys@useobject{currentmarker}{}%
\end{pgfscope}%
\end{pgfscope}%
\begin{pgfscope}%
\definecolor{textcolor}{rgb}{0.180392,0.180392,0.180392}%
\pgfsetstrokecolor{textcolor}%
\pgfsetfillcolor{textcolor}%
\pgftext[x=3.677487in,y=0.332064in,,top]{\color{textcolor}\rmfamily\fontsize{9.000000}{10.800000}\selectfont \(\displaystyle {3}\)}%
\end{pgfscope}%
\begin{pgfscope}%
\pgfpathrectangle{\pgfqpoint{0.564224in}{0.429287in}}{\pgfqpoint{5.534689in}{3.879221in}}%
\pgfusepath{clip}%
\pgfsetbuttcap%
\pgfsetroundjoin%
\pgfsetlinewidth{0.501875pt}%
\definecolor{currentstroke}{rgb}{0.698039,0.698039,0.698039}%
\pgfsetstrokecolor{currentstroke}%
\pgfsetdash{{1.850000pt}{0.800000pt}}{0.000000pt}%
\pgfpathmoveto{\pgfqpoint{4.369323in}{0.429287in}}%
\pgfpathlineto{\pgfqpoint{4.369323in}{4.308507in}}%
\pgfusepath{stroke}%
\end{pgfscope}%
\begin{pgfscope}%
\pgfsetbuttcap%
\pgfsetroundjoin%
\definecolor{currentfill}{rgb}{0.180392,0.180392,0.180392}%
\pgfsetfillcolor{currentfill}%
\pgfsetlinewidth{0.803000pt}%
\definecolor{currentstroke}{rgb}{0.180392,0.180392,0.180392}%
\pgfsetstrokecolor{currentstroke}%
\pgfsetdash{}{0pt}%
\pgfsys@defobject{currentmarker}{\pgfqpoint{0.000000in}{-0.048611in}}{\pgfqpoint{0.000000in}{0.000000in}}{%
\pgfpathmoveto{\pgfqpoint{0.000000in}{0.000000in}}%
\pgfpathlineto{\pgfqpoint{0.000000in}{-0.048611in}}%
\pgfusepath{stroke,fill}%
}%
\begin{pgfscope}%
\pgfsys@transformshift{4.369323in}{0.429287in}%
\pgfsys@useobject{currentmarker}{}%
\end{pgfscope}%
\end{pgfscope}%
\begin{pgfscope}%
\definecolor{textcolor}{rgb}{0.180392,0.180392,0.180392}%
\pgfsetstrokecolor{textcolor}%
\pgfsetfillcolor{textcolor}%
\pgftext[x=4.369323in,y=0.332064in,,top]{\color{textcolor}\rmfamily\fontsize{9.000000}{10.800000}\selectfont \(\displaystyle {4}\)}%
\end{pgfscope}%
\begin{pgfscope}%
\pgfpathrectangle{\pgfqpoint{0.564224in}{0.429287in}}{\pgfqpoint{5.534689in}{3.879221in}}%
\pgfusepath{clip}%
\pgfsetbuttcap%
\pgfsetroundjoin%
\pgfsetlinewidth{0.501875pt}%
\definecolor{currentstroke}{rgb}{0.698039,0.698039,0.698039}%
\pgfsetstrokecolor{currentstroke}%
\pgfsetdash{{1.850000pt}{0.800000pt}}{0.000000pt}%
\pgfpathmoveto{\pgfqpoint{5.061159in}{0.429287in}}%
\pgfpathlineto{\pgfqpoint{5.061159in}{4.308507in}}%
\pgfusepath{stroke}%
\end{pgfscope}%
\begin{pgfscope}%
\pgfsetbuttcap%
\pgfsetroundjoin%
\definecolor{currentfill}{rgb}{0.180392,0.180392,0.180392}%
\pgfsetfillcolor{currentfill}%
\pgfsetlinewidth{0.803000pt}%
\definecolor{currentstroke}{rgb}{0.180392,0.180392,0.180392}%
\pgfsetstrokecolor{currentstroke}%
\pgfsetdash{}{0pt}%
\pgfsys@defobject{currentmarker}{\pgfqpoint{0.000000in}{-0.048611in}}{\pgfqpoint{0.000000in}{0.000000in}}{%
\pgfpathmoveto{\pgfqpoint{0.000000in}{0.000000in}}%
\pgfpathlineto{\pgfqpoint{0.000000in}{-0.048611in}}%
\pgfusepath{stroke,fill}%
}%
\begin{pgfscope}%
\pgfsys@transformshift{5.061159in}{0.429287in}%
\pgfsys@useobject{currentmarker}{}%
\end{pgfscope}%
\end{pgfscope}%
\begin{pgfscope}%
\definecolor{textcolor}{rgb}{0.180392,0.180392,0.180392}%
\pgfsetstrokecolor{textcolor}%
\pgfsetfillcolor{textcolor}%
\pgftext[x=5.061159in,y=0.332064in,,top]{\color{textcolor}\rmfamily\fontsize{9.000000}{10.800000}\selectfont \(\displaystyle {5}\)}%
\end{pgfscope}%
\begin{pgfscope}%
\pgfpathrectangle{\pgfqpoint{0.564224in}{0.429287in}}{\pgfqpoint{5.534689in}{3.879221in}}%
\pgfusepath{clip}%
\pgfsetbuttcap%
\pgfsetroundjoin%
\pgfsetlinewidth{0.501875pt}%
\definecolor{currentstroke}{rgb}{0.698039,0.698039,0.698039}%
\pgfsetstrokecolor{currentstroke}%
\pgfsetdash{{1.850000pt}{0.800000pt}}{0.000000pt}%
\pgfpathmoveto{\pgfqpoint{5.752996in}{0.429287in}}%
\pgfpathlineto{\pgfqpoint{5.752996in}{4.308507in}}%
\pgfusepath{stroke}%
\end{pgfscope}%
\begin{pgfscope}%
\pgfsetbuttcap%
\pgfsetroundjoin%
\definecolor{currentfill}{rgb}{0.180392,0.180392,0.180392}%
\pgfsetfillcolor{currentfill}%
\pgfsetlinewidth{0.803000pt}%
\definecolor{currentstroke}{rgb}{0.180392,0.180392,0.180392}%
\pgfsetstrokecolor{currentstroke}%
\pgfsetdash{}{0pt}%
\pgfsys@defobject{currentmarker}{\pgfqpoint{0.000000in}{-0.048611in}}{\pgfqpoint{0.000000in}{0.000000in}}{%
\pgfpathmoveto{\pgfqpoint{0.000000in}{0.000000in}}%
\pgfpathlineto{\pgfqpoint{0.000000in}{-0.048611in}}%
\pgfusepath{stroke,fill}%
}%
\begin{pgfscope}%
\pgfsys@transformshift{5.752996in}{0.429287in}%
\pgfsys@useobject{currentmarker}{}%
\end{pgfscope}%
\end{pgfscope}%
\begin{pgfscope}%
\definecolor{textcolor}{rgb}{0.180392,0.180392,0.180392}%
\pgfsetstrokecolor{textcolor}%
\pgfsetfillcolor{textcolor}%
\pgftext[x=5.752996in,y=0.332064in,,top]{\color{textcolor}\rmfamily\fontsize{9.000000}{10.800000}\selectfont \(\displaystyle {6}\)}%
\end{pgfscope}%
\begin{pgfscope}%
\definecolor{textcolor}{rgb}{0.180392,0.180392,0.180392}%
\pgfsetstrokecolor{textcolor}%
\pgfsetfillcolor{textcolor}%
\pgftext[x=3.331569in,y=0.150823in,,top]{\color{textcolor}\rmfamily\fontsize{10.800000}{12.960000}\selectfont Time (Seconds)}%
\end{pgfscope}%
\begin{pgfscope}%
\pgfpathrectangle{\pgfqpoint{0.564224in}{0.429287in}}{\pgfqpoint{5.534689in}{3.879221in}}%
\pgfusepath{clip}%
\pgfsetbuttcap%
\pgfsetroundjoin%
\pgfsetlinewidth{0.501875pt}%
\definecolor{currentstroke}{rgb}{0.698039,0.698039,0.698039}%
\pgfsetstrokecolor{currentstroke}%
\pgfsetdash{{1.850000pt}{0.800000pt}}{0.000000pt}%
\pgfpathmoveto{\pgfqpoint{0.564224in}{0.605601in}}%
\pgfpathlineto{\pgfqpoint{6.098914in}{0.605601in}}%
\pgfusepath{stroke}%
\end{pgfscope}%
\begin{pgfscope}%
\pgfsetbuttcap%
\pgfsetroundjoin%
\definecolor{currentfill}{rgb}{0.180392,0.180392,0.180392}%
\pgfsetfillcolor{currentfill}%
\pgfsetlinewidth{0.803000pt}%
\definecolor{currentstroke}{rgb}{0.180392,0.180392,0.180392}%
\pgfsetstrokecolor{currentstroke}%
\pgfsetdash{}{0pt}%
\pgfsys@defobject{currentmarker}{\pgfqpoint{-0.048611in}{0.000000in}}{\pgfqpoint{-0.000000in}{0.000000in}}{%
\pgfpathmoveto{\pgfqpoint{-0.000000in}{0.000000in}}%
\pgfpathlineto{\pgfqpoint{-0.048611in}{0.000000in}}%
\pgfusepath{stroke,fill}%
}%
\begin{pgfscope}%
\pgfsys@transformshift{0.564224in}{0.605601in}%
\pgfsys@useobject{currentmarker}{}%
\end{pgfscope}%
\end{pgfscope}%
\begin{pgfscope}%
\definecolor{textcolor}{rgb}{0.180392,0.180392,0.180392}%
\pgfsetstrokecolor{textcolor}%
\pgfsetfillcolor{textcolor}%
\pgftext[x=0.206378in, y=0.560382in, left, base]{\color{textcolor}\rmfamily\fontsize{9.000000}{10.800000}\selectfont \(\displaystyle {\ensuremath{-}2.0}\)}%
\end{pgfscope}%
\begin{pgfscope}%
\pgfpathrectangle{\pgfqpoint{0.564224in}{0.429287in}}{\pgfqpoint{5.534689in}{3.879221in}}%
\pgfusepath{clip}%
\pgfsetbuttcap%
\pgfsetroundjoin%
\pgfsetlinewidth{0.501875pt}%
\definecolor{currentstroke}{rgb}{0.698039,0.698039,0.698039}%
\pgfsetstrokecolor{currentstroke}%
\pgfsetdash{{1.850000pt}{0.800000pt}}{0.000000pt}%
\pgfpathmoveto{\pgfqpoint{0.564224in}{1.046425in}}%
\pgfpathlineto{\pgfqpoint{6.098914in}{1.046425in}}%
\pgfusepath{stroke}%
\end{pgfscope}%
\begin{pgfscope}%
\pgfsetbuttcap%
\pgfsetroundjoin%
\definecolor{currentfill}{rgb}{0.180392,0.180392,0.180392}%
\pgfsetfillcolor{currentfill}%
\pgfsetlinewidth{0.803000pt}%
\definecolor{currentstroke}{rgb}{0.180392,0.180392,0.180392}%
\pgfsetstrokecolor{currentstroke}%
\pgfsetdash{}{0pt}%
\pgfsys@defobject{currentmarker}{\pgfqpoint{-0.048611in}{0.000000in}}{\pgfqpoint{-0.000000in}{0.000000in}}{%
\pgfpathmoveto{\pgfqpoint{-0.000000in}{0.000000in}}%
\pgfpathlineto{\pgfqpoint{-0.048611in}{0.000000in}}%
\pgfusepath{stroke,fill}%
}%
\begin{pgfscope}%
\pgfsys@transformshift{0.564224in}{1.046425in}%
\pgfsys@useobject{currentmarker}{}%
\end{pgfscope}%
\end{pgfscope}%
\begin{pgfscope}%
\definecolor{textcolor}{rgb}{0.180392,0.180392,0.180392}%
\pgfsetstrokecolor{textcolor}%
\pgfsetfillcolor{textcolor}%
\pgftext[x=0.206378in, y=1.001206in, left, base]{\color{textcolor}\rmfamily\fontsize{9.000000}{10.800000}\selectfont \(\displaystyle {\ensuremath{-}1.5}\)}%
\end{pgfscope}%
\begin{pgfscope}%
\pgfpathrectangle{\pgfqpoint{0.564224in}{0.429287in}}{\pgfqpoint{5.534689in}{3.879221in}}%
\pgfusepath{clip}%
\pgfsetbuttcap%
\pgfsetroundjoin%
\pgfsetlinewidth{0.501875pt}%
\definecolor{currentstroke}{rgb}{0.698039,0.698039,0.698039}%
\pgfsetstrokecolor{currentstroke}%
\pgfsetdash{{1.850000pt}{0.800000pt}}{0.000000pt}%
\pgfpathmoveto{\pgfqpoint{0.564224in}{1.487249in}}%
\pgfpathlineto{\pgfqpoint{6.098914in}{1.487249in}}%
\pgfusepath{stroke}%
\end{pgfscope}%
\begin{pgfscope}%
\pgfsetbuttcap%
\pgfsetroundjoin%
\definecolor{currentfill}{rgb}{0.180392,0.180392,0.180392}%
\pgfsetfillcolor{currentfill}%
\pgfsetlinewidth{0.803000pt}%
\definecolor{currentstroke}{rgb}{0.180392,0.180392,0.180392}%
\pgfsetstrokecolor{currentstroke}%
\pgfsetdash{}{0pt}%
\pgfsys@defobject{currentmarker}{\pgfqpoint{-0.048611in}{0.000000in}}{\pgfqpoint{-0.000000in}{0.000000in}}{%
\pgfpathmoveto{\pgfqpoint{-0.000000in}{0.000000in}}%
\pgfpathlineto{\pgfqpoint{-0.048611in}{0.000000in}}%
\pgfusepath{stroke,fill}%
}%
\begin{pgfscope}%
\pgfsys@transformshift{0.564224in}{1.487249in}%
\pgfsys@useobject{currentmarker}{}%
\end{pgfscope}%
\end{pgfscope}%
\begin{pgfscope}%
\definecolor{textcolor}{rgb}{0.180392,0.180392,0.180392}%
\pgfsetstrokecolor{textcolor}%
\pgfsetfillcolor{textcolor}%
\pgftext[x=0.206378in, y=1.442030in, left, base]{\color{textcolor}\rmfamily\fontsize{9.000000}{10.800000}\selectfont \(\displaystyle {\ensuremath{-}1.0}\)}%
\end{pgfscope}%
\begin{pgfscope}%
\pgfpathrectangle{\pgfqpoint{0.564224in}{0.429287in}}{\pgfqpoint{5.534689in}{3.879221in}}%
\pgfusepath{clip}%
\pgfsetbuttcap%
\pgfsetroundjoin%
\pgfsetlinewidth{0.501875pt}%
\definecolor{currentstroke}{rgb}{0.698039,0.698039,0.698039}%
\pgfsetstrokecolor{currentstroke}%
\pgfsetdash{{1.850000pt}{0.800000pt}}{0.000000pt}%
\pgfpathmoveto{\pgfqpoint{0.564224in}{1.928073in}}%
\pgfpathlineto{\pgfqpoint{6.098914in}{1.928073in}}%
\pgfusepath{stroke}%
\end{pgfscope}%
\begin{pgfscope}%
\pgfsetbuttcap%
\pgfsetroundjoin%
\definecolor{currentfill}{rgb}{0.180392,0.180392,0.180392}%
\pgfsetfillcolor{currentfill}%
\pgfsetlinewidth{0.803000pt}%
\definecolor{currentstroke}{rgb}{0.180392,0.180392,0.180392}%
\pgfsetstrokecolor{currentstroke}%
\pgfsetdash{}{0pt}%
\pgfsys@defobject{currentmarker}{\pgfqpoint{-0.048611in}{0.000000in}}{\pgfqpoint{-0.000000in}{0.000000in}}{%
\pgfpathmoveto{\pgfqpoint{-0.000000in}{0.000000in}}%
\pgfpathlineto{\pgfqpoint{-0.048611in}{0.000000in}}%
\pgfusepath{stroke,fill}%
}%
\begin{pgfscope}%
\pgfsys@transformshift{0.564224in}{1.928073in}%
\pgfsys@useobject{currentmarker}{}%
\end{pgfscope}%
\end{pgfscope}%
\begin{pgfscope}%
\definecolor{textcolor}{rgb}{0.180392,0.180392,0.180392}%
\pgfsetstrokecolor{textcolor}%
\pgfsetfillcolor{textcolor}%
\pgftext[x=0.206378in, y=1.882855in, left, base]{\color{textcolor}\rmfamily\fontsize{9.000000}{10.800000}\selectfont \(\displaystyle {\ensuremath{-}0.5}\)}%
\end{pgfscope}%
\begin{pgfscope}%
\pgfpathrectangle{\pgfqpoint{0.564224in}{0.429287in}}{\pgfqpoint{5.534689in}{3.879221in}}%
\pgfusepath{clip}%
\pgfsetbuttcap%
\pgfsetroundjoin%
\pgfsetlinewidth{0.501875pt}%
\definecolor{currentstroke}{rgb}{0.698039,0.698039,0.698039}%
\pgfsetstrokecolor{currentstroke}%
\pgfsetdash{{1.850000pt}{0.800000pt}}{0.000000pt}%
\pgfpathmoveto{\pgfqpoint{0.564224in}{2.368897in}}%
\pgfpathlineto{\pgfqpoint{6.098914in}{2.368897in}}%
\pgfusepath{stroke}%
\end{pgfscope}%
\begin{pgfscope}%
\pgfsetbuttcap%
\pgfsetroundjoin%
\definecolor{currentfill}{rgb}{0.180392,0.180392,0.180392}%
\pgfsetfillcolor{currentfill}%
\pgfsetlinewidth{0.803000pt}%
\definecolor{currentstroke}{rgb}{0.180392,0.180392,0.180392}%
\pgfsetstrokecolor{currentstroke}%
\pgfsetdash{}{0pt}%
\pgfsys@defobject{currentmarker}{\pgfqpoint{-0.048611in}{0.000000in}}{\pgfqpoint{-0.000000in}{0.000000in}}{%
\pgfpathmoveto{\pgfqpoint{-0.000000in}{0.000000in}}%
\pgfpathlineto{\pgfqpoint{-0.048611in}{0.000000in}}%
\pgfusepath{stroke,fill}%
}%
\begin{pgfscope}%
\pgfsys@transformshift{0.564224in}{2.368897in}%
\pgfsys@useobject{currentmarker}{}%
\end{pgfscope}%
\end{pgfscope}%
\begin{pgfscope}%
\definecolor{textcolor}{rgb}{0.180392,0.180392,0.180392}%
\pgfsetstrokecolor{textcolor}%
\pgfsetfillcolor{textcolor}%
\pgftext[x=0.310752in, y=2.323679in, left, base]{\color{textcolor}\rmfamily\fontsize{9.000000}{10.800000}\selectfont \(\displaystyle {0.0}\)}%
\end{pgfscope}%
\begin{pgfscope}%
\pgfpathrectangle{\pgfqpoint{0.564224in}{0.429287in}}{\pgfqpoint{5.534689in}{3.879221in}}%
\pgfusepath{clip}%
\pgfsetbuttcap%
\pgfsetroundjoin%
\pgfsetlinewidth{0.501875pt}%
\definecolor{currentstroke}{rgb}{0.698039,0.698039,0.698039}%
\pgfsetstrokecolor{currentstroke}%
\pgfsetdash{{1.850000pt}{0.800000pt}}{0.000000pt}%
\pgfpathmoveto{\pgfqpoint{0.564224in}{2.809721in}}%
\pgfpathlineto{\pgfqpoint{6.098914in}{2.809721in}}%
\pgfusepath{stroke}%
\end{pgfscope}%
\begin{pgfscope}%
\pgfsetbuttcap%
\pgfsetroundjoin%
\definecolor{currentfill}{rgb}{0.180392,0.180392,0.180392}%
\pgfsetfillcolor{currentfill}%
\pgfsetlinewidth{0.803000pt}%
\definecolor{currentstroke}{rgb}{0.180392,0.180392,0.180392}%
\pgfsetstrokecolor{currentstroke}%
\pgfsetdash{}{0pt}%
\pgfsys@defobject{currentmarker}{\pgfqpoint{-0.048611in}{0.000000in}}{\pgfqpoint{-0.000000in}{0.000000in}}{%
\pgfpathmoveto{\pgfqpoint{-0.000000in}{0.000000in}}%
\pgfpathlineto{\pgfqpoint{-0.048611in}{0.000000in}}%
\pgfusepath{stroke,fill}%
}%
\begin{pgfscope}%
\pgfsys@transformshift{0.564224in}{2.809721in}%
\pgfsys@useobject{currentmarker}{}%
\end{pgfscope}%
\end{pgfscope}%
\begin{pgfscope}%
\definecolor{textcolor}{rgb}{0.180392,0.180392,0.180392}%
\pgfsetstrokecolor{textcolor}%
\pgfsetfillcolor{textcolor}%
\pgftext[x=0.310752in, y=2.764503in, left, base]{\color{textcolor}\rmfamily\fontsize{9.000000}{10.800000}\selectfont \(\displaystyle {0.5}\)}%
\end{pgfscope}%
\begin{pgfscope}%
\pgfpathrectangle{\pgfqpoint{0.564224in}{0.429287in}}{\pgfqpoint{5.534689in}{3.879221in}}%
\pgfusepath{clip}%
\pgfsetbuttcap%
\pgfsetroundjoin%
\pgfsetlinewidth{0.501875pt}%
\definecolor{currentstroke}{rgb}{0.698039,0.698039,0.698039}%
\pgfsetstrokecolor{currentstroke}%
\pgfsetdash{{1.850000pt}{0.800000pt}}{0.000000pt}%
\pgfpathmoveto{\pgfqpoint{0.564224in}{3.250545in}}%
\pgfpathlineto{\pgfqpoint{6.098914in}{3.250545in}}%
\pgfusepath{stroke}%
\end{pgfscope}%
\begin{pgfscope}%
\pgfsetbuttcap%
\pgfsetroundjoin%
\definecolor{currentfill}{rgb}{0.180392,0.180392,0.180392}%
\pgfsetfillcolor{currentfill}%
\pgfsetlinewidth{0.803000pt}%
\definecolor{currentstroke}{rgb}{0.180392,0.180392,0.180392}%
\pgfsetstrokecolor{currentstroke}%
\pgfsetdash{}{0pt}%
\pgfsys@defobject{currentmarker}{\pgfqpoint{-0.048611in}{0.000000in}}{\pgfqpoint{-0.000000in}{0.000000in}}{%
\pgfpathmoveto{\pgfqpoint{-0.000000in}{0.000000in}}%
\pgfpathlineto{\pgfqpoint{-0.048611in}{0.000000in}}%
\pgfusepath{stroke,fill}%
}%
\begin{pgfscope}%
\pgfsys@transformshift{0.564224in}{3.250545in}%
\pgfsys@useobject{currentmarker}{}%
\end{pgfscope}%
\end{pgfscope}%
\begin{pgfscope}%
\definecolor{textcolor}{rgb}{0.180392,0.180392,0.180392}%
\pgfsetstrokecolor{textcolor}%
\pgfsetfillcolor{textcolor}%
\pgftext[x=0.310752in, y=3.205327in, left, base]{\color{textcolor}\rmfamily\fontsize{9.000000}{10.800000}\selectfont \(\displaystyle {1.0}\)}%
\end{pgfscope}%
\begin{pgfscope}%
\pgfpathrectangle{\pgfqpoint{0.564224in}{0.429287in}}{\pgfqpoint{5.534689in}{3.879221in}}%
\pgfusepath{clip}%
\pgfsetbuttcap%
\pgfsetroundjoin%
\pgfsetlinewidth{0.501875pt}%
\definecolor{currentstroke}{rgb}{0.698039,0.698039,0.698039}%
\pgfsetstrokecolor{currentstroke}%
\pgfsetdash{{1.850000pt}{0.800000pt}}{0.000000pt}%
\pgfpathmoveto{\pgfqpoint{0.564224in}{3.691369in}}%
\pgfpathlineto{\pgfqpoint{6.098914in}{3.691369in}}%
\pgfusepath{stroke}%
\end{pgfscope}%
\begin{pgfscope}%
\pgfsetbuttcap%
\pgfsetroundjoin%
\definecolor{currentfill}{rgb}{0.180392,0.180392,0.180392}%
\pgfsetfillcolor{currentfill}%
\pgfsetlinewidth{0.803000pt}%
\definecolor{currentstroke}{rgb}{0.180392,0.180392,0.180392}%
\pgfsetstrokecolor{currentstroke}%
\pgfsetdash{}{0pt}%
\pgfsys@defobject{currentmarker}{\pgfqpoint{-0.048611in}{0.000000in}}{\pgfqpoint{-0.000000in}{0.000000in}}{%
\pgfpathmoveto{\pgfqpoint{-0.000000in}{0.000000in}}%
\pgfpathlineto{\pgfqpoint{-0.048611in}{0.000000in}}%
\pgfusepath{stroke,fill}%
}%
\begin{pgfscope}%
\pgfsys@transformshift{0.564224in}{3.691369in}%
\pgfsys@useobject{currentmarker}{}%
\end{pgfscope}%
\end{pgfscope}%
\begin{pgfscope}%
\definecolor{textcolor}{rgb}{0.180392,0.180392,0.180392}%
\pgfsetstrokecolor{textcolor}%
\pgfsetfillcolor{textcolor}%
\pgftext[x=0.310752in, y=3.646151in, left, base]{\color{textcolor}\rmfamily\fontsize{9.000000}{10.800000}\selectfont \(\displaystyle {1.5}\)}%
\end{pgfscope}%
\begin{pgfscope}%
\pgfpathrectangle{\pgfqpoint{0.564224in}{0.429287in}}{\pgfqpoint{5.534689in}{3.879221in}}%
\pgfusepath{clip}%
\pgfsetbuttcap%
\pgfsetroundjoin%
\pgfsetlinewidth{0.501875pt}%
\definecolor{currentstroke}{rgb}{0.698039,0.698039,0.698039}%
\pgfsetstrokecolor{currentstroke}%
\pgfsetdash{{1.850000pt}{0.800000pt}}{0.000000pt}%
\pgfpathmoveto{\pgfqpoint{0.564224in}{4.132193in}}%
\pgfpathlineto{\pgfqpoint{6.098914in}{4.132193in}}%
\pgfusepath{stroke}%
\end{pgfscope}%
\begin{pgfscope}%
\pgfsetbuttcap%
\pgfsetroundjoin%
\definecolor{currentfill}{rgb}{0.180392,0.180392,0.180392}%
\pgfsetfillcolor{currentfill}%
\pgfsetlinewidth{0.803000pt}%
\definecolor{currentstroke}{rgb}{0.180392,0.180392,0.180392}%
\pgfsetstrokecolor{currentstroke}%
\pgfsetdash{}{0pt}%
\pgfsys@defobject{currentmarker}{\pgfqpoint{-0.048611in}{0.000000in}}{\pgfqpoint{-0.000000in}{0.000000in}}{%
\pgfpathmoveto{\pgfqpoint{-0.000000in}{0.000000in}}%
\pgfpathlineto{\pgfqpoint{-0.048611in}{0.000000in}}%
\pgfusepath{stroke,fill}%
}%
\begin{pgfscope}%
\pgfsys@transformshift{0.564224in}{4.132193in}%
\pgfsys@useobject{currentmarker}{}%
\end{pgfscope}%
\end{pgfscope}%
\begin{pgfscope}%
\definecolor{textcolor}{rgb}{0.180392,0.180392,0.180392}%
\pgfsetstrokecolor{textcolor}%
\pgfsetfillcolor{textcolor}%
\pgftext[x=0.310752in, y=4.086975in, left, base]{\color{textcolor}\rmfamily\fontsize{9.000000}{10.800000}\selectfont \(\displaystyle {2.0}\)}%
\end{pgfscope}%
\begin{pgfscope}%
\definecolor{textcolor}{rgb}{0.180392,0.180392,0.180392}%
\pgfsetstrokecolor{textcolor}%
\pgfsetfillcolor{textcolor}%
\pgftext[x=0.150823in,y=2.368897in,,bottom,rotate=90.000000]{\color{textcolor}\rmfamily\fontsize{10.800000}{12.960000}\selectfont Intensity (Unitless)}%
\end{pgfscope}%
\begin{pgfscope}%
\pgfpathrectangle{\pgfqpoint{0.564224in}{0.429287in}}{\pgfqpoint{5.534689in}{3.879221in}}%
\pgfusepath{clip}%
\pgfsetrectcap%
\pgfsetroundjoin%
\pgfsetlinewidth{2.007500pt}%
\definecolor{currentstroke}{rgb}{0.000000,0.501961,0.000000}%
\pgfsetstrokecolor{currentstroke}%
\pgfsetdash{}{0pt}%
\pgfpathmoveto{\pgfqpoint{0.554224in}{2.368962in}}%
\pgfpathlineto{\pgfqpoint{1.021058in}{2.368421in}}%
\pgfpathlineto{\pgfqpoint{1.112563in}{2.367326in}}%
\pgfpathlineto{\pgfqpoint{1.173567in}{2.369079in}}%
\pgfpathlineto{\pgfqpoint{1.265072in}{2.372880in}}%
\pgfpathlineto{\pgfqpoint{1.310825in}{2.371660in}}%
\pgfpathlineto{\pgfqpoint{1.356578in}{2.367296in}}%
\pgfpathlineto{\pgfqpoint{1.417581in}{2.360582in}}%
\pgfpathlineto{\pgfqpoint{1.448083in}{2.359674in}}%
\pgfpathlineto{\pgfqpoint{1.478585in}{2.361881in}}%
\pgfpathlineto{\pgfqpoint{1.509087in}{2.367389in}}%
\pgfpathlineto{\pgfqpoint{1.585341in}{2.386291in}}%
\pgfpathlineto{\pgfqpoint{1.600592in}{2.388362in}}%
\pgfpathlineto{\pgfqpoint{1.615843in}{2.389039in}}%
\pgfpathlineto{\pgfqpoint{1.631094in}{2.388068in}}%
\pgfpathlineto{\pgfqpoint{1.646345in}{2.385306in}}%
\pgfpathlineto{\pgfqpoint{1.661596in}{2.380745in}}%
\pgfpathlineto{\pgfqpoint{1.676847in}{2.374537in}}%
\pgfpathlineto{\pgfqpoint{1.707348in}{2.358610in}}%
\pgfpathlineto{\pgfqpoint{1.737850in}{2.341829in}}%
\pgfpathlineto{\pgfqpoint{1.753101in}{2.334923in}}%
\pgfpathlineto{\pgfqpoint{1.768352in}{2.330024in}}%
\pgfpathlineto{\pgfqpoint{1.783603in}{2.327813in}}%
\pgfpathlineto{\pgfqpoint{1.798854in}{2.328825in}}%
\pgfpathlineto{\pgfqpoint{1.814105in}{2.333382in}}%
\pgfpathlineto{\pgfqpoint{1.829356in}{2.341537in}}%
\pgfpathlineto{\pgfqpoint{1.844606in}{2.353039in}}%
\pgfpathlineto{\pgfqpoint{1.859857in}{2.367313in}}%
\pgfpathlineto{\pgfqpoint{1.920861in}{2.430809in}}%
\pgfpathlineto{\pgfqpoint{1.936112in}{2.441436in}}%
\pgfpathlineto{\pgfqpoint{1.951363in}{2.447191in}}%
\pgfpathlineto{\pgfqpoint{1.966614in}{2.447019in}}%
\pgfpathlineto{\pgfqpoint{1.981865in}{2.440253in}}%
\pgfpathlineto{\pgfqpoint{1.997115in}{2.426717in}}%
\pgfpathlineto{\pgfqpoint{2.012366in}{2.406797in}}%
\pgfpathlineto{\pgfqpoint{2.027617in}{2.381468in}}%
\pgfpathlineto{\pgfqpoint{2.058119in}{2.321274in}}%
\pgfpathlineto{\pgfqpoint{2.073370in}{2.290880in}}%
\pgfpathlineto{\pgfqpoint{2.088621in}{2.263728in}}%
\pgfpathlineto{\pgfqpoint{2.103872in}{2.242453in}}%
\pgfpathlineto{\pgfqpoint{2.119123in}{2.229468in}}%
\pgfpathlineto{\pgfqpoint{2.134374in}{2.226729in}}%
\pgfpathlineto{\pgfqpoint{2.149624in}{2.235524in}}%
\pgfpathlineto{\pgfqpoint{2.164875in}{2.256292in}}%
\pgfpathlineto{\pgfqpoint{2.180126in}{2.288501in}}%
\pgfpathlineto{\pgfqpoint{2.195377in}{2.330597in}}%
\pgfpathlineto{\pgfqpoint{2.210628in}{2.380031in}}%
\pgfpathlineto{\pgfqpoint{2.241130in}{2.486593in}}%
\pgfpathlineto{\pgfqpoint{2.256381in}{2.535186in}}%
\pgfpathlineto{\pgfqpoint{2.271632in}{2.574680in}}%
\pgfpathlineto{\pgfqpoint{2.286882in}{2.600935in}}%
\pgfpathlineto{\pgfqpoint{2.302133in}{2.610539in}}%
\pgfpathlineto{\pgfqpoint{2.317384in}{2.601169in}}%
\pgfpathlineto{\pgfqpoint{2.332635in}{2.571888in}}%
\pgfpathlineto{\pgfqpoint{2.347886in}{2.523344in}}%
\pgfpathlineto{\pgfqpoint{2.363137in}{2.457856in}}%
\pgfpathlineto{\pgfqpoint{2.378388in}{2.379368in}}%
\pgfpathlineto{\pgfqpoint{2.424141in}{2.124627in}}%
\pgfpathlineto{\pgfqpoint{2.439391in}{2.056432in}}%
\pgfpathlineto{\pgfqpoint{2.454642in}{2.008079in}}%
\pgfpathlineto{\pgfqpoint{2.469893in}{1.985157in}}%
\pgfpathlineto{\pgfqpoint{2.485144in}{1.991585in}}%
\pgfpathlineto{\pgfqpoint{2.500395in}{2.029153in}}%
\pgfpathlineto{\pgfqpoint{2.515646in}{2.097203in}}%
\pgfpathlineto{\pgfqpoint{2.530897in}{2.192498in}}%
\pgfpathlineto{\pgfqpoint{2.546148in}{2.309297in}}%
\pgfpathlineto{\pgfqpoint{2.591900in}{2.701340in}}%
\pgfpathlineto{\pgfqpoint{2.607151in}{2.811106in}}%
\pgfpathlineto{\pgfqpoint{2.622402in}{2.892988in}}%
\pgfpathlineto{\pgfqpoint{2.637653in}{2.938371in}}%
\pgfpathlineto{\pgfqpoint{2.652904in}{2.941058in}}%
\pgfpathlineto{\pgfqpoint{2.668155in}{2.897947in}}%
\pgfpathlineto{\pgfqpoint{2.683406in}{2.809494in}}%
\pgfpathlineto{\pgfqpoint{2.698657in}{2.679912in}}%
\pgfpathlineto{\pgfqpoint{2.713908in}{2.517063in}}%
\pgfpathlineto{\pgfqpoint{2.774911in}{1.786425in}}%
\pgfpathlineto{\pgfqpoint{2.790162in}{1.658180in}}%
\pgfpathlineto{\pgfqpoint{2.805413in}{1.579146in}}%
\pgfpathlineto{\pgfqpoint{2.820664in}{1.558514in}}%
\pgfpathlineto{\pgfqpoint{2.835915in}{1.601234in}}%
\pgfpathlineto{\pgfqpoint{2.851166in}{1.707367in}}%
\pgfpathlineto{\pgfqpoint{2.866417in}{1.871812in}}%
\pgfpathlineto{\pgfqpoint{2.881668in}{2.084453in}}%
\pgfpathlineto{\pgfqpoint{2.912169in}{2.592583in}}%
\pgfpathlineto{\pgfqpoint{2.927420in}{2.849809in}}%
\pgfpathlineto{\pgfqpoint{2.942671in}{3.081562in}}%
\pgfpathlineto{\pgfqpoint{2.957922in}{3.268059in}}%
\pgfpathlineto{\pgfqpoint{2.973173in}{3.392244in}}%
\pgfpathlineto{\pgfqpoint{2.988424in}{3.441315in}}%
\pgfpathlineto{\pgfqpoint{3.003675in}{3.407964in}}%
\pgfpathlineto{\pgfqpoint{3.018926in}{3.291218in}}%
\pgfpathlineto{\pgfqpoint{3.034176in}{3.096798in}}%
\pgfpathlineto{\pgfqpoint{3.049427in}{2.836933in}}%
\pgfpathlineto{\pgfqpoint{3.064678in}{2.529633in}}%
\pgfpathlineto{\pgfqpoint{3.095180in}{1.865860in}}%
\pgfpathlineto{\pgfqpoint{3.110431in}{1.561214in}}%
\pgfpathlineto{\pgfqpoint{3.125682in}{1.308717in}}%
\pgfpathlineto{\pgfqpoint{3.140933in}{1.130291in}}%
\pgfpathlineto{\pgfqpoint{3.156184in}{1.042684in}}%
\pgfpathlineto{\pgfqpoint{3.171435in}{1.055935in}}%
\pgfpathlineto{\pgfqpoint{3.186685in}{1.172328in}}%
\pgfpathlineto{\pgfqpoint{3.201936in}{1.385957in}}%
\pgfpathlineto{\pgfqpoint{3.217187in}{1.682945in}}%
\pgfpathlineto{\pgfqpoint{3.232438in}{2.042325in}}%
\pgfpathlineto{\pgfqpoint{3.278191in}{3.213312in}}%
\pgfpathlineto{\pgfqpoint{3.293442in}{3.532202in}}%
\pgfpathlineto{\pgfqpoint{3.308693in}{3.768476in}}%
\pgfpathlineto{\pgfqpoint{3.323944in}{3.901535in}}%
\pgfpathlineto{\pgfqpoint{3.339194in}{3.918521in}}%
\pgfpathlineto{\pgfqpoint{3.354445in}{3.815528in}}%
\pgfpathlineto{\pgfqpoint{3.369696in}{3.598112in}}%
\pgfpathlineto{\pgfqpoint{3.384947in}{3.281041in}}%
\pgfpathlineto{\pgfqpoint{3.400198in}{2.887294in}}%
\pgfpathlineto{\pgfqpoint{3.445951in}{1.559623in}}%
\pgfpathlineto{\pgfqpoint{3.461202in}{1.183393in}}%
\pgfpathlineto{\pgfqpoint{3.476453in}{0.893573in}}%
\pgfpathlineto{\pgfqpoint{3.491703in}{0.713984in}}%
\pgfpathlineto{\pgfqpoint{3.506954in}{0.660024in}}%
\pgfpathlineto{\pgfqpoint{3.522205in}{0.737335in}}%
\pgfpathlineto{\pgfqpoint{3.537456in}{0.941241in}}%
\pgfpathlineto{\pgfqpoint{3.552707in}{1.257015in}}%
\pgfpathlineto{\pgfqpoint{3.567958in}{1.660965in}}%
\pgfpathlineto{\pgfqpoint{3.628961in}{3.487712in}}%
\pgfpathlineto{\pgfqpoint{3.644212in}{3.818141in}}%
\pgfpathlineto{\pgfqpoint{3.659463in}{4.037974in}}%
\pgfpathlineto{\pgfqpoint{3.674714in}{4.129943in}}%
\pgfpathlineto{\pgfqpoint{3.689965in}{4.086818in}}%
\pgfpathlineto{\pgfqpoint{3.705216in}{3.911989in}}%
\pgfpathlineto{\pgfqpoint{3.720467in}{3.619194in}}%
\pgfpathlineto{\pgfqpoint{3.735718in}{3.231413in}}%
\pgfpathlineto{\pgfqpoint{3.766220in}{2.297349in}}%
\pgfpathlineto{\pgfqpoint{3.781470in}{1.823869in}}%
\pgfpathlineto{\pgfqpoint{3.796721in}{1.395194in}}%
\pgfpathlineto{\pgfqpoint{3.811972in}{1.044193in}}%
\pgfpathlineto{\pgfqpoint{3.827223in}{0.797401in}}%
\pgfpathlineto{\pgfqpoint{3.842474in}{0.672971in}}%
\pgfpathlineto{\pgfqpoint{3.857725in}{0.679310in}}%
\pgfpathlineto{\pgfqpoint{3.872976in}{0.814506in}}%
\pgfpathlineto{\pgfqpoint{3.888227in}{1.066592in}}%
\pgfpathlineto{\pgfqpoint{3.903478in}{1.414612in}}%
\pgfpathlineto{\pgfqpoint{3.918729in}{1.830386in}}%
\pgfpathlineto{\pgfqpoint{3.949230in}{2.730706in}}%
\pgfpathlineto{\pgfqpoint{3.964481in}{3.145300in}}%
\pgfpathlineto{\pgfqpoint{3.979732in}{3.493297in}}%
\pgfpathlineto{\pgfqpoint{3.994983in}{3.749121in}}%
\pgfpathlineto{\pgfqpoint{4.010234in}{3.894863in}}%
\pgfpathlineto{\pgfqpoint{4.025485in}{3.921548in}}%
\pgfpathlineto{\pgfqpoint{4.040736in}{3.829661in}}%
\pgfpathlineto{\pgfqpoint{4.055987in}{3.628910in}}%
\pgfpathlineto{\pgfqpoint{4.071238in}{3.337260in}}%
\pgfpathlineto{\pgfqpoint{4.086488in}{2.979325in}}%
\pgfpathlineto{\pgfqpoint{4.132241in}{1.807705in}}%
\pgfpathlineto{\pgfqpoint{4.147492in}{1.485271in}}%
\pgfpathlineto{\pgfqpoint{4.162743in}{1.239325in}}%
\pgfpathlineto{\pgfqpoint{4.177994in}{1.086459in}}%
\pgfpathlineto{\pgfqpoint{4.193245in}{1.035539in}}%
\pgfpathlineto{\pgfqpoint{4.208496in}{1.087240in}}%
\pgfpathlineto{\pgfqpoint{4.223746in}{1.234255in}}%
\pgfpathlineto{\pgfqpoint{4.238997in}{1.462119in}}%
\pgfpathlineto{\pgfqpoint{4.254248in}{1.750589in}}%
\pgfpathlineto{\pgfqpoint{4.300001in}{2.729572in}}%
\pgfpathlineto{\pgfqpoint{4.315252in}{3.009041in}}%
\pgfpathlineto{\pgfqpoint{4.330503in}{3.228999in}}%
\pgfpathlineto{\pgfqpoint{4.345754in}{3.375004in}}%
\pgfpathlineto{\pgfqpoint{4.361005in}{3.438917in}}%
\pgfpathlineto{\pgfqpoint{4.376255in}{3.419292in}}%
\pgfpathlineto{\pgfqpoint{4.391506in}{3.321206in}}%
\pgfpathlineto{\pgfqpoint{4.406757in}{3.155588in}}%
\pgfpathlineto{\pgfqpoint{4.422008in}{2.938111in}}%
\pgfpathlineto{\pgfqpoint{4.483012in}{1.944238in}}%
\pgfpathlineto{\pgfqpoint{4.498263in}{1.760846in}}%
\pgfpathlineto{\pgfqpoint{4.513514in}{1.632648in}}%
\pgfpathlineto{\pgfqpoint{4.528764in}{1.566606in}}%
\pgfpathlineto{\pgfqpoint{4.544015in}{1.564526in}}%
\pgfpathlineto{\pgfqpoint{4.559266in}{1.623181in}}%
\pgfpathlineto{\pgfqpoint{4.574517in}{1.734826in}}%
\pgfpathlineto{\pgfqpoint{4.589768in}{1.888044in}}%
\pgfpathlineto{\pgfqpoint{4.620270in}{2.261851in}}%
\pgfpathlineto{\pgfqpoint{4.635521in}{2.451676in}}%
\pgfpathlineto{\pgfqpoint{4.650772in}{2.624024in}}%
\pgfpathlineto{\pgfqpoint{4.666023in}{2.766795in}}%
\pgfpathlineto{\pgfqpoint{4.681273in}{2.870888in}}%
\pgfpathlineto{\pgfqpoint{4.696524in}{2.930736in}}%
\pgfpathlineto{\pgfqpoint{4.711775in}{2.944524in}}%
\pgfpathlineto{\pgfqpoint{4.727026in}{2.914099in}}%
\pgfpathlineto{\pgfqpoint{4.742277in}{2.844604in}}%
\pgfpathlineto{\pgfqpoint{4.757528in}{2.743862in}}%
\pgfpathlineto{\pgfqpoint{4.772779in}{2.621599in}}%
\pgfpathlineto{\pgfqpoint{4.803281in}{2.355607in}}%
\pgfpathlineto{\pgfqpoint{4.818531in}{2.232857in}}%
\pgfpathlineto{\pgfqpoint{4.833782in}{2.128954in}}%
\pgfpathlineto{\pgfqpoint{4.849033in}{2.050486in}}%
\pgfpathlineto{\pgfqpoint{4.864284in}{2.001623in}}%
\pgfpathlineto{\pgfqpoint{4.879535in}{1.983955in}}%
\pgfpathlineto{\pgfqpoint{4.894786in}{1.996554in}}%
\pgfpathlineto{\pgfqpoint{4.910037in}{2.036229in}}%
\pgfpathlineto{\pgfqpoint{4.925288in}{2.097938in}}%
\pgfpathlineto{\pgfqpoint{4.940539in}{2.175318in}}%
\pgfpathlineto{\pgfqpoint{4.986291in}{2.430565in}}%
\pgfpathlineto{\pgfqpoint{5.001542in}{2.501317in}}%
\pgfpathlineto{\pgfqpoint{5.016793in}{2.556379in}}%
\pgfpathlineto{\pgfqpoint{5.032044in}{2.592827in}}%
\pgfpathlineto{\pgfqpoint{5.047295in}{2.609409in}}%
\pgfpathlineto{\pgfqpoint{5.062546in}{2.606503in}}%
\pgfpathlineto{\pgfqpoint{5.077797in}{2.585958in}}%
\pgfpathlineto{\pgfqpoint{5.093048in}{2.550829in}}%
\pgfpathlineto{\pgfqpoint{5.108299in}{2.505036in}}%
\pgfpathlineto{\pgfqpoint{5.154051in}{2.347886in}}%
\pgfpathlineto{\pgfqpoint{5.169302in}{2.302778in}}%
\pgfpathlineto{\pgfqpoint{5.184553in}{2.266740in}}%
\pgfpathlineto{\pgfqpoint{5.199804in}{2.241697in}}%
\pgfpathlineto{\pgfqpoint{5.215055in}{2.228549in}}%
\pgfpathlineto{\pgfqpoint{5.230306in}{2.227202in}}%
\pgfpathlineto{\pgfqpoint{5.245557in}{2.236657in}}%
\pgfpathlineto{\pgfqpoint{5.260808in}{2.255178in}}%
\pgfpathlineto{\pgfqpoint{5.276058in}{2.280491in}}%
\pgfpathlineto{\pgfqpoint{5.337062in}{2.398142in}}%
\pgfpathlineto{\pgfqpoint{5.352313in}{2.420173in}}%
\pgfpathlineto{\pgfqpoint{5.367564in}{2.436106in}}%
\pgfpathlineto{\pgfqpoint{5.382815in}{2.445342in}}%
\pgfpathlineto{\pgfqpoint{5.398066in}{2.447859in}}%
\pgfpathlineto{\pgfqpoint{5.413316in}{2.444155in}}%
\pgfpathlineto{\pgfqpoint{5.428567in}{2.435158in}}%
\pgfpathlineto{\pgfqpoint{5.443818in}{2.422109in}}%
\pgfpathlineto{\pgfqpoint{5.474320in}{2.389600in}}%
\pgfpathlineto{\pgfqpoint{5.504822in}{2.357949in}}%
\pgfpathlineto{\pgfqpoint{5.520073in}{2.345356in}}%
\pgfpathlineto{\pgfqpoint{5.535324in}{2.335938in}}%
\pgfpathlineto{\pgfqpoint{5.550575in}{2.330063in}}%
\pgfpathlineto{\pgfqpoint{5.565825in}{2.327786in}}%
\pgfpathlineto{\pgfqpoint{5.581076in}{2.328877in}}%
\pgfpathlineto{\pgfqpoint{5.596327in}{2.332871in}}%
\pgfpathlineto{\pgfqpoint{5.611578in}{2.339133in}}%
\pgfpathlineto{\pgfqpoint{5.642080in}{2.355464in}}%
\pgfpathlineto{\pgfqpoint{5.672582in}{2.371929in}}%
\pgfpathlineto{\pgfqpoint{5.687833in}{2.378665in}}%
\pgfpathlineto{\pgfqpoint{5.703084in}{2.383851in}}%
\pgfpathlineto{\pgfqpoint{5.718334in}{2.387274in}}%
\pgfpathlineto{\pgfqpoint{5.733585in}{2.388887in}}%
\pgfpathlineto{\pgfqpoint{5.748836in}{2.388785in}}%
\pgfpathlineto{\pgfqpoint{5.764087in}{2.387186in}}%
\pgfpathlineto{\pgfqpoint{5.794589in}{2.380781in}}%
\pgfpathlineto{\pgfqpoint{5.855593in}{2.365313in}}%
\pgfpathlineto{\pgfqpoint{5.886094in}{2.360808in}}%
\pgfpathlineto{\pgfqpoint{5.916596in}{2.359661in}}%
\pgfpathlineto{\pgfqpoint{5.947098in}{2.361372in}}%
\pgfpathlineto{\pgfqpoint{6.053854in}{2.372201in}}%
\pgfpathlineto{\pgfqpoint{6.099607in}{2.372740in}}%
\pgfpathlineto{\pgfqpoint{6.108914in}{2.372475in}}%
\pgfpathlineto{\pgfqpoint{6.108914in}{2.372475in}}%
\pgfusepath{stroke}%
\end{pgfscope}%
\begin{pgfscope}%
\pgfpathrectangle{\pgfqpoint{0.564224in}{0.429287in}}{\pgfqpoint{5.534689in}{3.879221in}}%
\pgfusepath{clip}%
\pgfsetbuttcap%
\pgfsetroundjoin%
\pgfsetlinewidth{2.007500pt}%
\definecolor{currentstroke}{rgb}{1.000000,0.647059,0.000000}%
\pgfsetstrokecolor{currentstroke}%
\pgfsetdash{{2.000000pt}{3.300000pt}}{0.000000pt}%
\pgfpathmoveto{\pgfqpoint{0.554224in}{2.368963in}}%
\pgfpathlineto{\pgfqpoint{1.051560in}{2.370209in}}%
\pgfpathlineto{\pgfqpoint{1.249821in}{2.372634in}}%
\pgfpathlineto{\pgfqpoint{1.387080in}{2.376248in}}%
\pgfpathlineto{\pgfqpoint{1.493836in}{2.381004in}}%
\pgfpathlineto{\pgfqpoint{1.585341in}{2.387117in}}%
\pgfpathlineto{\pgfqpoint{1.661596in}{2.394170in}}%
\pgfpathlineto{\pgfqpoint{1.737850in}{2.403531in}}%
\pgfpathlineto{\pgfqpoint{1.798854in}{2.413072in}}%
\pgfpathlineto{\pgfqpoint{1.859857in}{2.424805in}}%
\pgfpathlineto{\pgfqpoint{1.920861in}{2.439106in}}%
\pgfpathlineto{\pgfqpoint{1.966614in}{2.451762in}}%
\pgfpathlineto{\pgfqpoint{2.012366in}{2.466271in}}%
\pgfpathlineto{\pgfqpoint{2.058119in}{2.482822in}}%
\pgfpathlineto{\pgfqpoint{2.103872in}{2.501605in}}%
\pgfpathlineto{\pgfqpoint{2.149624in}{2.522810in}}%
\pgfpathlineto{\pgfqpoint{2.195377in}{2.546624in}}%
\pgfpathlineto{\pgfqpoint{2.241130in}{2.573227in}}%
\pgfpathlineto{\pgfqpoint{2.286882in}{2.602787in}}%
\pgfpathlineto{\pgfqpoint{2.332635in}{2.635454in}}%
\pgfpathlineto{\pgfqpoint{2.378388in}{2.671359in}}%
\pgfpathlineto{\pgfqpoint{2.424141in}{2.710603in}}%
\pgfpathlineto{\pgfqpoint{2.469893in}{2.753253in}}%
\pgfpathlineto{\pgfqpoint{2.515646in}{2.799341in}}%
\pgfpathlineto{\pgfqpoint{2.561399in}{2.848851in}}%
\pgfpathlineto{\pgfqpoint{2.607151in}{2.901720in}}%
\pgfpathlineto{\pgfqpoint{2.652904in}{2.957832in}}%
\pgfpathlineto{\pgfqpoint{2.698657in}{3.017012in}}%
\pgfpathlineto{\pgfqpoint{2.759660in}{3.100279in}}%
\pgfpathlineto{\pgfqpoint{2.820664in}{3.187851in}}%
\pgfpathlineto{\pgfqpoint{2.896918in}{3.301947in}}%
\pgfpathlineto{\pgfqpoint{3.034176in}{3.513280in}}%
\pgfpathlineto{\pgfqpoint{3.125682in}{3.651775in}}%
\pgfpathlineto{\pgfqpoint{3.186685in}{3.739906in}}%
\pgfpathlineto{\pgfqpoint{3.232438in}{3.802622in}}%
\pgfpathlineto{\pgfqpoint{3.278191in}{3.861664in}}%
\pgfpathlineto{\pgfqpoint{3.323944in}{3.916354in}}%
\pgfpathlineto{\pgfqpoint{3.354445in}{3.950077in}}%
\pgfpathlineto{\pgfqpoint{3.384947in}{3.981397in}}%
\pgfpathlineto{\pgfqpoint{3.415449in}{4.010144in}}%
\pgfpathlineto{\pgfqpoint{3.445951in}{4.036160in}}%
\pgfpathlineto{\pgfqpoint{3.476453in}{4.059299in}}%
\pgfpathlineto{\pgfqpoint{3.506954in}{4.079431in}}%
\pgfpathlineto{\pgfqpoint{3.537456in}{4.096441in}}%
\pgfpathlineto{\pgfqpoint{3.567958in}{4.110233in}}%
\pgfpathlineto{\pgfqpoint{3.598460in}{4.120727in}}%
\pgfpathlineto{\pgfqpoint{3.628961in}{4.127861in}}%
\pgfpathlineto{\pgfqpoint{3.659463in}{4.131595in}}%
\pgfpathlineto{\pgfqpoint{3.689965in}{4.131906in}}%
\pgfpathlineto{\pgfqpoint{3.720467in}{4.128794in}}%
\pgfpathlineto{\pgfqpoint{3.750969in}{4.122275in}}%
\pgfpathlineto{\pgfqpoint{3.781470in}{4.112389in}}%
\pgfpathlineto{\pgfqpoint{3.811972in}{4.099191in}}%
\pgfpathlineto{\pgfqpoint{3.842474in}{4.082759in}}%
\pgfpathlineto{\pgfqpoint{3.872976in}{4.063186in}}%
\pgfpathlineto{\pgfqpoint{3.903478in}{4.040585in}}%
\pgfpathlineto{\pgfqpoint{3.933979in}{4.015082in}}%
\pgfpathlineto{\pgfqpoint{3.964481in}{3.986820in}}%
\pgfpathlineto{\pgfqpoint{3.994983in}{3.955955in}}%
\pgfpathlineto{\pgfqpoint{4.025485in}{3.922656in}}%
\pgfpathlineto{\pgfqpoint{4.071238in}{3.868538in}}%
\pgfpathlineto{\pgfqpoint{4.116990in}{3.809988in}}%
\pgfpathlineto{\pgfqpoint{4.162743in}{3.747681in}}%
\pgfpathlineto{\pgfqpoint{4.223746in}{3.659962in}}%
\pgfpathlineto{\pgfqpoint{4.300001in}{3.545187in}}%
\pgfpathlineto{\pgfqpoint{4.528764in}{3.195995in}}%
\pgfpathlineto{\pgfqpoint{4.589768in}{3.108074in}}%
\pgfpathlineto{\pgfqpoint{4.650772in}{3.024384in}}%
\pgfpathlineto{\pgfqpoint{4.696524in}{2.964846in}}%
\pgfpathlineto{\pgfqpoint{4.742277in}{2.908352in}}%
\pgfpathlineto{\pgfqpoint{4.788030in}{2.855082in}}%
\pgfpathlineto{\pgfqpoint{4.833782in}{2.805161in}}%
\pgfpathlineto{\pgfqpoint{4.879535in}{2.758657in}}%
\pgfpathlineto{\pgfqpoint{4.925288in}{2.715590in}}%
\pgfpathlineto{\pgfqpoint{4.971040in}{2.675937in}}%
\pgfpathlineto{\pgfqpoint{5.016793in}{2.639632in}}%
\pgfpathlineto{\pgfqpoint{5.062546in}{2.606578in}}%
\pgfpathlineto{\pgfqpoint{5.108299in}{2.576649in}}%
\pgfpathlineto{\pgfqpoint{5.154051in}{2.549696in}}%
\pgfpathlineto{\pgfqpoint{5.199804in}{2.525554in}}%
\pgfpathlineto{\pgfqpoint{5.245557in}{2.504043in}}%
\pgfpathlineto{\pgfqpoint{5.291309in}{2.484976in}}%
\pgfpathlineto{\pgfqpoint{5.337062in}{2.468165in}}%
\pgfpathlineto{\pgfqpoint{5.382815in}{2.453418in}}%
\pgfpathlineto{\pgfqpoint{5.428567in}{2.440548in}}%
\pgfpathlineto{\pgfqpoint{5.489571in}{2.425993in}}%
\pgfpathlineto{\pgfqpoint{5.550575in}{2.414042in}}%
\pgfpathlineto{\pgfqpoint{5.611578in}{2.404317in}}%
\pgfpathlineto{\pgfqpoint{5.687833in}{2.394766in}}%
\pgfpathlineto{\pgfqpoint{5.764087in}{2.387563in}}%
\pgfpathlineto{\pgfqpoint{5.855593in}{2.381314in}}%
\pgfpathlineto{\pgfqpoint{5.962349in}{2.376445in}}%
\pgfpathlineto{\pgfqpoint{6.099607in}{2.372741in}}%
\pgfpathlineto{\pgfqpoint{6.108914in}{2.372566in}}%
\pgfpathlineto{\pgfqpoint{6.108914in}{2.372566in}}%
\pgfusepath{stroke}%
\end{pgfscope}%
\begin{pgfscope}%
\pgfpathrectangle{\pgfqpoint{0.564224in}{0.429287in}}{\pgfqpoint{5.534689in}{3.879221in}}%
\pgfusepath{clip}%
\pgfsetbuttcap%
\pgfsetroundjoin%
\pgfsetlinewidth{2.007500pt}%
\definecolor{currentstroke}{rgb}{1.000000,0.647059,0.000000}%
\pgfsetstrokecolor{currentstroke}%
\pgfsetdash{{2.000000pt}{3.300000pt}}{0.000000pt}%
\pgfpathmoveto{\pgfqpoint{0.554224in}{2.368831in}}%
\pgfpathlineto{\pgfqpoint{1.051560in}{2.367585in}}%
\pgfpathlineto{\pgfqpoint{1.249821in}{2.365160in}}%
\pgfpathlineto{\pgfqpoint{1.387080in}{2.361546in}}%
\pgfpathlineto{\pgfqpoint{1.493836in}{2.356790in}}%
\pgfpathlineto{\pgfqpoint{1.585341in}{2.350677in}}%
\pgfpathlineto{\pgfqpoint{1.661596in}{2.343624in}}%
\pgfpathlineto{\pgfqpoint{1.737850in}{2.334263in}}%
\pgfpathlineto{\pgfqpoint{1.798854in}{2.324722in}}%
\pgfpathlineto{\pgfqpoint{1.859857in}{2.312989in}}%
\pgfpathlineto{\pgfqpoint{1.920861in}{2.298688in}}%
\pgfpathlineto{\pgfqpoint{1.966614in}{2.286032in}}%
\pgfpathlineto{\pgfqpoint{2.012366in}{2.271522in}}%
\pgfpathlineto{\pgfqpoint{2.058119in}{2.254971in}}%
\pgfpathlineto{\pgfqpoint{2.103872in}{2.236189in}}%
\pgfpathlineto{\pgfqpoint{2.149624in}{2.214984in}}%
\pgfpathlineto{\pgfqpoint{2.195377in}{2.191170in}}%
\pgfpathlineto{\pgfqpoint{2.241130in}{2.164567in}}%
\pgfpathlineto{\pgfqpoint{2.286882in}{2.135007in}}%
\pgfpathlineto{\pgfqpoint{2.332635in}{2.102339in}}%
\pgfpathlineto{\pgfqpoint{2.378388in}{2.066434in}}%
\pgfpathlineto{\pgfqpoint{2.424141in}{2.027191in}}%
\pgfpathlineto{\pgfqpoint{2.469893in}{1.984540in}}%
\pgfpathlineto{\pgfqpoint{2.515646in}{1.938453in}}%
\pgfpathlineto{\pgfqpoint{2.561399in}{1.888943in}}%
\pgfpathlineto{\pgfqpoint{2.607151in}{1.836073in}}%
\pgfpathlineto{\pgfqpoint{2.652904in}{1.779962in}}%
\pgfpathlineto{\pgfqpoint{2.698657in}{1.720781in}}%
\pgfpathlineto{\pgfqpoint{2.759660in}{1.637515in}}%
\pgfpathlineto{\pgfqpoint{2.820664in}{1.549943in}}%
\pgfpathlineto{\pgfqpoint{2.896918in}{1.435847in}}%
\pgfpathlineto{\pgfqpoint{3.034176in}{1.224514in}}%
\pgfpathlineto{\pgfqpoint{3.125682in}{1.086019in}}%
\pgfpathlineto{\pgfqpoint{3.186685in}{0.997887in}}%
\pgfpathlineto{\pgfqpoint{3.232438in}{0.935172in}}%
\pgfpathlineto{\pgfqpoint{3.278191in}{0.876130in}}%
\pgfpathlineto{\pgfqpoint{3.323944in}{0.821439in}}%
\pgfpathlineto{\pgfqpoint{3.354445in}{0.787717in}}%
\pgfpathlineto{\pgfqpoint{3.384947in}{0.756397in}}%
\pgfpathlineto{\pgfqpoint{3.415449in}{0.727650in}}%
\pgfpathlineto{\pgfqpoint{3.445951in}{0.701634in}}%
\pgfpathlineto{\pgfqpoint{3.476453in}{0.678495in}}%
\pgfpathlineto{\pgfqpoint{3.506954in}{0.658363in}}%
\pgfpathlineto{\pgfqpoint{3.537456in}{0.641352in}}%
\pgfpathlineto{\pgfqpoint{3.567958in}{0.627560in}}%
\pgfpathlineto{\pgfqpoint{3.598460in}{0.617067in}}%
\pgfpathlineto{\pgfqpoint{3.628961in}{0.609933in}}%
\pgfpathlineto{\pgfqpoint{3.659463in}{0.606199in}}%
\pgfpathlineto{\pgfqpoint{3.689965in}{0.605887in}}%
\pgfpathlineto{\pgfqpoint{3.720467in}{0.609000in}}%
\pgfpathlineto{\pgfqpoint{3.750969in}{0.615519in}}%
\pgfpathlineto{\pgfqpoint{3.781470in}{0.625405in}}%
\pgfpathlineto{\pgfqpoint{3.811972in}{0.638603in}}%
\pgfpathlineto{\pgfqpoint{3.842474in}{0.655035in}}%
\pgfpathlineto{\pgfqpoint{3.872976in}{0.674607in}}%
\pgfpathlineto{\pgfqpoint{3.903478in}{0.697209in}}%
\pgfpathlineto{\pgfqpoint{3.933979in}{0.722712in}}%
\pgfpathlineto{\pgfqpoint{3.964481in}{0.750974in}}%
\pgfpathlineto{\pgfqpoint{3.994983in}{0.781839in}}%
\pgfpathlineto{\pgfqpoint{4.025485in}{0.815137in}}%
\pgfpathlineto{\pgfqpoint{4.071238in}{0.869256in}}%
\pgfpathlineto{\pgfqpoint{4.116990in}{0.927806in}}%
\pgfpathlineto{\pgfqpoint{4.162743in}{0.990113in}}%
\pgfpathlineto{\pgfqpoint{4.223746in}{1.077832in}}%
\pgfpathlineto{\pgfqpoint{4.300001in}{1.192607in}}%
\pgfpathlineto{\pgfqpoint{4.528764in}{1.541799in}}%
\pgfpathlineto{\pgfqpoint{4.589768in}{1.629719in}}%
\pgfpathlineto{\pgfqpoint{4.650772in}{1.713410in}}%
\pgfpathlineto{\pgfqpoint{4.696524in}{1.772948in}}%
\pgfpathlineto{\pgfqpoint{4.742277in}{1.829442in}}%
\pgfpathlineto{\pgfqpoint{4.788030in}{1.882711in}}%
\pgfpathlineto{\pgfqpoint{4.833782in}{1.932633in}}%
\pgfpathlineto{\pgfqpoint{4.879535in}{1.979137in}}%
\pgfpathlineto{\pgfqpoint{4.925288in}{2.022203in}}%
\pgfpathlineto{\pgfqpoint{4.971040in}{2.061857in}}%
\pgfpathlineto{\pgfqpoint{5.016793in}{2.098162in}}%
\pgfpathlineto{\pgfqpoint{5.062546in}{2.131216in}}%
\pgfpathlineto{\pgfqpoint{5.108299in}{2.161145in}}%
\pgfpathlineto{\pgfqpoint{5.154051in}{2.188097in}}%
\pgfpathlineto{\pgfqpoint{5.199804in}{2.212240in}}%
\pgfpathlineto{\pgfqpoint{5.245557in}{2.233751in}}%
\pgfpathlineto{\pgfqpoint{5.291309in}{2.252817in}}%
\pgfpathlineto{\pgfqpoint{5.337062in}{2.269629in}}%
\pgfpathlineto{\pgfqpoint{5.382815in}{2.284376in}}%
\pgfpathlineto{\pgfqpoint{5.428567in}{2.297246in}}%
\pgfpathlineto{\pgfqpoint{5.489571in}{2.311801in}}%
\pgfpathlineto{\pgfqpoint{5.550575in}{2.323751in}}%
\pgfpathlineto{\pgfqpoint{5.611578in}{2.333477in}}%
\pgfpathlineto{\pgfqpoint{5.687833in}{2.343027in}}%
\pgfpathlineto{\pgfqpoint{5.764087in}{2.350231in}}%
\pgfpathlineto{\pgfqpoint{5.855593in}{2.356480in}}%
\pgfpathlineto{\pgfqpoint{5.962349in}{2.361348in}}%
\pgfpathlineto{\pgfqpoint{6.099607in}{2.365053in}}%
\pgfpathlineto{\pgfqpoint{6.108914in}{2.365228in}}%
\pgfpathlineto{\pgfqpoint{6.108914in}{2.365228in}}%
\pgfusepath{stroke}%
\end{pgfscope}%
\begin{pgfscope}%
\pgfsetrectcap%
\pgfsetmiterjoin%
\pgfsetlinewidth{0.803000pt}%
\definecolor{currentstroke}{rgb}{0.737255,0.737255,0.737255}%
\pgfsetstrokecolor{currentstroke}%
\pgfsetdash{}{0pt}%
\pgfpathmoveto{\pgfqpoint{0.564224in}{0.429287in}}%
\pgfpathlineto{\pgfqpoint{0.564224in}{4.308507in}}%
\pgfusepath{stroke}%
\end{pgfscope}%
\begin{pgfscope}%
\pgfsetrectcap%
\pgfsetmiterjoin%
\pgfsetlinewidth{0.803000pt}%
\definecolor{currentstroke}{rgb}{0.737255,0.737255,0.737255}%
\pgfsetstrokecolor{currentstroke}%
\pgfsetdash{}{0pt}%
\pgfpathmoveto{\pgfqpoint{6.098914in}{0.429287in}}%
\pgfpathlineto{\pgfqpoint{6.098914in}{4.308507in}}%
\pgfusepath{stroke}%
\end{pgfscope}%
\begin{pgfscope}%
\pgfsetrectcap%
\pgfsetmiterjoin%
\pgfsetlinewidth{0.803000pt}%
\definecolor{currentstroke}{rgb}{0.737255,0.737255,0.737255}%
\pgfsetstrokecolor{currentstroke}%
\pgfsetdash{}{0pt}%
\pgfpathmoveto{\pgfqpoint{0.564224in}{0.429287in}}%
\pgfpathlineto{\pgfqpoint{6.098914in}{0.429287in}}%
\pgfusepath{stroke}%
\end{pgfscope}%
\begin{pgfscope}%
\pgfsetrectcap%
\pgfsetmiterjoin%
\pgfsetlinewidth{0.803000pt}%
\definecolor{currentstroke}{rgb}{0.737255,0.737255,0.737255}%
\pgfsetstrokecolor{currentstroke}%
\pgfsetdash{}{0pt}%
\pgfpathmoveto{\pgfqpoint{0.564224in}{4.308507in}}%
\pgfpathlineto{\pgfqpoint{6.098914in}{4.308507in}}%
\pgfusepath{stroke}%
\end{pgfscope}%
\begin{pgfscope}%
\pgfsetroundcap%
\pgfsetroundjoin%
\pgfsetlinewidth{0.501875pt}%
\definecolor{currentstroke}{rgb}{0.933333,0.933333,0.933333}%
\pgfsetstrokecolor{currentstroke}%
\pgfsetdash{}{0pt}%
\pgfpathmoveto{\pgfqpoint{5.005941in}{3.571778in}}%
\pgfpathquadraticcurveto{\pgfqpoint{4.614914in}{3.629541in}}{\pgfqpoint{4.231568in}{3.686169in}}%
\pgfusepath{stroke}%
\end{pgfscope}%
\begin{pgfscope}%
\pgfsetroundcap%
\pgfsetroundjoin%
\pgfsetlinewidth{0.501875pt}%
\definecolor{currentstroke}{rgb}{0.933333,0.933333,0.933333}%
\pgfsetstrokecolor{currentstroke}%
\pgfsetdash{}{0pt}%
\pgfpathmoveto{\pgfqpoint{4.277378in}{3.654130in}}%
\pgfpathlineto{\pgfqpoint{4.231568in}{3.686169in}}%
\pgfpathlineto{\pgfqpoint{4.284684in}{3.703594in}}%
\pgfusepath{stroke}%
\end{pgfscope}%
\begin{pgfscope}%
\definecolor{textcolor}{rgb}{0.180392,0.180392,0.180392}%
\pgfsetstrokecolor{textcolor}%
\pgfsetfillcolor{textcolor}%
\pgftext[x=5.061159in,y=3.470957in,left,base]{\color{textcolor}\rmfamily\fontsize{9.000000}{10.800000}\selectfont Envelope signal}%
\end{pgfscope}%
\begin{pgfscope}%
\pgfsetroundcap%
\pgfsetroundjoin%
\pgfsetlinewidth{0.501875pt}%
\definecolor{currentstroke}{rgb}{0.933333,0.933333,0.933333}%
\pgfsetstrokecolor{currentstroke}%
\pgfsetdash{}{0pt}%
\pgfpathmoveto{\pgfqpoint{5.440371in}{3.383426in}}%
\pgfpathquadraticcurveto{\pgfqpoint{4.824888in}{2.227174in}}{\pgfqpoint{4.213053in}{1.077777in}}%
\pgfusepath{stroke}%
\end{pgfscope}%
\begin{pgfscope}%
\pgfsetroundcap%
\pgfsetroundjoin%
\pgfsetlinewidth{0.501875pt}%
\definecolor{currentstroke}{rgb}{0.933333,0.933333,0.933333}%
\pgfsetstrokecolor{currentstroke}%
\pgfsetdash{}{0pt}%
\pgfpathmoveto{\pgfqpoint{4.258616in}{1.110166in}}%
\pgfpathlineto{\pgfqpoint{4.213053in}{1.077777in}}%
\pgfpathlineto{\pgfqpoint{4.214479in}{1.133660in}}%
\pgfusepath{stroke}%
\end{pgfscope}%
\begin{pgfscope}%
\definecolor{textcolor}{rgb}{0.180392,0.180392,0.180392}%
\pgfsetstrokecolor{textcolor}%
\pgfsetfillcolor{textcolor}%
\pgftext[x=5.061159in,y=3.470957in,left,base]{\color{textcolor}\rmfamily\fontsize{9.000000}{10.800000}\selectfont Envelope signal}%
\end{pgfscope}%
\begin{pgfscope}%
\definecolor{textcolor}{rgb}{0.180392,0.180392,0.180392}%
\pgfsetstrokecolor{textcolor}%
\pgfsetfillcolor{textcolor}%
\pgftext[x=3.331569in,y=4.391841in,,base]{\color{textcolor}\rmfamily\fontsize{12.960000}{15.552000}\selectfont Pulsed Gaussian Signal}%
\end{pgfscope}%
\end{pgfpicture}%
\makeatother%
\endgroup%
}
		\caption{\centering{An example graph of the Gaussian Pulse, the envelope signal is shown as a dotted line, the signal propagated in the medium is the solid line.}}
		\label{fig:gaussian_demo}
	\end{figure}	
	
	\pagebreak
	
	It is also important to note here that the envelope signal is itself comprised of two components, a positive and a negative portion, this is easily displayed in the frequency domain, observe the spectrum (Figure \ref{fig:gaussian_spectrum}) of the propagating signal (displayed before in Figure \ref{fig:gaussian_demo}), there are a positive and negative frequency components: \linebreak
	
	\begin{figure}[h]
		\centering
		\scalebox{0.85}{%% Creator: Matplotlib, PGF backend
%%
%% To include the figure in your LaTeX document, write
%%   \input{<filename>.pgf}
%%
%% Make sure the required packages are loaded in your preamble
%%   \usepackage{pgf}
%%
%% Also ensure that all the required font packages are loaded; for instance,
%% the lmodern package is sometimes necessary when using math font.
%%   \usepackage{lmodern}
%%
%% Figures using additional raster images can only be included by \input if
%% they are in the same directory as the main LaTeX file. For loading figures
%% from other directories you can use the `import` package
%%   \usepackage{import}
%%
%% and then include the figures with
%%   \import{<path to file>}{<filename>.pgf}
%%
%% Matplotlib used the following preamble
%%   \usepackage[T1]{fontenc} \usepackage{mathpazo}
%%
\begingroup%
\makeatletter%
\begin{pgfpicture}%
\pgfpathrectangle{\pgfpointorigin}{\pgfqpoint{6.020575in}{2.250534in}}%
\pgfusepath{use as bounding box, clip}%
\begin{pgfscope}%
\pgfsetbuttcap%
\pgfsetmiterjoin%
\definecolor{currentfill}{rgb}{1.000000,1.000000,1.000000}%
\pgfsetfillcolor{currentfill}%
\pgfsetlinewidth{0.000000pt}%
\definecolor{currentstroke}{rgb}{1.000000,1.000000,1.000000}%
\pgfsetstrokecolor{currentstroke}%
\pgfsetdash{}{0pt}%
\pgfpathmoveto{\pgfqpoint{0.000000in}{0.000000in}}%
\pgfpathlineto{\pgfqpoint{6.020575in}{0.000000in}}%
\pgfpathlineto{\pgfqpoint{6.020575in}{2.250534in}}%
\pgfpathlineto{\pgfqpoint{0.000000in}{2.250534in}}%
\pgfpathlineto{\pgfqpoint{0.000000in}{0.000000in}}%
\pgfpathclose%
\pgfusepath{fill}%
\end{pgfscope}%
\begin{pgfscope}%
\pgfsetbuttcap%
\pgfsetmiterjoin%
\definecolor{currentfill}{rgb}{0.933333,0.933333,0.933333}%
\pgfsetfillcolor{currentfill}%
\pgfsetlinewidth{0.000000pt}%
\definecolor{currentstroke}{rgb}{0.000000,0.000000,0.000000}%
\pgfsetstrokecolor{currentstroke}%
\pgfsetstrokeopacity{0.000000}%
\pgfsetdash{}{0pt}%
\pgfpathmoveto{\pgfqpoint{0.477328in}{0.429287in}}%
\pgfpathlineto{\pgfqpoint{5.958075in}{0.429287in}}%
\pgfpathlineto{\pgfqpoint{5.958075in}{2.036972in}}%
\pgfpathlineto{\pgfqpoint{0.477328in}{2.036972in}}%
\pgfpathlineto{\pgfqpoint{0.477328in}{0.429287in}}%
\pgfpathclose%
\pgfusepath{fill}%
\end{pgfscope}%
\begin{pgfscope}%
\pgfpathrectangle{\pgfqpoint{0.477328in}{0.429287in}}{\pgfqpoint{5.480747in}{1.607685in}}%
\pgfusepath{clip}%
\pgfsetbuttcap%
\pgfsetroundjoin%
\pgfsetlinewidth{0.501875pt}%
\definecolor{currentstroke}{rgb}{0.698039,0.698039,0.698039}%
\pgfsetstrokecolor{currentstroke}%
\pgfsetdash{{1.850000pt}{0.800000pt}}{0.000000pt}%
\pgfpathmoveto{\pgfqpoint{0.477328in}{0.429287in}}%
\pgfpathlineto{\pgfqpoint{0.477328in}{2.036972in}}%
\pgfusepath{stroke}%
\end{pgfscope}%
\begin{pgfscope}%
\pgfsetbuttcap%
\pgfsetroundjoin%
\definecolor{currentfill}{rgb}{0.180392,0.180392,0.180392}%
\pgfsetfillcolor{currentfill}%
\pgfsetlinewidth{0.803000pt}%
\definecolor{currentstroke}{rgb}{0.180392,0.180392,0.180392}%
\pgfsetstrokecolor{currentstroke}%
\pgfsetdash{}{0pt}%
\pgfsys@defobject{currentmarker}{\pgfqpoint{0.000000in}{-0.048611in}}{\pgfqpoint{0.000000in}{0.000000in}}{%
\pgfpathmoveto{\pgfqpoint{0.000000in}{0.000000in}}%
\pgfpathlineto{\pgfqpoint{0.000000in}{-0.048611in}}%
\pgfusepath{stroke,fill}%
}%
\begin{pgfscope}%
\pgfsys@transformshift{0.477328in}{0.429287in}%
\pgfsys@useobject{currentmarker}{}%
\end{pgfscope}%
\end{pgfscope}%
\begin{pgfscope}%
\definecolor{textcolor}{rgb}{0.180392,0.180392,0.180392}%
\pgfsetstrokecolor{textcolor}%
\pgfsetfillcolor{textcolor}%
\pgftext[x=0.477328in,y=0.332064in,,top]{\color{textcolor}\rmfamily\fontsize{9.000000}{10.800000}\selectfont \(\displaystyle {\ensuremath{-}30}\)}%
\end{pgfscope}%
\begin{pgfscope}%
\pgfpathrectangle{\pgfqpoint{0.477328in}{0.429287in}}{\pgfqpoint{5.480747in}{1.607685in}}%
\pgfusepath{clip}%
\pgfsetbuttcap%
\pgfsetroundjoin%
\pgfsetlinewidth{0.501875pt}%
\definecolor{currentstroke}{rgb}{0.698039,0.698039,0.698039}%
\pgfsetstrokecolor{currentstroke}%
\pgfsetdash{{1.850000pt}{0.800000pt}}{0.000000pt}%
\pgfpathmoveto{\pgfqpoint{1.390785in}{0.429287in}}%
\pgfpathlineto{\pgfqpoint{1.390785in}{2.036972in}}%
\pgfusepath{stroke}%
\end{pgfscope}%
\begin{pgfscope}%
\pgfsetbuttcap%
\pgfsetroundjoin%
\definecolor{currentfill}{rgb}{0.180392,0.180392,0.180392}%
\pgfsetfillcolor{currentfill}%
\pgfsetlinewidth{0.803000pt}%
\definecolor{currentstroke}{rgb}{0.180392,0.180392,0.180392}%
\pgfsetstrokecolor{currentstroke}%
\pgfsetdash{}{0pt}%
\pgfsys@defobject{currentmarker}{\pgfqpoint{0.000000in}{-0.048611in}}{\pgfqpoint{0.000000in}{0.000000in}}{%
\pgfpathmoveto{\pgfqpoint{0.000000in}{0.000000in}}%
\pgfpathlineto{\pgfqpoint{0.000000in}{-0.048611in}}%
\pgfusepath{stroke,fill}%
}%
\begin{pgfscope}%
\pgfsys@transformshift{1.390785in}{0.429287in}%
\pgfsys@useobject{currentmarker}{}%
\end{pgfscope}%
\end{pgfscope}%
\begin{pgfscope}%
\definecolor{textcolor}{rgb}{0.180392,0.180392,0.180392}%
\pgfsetstrokecolor{textcolor}%
\pgfsetfillcolor{textcolor}%
\pgftext[x=1.390785in,y=0.332064in,,top]{\color{textcolor}\rmfamily\fontsize{9.000000}{10.800000}\selectfont \(\displaystyle {\ensuremath{-}20}\)}%
\end{pgfscope}%
\begin{pgfscope}%
\pgfpathrectangle{\pgfqpoint{0.477328in}{0.429287in}}{\pgfqpoint{5.480747in}{1.607685in}}%
\pgfusepath{clip}%
\pgfsetbuttcap%
\pgfsetroundjoin%
\pgfsetlinewidth{0.501875pt}%
\definecolor{currentstroke}{rgb}{0.698039,0.698039,0.698039}%
\pgfsetstrokecolor{currentstroke}%
\pgfsetdash{{1.850000pt}{0.800000pt}}{0.000000pt}%
\pgfpathmoveto{\pgfqpoint{2.304243in}{0.429287in}}%
\pgfpathlineto{\pgfqpoint{2.304243in}{2.036972in}}%
\pgfusepath{stroke}%
\end{pgfscope}%
\begin{pgfscope}%
\pgfsetbuttcap%
\pgfsetroundjoin%
\definecolor{currentfill}{rgb}{0.180392,0.180392,0.180392}%
\pgfsetfillcolor{currentfill}%
\pgfsetlinewidth{0.803000pt}%
\definecolor{currentstroke}{rgb}{0.180392,0.180392,0.180392}%
\pgfsetstrokecolor{currentstroke}%
\pgfsetdash{}{0pt}%
\pgfsys@defobject{currentmarker}{\pgfqpoint{0.000000in}{-0.048611in}}{\pgfqpoint{0.000000in}{0.000000in}}{%
\pgfpathmoveto{\pgfqpoint{0.000000in}{0.000000in}}%
\pgfpathlineto{\pgfqpoint{0.000000in}{-0.048611in}}%
\pgfusepath{stroke,fill}%
}%
\begin{pgfscope}%
\pgfsys@transformshift{2.304243in}{0.429287in}%
\pgfsys@useobject{currentmarker}{}%
\end{pgfscope}%
\end{pgfscope}%
\begin{pgfscope}%
\definecolor{textcolor}{rgb}{0.180392,0.180392,0.180392}%
\pgfsetstrokecolor{textcolor}%
\pgfsetfillcolor{textcolor}%
\pgftext[x=2.304243in,y=0.332064in,,top]{\color{textcolor}\rmfamily\fontsize{9.000000}{10.800000}\selectfont \(\displaystyle {\ensuremath{-}10}\)}%
\end{pgfscope}%
\begin{pgfscope}%
\pgfpathrectangle{\pgfqpoint{0.477328in}{0.429287in}}{\pgfqpoint{5.480747in}{1.607685in}}%
\pgfusepath{clip}%
\pgfsetbuttcap%
\pgfsetroundjoin%
\pgfsetlinewidth{0.501875pt}%
\definecolor{currentstroke}{rgb}{0.698039,0.698039,0.698039}%
\pgfsetstrokecolor{currentstroke}%
\pgfsetdash{{1.850000pt}{0.800000pt}}{0.000000pt}%
\pgfpathmoveto{\pgfqpoint{3.217701in}{0.429287in}}%
\pgfpathlineto{\pgfqpoint{3.217701in}{2.036972in}}%
\pgfusepath{stroke}%
\end{pgfscope}%
\begin{pgfscope}%
\pgfsetbuttcap%
\pgfsetroundjoin%
\definecolor{currentfill}{rgb}{0.180392,0.180392,0.180392}%
\pgfsetfillcolor{currentfill}%
\pgfsetlinewidth{0.803000pt}%
\definecolor{currentstroke}{rgb}{0.180392,0.180392,0.180392}%
\pgfsetstrokecolor{currentstroke}%
\pgfsetdash{}{0pt}%
\pgfsys@defobject{currentmarker}{\pgfqpoint{0.000000in}{-0.048611in}}{\pgfqpoint{0.000000in}{0.000000in}}{%
\pgfpathmoveto{\pgfqpoint{0.000000in}{0.000000in}}%
\pgfpathlineto{\pgfqpoint{0.000000in}{-0.048611in}}%
\pgfusepath{stroke,fill}%
}%
\begin{pgfscope}%
\pgfsys@transformshift{3.217701in}{0.429287in}%
\pgfsys@useobject{currentmarker}{}%
\end{pgfscope}%
\end{pgfscope}%
\begin{pgfscope}%
\definecolor{textcolor}{rgb}{0.180392,0.180392,0.180392}%
\pgfsetstrokecolor{textcolor}%
\pgfsetfillcolor{textcolor}%
\pgftext[x=3.217701in,y=0.332064in,,top]{\color{textcolor}\rmfamily\fontsize{9.000000}{10.800000}\selectfont \(\displaystyle {0}\)}%
\end{pgfscope}%
\begin{pgfscope}%
\pgfpathrectangle{\pgfqpoint{0.477328in}{0.429287in}}{\pgfqpoint{5.480747in}{1.607685in}}%
\pgfusepath{clip}%
\pgfsetbuttcap%
\pgfsetroundjoin%
\pgfsetlinewidth{0.501875pt}%
\definecolor{currentstroke}{rgb}{0.698039,0.698039,0.698039}%
\pgfsetstrokecolor{currentstroke}%
\pgfsetdash{{1.850000pt}{0.800000pt}}{0.000000pt}%
\pgfpathmoveto{\pgfqpoint{4.131159in}{0.429287in}}%
\pgfpathlineto{\pgfqpoint{4.131159in}{2.036972in}}%
\pgfusepath{stroke}%
\end{pgfscope}%
\begin{pgfscope}%
\pgfsetbuttcap%
\pgfsetroundjoin%
\definecolor{currentfill}{rgb}{0.180392,0.180392,0.180392}%
\pgfsetfillcolor{currentfill}%
\pgfsetlinewidth{0.803000pt}%
\definecolor{currentstroke}{rgb}{0.180392,0.180392,0.180392}%
\pgfsetstrokecolor{currentstroke}%
\pgfsetdash{}{0pt}%
\pgfsys@defobject{currentmarker}{\pgfqpoint{0.000000in}{-0.048611in}}{\pgfqpoint{0.000000in}{0.000000in}}{%
\pgfpathmoveto{\pgfqpoint{0.000000in}{0.000000in}}%
\pgfpathlineto{\pgfqpoint{0.000000in}{-0.048611in}}%
\pgfusepath{stroke,fill}%
}%
\begin{pgfscope}%
\pgfsys@transformshift{4.131159in}{0.429287in}%
\pgfsys@useobject{currentmarker}{}%
\end{pgfscope}%
\end{pgfscope}%
\begin{pgfscope}%
\definecolor{textcolor}{rgb}{0.180392,0.180392,0.180392}%
\pgfsetstrokecolor{textcolor}%
\pgfsetfillcolor{textcolor}%
\pgftext[x=4.131159in,y=0.332064in,,top]{\color{textcolor}\rmfamily\fontsize{9.000000}{10.800000}\selectfont \(\displaystyle {10}\)}%
\end{pgfscope}%
\begin{pgfscope}%
\pgfpathrectangle{\pgfqpoint{0.477328in}{0.429287in}}{\pgfqpoint{5.480747in}{1.607685in}}%
\pgfusepath{clip}%
\pgfsetbuttcap%
\pgfsetroundjoin%
\pgfsetlinewidth{0.501875pt}%
\definecolor{currentstroke}{rgb}{0.698039,0.698039,0.698039}%
\pgfsetstrokecolor{currentstroke}%
\pgfsetdash{{1.850000pt}{0.800000pt}}{0.000000pt}%
\pgfpathmoveto{\pgfqpoint{5.044617in}{0.429287in}}%
\pgfpathlineto{\pgfqpoint{5.044617in}{2.036972in}}%
\pgfusepath{stroke}%
\end{pgfscope}%
\begin{pgfscope}%
\pgfsetbuttcap%
\pgfsetroundjoin%
\definecolor{currentfill}{rgb}{0.180392,0.180392,0.180392}%
\pgfsetfillcolor{currentfill}%
\pgfsetlinewidth{0.803000pt}%
\definecolor{currentstroke}{rgb}{0.180392,0.180392,0.180392}%
\pgfsetstrokecolor{currentstroke}%
\pgfsetdash{}{0pt}%
\pgfsys@defobject{currentmarker}{\pgfqpoint{0.000000in}{-0.048611in}}{\pgfqpoint{0.000000in}{0.000000in}}{%
\pgfpathmoveto{\pgfqpoint{0.000000in}{0.000000in}}%
\pgfpathlineto{\pgfqpoint{0.000000in}{-0.048611in}}%
\pgfusepath{stroke,fill}%
}%
\begin{pgfscope}%
\pgfsys@transformshift{5.044617in}{0.429287in}%
\pgfsys@useobject{currentmarker}{}%
\end{pgfscope}%
\end{pgfscope}%
\begin{pgfscope}%
\definecolor{textcolor}{rgb}{0.180392,0.180392,0.180392}%
\pgfsetstrokecolor{textcolor}%
\pgfsetfillcolor{textcolor}%
\pgftext[x=5.044617in,y=0.332064in,,top]{\color{textcolor}\rmfamily\fontsize{9.000000}{10.800000}\selectfont \(\displaystyle {20}\)}%
\end{pgfscope}%
\begin{pgfscope}%
\pgfpathrectangle{\pgfqpoint{0.477328in}{0.429287in}}{\pgfqpoint{5.480747in}{1.607685in}}%
\pgfusepath{clip}%
\pgfsetbuttcap%
\pgfsetroundjoin%
\pgfsetlinewidth{0.501875pt}%
\definecolor{currentstroke}{rgb}{0.698039,0.698039,0.698039}%
\pgfsetstrokecolor{currentstroke}%
\pgfsetdash{{1.850000pt}{0.800000pt}}{0.000000pt}%
\pgfpathmoveto{\pgfqpoint{5.958075in}{0.429287in}}%
\pgfpathlineto{\pgfqpoint{5.958075in}{2.036972in}}%
\pgfusepath{stroke}%
\end{pgfscope}%
\begin{pgfscope}%
\pgfsetbuttcap%
\pgfsetroundjoin%
\definecolor{currentfill}{rgb}{0.180392,0.180392,0.180392}%
\pgfsetfillcolor{currentfill}%
\pgfsetlinewidth{0.803000pt}%
\definecolor{currentstroke}{rgb}{0.180392,0.180392,0.180392}%
\pgfsetstrokecolor{currentstroke}%
\pgfsetdash{}{0pt}%
\pgfsys@defobject{currentmarker}{\pgfqpoint{0.000000in}{-0.048611in}}{\pgfqpoint{0.000000in}{0.000000in}}{%
\pgfpathmoveto{\pgfqpoint{0.000000in}{0.000000in}}%
\pgfpathlineto{\pgfqpoint{0.000000in}{-0.048611in}}%
\pgfusepath{stroke,fill}%
}%
\begin{pgfscope}%
\pgfsys@transformshift{5.958075in}{0.429287in}%
\pgfsys@useobject{currentmarker}{}%
\end{pgfscope}%
\end{pgfscope}%
\begin{pgfscope}%
\definecolor{textcolor}{rgb}{0.180392,0.180392,0.180392}%
\pgfsetstrokecolor{textcolor}%
\pgfsetfillcolor{textcolor}%
\pgftext[x=5.958075in,y=0.332064in,,top]{\color{textcolor}\rmfamily\fontsize{9.000000}{10.800000}\selectfont \(\displaystyle {30}\)}%
\end{pgfscope}%
\begin{pgfscope}%
\definecolor{textcolor}{rgb}{0.180392,0.180392,0.180392}%
\pgfsetstrokecolor{textcolor}%
\pgfsetfillcolor{textcolor}%
\pgftext[x=3.217701in,y=0.150823in,,top]{\color{textcolor}\rmfamily\fontsize{10.800000}{12.960000}\selectfont Angular Frequency, \(\displaystyle \omega\)}%
\end{pgfscope}%
\begin{pgfscope}%
\pgfpathrectangle{\pgfqpoint{0.477328in}{0.429287in}}{\pgfqpoint{5.480747in}{1.607685in}}%
\pgfusepath{clip}%
\pgfsetbuttcap%
\pgfsetroundjoin%
\pgfsetlinewidth{0.501875pt}%
\definecolor{currentstroke}{rgb}{0.698039,0.698039,0.698039}%
\pgfsetstrokecolor{currentstroke}%
\pgfsetdash{{1.850000pt}{0.800000pt}}{0.000000pt}%
\pgfpathmoveto{\pgfqpoint{0.477328in}{0.502363in}}%
\pgfpathlineto{\pgfqpoint{5.958075in}{0.502363in}}%
\pgfusepath{stroke}%
\end{pgfscope}%
\begin{pgfscope}%
\pgfsetbuttcap%
\pgfsetroundjoin%
\definecolor{currentfill}{rgb}{0.180392,0.180392,0.180392}%
\pgfsetfillcolor{currentfill}%
\pgfsetlinewidth{0.803000pt}%
\definecolor{currentstroke}{rgb}{0.180392,0.180392,0.180392}%
\pgfsetstrokecolor{currentstroke}%
\pgfsetdash{}{0pt}%
\pgfsys@defobject{currentmarker}{\pgfqpoint{-0.048611in}{0.000000in}}{\pgfqpoint{-0.000000in}{0.000000in}}{%
\pgfpathmoveto{\pgfqpoint{-0.000000in}{0.000000in}}%
\pgfpathlineto{\pgfqpoint{-0.048611in}{0.000000in}}%
\pgfusepath{stroke,fill}%
}%
\begin{pgfscope}%
\pgfsys@transformshift{0.477328in}{0.502363in}%
\pgfsys@useobject{currentmarker}{}%
\end{pgfscope}%
\end{pgfscope}%
\begin{pgfscope}%
\definecolor{textcolor}{rgb}{0.180392,0.180392,0.180392}%
\pgfsetstrokecolor{textcolor}%
\pgfsetfillcolor{textcolor}%
\pgftext[x=0.223855in, y=0.457145in, left, base]{\color{textcolor}\rmfamily\fontsize{9.000000}{10.800000}\selectfont \(\displaystyle {0.0}\)}%
\end{pgfscope}%
\begin{pgfscope}%
\pgfpathrectangle{\pgfqpoint{0.477328in}{0.429287in}}{\pgfqpoint{5.480747in}{1.607685in}}%
\pgfusepath{clip}%
\pgfsetbuttcap%
\pgfsetroundjoin%
\pgfsetlinewidth{0.501875pt}%
\definecolor{currentstroke}{rgb}{0.698039,0.698039,0.698039}%
\pgfsetstrokecolor{currentstroke}%
\pgfsetdash{{1.850000pt}{0.800000pt}}{0.000000pt}%
\pgfpathmoveto{\pgfqpoint{0.477328in}{0.794670in}}%
\pgfpathlineto{\pgfqpoint{5.958075in}{0.794670in}}%
\pgfusepath{stroke}%
\end{pgfscope}%
\begin{pgfscope}%
\pgfsetbuttcap%
\pgfsetroundjoin%
\definecolor{currentfill}{rgb}{0.180392,0.180392,0.180392}%
\pgfsetfillcolor{currentfill}%
\pgfsetlinewidth{0.803000pt}%
\definecolor{currentstroke}{rgb}{0.180392,0.180392,0.180392}%
\pgfsetstrokecolor{currentstroke}%
\pgfsetdash{}{0pt}%
\pgfsys@defobject{currentmarker}{\pgfqpoint{-0.048611in}{0.000000in}}{\pgfqpoint{-0.000000in}{0.000000in}}{%
\pgfpathmoveto{\pgfqpoint{-0.000000in}{0.000000in}}%
\pgfpathlineto{\pgfqpoint{-0.048611in}{0.000000in}}%
\pgfusepath{stroke,fill}%
}%
\begin{pgfscope}%
\pgfsys@transformshift{0.477328in}{0.794670in}%
\pgfsys@useobject{currentmarker}{}%
\end{pgfscope}%
\end{pgfscope}%
\begin{pgfscope}%
\definecolor{textcolor}{rgb}{0.180392,0.180392,0.180392}%
\pgfsetstrokecolor{textcolor}%
\pgfsetfillcolor{textcolor}%
\pgftext[x=0.223855in, y=0.749451in, left, base]{\color{textcolor}\rmfamily\fontsize{9.000000}{10.800000}\selectfont \(\displaystyle {0.2}\)}%
\end{pgfscope}%
\begin{pgfscope}%
\pgfpathrectangle{\pgfqpoint{0.477328in}{0.429287in}}{\pgfqpoint{5.480747in}{1.607685in}}%
\pgfusepath{clip}%
\pgfsetbuttcap%
\pgfsetroundjoin%
\pgfsetlinewidth{0.501875pt}%
\definecolor{currentstroke}{rgb}{0.698039,0.698039,0.698039}%
\pgfsetstrokecolor{currentstroke}%
\pgfsetdash{{1.850000pt}{0.800000pt}}{0.000000pt}%
\pgfpathmoveto{\pgfqpoint{0.477328in}{1.086976in}}%
\pgfpathlineto{\pgfqpoint{5.958075in}{1.086976in}}%
\pgfusepath{stroke}%
\end{pgfscope}%
\begin{pgfscope}%
\pgfsetbuttcap%
\pgfsetroundjoin%
\definecolor{currentfill}{rgb}{0.180392,0.180392,0.180392}%
\pgfsetfillcolor{currentfill}%
\pgfsetlinewidth{0.803000pt}%
\definecolor{currentstroke}{rgb}{0.180392,0.180392,0.180392}%
\pgfsetstrokecolor{currentstroke}%
\pgfsetdash{}{0pt}%
\pgfsys@defobject{currentmarker}{\pgfqpoint{-0.048611in}{0.000000in}}{\pgfqpoint{-0.000000in}{0.000000in}}{%
\pgfpathmoveto{\pgfqpoint{-0.000000in}{0.000000in}}%
\pgfpathlineto{\pgfqpoint{-0.048611in}{0.000000in}}%
\pgfusepath{stroke,fill}%
}%
\begin{pgfscope}%
\pgfsys@transformshift{0.477328in}{1.086976in}%
\pgfsys@useobject{currentmarker}{}%
\end{pgfscope}%
\end{pgfscope}%
\begin{pgfscope}%
\definecolor{textcolor}{rgb}{0.180392,0.180392,0.180392}%
\pgfsetstrokecolor{textcolor}%
\pgfsetfillcolor{textcolor}%
\pgftext[x=0.223855in, y=1.041758in, left, base]{\color{textcolor}\rmfamily\fontsize{9.000000}{10.800000}\selectfont \(\displaystyle {0.4}\)}%
\end{pgfscope}%
\begin{pgfscope}%
\pgfpathrectangle{\pgfqpoint{0.477328in}{0.429287in}}{\pgfqpoint{5.480747in}{1.607685in}}%
\pgfusepath{clip}%
\pgfsetbuttcap%
\pgfsetroundjoin%
\pgfsetlinewidth{0.501875pt}%
\definecolor{currentstroke}{rgb}{0.698039,0.698039,0.698039}%
\pgfsetstrokecolor{currentstroke}%
\pgfsetdash{{1.850000pt}{0.800000pt}}{0.000000pt}%
\pgfpathmoveto{\pgfqpoint{0.477328in}{1.379283in}}%
\pgfpathlineto{\pgfqpoint{5.958075in}{1.379283in}}%
\pgfusepath{stroke}%
\end{pgfscope}%
\begin{pgfscope}%
\pgfsetbuttcap%
\pgfsetroundjoin%
\definecolor{currentfill}{rgb}{0.180392,0.180392,0.180392}%
\pgfsetfillcolor{currentfill}%
\pgfsetlinewidth{0.803000pt}%
\definecolor{currentstroke}{rgb}{0.180392,0.180392,0.180392}%
\pgfsetstrokecolor{currentstroke}%
\pgfsetdash{}{0pt}%
\pgfsys@defobject{currentmarker}{\pgfqpoint{-0.048611in}{0.000000in}}{\pgfqpoint{-0.000000in}{0.000000in}}{%
\pgfpathmoveto{\pgfqpoint{-0.000000in}{0.000000in}}%
\pgfpathlineto{\pgfqpoint{-0.048611in}{0.000000in}}%
\pgfusepath{stroke,fill}%
}%
\begin{pgfscope}%
\pgfsys@transformshift{0.477328in}{1.379283in}%
\pgfsys@useobject{currentmarker}{}%
\end{pgfscope}%
\end{pgfscope}%
\begin{pgfscope}%
\definecolor{textcolor}{rgb}{0.180392,0.180392,0.180392}%
\pgfsetstrokecolor{textcolor}%
\pgfsetfillcolor{textcolor}%
\pgftext[x=0.223855in, y=1.334064in, left, base]{\color{textcolor}\rmfamily\fontsize{9.000000}{10.800000}\selectfont \(\displaystyle {0.6}\)}%
\end{pgfscope}%
\begin{pgfscope}%
\pgfpathrectangle{\pgfqpoint{0.477328in}{0.429287in}}{\pgfqpoint{5.480747in}{1.607685in}}%
\pgfusepath{clip}%
\pgfsetbuttcap%
\pgfsetroundjoin%
\pgfsetlinewidth{0.501875pt}%
\definecolor{currentstroke}{rgb}{0.698039,0.698039,0.698039}%
\pgfsetstrokecolor{currentstroke}%
\pgfsetdash{{1.850000pt}{0.800000pt}}{0.000000pt}%
\pgfpathmoveto{\pgfqpoint{0.477328in}{1.671589in}}%
\pgfpathlineto{\pgfqpoint{5.958075in}{1.671589in}}%
\pgfusepath{stroke}%
\end{pgfscope}%
\begin{pgfscope}%
\pgfsetbuttcap%
\pgfsetroundjoin%
\definecolor{currentfill}{rgb}{0.180392,0.180392,0.180392}%
\pgfsetfillcolor{currentfill}%
\pgfsetlinewidth{0.803000pt}%
\definecolor{currentstroke}{rgb}{0.180392,0.180392,0.180392}%
\pgfsetstrokecolor{currentstroke}%
\pgfsetdash{}{0pt}%
\pgfsys@defobject{currentmarker}{\pgfqpoint{-0.048611in}{0.000000in}}{\pgfqpoint{-0.000000in}{0.000000in}}{%
\pgfpathmoveto{\pgfqpoint{-0.000000in}{0.000000in}}%
\pgfpathlineto{\pgfqpoint{-0.048611in}{0.000000in}}%
\pgfusepath{stroke,fill}%
}%
\begin{pgfscope}%
\pgfsys@transformshift{0.477328in}{1.671589in}%
\pgfsys@useobject{currentmarker}{}%
\end{pgfscope}%
\end{pgfscope}%
\begin{pgfscope}%
\definecolor{textcolor}{rgb}{0.180392,0.180392,0.180392}%
\pgfsetstrokecolor{textcolor}%
\pgfsetfillcolor{textcolor}%
\pgftext[x=0.223855in, y=1.626371in, left, base]{\color{textcolor}\rmfamily\fontsize{9.000000}{10.800000}\selectfont \(\displaystyle {0.8}\)}%
\end{pgfscope}%
\begin{pgfscope}%
\pgfpathrectangle{\pgfqpoint{0.477328in}{0.429287in}}{\pgfqpoint{5.480747in}{1.607685in}}%
\pgfusepath{clip}%
\pgfsetbuttcap%
\pgfsetroundjoin%
\pgfsetlinewidth{0.501875pt}%
\definecolor{currentstroke}{rgb}{0.698039,0.698039,0.698039}%
\pgfsetstrokecolor{currentstroke}%
\pgfsetdash{{1.850000pt}{0.800000pt}}{0.000000pt}%
\pgfpathmoveto{\pgfqpoint{0.477328in}{1.963896in}}%
\pgfpathlineto{\pgfqpoint{5.958075in}{1.963896in}}%
\pgfusepath{stroke}%
\end{pgfscope}%
\begin{pgfscope}%
\pgfsetbuttcap%
\pgfsetroundjoin%
\definecolor{currentfill}{rgb}{0.180392,0.180392,0.180392}%
\pgfsetfillcolor{currentfill}%
\pgfsetlinewidth{0.803000pt}%
\definecolor{currentstroke}{rgb}{0.180392,0.180392,0.180392}%
\pgfsetstrokecolor{currentstroke}%
\pgfsetdash{}{0pt}%
\pgfsys@defobject{currentmarker}{\pgfqpoint{-0.048611in}{0.000000in}}{\pgfqpoint{-0.000000in}{0.000000in}}{%
\pgfpathmoveto{\pgfqpoint{-0.000000in}{0.000000in}}%
\pgfpathlineto{\pgfqpoint{-0.048611in}{0.000000in}}%
\pgfusepath{stroke,fill}%
}%
\begin{pgfscope}%
\pgfsys@transformshift{0.477328in}{1.963896in}%
\pgfsys@useobject{currentmarker}{}%
\end{pgfscope}%
\end{pgfscope}%
\begin{pgfscope}%
\definecolor{textcolor}{rgb}{0.180392,0.180392,0.180392}%
\pgfsetstrokecolor{textcolor}%
\pgfsetfillcolor{textcolor}%
\pgftext[x=0.223855in, y=1.918677in, left, base]{\color{textcolor}\rmfamily\fontsize{9.000000}{10.800000}\selectfont \(\displaystyle {1.0}\)}%
\end{pgfscope}%
\begin{pgfscope}%
\definecolor{textcolor}{rgb}{0.180392,0.180392,0.180392}%
\pgfsetstrokecolor{textcolor}%
\pgfsetfillcolor{textcolor}%
\pgftext[x=0.168300in,y=1.233129in,,bottom,rotate=90.000000]{\color{textcolor}\rmfamily\fontsize{10.800000}{12.960000}\selectfont \(\displaystyle \tilde{E}(\omega)\)}%
\end{pgfscope}%
\begin{pgfscope}%
\pgfpathrectangle{\pgfqpoint{0.477328in}{0.429287in}}{\pgfqpoint{5.480747in}{1.607685in}}%
\pgfusepath{clip}%
\pgfsetrectcap%
\pgfsetroundjoin%
\pgfsetlinewidth{2.007500pt}%
\definecolor{currentstroke}{rgb}{0.000000,0.501961,0.000000}%
\pgfsetstrokecolor{currentstroke}%
\pgfsetdash{}{0pt}%
\pgfpathmoveto{\pgfqpoint{0.474256in}{0.502372in}}%
\pgfpathlineto{\pgfqpoint{0.566087in}{0.503104in}}%
\pgfpathlineto{\pgfqpoint{0.600523in}{0.505358in}}%
\pgfpathlineto{\pgfqpoint{0.623481in}{0.509390in}}%
\pgfpathlineto{\pgfqpoint{0.634960in}{0.512874in}}%
\pgfpathlineto{\pgfqpoint{0.646439in}{0.517840in}}%
\pgfpathlineto{\pgfqpoint{0.657918in}{0.524794in}}%
\pgfpathlineto{\pgfqpoint{0.669397in}{0.534363in}}%
\pgfpathlineto{\pgfqpoint{0.680875in}{0.547300in}}%
\pgfpathlineto{\pgfqpoint{0.692354in}{0.564478in}}%
\pgfpathlineto{\pgfqpoint{0.703833in}{0.586877in}}%
\pgfpathlineto{\pgfqpoint{0.715312in}{0.615551in}}%
\pgfpathlineto{\pgfqpoint{0.726791in}{0.651580in}}%
\pgfpathlineto{\pgfqpoint{0.738270in}{0.695995in}}%
\pgfpathlineto{\pgfqpoint{0.749748in}{0.749693in}}%
\pgfpathlineto{\pgfqpoint{0.761227in}{0.813333in}}%
\pgfpathlineto{\pgfqpoint{0.772706in}{0.887223in}}%
\pgfpathlineto{\pgfqpoint{0.784185in}{0.971207in}}%
\pgfpathlineto{\pgfqpoint{0.807143in}{1.165963in}}%
\pgfpathlineto{\pgfqpoint{0.864537in}{1.702093in}}%
\pgfpathlineto{\pgfqpoint{0.876016in}{1.790449in}}%
\pgfpathlineto{\pgfqpoint{0.887495in}{1.863644in}}%
\pgfpathlineto{\pgfqpoint{0.898973in}{1.918459in}}%
\pgfpathlineto{\pgfqpoint{0.910452in}{1.952402in}}%
\pgfpathlineto{\pgfqpoint{0.921931in}{1.963895in}}%
\pgfpathlineto{\pgfqpoint{0.933410in}{1.952400in}}%
\pgfpathlineto{\pgfqpoint{0.944889in}{1.918456in}}%
\pgfpathlineto{\pgfqpoint{0.956368in}{1.863639in}}%
\pgfpathlineto{\pgfqpoint{0.967847in}{1.790442in}}%
\pgfpathlineto{\pgfqpoint{0.979325in}{1.702086in}}%
\pgfpathlineto{\pgfqpoint{1.002283in}{1.494983in}}%
\pgfpathlineto{\pgfqpoint{1.036720in}{1.165954in}}%
\pgfpathlineto{\pgfqpoint{1.059677in}{0.971200in}}%
\pgfpathlineto{\pgfqpoint{1.071156in}{0.887217in}}%
\pgfpathlineto{\pgfqpoint{1.082635in}{0.813328in}}%
\pgfpathlineto{\pgfqpoint{1.094114in}{0.749688in}}%
\pgfpathlineto{\pgfqpoint{1.105593in}{0.695991in}}%
\pgfpathlineto{\pgfqpoint{1.117072in}{0.651577in}}%
\pgfpathlineto{\pgfqpoint{1.128550in}{0.615549in}}%
\pgfpathlineto{\pgfqpoint{1.140029in}{0.586875in}}%
\pgfpathlineto{\pgfqpoint{1.151508in}{0.564476in}}%
\pgfpathlineto{\pgfqpoint{1.162987in}{0.547299in}}%
\pgfpathlineto{\pgfqpoint{1.174466in}{0.534363in}}%
\pgfpathlineto{\pgfqpoint{1.185945in}{0.524793in}}%
\pgfpathlineto{\pgfqpoint{1.197424in}{0.517839in}}%
\pgfpathlineto{\pgfqpoint{1.208902in}{0.512874in}}%
\pgfpathlineto{\pgfqpoint{1.231860in}{0.506987in}}%
\pgfpathlineto{\pgfqpoint{1.254818in}{0.504273in}}%
\pgfpathlineto{\pgfqpoint{1.289254in}{0.502813in}}%
\pgfpathlineto{\pgfqpoint{1.381085in}{0.502368in}}%
\pgfpathlineto{\pgfqpoint{5.134669in}{0.502633in}}%
\pgfpathlineto{\pgfqpoint{5.180584in}{0.504273in}}%
\pgfpathlineto{\pgfqpoint{5.203542in}{0.506987in}}%
\pgfpathlineto{\pgfqpoint{5.226500in}{0.512874in}}%
\pgfpathlineto{\pgfqpoint{5.237979in}{0.517839in}}%
\pgfpathlineto{\pgfqpoint{5.249457in}{0.524793in}}%
\pgfpathlineto{\pgfqpoint{5.260936in}{0.534363in}}%
\pgfpathlineto{\pgfqpoint{5.272415in}{0.547299in}}%
\pgfpathlineto{\pgfqpoint{5.283894in}{0.564476in}}%
\pgfpathlineto{\pgfqpoint{5.295373in}{0.586875in}}%
\pgfpathlineto{\pgfqpoint{5.306852in}{0.615549in}}%
\pgfpathlineto{\pgfqpoint{5.318331in}{0.651577in}}%
\pgfpathlineto{\pgfqpoint{5.329809in}{0.695991in}}%
\pgfpathlineto{\pgfqpoint{5.341288in}{0.749688in}}%
\pgfpathlineto{\pgfqpoint{5.352767in}{0.813328in}}%
\pgfpathlineto{\pgfqpoint{5.364246in}{0.887217in}}%
\pgfpathlineto{\pgfqpoint{5.375725in}{0.971200in}}%
\pgfpathlineto{\pgfqpoint{5.398682in}{1.165954in}}%
\pgfpathlineto{\pgfqpoint{5.456077in}{1.702086in}}%
\pgfpathlineto{\pgfqpoint{5.467556in}{1.790442in}}%
\pgfpathlineto{\pgfqpoint{5.479034in}{1.863639in}}%
\pgfpathlineto{\pgfqpoint{5.490513in}{1.918456in}}%
\pgfpathlineto{\pgfqpoint{5.501992in}{1.952400in}}%
\pgfpathlineto{\pgfqpoint{5.513471in}{1.963895in}}%
\pgfpathlineto{\pgfqpoint{5.524950in}{1.952402in}}%
\pgfpathlineto{\pgfqpoint{5.536429in}{1.918459in}}%
\pgfpathlineto{\pgfqpoint{5.547908in}{1.863644in}}%
\pgfpathlineto{\pgfqpoint{5.559386in}{1.790449in}}%
\pgfpathlineto{\pgfqpoint{5.570865in}{1.702093in}}%
\pgfpathlineto{\pgfqpoint{5.593823in}{1.494992in}}%
\pgfpathlineto{\pgfqpoint{5.628259in}{1.165963in}}%
\pgfpathlineto{\pgfqpoint{5.651217in}{0.971207in}}%
\pgfpathlineto{\pgfqpoint{5.662696in}{0.887223in}}%
\pgfpathlineto{\pgfqpoint{5.674175in}{0.813333in}}%
\pgfpathlineto{\pgfqpoint{5.685654in}{0.749693in}}%
\pgfpathlineto{\pgfqpoint{5.697133in}{0.695995in}}%
\pgfpathlineto{\pgfqpoint{5.708611in}{0.651580in}}%
\pgfpathlineto{\pgfqpoint{5.720090in}{0.615551in}}%
\pgfpathlineto{\pgfqpoint{5.731569in}{0.586877in}}%
\pgfpathlineto{\pgfqpoint{5.743048in}{0.564478in}}%
\pgfpathlineto{\pgfqpoint{5.754527in}{0.547300in}}%
\pgfpathlineto{\pgfqpoint{5.766006in}{0.534363in}}%
\pgfpathlineto{\pgfqpoint{5.777484in}{0.524794in}}%
\pgfpathlineto{\pgfqpoint{5.788963in}{0.517840in}}%
\pgfpathlineto{\pgfqpoint{5.800442in}{0.512874in}}%
\pgfpathlineto{\pgfqpoint{5.823400in}{0.506987in}}%
\pgfpathlineto{\pgfqpoint{5.846358in}{0.504273in}}%
\pgfpathlineto{\pgfqpoint{5.880794in}{0.502813in}}%
\pgfpathlineto{\pgfqpoint{5.961146in}{0.502372in}}%
\pgfpathlineto{\pgfqpoint{5.961146in}{0.502372in}}%
\pgfusepath{stroke}%
\end{pgfscope}%
\begin{pgfscope}%
\pgfsetrectcap%
\pgfsetmiterjoin%
\pgfsetlinewidth{0.803000pt}%
\definecolor{currentstroke}{rgb}{0.737255,0.737255,0.737255}%
\pgfsetstrokecolor{currentstroke}%
\pgfsetdash{}{0pt}%
\pgfpathmoveto{\pgfqpoint{0.477328in}{0.429287in}}%
\pgfpathlineto{\pgfqpoint{0.477328in}{2.036972in}}%
\pgfusepath{stroke}%
\end{pgfscope}%
\begin{pgfscope}%
\pgfsetrectcap%
\pgfsetmiterjoin%
\pgfsetlinewidth{0.803000pt}%
\definecolor{currentstroke}{rgb}{0.737255,0.737255,0.737255}%
\pgfsetstrokecolor{currentstroke}%
\pgfsetdash{}{0pt}%
\pgfpathmoveto{\pgfqpoint{5.958075in}{0.429287in}}%
\pgfpathlineto{\pgfqpoint{5.958075in}{2.036972in}}%
\pgfusepath{stroke}%
\end{pgfscope}%
\begin{pgfscope}%
\pgfsetrectcap%
\pgfsetmiterjoin%
\pgfsetlinewidth{0.803000pt}%
\definecolor{currentstroke}{rgb}{0.737255,0.737255,0.737255}%
\pgfsetstrokecolor{currentstroke}%
\pgfsetdash{}{0pt}%
\pgfpathmoveto{\pgfqpoint{0.477328in}{0.429287in}}%
\pgfpathlineto{\pgfqpoint{5.958075in}{0.429287in}}%
\pgfusepath{stroke}%
\end{pgfscope}%
\begin{pgfscope}%
\pgfsetrectcap%
\pgfsetmiterjoin%
\pgfsetlinewidth{0.803000pt}%
\definecolor{currentstroke}{rgb}{0.737255,0.737255,0.737255}%
\pgfsetstrokecolor{currentstroke}%
\pgfsetdash{}{0pt}%
\pgfpathmoveto{\pgfqpoint{0.477328in}{2.036972in}}%
\pgfpathlineto{\pgfqpoint{5.958075in}{2.036972in}}%
\pgfusepath{stroke}%
\end{pgfscope}%
\begin{pgfscope}%
\definecolor{textcolor}{rgb}{0.180392,0.180392,0.180392}%
\pgfsetstrokecolor{textcolor}%
\pgfsetfillcolor{textcolor}%
\pgftext[x=3.217701in,y=2.120305in,,base]{\color{textcolor}\rmfamily\fontsize{12.960000}{15.552000}\selectfont Pulsed Gaussian Spectrum}%
\end{pgfscope}%
\end{pgfpicture}%
\makeatother%
\endgroup%
}
		\caption{\centering{The Gaussian Pulse spectrum}}
		\label{fig:gaussian_spectrum}
	\end{figure}
	
	\vspace{5mm}
	
	Note that the two Gaussians are centred at the carrier frequency and that the waist frequency interval at 1/e intensity ($\Delta\omega$) must be significantly smaller than the carrier frequency of the signal ($\omega_0$), i.e. ${\Delta\omega}\ll{\omega_0}$. This is shown in the normalised frequency spectrum graph, Figure \ref{fig:gaussian_spectrum_normalised}: \linebreak
	
	\begin{figure}[h]
		\centering
		\scalebox{0.85}{%% Creator: Matplotlib, PGF backend
%%
%% To include the figure in your LaTeX document, write
%%   \input{<filename>.pgf}
%%
%% Make sure the required packages are loaded in your preamble
%%   \usepackage{pgf}
%%
%% Also ensure that all the required font packages are loaded; for instance,
%% the lmodern package is sometimes necessary when using math font.
%%   \usepackage{lmodern}
%%
%% Figures using additional raster images can only be included by \input if
%% they are in the same directory as the main LaTeX file. For loading figures
%% from other directories you can use the `import` package
%%   \usepackage{import}
%%
%% and then include the figures with
%%   \import{<path to file>}{<filename>.pgf}
%%
%% Matplotlib used the following preamble
%%   \usepackage[T1]{fontenc} \usepackage{mathpazo}
%%
\begingroup%
\makeatletter%
\begin{pgfpicture}%
\pgfpathrectangle{\pgfpointorigin}{\pgfqpoint{6.086066in}{4.655390in}}%
\pgfusepath{use as bounding box, clip}%
\begin{pgfscope}%
\pgfsetbuttcap%
\pgfsetmiterjoin%
\definecolor{currentfill}{rgb}{1.000000,1.000000,1.000000}%
\pgfsetfillcolor{currentfill}%
\pgfsetlinewidth{0.000000pt}%
\definecolor{currentstroke}{rgb}{1.000000,1.000000,1.000000}%
\pgfsetstrokecolor{currentstroke}%
\pgfsetdash{}{0pt}%
\pgfpathmoveto{\pgfqpoint{0.000000in}{0.000000in}}%
\pgfpathlineto{\pgfqpoint{6.086066in}{0.000000in}}%
\pgfpathlineto{\pgfqpoint{6.086066in}{4.655390in}}%
\pgfpathlineto{\pgfqpoint{0.000000in}{4.655390in}}%
\pgfpathlineto{\pgfqpoint{0.000000in}{0.000000in}}%
\pgfpathclose%
\pgfusepath{fill}%
\end{pgfscope}%
\begin{pgfscope}%
\pgfsetbuttcap%
\pgfsetmiterjoin%
\definecolor{currentfill}{rgb}{0.933333,0.933333,0.933333}%
\pgfsetfillcolor{currentfill}%
\pgfsetlinewidth{0.000000pt}%
\definecolor{currentstroke}{rgb}{0.000000,0.000000,0.000000}%
\pgfsetstrokecolor{currentstroke}%
\pgfsetstrokeopacity{0.000000}%
\pgfsetdash{}{0pt}%
\pgfpathmoveto{\pgfqpoint{0.477328in}{0.583522in}}%
\pgfpathlineto{\pgfqpoint{6.086066in}{0.583522in}}%
\pgfpathlineto{\pgfqpoint{6.086066in}{4.441828in}}%
\pgfpathlineto{\pgfqpoint{0.477328in}{4.441828in}}%
\pgfpathlineto{\pgfqpoint{0.477328in}{0.583522in}}%
\pgfpathclose%
\pgfusepath{fill}%
\end{pgfscope}%
\begin{pgfscope}%
\pgfpathrectangle{\pgfqpoint{0.477328in}{0.583522in}}{\pgfqpoint{5.608738in}{3.858305in}}%
\pgfusepath{clip}%
\pgfsetbuttcap%
\pgfsetroundjoin%
\pgfsetlinewidth{0.501875pt}%
\definecolor{currentstroke}{rgb}{0.698039,0.698039,0.698039}%
\pgfsetstrokecolor{currentstroke}%
\pgfsetdash{{1.850000pt}{0.800000pt}}{0.000000pt}%
\pgfpathmoveto{\pgfqpoint{1.124490in}{0.583522in}}%
\pgfpathlineto{\pgfqpoint{1.124490in}{4.441828in}}%
\pgfusepath{stroke}%
\end{pgfscope}%
\begin{pgfscope}%
\pgfsetbuttcap%
\pgfsetroundjoin%
\definecolor{currentfill}{rgb}{0.180392,0.180392,0.180392}%
\pgfsetfillcolor{currentfill}%
\pgfsetlinewidth{0.803000pt}%
\definecolor{currentstroke}{rgb}{0.180392,0.180392,0.180392}%
\pgfsetstrokecolor{currentstroke}%
\pgfsetdash{}{0pt}%
\pgfsys@defobject{currentmarker}{\pgfqpoint{0.000000in}{-0.048611in}}{\pgfqpoint{0.000000in}{0.000000in}}{%
\pgfpathmoveto{\pgfqpoint{0.000000in}{0.000000in}}%
\pgfpathlineto{\pgfqpoint{0.000000in}{-0.048611in}}%
\pgfusepath{stroke,fill}%
}%
\begin{pgfscope}%
\pgfsys@transformshift{1.124490in}{0.583522in}%
\pgfsys@useobject{currentmarker}{}%
\end{pgfscope}%
\end{pgfscope}%
\begin{pgfscope}%
\definecolor{textcolor}{rgb}{0.180392,0.180392,0.180392}%
\pgfsetstrokecolor{textcolor}%
\pgfsetfillcolor{textcolor}%
\pgftext[x=1.124490in,y=0.486300in,,top]{\color{textcolor}\rmfamily\fontsize{9.000000}{10.800000}\selectfont \(\displaystyle {\ensuremath{-}1.0}\)}%
\end{pgfscope}%
\begin{pgfscope}%
\pgfpathrectangle{\pgfqpoint{0.477328in}{0.583522in}}{\pgfqpoint{5.608738in}{3.858305in}}%
\pgfusepath{clip}%
\pgfsetbuttcap%
\pgfsetroundjoin%
\pgfsetlinewidth{0.501875pt}%
\definecolor{currentstroke}{rgb}{0.698039,0.698039,0.698039}%
\pgfsetstrokecolor{currentstroke}%
\pgfsetdash{{1.850000pt}{0.800000pt}}{0.000000pt}%
\pgfpathmoveto{\pgfqpoint{2.203093in}{0.583522in}}%
\pgfpathlineto{\pgfqpoint{2.203093in}{4.441828in}}%
\pgfusepath{stroke}%
\end{pgfscope}%
\begin{pgfscope}%
\pgfsetbuttcap%
\pgfsetroundjoin%
\definecolor{currentfill}{rgb}{0.180392,0.180392,0.180392}%
\pgfsetfillcolor{currentfill}%
\pgfsetlinewidth{0.803000pt}%
\definecolor{currentstroke}{rgb}{0.180392,0.180392,0.180392}%
\pgfsetstrokecolor{currentstroke}%
\pgfsetdash{}{0pt}%
\pgfsys@defobject{currentmarker}{\pgfqpoint{0.000000in}{-0.048611in}}{\pgfqpoint{0.000000in}{0.000000in}}{%
\pgfpathmoveto{\pgfqpoint{0.000000in}{0.000000in}}%
\pgfpathlineto{\pgfqpoint{0.000000in}{-0.048611in}}%
\pgfusepath{stroke,fill}%
}%
\begin{pgfscope}%
\pgfsys@transformshift{2.203093in}{0.583522in}%
\pgfsys@useobject{currentmarker}{}%
\end{pgfscope}%
\end{pgfscope}%
\begin{pgfscope}%
\definecolor{textcolor}{rgb}{0.180392,0.180392,0.180392}%
\pgfsetstrokecolor{textcolor}%
\pgfsetfillcolor{textcolor}%
\pgftext[x=2.203093in,y=0.486300in,,top]{\color{textcolor}\rmfamily\fontsize{9.000000}{10.800000}\selectfont \(\displaystyle {\ensuremath{-}0.5}\)}%
\end{pgfscope}%
\begin{pgfscope}%
\pgfpathrectangle{\pgfqpoint{0.477328in}{0.583522in}}{\pgfqpoint{5.608738in}{3.858305in}}%
\pgfusepath{clip}%
\pgfsetbuttcap%
\pgfsetroundjoin%
\pgfsetlinewidth{0.501875pt}%
\definecolor{currentstroke}{rgb}{0.698039,0.698039,0.698039}%
\pgfsetstrokecolor{currentstroke}%
\pgfsetdash{{1.850000pt}{0.800000pt}}{0.000000pt}%
\pgfpathmoveto{\pgfqpoint{3.281697in}{0.583522in}}%
\pgfpathlineto{\pgfqpoint{3.281697in}{4.441828in}}%
\pgfusepath{stroke}%
\end{pgfscope}%
\begin{pgfscope}%
\pgfsetbuttcap%
\pgfsetroundjoin%
\definecolor{currentfill}{rgb}{0.180392,0.180392,0.180392}%
\pgfsetfillcolor{currentfill}%
\pgfsetlinewidth{0.803000pt}%
\definecolor{currentstroke}{rgb}{0.180392,0.180392,0.180392}%
\pgfsetstrokecolor{currentstroke}%
\pgfsetdash{}{0pt}%
\pgfsys@defobject{currentmarker}{\pgfqpoint{0.000000in}{-0.048611in}}{\pgfqpoint{0.000000in}{0.000000in}}{%
\pgfpathmoveto{\pgfqpoint{0.000000in}{0.000000in}}%
\pgfpathlineto{\pgfqpoint{0.000000in}{-0.048611in}}%
\pgfusepath{stroke,fill}%
}%
\begin{pgfscope}%
\pgfsys@transformshift{3.281697in}{0.583522in}%
\pgfsys@useobject{currentmarker}{}%
\end{pgfscope}%
\end{pgfscope}%
\begin{pgfscope}%
\definecolor{textcolor}{rgb}{0.180392,0.180392,0.180392}%
\pgfsetstrokecolor{textcolor}%
\pgfsetfillcolor{textcolor}%
\pgftext[x=3.281697in,y=0.486300in,,top]{\color{textcolor}\rmfamily\fontsize{9.000000}{10.800000}\selectfont \(\displaystyle {0.0}\)}%
\end{pgfscope}%
\begin{pgfscope}%
\pgfpathrectangle{\pgfqpoint{0.477328in}{0.583522in}}{\pgfqpoint{5.608738in}{3.858305in}}%
\pgfusepath{clip}%
\pgfsetbuttcap%
\pgfsetroundjoin%
\pgfsetlinewidth{0.501875pt}%
\definecolor{currentstroke}{rgb}{0.698039,0.698039,0.698039}%
\pgfsetstrokecolor{currentstroke}%
\pgfsetdash{{1.850000pt}{0.800000pt}}{0.000000pt}%
\pgfpathmoveto{\pgfqpoint{4.360300in}{0.583522in}}%
\pgfpathlineto{\pgfqpoint{4.360300in}{4.441828in}}%
\pgfusepath{stroke}%
\end{pgfscope}%
\begin{pgfscope}%
\pgfsetbuttcap%
\pgfsetroundjoin%
\definecolor{currentfill}{rgb}{0.180392,0.180392,0.180392}%
\pgfsetfillcolor{currentfill}%
\pgfsetlinewidth{0.803000pt}%
\definecolor{currentstroke}{rgb}{0.180392,0.180392,0.180392}%
\pgfsetstrokecolor{currentstroke}%
\pgfsetdash{}{0pt}%
\pgfsys@defobject{currentmarker}{\pgfqpoint{0.000000in}{-0.048611in}}{\pgfqpoint{0.000000in}{0.000000in}}{%
\pgfpathmoveto{\pgfqpoint{0.000000in}{0.000000in}}%
\pgfpathlineto{\pgfqpoint{0.000000in}{-0.048611in}}%
\pgfusepath{stroke,fill}%
}%
\begin{pgfscope}%
\pgfsys@transformshift{4.360300in}{0.583522in}%
\pgfsys@useobject{currentmarker}{}%
\end{pgfscope}%
\end{pgfscope}%
\begin{pgfscope}%
\definecolor{textcolor}{rgb}{0.180392,0.180392,0.180392}%
\pgfsetstrokecolor{textcolor}%
\pgfsetfillcolor{textcolor}%
\pgftext[x=4.360300in,y=0.486300in,,top]{\color{textcolor}\rmfamily\fontsize{9.000000}{10.800000}\selectfont \(\displaystyle {0.5}\)}%
\end{pgfscope}%
\begin{pgfscope}%
\pgfpathrectangle{\pgfqpoint{0.477328in}{0.583522in}}{\pgfqpoint{5.608738in}{3.858305in}}%
\pgfusepath{clip}%
\pgfsetbuttcap%
\pgfsetroundjoin%
\pgfsetlinewidth{0.501875pt}%
\definecolor{currentstroke}{rgb}{0.698039,0.698039,0.698039}%
\pgfsetstrokecolor{currentstroke}%
\pgfsetdash{{1.850000pt}{0.800000pt}}{0.000000pt}%
\pgfpathmoveto{\pgfqpoint{5.438904in}{0.583522in}}%
\pgfpathlineto{\pgfqpoint{5.438904in}{4.441828in}}%
\pgfusepath{stroke}%
\end{pgfscope}%
\begin{pgfscope}%
\pgfsetbuttcap%
\pgfsetroundjoin%
\definecolor{currentfill}{rgb}{0.180392,0.180392,0.180392}%
\pgfsetfillcolor{currentfill}%
\pgfsetlinewidth{0.803000pt}%
\definecolor{currentstroke}{rgb}{0.180392,0.180392,0.180392}%
\pgfsetstrokecolor{currentstroke}%
\pgfsetdash{}{0pt}%
\pgfsys@defobject{currentmarker}{\pgfqpoint{0.000000in}{-0.048611in}}{\pgfqpoint{0.000000in}{0.000000in}}{%
\pgfpathmoveto{\pgfqpoint{0.000000in}{0.000000in}}%
\pgfpathlineto{\pgfqpoint{0.000000in}{-0.048611in}}%
\pgfusepath{stroke,fill}%
}%
\begin{pgfscope}%
\pgfsys@transformshift{5.438904in}{0.583522in}%
\pgfsys@useobject{currentmarker}{}%
\end{pgfscope}%
\end{pgfscope}%
\begin{pgfscope}%
\definecolor{textcolor}{rgb}{0.180392,0.180392,0.180392}%
\pgfsetstrokecolor{textcolor}%
\pgfsetfillcolor{textcolor}%
\pgftext[x=5.438904in,y=0.486300in,,top]{\color{textcolor}\rmfamily\fontsize{9.000000}{10.800000}\selectfont \(\displaystyle {1.0}\)}%
\end{pgfscope}%
\begin{pgfscope}%
\definecolor{textcolor}{rgb}{0.180392,0.180392,0.180392}%
\pgfsetstrokecolor{textcolor}%
\pgfsetfillcolor{textcolor}%
\pgftext[x=3.281697in,y=0.305059in,,top]{\color{textcolor}\rmfamily\fontsize{10.800000}{12.960000}\selectfont Normalised Angular Frequency, \(\displaystyle \frac{\omega}{\omega_0}\)}%
\end{pgfscope}%
\begin{pgfscope}%
\pgfpathrectangle{\pgfqpoint{0.477328in}{0.583522in}}{\pgfqpoint{5.608738in}{3.858305in}}%
\pgfusepath{clip}%
\pgfsetbuttcap%
\pgfsetroundjoin%
\pgfsetlinewidth{0.501875pt}%
\definecolor{currentstroke}{rgb}{0.698039,0.698039,0.698039}%
\pgfsetstrokecolor{currentstroke}%
\pgfsetdash{{1.850000pt}{0.800000pt}}{0.000000pt}%
\pgfpathmoveto{\pgfqpoint{0.477328in}{0.758900in}}%
\pgfpathlineto{\pgfqpoint{6.086066in}{0.758900in}}%
\pgfusepath{stroke}%
\end{pgfscope}%
\begin{pgfscope}%
\pgfsetbuttcap%
\pgfsetroundjoin%
\definecolor{currentfill}{rgb}{0.180392,0.180392,0.180392}%
\pgfsetfillcolor{currentfill}%
\pgfsetlinewidth{0.803000pt}%
\definecolor{currentstroke}{rgb}{0.180392,0.180392,0.180392}%
\pgfsetstrokecolor{currentstroke}%
\pgfsetdash{}{0pt}%
\pgfsys@defobject{currentmarker}{\pgfqpoint{-0.048611in}{0.000000in}}{\pgfqpoint{-0.000000in}{0.000000in}}{%
\pgfpathmoveto{\pgfqpoint{-0.000000in}{0.000000in}}%
\pgfpathlineto{\pgfqpoint{-0.048611in}{0.000000in}}%
\pgfusepath{stroke,fill}%
}%
\begin{pgfscope}%
\pgfsys@transformshift{0.477328in}{0.758900in}%
\pgfsys@useobject{currentmarker}{}%
\end{pgfscope}%
\end{pgfscope}%
\begin{pgfscope}%
\definecolor{textcolor}{rgb}{0.180392,0.180392,0.180392}%
\pgfsetstrokecolor{textcolor}%
\pgfsetfillcolor{textcolor}%
\pgftext[x=0.223855in, y=0.713681in, left, base]{\color{textcolor}\rmfamily\fontsize{9.000000}{10.800000}\selectfont \(\displaystyle {0.0}\)}%
\end{pgfscope}%
\begin{pgfscope}%
\pgfpathrectangle{\pgfqpoint{0.477328in}{0.583522in}}{\pgfqpoint{5.608738in}{3.858305in}}%
\pgfusepath{clip}%
\pgfsetbuttcap%
\pgfsetroundjoin%
\pgfsetlinewidth{0.501875pt}%
\definecolor{currentstroke}{rgb}{0.698039,0.698039,0.698039}%
\pgfsetstrokecolor{currentstroke}%
\pgfsetdash{{1.850000pt}{0.800000pt}}{0.000000pt}%
\pgfpathmoveto{\pgfqpoint{0.477328in}{1.460410in}}%
\pgfpathlineto{\pgfqpoint{6.086066in}{1.460410in}}%
\pgfusepath{stroke}%
\end{pgfscope}%
\begin{pgfscope}%
\pgfsetbuttcap%
\pgfsetroundjoin%
\definecolor{currentfill}{rgb}{0.180392,0.180392,0.180392}%
\pgfsetfillcolor{currentfill}%
\pgfsetlinewidth{0.803000pt}%
\definecolor{currentstroke}{rgb}{0.180392,0.180392,0.180392}%
\pgfsetstrokecolor{currentstroke}%
\pgfsetdash{}{0pt}%
\pgfsys@defobject{currentmarker}{\pgfqpoint{-0.048611in}{0.000000in}}{\pgfqpoint{-0.000000in}{0.000000in}}{%
\pgfpathmoveto{\pgfqpoint{-0.000000in}{0.000000in}}%
\pgfpathlineto{\pgfqpoint{-0.048611in}{0.000000in}}%
\pgfusepath{stroke,fill}%
}%
\begin{pgfscope}%
\pgfsys@transformshift{0.477328in}{1.460410in}%
\pgfsys@useobject{currentmarker}{}%
\end{pgfscope}%
\end{pgfscope}%
\begin{pgfscope}%
\definecolor{textcolor}{rgb}{0.180392,0.180392,0.180392}%
\pgfsetstrokecolor{textcolor}%
\pgfsetfillcolor{textcolor}%
\pgftext[x=0.223855in, y=1.415192in, left, base]{\color{textcolor}\rmfamily\fontsize{9.000000}{10.800000}\selectfont \(\displaystyle {0.2}\)}%
\end{pgfscope}%
\begin{pgfscope}%
\pgfpathrectangle{\pgfqpoint{0.477328in}{0.583522in}}{\pgfqpoint{5.608738in}{3.858305in}}%
\pgfusepath{clip}%
\pgfsetbuttcap%
\pgfsetroundjoin%
\pgfsetlinewidth{0.501875pt}%
\definecolor{currentstroke}{rgb}{0.698039,0.698039,0.698039}%
\pgfsetstrokecolor{currentstroke}%
\pgfsetdash{{1.850000pt}{0.800000pt}}{0.000000pt}%
\pgfpathmoveto{\pgfqpoint{0.477328in}{2.161920in}}%
\pgfpathlineto{\pgfqpoint{6.086066in}{2.161920in}}%
\pgfusepath{stroke}%
\end{pgfscope}%
\begin{pgfscope}%
\pgfsetbuttcap%
\pgfsetroundjoin%
\definecolor{currentfill}{rgb}{0.180392,0.180392,0.180392}%
\pgfsetfillcolor{currentfill}%
\pgfsetlinewidth{0.803000pt}%
\definecolor{currentstroke}{rgb}{0.180392,0.180392,0.180392}%
\pgfsetstrokecolor{currentstroke}%
\pgfsetdash{}{0pt}%
\pgfsys@defobject{currentmarker}{\pgfqpoint{-0.048611in}{0.000000in}}{\pgfqpoint{-0.000000in}{0.000000in}}{%
\pgfpathmoveto{\pgfqpoint{-0.000000in}{0.000000in}}%
\pgfpathlineto{\pgfqpoint{-0.048611in}{0.000000in}}%
\pgfusepath{stroke,fill}%
}%
\begin{pgfscope}%
\pgfsys@transformshift{0.477328in}{2.161920in}%
\pgfsys@useobject{currentmarker}{}%
\end{pgfscope}%
\end{pgfscope}%
\begin{pgfscope}%
\definecolor{textcolor}{rgb}{0.180392,0.180392,0.180392}%
\pgfsetstrokecolor{textcolor}%
\pgfsetfillcolor{textcolor}%
\pgftext[x=0.223855in, y=2.116702in, left, base]{\color{textcolor}\rmfamily\fontsize{9.000000}{10.800000}\selectfont \(\displaystyle {0.4}\)}%
\end{pgfscope}%
\begin{pgfscope}%
\pgfpathrectangle{\pgfqpoint{0.477328in}{0.583522in}}{\pgfqpoint{5.608738in}{3.858305in}}%
\pgfusepath{clip}%
\pgfsetbuttcap%
\pgfsetroundjoin%
\pgfsetlinewidth{0.501875pt}%
\definecolor{currentstroke}{rgb}{0.698039,0.698039,0.698039}%
\pgfsetstrokecolor{currentstroke}%
\pgfsetdash{{1.850000pt}{0.800000pt}}{0.000000pt}%
\pgfpathmoveto{\pgfqpoint{0.477328in}{2.863430in}}%
\pgfpathlineto{\pgfqpoint{6.086066in}{2.863430in}}%
\pgfusepath{stroke}%
\end{pgfscope}%
\begin{pgfscope}%
\pgfsetbuttcap%
\pgfsetroundjoin%
\definecolor{currentfill}{rgb}{0.180392,0.180392,0.180392}%
\pgfsetfillcolor{currentfill}%
\pgfsetlinewidth{0.803000pt}%
\definecolor{currentstroke}{rgb}{0.180392,0.180392,0.180392}%
\pgfsetstrokecolor{currentstroke}%
\pgfsetdash{}{0pt}%
\pgfsys@defobject{currentmarker}{\pgfqpoint{-0.048611in}{0.000000in}}{\pgfqpoint{-0.000000in}{0.000000in}}{%
\pgfpathmoveto{\pgfqpoint{-0.000000in}{0.000000in}}%
\pgfpathlineto{\pgfqpoint{-0.048611in}{0.000000in}}%
\pgfusepath{stroke,fill}%
}%
\begin{pgfscope}%
\pgfsys@transformshift{0.477328in}{2.863430in}%
\pgfsys@useobject{currentmarker}{}%
\end{pgfscope}%
\end{pgfscope}%
\begin{pgfscope}%
\definecolor{textcolor}{rgb}{0.180392,0.180392,0.180392}%
\pgfsetstrokecolor{textcolor}%
\pgfsetfillcolor{textcolor}%
\pgftext[x=0.223855in, y=2.818212in, left, base]{\color{textcolor}\rmfamily\fontsize{9.000000}{10.800000}\selectfont \(\displaystyle {0.6}\)}%
\end{pgfscope}%
\begin{pgfscope}%
\pgfpathrectangle{\pgfqpoint{0.477328in}{0.583522in}}{\pgfqpoint{5.608738in}{3.858305in}}%
\pgfusepath{clip}%
\pgfsetbuttcap%
\pgfsetroundjoin%
\pgfsetlinewidth{0.501875pt}%
\definecolor{currentstroke}{rgb}{0.698039,0.698039,0.698039}%
\pgfsetstrokecolor{currentstroke}%
\pgfsetdash{{1.850000pt}{0.800000pt}}{0.000000pt}%
\pgfpathmoveto{\pgfqpoint{0.477328in}{3.564941in}}%
\pgfpathlineto{\pgfqpoint{6.086066in}{3.564941in}}%
\pgfusepath{stroke}%
\end{pgfscope}%
\begin{pgfscope}%
\pgfsetbuttcap%
\pgfsetroundjoin%
\definecolor{currentfill}{rgb}{0.180392,0.180392,0.180392}%
\pgfsetfillcolor{currentfill}%
\pgfsetlinewidth{0.803000pt}%
\definecolor{currentstroke}{rgb}{0.180392,0.180392,0.180392}%
\pgfsetstrokecolor{currentstroke}%
\pgfsetdash{}{0pt}%
\pgfsys@defobject{currentmarker}{\pgfqpoint{-0.048611in}{0.000000in}}{\pgfqpoint{-0.000000in}{0.000000in}}{%
\pgfpathmoveto{\pgfqpoint{-0.000000in}{0.000000in}}%
\pgfpathlineto{\pgfqpoint{-0.048611in}{0.000000in}}%
\pgfusepath{stroke,fill}%
}%
\begin{pgfscope}%
\pgfsys@transformshift{0.477328in}{3.564941in}%
\pgfsys@useobject{currentmarker}{}%
\end{pgfscope}%
\end{pgfscope}%
\begin{pgfscope}%
\definecolor{textcolor}{rgb}{0.180392,0.180392,0.180392}%
\pgfsetstrokecolor{textcolor}%
\pgfsetfillcolor{textcolor}%
\pgftext[x=0.223855in, y=3.519722in, left, base]{\color{textcolor}\rmfamily\fontsize{9.000000}{10.800000}\selectfont \(\displaystyle {0.8}\)}%
\end{pgfscope}%
\begin{pgfscope}%
\pgfpathrectangle{\pgfqpoint{0.477328in}{0.583522in}}{\pgfqpoint{5.608738in}{3.858305in}}%
\pgfusepath{clip}%
\pgfsetbuttcap%
\pgfsetroundjoin%
\pgfsetlinewidth{0.501875pt}%
\definecolor{currentstroke}{rgb}{0.698039,0.698039,0.698039}%
\pgfsetstrokecolor{currentstroke}%
\pgfsetdash{{1.850000pt}{0.800000pt}}{0.000000pt}%
\pgfpathmoveto{\pgfqpoint{0.477328in}{4.266451in}}%
\pgfpathlineto{\pgfqpoint{6.086066in}{4.266451in}}%
\pgfusepath{stroke}%
\end{pgfscope}%
\begin{pgfscope}%
\pgfsetbuttcap%
\pgfsetroundjoin%
\definecolor{currentfill}{rgb}{0.180392,0.180392,0.180392}%
\pgfsetfillcolor{currentfill}%
\pgfsetlinewidth{0.803000pt}%
\definecolor{currentstroke}{rgb}{0.180392,0.180392,0.180392}%
\pgfsetstrokecolor{currentstroke}%
\pgfsetdash{}{0pt}%
\pgfsys@defobject{currentmarker}{\pgfqpoint{-0.048611in}{0.000000in}}{\pgfqpoint{-0.000000in}{0.000000in}}{%
\pgfpathmoveto{\pgfqpoint{-0.000000in}{0.000000in}}%
\pgfpathlineto{\pgfqpoint{-0.048611in}{0.000000in}}%
\pgfusepath{stroke,fill}%
}%
\begin{pgfscope}%
\pgfsys@transformshift{0.477328in}{4.266451in}%
\pgfsys@useobject{currentmarker}{}%
\end{pgfscope}%
\end{pgfscope}%
\begin{pgfscope}%
\definecolor{textcolor}{rgb}{0.180392,0.180392,0.180392}%
\pgfsetstrokecolor{textcolor}%
\pgfsetfillcolor{textcolor}%
\pgftext[x=0.223855in, y=4.221232in, left, base]{\color{textcolor}\rmfamily\fontsize{9.000000}{10.800000}\selectfont \(\displaystyle {1.0}\)}%
\end{pgfscope}%
\begin{pgfscope}%
\definecolor{textcolor}{rgb}{0.180392,0.180392,0.180392}%
\pgfsetstrokecolor{textcolor}%
\pgfsetfillcolor{textcolor}%
\pgftext[x=0.168300in,y=2.512675in,,bottom,rotate=90.000000]{\color{textcolor}\rmfamily\fontsize{10.800000}{12.960000}\selectfont \(\displaystyle \tilde{E}(\omega)\)}%
\end{pgfscope}%
\begin{pgfscope}%
\pgfpathrectangle{\pgfqpoint{0.477328in}{0.583522in}}{\pgfqpoint{5.608738in}{3.858305in}}%
\pgfusepath{clip}%
\pgfsetrectcap%
\pgfsetroundjoin%
\pgfsetlinewidth{2.007500pt}%
\definecolor{currentstroke}{rgb}{0.000000,0.501961,0.000000}%
\pgfsetstrokecolor{currentstroke}%
\pgfsetdash{}{0pt}%
\pgfpathmoveto{\pgfqpoint{0.467328in}{0.758900in}}%
\pgfpathlineto{\pgfqpoint{0.779337in}{0.759980in}}%
\pgfpathlineto{\pgfqpoint{0.800909in}{0.761776in}}%
\pgfpathlineto{\pgfqpoint{0.822481in}{0.766088in}}%
\pgfpathlineto{\pgfqpoint{0.833267in}{0.769997in}}%
\pgfpathlineto{\pgfqpoint{0.844053in}{0.775764in}}%
\pgfpathlineto{\pgfqpoint{0.854839in}{0.784126in}}%
\pgfpathlineto{\pgfqpoint{0.865625in}{0.796043in}}%
\pgfpathlineto{\pgfqpoint{0.876411in}{0.812732in}}%
\pgfpathlineto{\pgfqpoint{0.887197in}{0.835698in}}%
\pgfpathlineto{\pgfqpoint{0.897983in}{0.866744in}}%
\pgfpathlineto{\pgfqpoint{0.908769in}{0.907969in}}%
\pgfpathlineto{\pgfqpoint{0.919555in}{0.961725in}}%
\pgfpathlineto{\pgfqpoint{0.930341in}{1.030541in}}%
\pgfpathlineto{\pgfqpoint{0.941127in}{1.117007in}}%
\pgfpathlineto{\pgfqpoint{0.951913in}{1.223598in}}%
\pgfpathlineto{\pgfqpoint{0.962699in}{1.352469in}}%
\pgfpathlineto{\pgfqpoint{0.973485in}{1.505200in}}%
\pgfpathlineto{\pgfqpoint{0.984271in}{1.682530in}}%
\pgfpathlineto{\pgfqpoint{0.995057in}{1.884085in}}%
\pgfpathlineto{\pgfqpoint{1.016629in}{2.351481in}}%
\pgfpathlineto{\pgfqpoint{1.070560in}{3.638147in}}%
\pgfpathlineto{\pgfqpoint{1.081346in}{3.850193in}}%
\pgfpathlineto{\pgfqpoint{1.092132in}{4.025856in}}%
\pgfpathlineto{\pgfqpoint{1.102918in}{4.157407in}}%
\pgfpathlineto{\pgfqpoint{1.113704in}{4.238867in}}%
\pgfpathlineto{\pgfqpoint{1.124490in}{4.266450in}}%
\pgfpathlineto{\pgfqpoint{1.135276in}{4.238862in}}%
\pgfpathlineto{\pgfqpoint{1.146062in}{4.157399in}}%
\pgfpathlineto{\pgfqpoint{1.156848in}{4.025843in}}%
\pgfpathlineto{\pgfqpoint{1.167634in}{3.850177in}}%
\pgfpathlineto{\pgfqpoint{1.178420in}{3.638129in}}%
\pgfpathlineto{\pgfqpoint{1.199992in}{3.141101in}}%
\pgfpathlineto{\pgfqpoint{1.232350in}{2.351461in}}%
\pgfpathlineto{\pgfqpoint{1.253922in}{1.884068in}}%
\pgfpathlineto{\pgfqpoint{1.264708in}{1.682514in}}%
\pgfpathlineto{\pgfqpoint{1.275494in}{1.505187in}}%
\pgfpathlineto{\pgfqpoint{1.286280in}{1.352458in}}%
\pgfpathlineto{\pgfqpoint{1.297066in}{1.223589in}}%
\pgfpathlineto{\pgfqpoint{1.307852in}{1.116999in}}%
\pgfpathlineto{\pgfqpoint{1.318638in}{1.030535in}}%
\pgfpathlineto{\pgfqpoint{1.329424in}{0.961720in}}%
\pgfpathlineto{\pgfqpoint{1.340210in}{0.907966in}}%
\pgfpathlineto{\pgfqpoint{1.350996in}{0.866741in}}%
\pgfpathlineto{\pgfqpoint{1.361783in}{0.835696in}}%
\pgfpathlineto{\pgfqpoint{1.372569in}{0.812730in}}%
\pgfpathlineto{\pgfqpoint{1.383355in}{0.796042in}}%
\pgfpathlineto{\pgfqpoint{1.394141in}{0.784125in}}%
\pgfpathlineto{\pgfqpoint{1.404927in}{0.775764in}}%
\pgfpathlineto{\pgfqpoint{1.415713in}{0.769997in}}%
\pgfpathlineto{\pgfqpoint{1.426499in}{0.766088in}}%
\pgfpathlineto{\pgfqpoint{1.448071in}{0.761776in}}%
\pgfpathlineto{\pgfqpoint{1.480429in}{0.759547in}}%
\pgfpathlineto{\pgfqpoint{1.545145in}{0.758921in}}%
\pgfpathlineto{\pgfqpoint{2.774753in}{0.758900in}}%
\pgfpathlineto{\pgfqpoint{5.093751in}{0.759980in}}%
\pgfpathlineto{\pgfqpoint{5.115323in}{0.761776in}}%
\pgfpathlineto{\pgfqpoint{5.136895in}{0.766088in}}%
\pgfpathlineto{\pgfqpoint{5.147681in}{0.769997in}}%
\pgfpathlineto{\pgfqpoint{5.158467in}{0.775764in}}%
\pgfpathlineto{\pgfqpoint{5.169253in}{0.784125in}}%
\pgfpathlineto{\pgfqpoint{5.180039in}{0.796042in}}%
\pgfpathlineto{\pgfqpoint{5.190825in}{0.812730in}}%
\pgfpathlineto{\pgfqpoint{5.201611in}{0.835696in}}%
\pgfpathlineto{\pgfqpoint{5.212397in}{0.866741in}}%
\pgfpathlineto{\pgfqpoint{5.223183in}{0.907966in}}%
\pgfpathlineto{\pgfqpoint{5.233969in}{0.961720in}}%
\pgfpathlineto{\pgfqpoint{5.244755in}{1.030535in}}%
\pgfpathlineto{\pgfqpoint{5.255541in}{1.116999in}}%
\pgfpathlineto{\pgfqpoint{5.266327in}{1.223589in}}%
\pgfpathlineto{\pgfqpoint{5.277113in}{1.352458in}}%
\pgfpathlineto{\pgfqpoint{5.287899in}{1.505187in}}%
\pgfpathlineto{\pgfqpoint{5.298685in}{1.682514in}}%
\pgfpathlineto{\pgfqpoint{5.309471in}{1.884068in}}%
\pgfpathlineto{\pgfqpoint{5.331043in}{2.351461in}}%
\pgfpathlineto{\pgfqpoint{5.384974in}{3.638129in}}%
\pgfpathlineto{\pgfqpoint{5.395760in}{3.850177in}}%
\pgfpathlineto{\pgfqpoint{5.406546in}{4.025843in}}%
\pgfpathlineto{\pgfqpoint{5.417332in}{4.157399in}}%
\pgfpathlineto{\pgfqpoint{5.428118in}{4.238862in}}%
\pgfpathlineto{\pgfqpoint{5.438904in}{4.266450in}}%
\pgfpathlineto{\pgfqpoint{5.449690in}{4.238867in}}%
\pgfpathlineto{\pgfqpoint{5.460476in}{4.157407in}}%
\pgfpathlineto{\pgfqpoint{5.471262in}{4.025856in}}%
\pgfpathlineto{\pgfqpoint{5.482048in}{3.850193in}}%
\pgfpathlineto{\pgfqpoint{5.492834in}{3.638147in}}%
\pgfpathlineto{\pgfqpoint{5.514406in}{3.141122in}}%
\pgfpathlineto{\pgfqpoint{5.546764in}{2.351481in}}%
\pgfpathlineto{\pgfqpoint{5.568336in}{1.884085in}}%
\pgfpathlineto{\pgfqpoint{5.579122in}{1.682530in}}%
\pgfpathlineto{\pgfqpoint{5.589908in}{1.505200in}}%
\pgfpathlineto{\pgfqpoint{5.600694in}{1.352469in}}%
\pgfpathlineto{\pgfqpoint{5.611480in}{1.223598in}}%
\pgfpathlineto{\pgfqpoint{5.622266in}{1.117007in}}%
\pgfpathlineto{\pgfqpoint{5.633052in}{1.030541in}}%
\pgfpathlineto{\pgfqpoint{5.643838in}{0.961725in}}%
\pgfpathlineto{\pgfqpoint{5.654625in}{0.907969in}}%
\pgfpathlineto{\pgfqpoint{5.665411in}{0.866744in}}%
\pgfpathlineto{\pgfqpoint{5.676197in}{0.835698in}}%
\pgfpathlineto{\pgfqpoint{5.686983in}{0.812732in}}%
\pgfpathlineto{\pgfqpoint{5.697769in}{0.796043in}}%
\pgfpathlineto{\pgfqpoint{5.708555in}{0.784126in}}%
\pgfpathlineto{\pgfqpoint{5.719341in}{0.775764in}}%
\pgfpathlineto{\pgfqpoint{5.730127in}{0.769997in}}%
\pgfpathlineto{\pgfqpoint{5.740913in}{0.766088in}}%
\pgfpathlineto{\pgfqpoint{5.762485in}{0.761776in}}%
\pgfpathlineto{\pgfqpoint{5.794843in}{0.759547in}}%
\pgfpathlineto{\pgfqpoint{5.859559in}{0.758921in}}%
\pgfpathlineto{\pgfqpoint{6.096066in}{0.758900in}}%
\pgfpathlineto{\pgfqpoint{6.096066in}{0.758900in}}%
\pgfusepath{stroke}%
\end{pgfscope}%
\begin{pgfscope}%
\pgfsetrectcap%
\pgfsetmiterjoin%
\pgfsetlinewidth{0.803000pt}%
\definecolor{currentstroke}{rgb}{0.737255,0.737255,0.737255}%
\pgfsetstrokecolor{currentstroke}%
\pgfsetdash{}{0pt}%
\pgfpathmoveto{\pgfqpoint{0.477328in}{0.583522in}}%
\pgfpathlineto{\pgfqpoint{0.477328in}{4.441828in}}%
\pgfusepath{stroke}%
\end{pgfscope}%
\begin{pgfscope}%
\pgfsetrectcap%
\pgfsetmiterjoin%
\pgfsetlinewidth{0.803000pt}%
\definecolor{currentstroke}{rgb}{0.737255,0.737255,0.737255}%
\pgfsetstrokecolor{currentstroke}%
\pgfsetdash{}{0pt}%
\pgfpathmoveto{\pgfqpoint{6.086066in}{0.583522in}}%
\pgfpathlineto{\pgfqpoint{6.086066in}{4.441828in}}%
\pgfusepath{stroke}%
\end{pgfscope}%
\begin{pgfscope}%
\pgfsetrectcap%
\pgfsetmiterjoin%
\pgfsetlinewidth{0.803000pt}%
\definecolor{currentstroke}{rgb}{0.737255,0.737255,0.737255}%
\pgfsetstrokecolor{currentstroke}%
\pgfsetdash{}{0pt}%
\pgfpathmoveto{\pgfqpoint{0.477328in}{0.583522in}}%
\pgfpathlineto{\pgfqpoint{6.086066in}{0.583522in}}%
\pgfusepath{stroke}%
\end{pgfscope}%
\begin{pgfscope}%
\pgfsetrectcap%
\pgfsetmiterjoin%
\pgfsetlinewidth{0.803000pt}%
\definecolor{currentstroke}{rgb}{0.737255,0.737255,0.737255}%
\pgfsetstrokecolor{currentstroke}%
\pgfsetdash{}{0pt}%
\pgfpathmoveto{\pgfqpoint{0.477328in}{4.441828in}}%
\pgfpathlineto{\pgfqpoint{6.086066in}{4.441828in}}%
\pgfusepath{stroke}%
\end{pgfscope}%
\begin{pgfscope}%
\pgfsetroundcap%
\pgfsetroundjoin%
\pgfsetlinewidth{0.501875pt}%
\definecolor{currentstroke}{rgb}{0.000000,0.000000,0.000000}%
\pgfsetstrokecolor{currentstroke}%
\pgfsetdash{}{0pt}%
\pgfpathmoveto{\pgfqpoint{1.338215in}{2.046171in}}%
\pgfpathquadraticcurveto{\pgfqpoint{1.169719in}{2.046171in}}{\pgfqpoint{1.008987in}{2.046171in}}%
\pgfusepath{stroke}%
\end{pgfscope}%
\begin{pgfscope}%
\pgfsetroundcap%
\pgfsetroundjoin%
\pgfsetlinewidth{0.501875pt}%
\definecolor{currentstroke}{rgb}{0.000000,0.000000,0.000000}%
\pgfsetstrokecolor{currentstroke}%
\pgfsetdash{}{0pt}%
\pgfpathmoveto{\pgfqpoint{1.058987in}{2.021171in}}%
\pgfpathlineto{\pgfqpoint{1.008987in}{2.046171in}}%
\pgfpathlineto{\pgfqpoint{1.058987in}{2.071171in}}%
\pgfusepath{stroke}%
\end{pgfscope}%
\begin{pgfscope}%
\definecolor{textcolor}{rgb}{0.180392,0.180392,0.180392}%
\pgfsetstrokecolor{textcolor}%
\pgfsetfillcolor{textcolor}%
\pgftext[x=1.663792in,y=2.046171in,,]{\color{textcolor}\rmfamily\fontsize{9.000000}{10.800000}\selectfont \textbf{\(\displaystyle {\Delta\omega}\ll{\omega_0}\)}}%
\end{pgfscope}%
\begin{pgfscope}%
\pgfsetroundcap%
\pgfsetroundjoin%
\pgfsetlinewidth{0.501875pt}%
\definecolor{currentstroke}{rgb}{0.000000,0.000000,0.000000}%
\pgfsetstrokecolor{currentstroke}%
\pgfsetdash{}{0pt}%
\pgfpathmoveto{\pgfqpoint{1.152250in}{2.046171in}}%
\pgfpathquadraticcurveto{\pgfqpoint{1.199986in}{2.046171in}}{\pgfqpoint{1.239958in}{2.046171in}}%
\pgfusepath{stroke}%
\end{pgfscope}%
\begin{pgfscope}%
\pgfsetroundcap%
\pgfsetroundjoin%
\pgfsetlinewidth{0.501875pt}%
\definecolor{currentstroke}{rgb}{0.000000,0.000000,0.000000}%
\pgfsetstrokecolor{currentstroke}%
\pgfsetdash{}{0pt}%
\pgfpathmoveto{\pgfqpoint{1.189958in}{2.071171in}}%
\pgfpathlineto{\pgfqpoint{1.239958in}{2.046171in}}%
\pgfpathlineto{\pgfqpoint{1.189958in}{2.021171in}}%
\pgfusepath{stroke}%
\end{pgfscope}%
\begin{pgfscope}%
\definecolor{textcolor}{rgb}{0.180392,0.180392,0.180392}%
\pgfsetstrokecolor{textcolor}%
\pgfsetfillcolor{textcolor}%
\pgftext[x=3.281697in,y=4.525161in,,base]{\color{textcolor}\rmfamily\fontsize{12.960000}{15.552000}\selectfont Pulsed Gaussian Normalised Spectrum}%
\end{pgfscope}%
\end{pgfpicture}%
\makeatother%
\endgroup%
}
		\caption{\centering{The Gaussian Pulse on the normalised frequency spectrum}}
		\label{fig:gaussian_spectrum_normalised}
	\end{figure}
	
	\pagebreak
	
	
	
	\subsubsection{The Components of the NLSE}
	
	The various components of the NLSE shown before (in Equation \ref{eq:nlse} and repeated below for completeness), can be summarised as follows:
	\begin{equation*}
       \mathlarger{ j \frac{\partial A(r,t)}{\partial z} + \frac{1}{2 \beta} \frac{\partial^2 A(r,t)}{\partial x^2} + \frac{1}{2 \beta} \frac{\partial^2 A(r,t)}{\partial y^2} - \frac{ \beta ''}{2} \frac{\partial^2 A(r,t)}{\partial t^2} + \chi^{(3)} |A(r,t)|^2 A(r,t) = 0 }
	\end{equation*}
	
	\begin{center}
	\begin{table}[htbp!]
	    \centering
	    %\def\arraystretch{0.1}
	    \begin{tabular}{p{5cm} p{8cm}}
	        $\mathlarger{ \mathlarger{ j \frac{\partial A(r,t)}{\partial z} } }$ \vspace{5mm} & This takes into account the propagation along the z-axis \\
	        $\mathlarger{ \mathlarger{ \frac{1}{2 \beta} \frac{\partial^2 A(r,t)}{\partial x^2} } }$ \vspace{5mm} & Is the dispersion term along the x-axis \\
	        $\mathlarger{ \mathlarger{ \frac{1}{2 \beta} \frac{\partial^2 A(r,t)}{\partial y^2} } }$ \vspace{5mm} & Is the dispersion term along the y-axis \\
	        $\mathlarger{ \mathlarger{ - \frac{ \beta ''}{2} \frac{\partial^2 A(r,t)}{\partial t^2} } }$ \vspace{5mm} & Is the dispersion term along the t-axis \\
	        $\mathlarger{ \mathlarger{ \chi^{(3)} |A(r,t)|^2 A(r,t) } }$ \vspace{5mm} & Is the non-linear term \\
	        $\mathlarger{ \mathlarger{ \beta = \frac{\omega}{c} n(\omega) } }$ \vspace{5mm} & Is the propagation constant \\
	        $\mathlarger{ \mathlarger{ \beta '' } }$ \vspace{5mm} & Is the group velocity dispersion \\
	        $\mathlarger{ \mathlarger{ \chi^{(3)} } \quad n = \sqrt{1+\chi^{(\omega)}}}$ \vspace{5mm} & Accounts for non-linear cubic response of the medium \\
	    \end{tabular}
	    \label{table_lin_eq}
	\end{table}
	\end{center}
	
	In general, the NLSE is a (3+1) Dimension ($z, y, z, t$) model ((3+1)D model) which describes the evolution of an optical light pulse in space and time.
	In optical fibres, the envelope of the pulse, $A(r,t)$ can be expressed (using separation of variables) as shown below in Equation \ref{eq:sep_of_vars_envelope}: \linebreak
	
	\begin{equation} \label{eq:sep_of_vars_envelope}
		\mathlarger{ \mathlarger{ A(r,t) = A(x, y, z, t) = F(z, t) \cdot M(x, y) \cdot e^{i \delta \beta z} } }
	\end{equation}
	
	\begin{align*}
		\text{Where:}& \\
		M(z, y) & \text{, defines the mode profiles} \\
		\partial \beta \text{, is } & \text{the correction of the propagation constant } \implies \beta = \beta_0 + \partial \beta \\
		\therefore M(z, &y) \cdot e^{i \delta \beta z} \text{, describes the modal distribution in the x-y plane at a given z, and;} \\
		F(z, t) & \text{, describes the slowly varying (compared to the carrier) pulse envelope in the z-t plane}
	\end{align*}
	
	\pagebreak
	
	
	
	
	\subsection{Spatio-Temporal Regime, the NLSE (1+1D)}
	
	The first regime we will focus on is the spatio-temporal evolution of $F(z,t)$, ignoring other sections of our pulse. From the NLSE (3+1D) one can derive a model for $F(z,t)$ to obtain the NLSE (1+1D), given by Equation \ref{eq:nlse_1_1d}: \linebreak
	
	\begin{equation} \label{eq:nlse_1_1d}
       j \frac{\partial F(z,t)}{\partial z} - \frac{ \beta ''}{2} \frac{\partial^2 F(z,t)}{\partial t^2} + \gamma |F(z,t)|^2 F(z,t) = 0
	\end{equation}
	
	\begin{align}
		\text{Where:}& \nonumber\\
		z & \text{, is the propagation or evolution coordinate} \nonumber\\
		t & \text{, is the temporal coordinate} \nonumber\\
		\beta '' & \text{, is the Group Velocity Dispersion (GVD) inside the fibre} \nonumber\\
		\gamma & \text{, is the effective non-linear term, related to } \chi^{(3)} \nonumber
	\end{align}
	
	\vspace{5mm}
	
	It should be noted that both the (3+1D) and (1+1D) NLSEs are non-linear, partial differential equations that do not trivially lend themselves to analytical solutions, except in some very specific cases. It is more likely that one could successfully perform an analytical study of the (1+1D) NLSE as compared to the (3+1D) NLSE for obvious reasons, however it is even more practical to take a numerical approach to each of these equations to understand their linear and non-linear effects in optical fibres. \linebreak
	
	A large number of specific numerical methods may be performed, but they all fall into two general families: \linebreak
	\begin{itemize}[leftmargin=1cm]
		\item Finite-Difference methods
		\item Pseudo-Spectral methods (these are generally faster by an order of magnitude whilst achieving the same accuracy as the above methods)
	\end{itemize}
	
	\vspace{5mm}
	
	One pseudo-spectral method which has been used extensively to solve pulse propagation in non-linear, dispersive media is the \textbf{split-step Fourier method} (also known as the Beam Propagation Method - BPM), the relative speed of this method when compared with finite-difference schemes can be attributed to the Fast Fourier Transform (FFT) algorithm.
	
	\pagebreak

	
	
	
\section{Numerical Techniques}

	As mentioned in the previous section one of the most practical and efficient methods to solve pulse propagation in optical fibres, and the one we will heavily focus on in the course, is the split-step Fourier method. To understand the methodology of, and use, the split-step Fourier method one must write the NLSE in the form shown below in Equation \ref{eq:split_step_fourier_nlse}:
	
	\begin{equation} \label{eq:split_step_fourier_nlse}
       \mathlarger{ \frac{\partial F(z,t)}{\partial z} = (\tilde{D} + \tilde{N}) F }
	\end{equation}
	
	
	\vspace{5mm}
	
	
	$\tilde{D}$ is a differential operator that accounts for dispersion in a linear medium (however it can account for absorption and/or higher order dispersions) and is given by Equation \ref{eq:disp_operator}:
	
	\begin{equation} \label{eq:disp_operator}
       \mathlarger{ \tilde{D} = - j \frac{ \beta ''}{2} \frac{\partial^2}{\partial t^2} }
	\end{equation}
	
	
	\vspace{5mm}
	
	
	$\tilde{N}$ is a non-linear operator that governs the effect of fibre non-linearity on pulse propagation, e.g. Kerr non-linearity (but it can account for other non-linear phenomena like Raman effects). $\tilde{N}$ is given by Equation \ref{eq:non_lin_operator}:
	
	\begin{equation} \label{eq:non_lin_operator}
       \mathlarger{ \tilde{N} = - j \gamma |F|^2 }
	\end{equation}
	
	
	\vspace{5mm}
	
	
	In general the dispersion, $\tilde{D}$, and non-linearity, $\tilde{N}$, act together along the fibre, but the Split-Step Fourier method obtains an \textbf{approximate solution} by assuming that by propagating the envelope over a small distance, $h$, the dispersive and non-linear effects can be considered to act independently. \linebreak 
	More specifically, the propagation from a point, $z$, to another, $z+h$, can be described via two steps of calculation (the astute reader will note this is why the method is called split-step): \vspace{5mm}
	
	\begin{enumerate}[leftmargin=1cm]
		\item Calculating the propagation accounting only for non-linear effects, $\tilde{D} = 0$
		\item Calculating the propagation accounting only for dispersive effects, $\tilde{N} = 0$
	\end{enumerate}
	
	
	\vspace{5mm}
	
	
	Mathematically this process is described by Equation \ref{eq:split_step_fourier_nlse_2}:
	
	\begin{equation} \label{eq:split_step_fourier_nlse_2}
		\mathlarger{ F(z + h, t) \approx e^{h \tilde{D}} \cdot e^{h \tilde{N}} \cdot F(z, t) }
	\end{equation}
	
	
	\vspace{5mm}
	
	
	Later we will show that the term $e^{h \tilde{D}}$  can be evaluated in the Fourier domain using Equation \ref{eq:split_step_dispersive_approx}:
	
	\begin{equation} \label{eq:split_step_dispersive_approx}
		\mathlarger{ e^{h \tilde{D}} \cdot F(z, t) = \mathcal{FT}^{-1}[e^{h \tilde{D}(j\omega)}] \quad \mathcal{FT}[F(z, t)] }
	\end{equation}
	
	
	\vspace{5mm}
	
	
	For full clarity (to avoid any confusion with notation used) $\mathcal{FT}$ denotes the Fourier Transform, and $\tilde{D}(j\omega)$ is obtained by replacing the differential operators in time domain; $\frac{\partial}{\partial t} \to j\omega$ and $\frac{\partial^2}{\partial ^2t} \to (j\omega)^2 = -\omega^2$. As $\tilde{D}(j\omega)$ is a number in Fourier space, the evaluation of the term mathematically, and numerically with the FFT, is fast and simple. \linebreak
	
	
	\pagebreak
	
	
	To estimate the accuracy of the split-step method, one must observe the formal solution of the NLSE compared to the approximate solution we have formed, the formal solution is given by Equation \ref{eq:split_step_dispersive_exact} (assuming $\tilde{N}$ is z-independent):
	
	\begin{equation} \label{eq:split_step_dispersive_exact}
		\mathlarger{ F(z+h, t) = e^{h (\tilde{D} + \tilde{N})} F(z, t) }
	\end{equation}
	
	
	\vspace{5mm}
	
	
	The comparison of this equation and the approximate solution leads us to a topic of mathematics you may not be familiar with, the Baker-Campbell-Hausdorff proof; \linebreak
	
	\subsection{The Baker-Campbell-Hausdorff Formula}
		Now is a good time to recall the Baker-Campbell-Hausdorff formula \cite{baker_campbell_hausdorff_wiki} which is a solution for an equation such as that shown below in Equation \ref{eq:baker_campbell_hausdorff}:
			
		\begin{equation} \label{eq:baker_campbell_hausdorff}
			\mathlarger{\mathlarger{ e^{\tilde{a}} e^{\tilde{b}} = e^{\tilde{c}} }}
		\end{equation}
		
		
		\vspace{5mm}
		
		
		The solution to which is given by Equation \ref{eq:baker_campbell_hausdorff_solution}:
		
		\begin{equation} \label{eq:baker_campbell_hausdorff_solution}
			\mathlarger{\mathlarger{ e^{\tilde{c}} = e^{\tilde{a} + \tilde{b} + \frac{1}{2}[\tilde{a}, \tilde{b}] + \frac{1}{12}[\tilde{a}, [\tilde{a}, \tilde{b}]] - \frac{1}{12}[\tilde{b}, [\tilde{a}, \tilde{b}]] + \dots} }}
		\end{equation}
		
		
		\vspace{5mm}
		
		 
		Where:
		\begin{equation}
			\mathlarger{ [\tilde{a}, \tilde{b}] = \tilde{a} \tilde{b} - \tilde{b} \tilde{a} } \nonumber
		\end{equation}
		
		And:		
		\begin{align}
			\text{If } [\tilde{a}, \tilde{b}] & = 0 \text{ the operators commutate} \nonumber\\
			\text{If } [\tilde{a}, \tilde{b}] & \neq 0 \text{ the operators do not commutate} \nonumber
		\end{align}
		
		
		\vspace{5mm}
		
		
		Comparing Equations \ref{eq:split_step_fourier_nlse_2} and \ref{eq:split_step_dispersive_exact} exactly indicates that the Split-Step method ignores the non-commutating nature of the $\tilde{D}$ and $\tilde{N}$ operators. However, if we cheat a bit and insert $\tilde{a} = h \tilde{D} $ and $\tilde{b} = h \tilde{N} $ into Equation \ref{eq:baker_campbell_hausdorff_solution} and compare it to Equation \ref{eq:split_step_dispersive_exact} the third exponent term will be $\frac{1}{2}[\tilde{a}, \tilde{b}] = \frac{1}{2} h^2 [\tilde{D}, \tilde{N}]$. Thus, the Split-Step Fourier method is accurate to the second order when using the step size of $h$ (i.e. the equations match until the third exponential term and thus can be said to be approximate to a certain degree), one must emphasise what this signifies: \textbf{as long as $h$ is small the errors of calculation will also be small (insignificant).}
	
	
	
	
	\pagebreak
	
	
	
	
	\subsection{Numeric Implementation}
		To summarise the previous section, the general implementation of the Split-Step Fourier method is given by Equation \ref{eqn:split_step_fourier_general}: \linebreak
		
		\begin{equation} \label{eqn:split_step_fourier_general}
			\mathlarger{ F(z+h, t) = e^{h (\tilde{D} + \tilde{N})} F(z, t) \approx e^{h \tilde{D}} \cdot e^{h \tilde{N}} \cdot F(z, t)  }
		\end{equation}
		
		\vspace{5mm}
		
		And thus, to simulate the propagation from $z$ to $z+h$ we will follow the previously set out steps, repeated below for completeness (this could be completed in reverse order just as effectively): \linebreak
		
		\begin{enumerate}[leftmargin=1cm]
			\item Calculating the propagation accounting only for non-linear effects, $\tilde{D} = 0$
			\item Calculating the propagation accounting only for dispersive effects, $\tilde{N} = 0$
		\end{enumerate}
		
		\vspace{5mm}
		
		
		\subsubsection{Considerations for Numeric Implementation}
		
			For the envelope $F(z, t)$, we must take into account that it's profile, for numerical applications, will be recorded in a matrix (or vector), via appropriate sampling of the signal, $F$, e.g. Shannon-Nyquist Sampling Theorem \cite{MechatronicsBolton2015} \cite{shannon_nyquist_sampling_theorem}. This is displayed below in Figure \ref{fig:envelope_continuous_discrete}, the dotted line displays the discrete sampling of the signal and the solid line the continuous signal that exists in reality: \linebreak
			
			
			\begin{figure}[h]
				\centering
				\scalebox{0.85}{%% Creator: Matplotlib, PGF backend
%%
%% To include the figure in your LaTeX document, write
%%   \input{<filename>.pgf}
%%
%% Make sure the required packages are loaded in your preamble
%%   \usepackage{pgf}
%%
%% Also ensure that all the required font packages are loaded; for instance,
%% the lmodern package is sometimes necessary when using math font.
%%   \usepackage{lmodern}
%%
%% Figures using additional raster images can only be included by \input if
%% they are in the same directory as the main LaTeX file. For loading figures
%% from other directories you can use the `import` package
%%   \usepackage{import}
%%
%% and then include the figures with
%%   \import{<path to file>}{<filename>.pgf}
%%
%% Matplotlib used the following preamble
%%   \usepackage[T1]{fontenc} \usepackage{mathpazo}
%%
\begingroup%
\makeatletter%
\begin{pgfpicture}%
\pgfpathrectangle{\pgfpointorigin}{\pgfqpoint{6.198463in}{2.944837in}}%
\pgfusepath{use as bounding box, clip}%
\begin{pgfscope}%
\pgfsetbuttcap%
\pgfsetmiterjoin%
\definecolor{currentfill}{rgb}{1.000000,1.000000,1.000000}%
\pgfsetfillcolor{currentfill}%
\pgfsetlinewidth{0.000000pt}%
\definecolor{currentstroke}{rgb}{1.000000,1.000000,1.000000}%
\pgfsetstrokecolor{currentstroke}%
\pgfsetdash{}{0pt}%
\pgfpathmoveto{\pgfqpoint{0.000000in}{0.000000in}}%
\pgfpathlineto{\pgfqpoint{6.198463in}{0.000000in}}%
\pgfpathlineto{\pgfqpoint{6.198463in}{2.944837in}}%
\pgfpathlineto{\pgfqpoint{0.000000in}{2.944837in}}%
\pgfpathlineto{\pgfqpoint{0.000000in}{0.000000in}}%
\pgfpathclose%
\pgfusepath{fill}%
\end{pgfscope}%
\begin{pgfscope}%
\pgfsetbuttcap%
\pgfsetmiterjoin%
\definecolor{currentfill}{rgb}{0.933333,0.933333,0.933333}%
\pgfsetfillcolor{currentfill}%
\pgfsetlinewidth{0.000000pt}%
\definecolor{currentstroke}{rgb}{0.000000,0.000000,0.000000}%
\pgfsetstrokecolor{currentstroke}%
\pgfsetstrokeopacity{0.000000}%
\pgfsetdash{}{0pt}%
\pgfpathmoveto{\pgfqpoint{0.564224in}{0.429287in}}%
\pgfpathlineto{\pgfqpoint{6.198463in}{0.429287in}}%
\pgfpathlineto{\pgfqpoint{6.198463in}{2.731275in}}%
\pgfpathlineto{\pgfqpoint{0.564224in}{2.731275in}}%
\pgfpathlineto{\pgfqpoint{0.564224in}{0.429287in}}%
\pgfpathclose%
\pgfusepath{fill}%
\end{pgfscope}%
\begin{pgfscope}%
\pgfpathrectangle{\pgfqpoint{0.564224in}{0.429287in}}{\pgfqpoint{5.634238in}{2.301989in}}%
\pgfusepath{clip}%
\pgfsetbuttcap%
\pgfsetroundjoin%
\pgfsetlinewidth{0.501875pt}%
\definecolor{currentstroke}{rgb}{0.698039,0.698039,0.698039}%
\pgfsetstrokecolor{currentstroke}%
\pgfsetdash{{1.850000pt}{0.800000pt}}{0.000000pt}%
\pgfpathmoveto{\pgfqpoint{0.966670in}{0.429287in}}%
\pgfpathlineto{\pgfqpoint{0.966670in}{2.731275in}}%
\pgfusepath{stroke}%
\end{pgfscope}%
\begin{pgfscope}%
\pgfsetbuttcap%
\pgfsetroundjoin%
\definecolor{currentfill}{rgb}{0.180392,0.180392,0.180392}%
\pgfsetfillcolor{currentfill}%
\pgfsetlinewidth{0.803000pt}%
\definecolor{currentstroke}{rgb}{0.180392,0.180392,0.180392}%
\pgfsetstrokecolor{currentstroke}%
\pgfsetdash{}{0pt}%
\pgfsys@defobject{currentmarker}{\pgfqpoint{0.000000in}{-0.048611in}}{\pgfqpoint{0.000000in}{0.000000in}}{%
\pgfpathmoveto{\pgfqpoint{0.000000in}{0.000000in}}%
\pgfpathlineto{\pgfqpoint{0.000000in}{-0.048611in}}%
\pgfusepath{stroke,fill}%
}%
\begin{pgfscope}%
\pgfsys@transformshift{0.966670in}{0.429287in}%
\pgfsys@useobject{currentmarker}{}%
\end{pgfscope}%
\end{pgfscope}%
\begin{pgfscope}%
\definecolor{textcolor}{rgb}{0.180392,0.180392,0.180392}%
\pgfsetstrokecolor{textcolor}%
\pgfsetfillcolor{textcolor}%
\pgftext[x=0.966670in,y=0.332064in,,top]{\color{textcolor}\rmfamily\fontsize{9.000000}{10.800000}\selectfont \(\displaystyle {0}\)}%
\end{pgfscope}%
\begin{pgfscope}%
\pgfpathrectangle{\pgfqpoint{0.564224in}{0.429287in}}{\pgfqpoint{5.634238in}{2.301989in}}%
\pgfusepath{clip}%
\pgfsetbuttcap%
\pgfsetroundjoin%
\pgfsetlinewidth{0.501875pt}%
\definecolor{currentstroke}{rgb}{0.698039,0.698039,0.698039}%
\pgfsetstrokecolor{currentstroke}%
\pgfsetdash{{1.850000pt}{0.800000pt}}{0.000000pt}%
\pgfpathmoveto{\pgfqpoint{1.771561in}{0.429287in}}%
\pgfpathlineto{\pgfqpoint{1.771561in}{2.731275in}}%
\pgfusepath{stroke}%
\end{pgfscope}%
\begin{pgfscope}%
\pgfsetbuttcap%
\pgfsetroundjoin%
\definecolor{currentfill}{rgb}{0.180392,0.180392,0.180392}%
\pgfsetfillcolor{currentfill}%
\pgfsetlinewidth{0.803000pt}%
\definecolor{currentstroke}{rgb}{0.180392,0.180392,0.180392}%
\pgfsetstrokecolor{currentstroke}%
\pgfsetdash{}{0pt}%
\pgfsys@defobject{currentmarker}{\pgfqpoint{0.000000in}{-0.048611in}}{\pgfqpoint{0.000000in}{0.000000in}}{%
\pgfpathmoveto{\pgfqpoint{0.000000in}{0.000000in}}%
\pgfpathlineto{\pgfqpoint{0.000000in}{-0.048611in}}%
\pgfusepath{stroke,fill}%
}%
\begin{pgfscope}%
\pgfsys@transformshift{1.771561in}{0.429287in}%
\pgfsys@useobject{currentmarker}{}%
\end{pgfscope}%
\end{pgfscope}%
\begin{pgfscope}%
\definecolor{textcolor}{rgb}{0.180392,0.180392,0.180392}%
\pgfsetstrokecolor{textcolor}%
\pgfsetfillcolor{textcolor}%
\pgftext[x=1.771561in,y=0.332064in,,top]{\color{textcolor}\rmfamily\fontsize{9.000000}{10.800000}\selectfont \(\displaystyle {1}\)}%
\end{pgfscope}%
\begin{pgfscope}%
\pgfpathrectangle{\pgfqpoint{0.564224in}{0.429287in}}{\pgfqpoint{5.634238in}{2.301989in}}%
\pgfusepath{clip}%
\pgfsetbuttcap%
\pgfsetroundjoin%
\pgfsetlinewidth{0.501875pt}%
\definecolor{currentstroke}{rgb}{0.698039,0.698039,0.698039}%
\pgfsetstrokecolor{currentstroke}%
\pgfsetdash{{1.850000pt}{0.800000pt}}{0.000000pt}%
\pgfpathmoveto{\pgfqpoint{2.576452in}{0.429287in}}%
\pgfpathlineto{\pgfqpoint{2.576452in}{2.731275in}}%
\pgfusepath{stroke}%
\end{pgfscope}%
\begin{pgfscope}%
\pgfsetbuttcap%
\pgfsetroundjoin%
\definecolor{currentfill}{rgb}{0.180392,0.180392,0.180392}%
\pgfsetfillcolor{currentfill}%
\pgfsetlinewidth{0.803000pt}%
\definecolor{currentstroke}{rgb}{0.180392,0.180392,0.180392}%
\pgfsetstrokecolor{currentstroke}%
\pgfsetdash{}{0pt}%
\pgfsys@defobject{currentmarker}{\pgfqpoint{0.000000in}{-0.048611in}}{\pgfqpoint{0.000000in}{0.000000in}}{%
\pgfpathmoveto{\pgfqpoint{0.000000in}{0.000000in}}%
\pgfpathlineto{\pgfqpoint{0.000000in}{-0.048611in}}%
\pgfusepath{stroke,fill}%
}%
\begin{pgfscope}%
\pgfsys@transformshift{2.576452in}{0.429287in}%
\pgfsys@useobject{currentmarker}{}%
\end{pgfscope}%
\end{pgfscope}%
\begin{pgfscope}%
\definecolor{textcolor}{rgb}{0.180392,0.180392,0.180392}%
\pgfsetstrokecolor{textcolor}%
\pgfsetfillcolor{textcolor}%
\pgftext[x=2.576452in,y=0.332064in,,top]{\color{textcolor}\rmfamily\fontsize{9.000000}{10.800000}\selectfont \(\displaystyle {2}\)}%
\end{pgfscope}%
\begin{pgfscope}%
\pgfpathrectangle{\pgfqpoint{0.564224in}{0.429287in}}{\pgfqpoint{5.634238in}{2.301989in}}%
\pgfusepath{clip}%
\pgfsetbuttcap%
\pgfsetroundjoin%
\pgfsetlinewidth{0.501875pt}%
\definecolor{currentstroke}{rgb}{0.698039,0.698039,0.698039}%
\pgfsetstrokecolor{currentstroke}%
\pgfsetdash{{1.850000pt}{0.800000pt}}{0.000000pt}%
\pgfpathmoveto{\pgfqpoint{3.381343in}{0.429287in}}%
\pgfpathlineto{\pgfqpoint{3.381343in}{2.731275in}}%
\pgfusepath{stroke}%
\end{pgfscope}%
\begin{pgfscope}%
\pgfsetbuttcap%
\pgfsetroundjoin%
\definecolor{currentfill}{rgb}{0.180392,0.180392,0.180392}%
\pgfsetfillcolor{currentfill}%
\pgfsetlinewidth{0.803000pt}%
\definecolor{currentstroke}{rgb}{0.180392,0.180392,0.180392}%
\pgfsetstrokecolor{currentstroke}%
\pgfsetdash{}{0pt}%
\pgfsys@defobject{currentmarker}{\pgfqpoint{0.000000in}{-0.048611in}}{\pgfqpoint{0.000000in}{0.000000in}}{%
\pgfpathmoveto{\pgfqpoint{0.000000in}{0.000000in}}%
\pgfpathlineto{\pgfqpoint{0.000000in}{-0.048611in}}%
\pgfusepath{stroke,fill}%
}%
\begin{pgfscope}%
\pgfsys@transformshift{3.381343in}{0.429287in}%
\pgfsys@useobject{currentmarker}{}%
\end{pgfscope}%
\end{pgfscope}%
\begin{pgfscope}%
\definecolor{textcolor}{rgb}{0.180392,0.180392,0.180392}%
\pgfsetstrokecolor{textcolor}%
\pgfsetfillcolor{textcolor}%
\pgftext[x=3.381343in,y=0.332064in,,top]{\color{textcolor}\rmfamily\fontsize{9.000000}{10.800000}\selectfont \(\displaystyle {3}\)}%
\end{pgfscope}%
\begin{pgfscope}%
\pgfpathrectangle{\pgfqpoint{0.564224in}{0.429287in}}{\pgfqpoint{5.634238in}{2.301989in}}%
\pgfusepath{clip}%
\pgfsetbuttcap%
\pgfsetroundjoin%
\pgfsetlinewidth{0.501875pt}%
\definecolor{currentstroke}{rgb}{0.698039,0.698039,0.698039}%
\pgfsetstrokecolor{currentstroke}%
\pgfsetdash{{1.850000pt}{0.800000pt}}{0.000000pt}%
\pgfpathmoveto{\pgfqpoint{4.186235in}{0.429287in}}%
\pgfpathlineto{\pgfqpoint{4.186235in}{2.731275in}}%
\pgfusepath{stroke}%
\end{pgfscope}%
\begin{pgfscope}%
\pgfsetbuttcap%
\pgfsetroundjoin%
\definecolor{currentfill}{rgb}{0.180392,0.180392,0.180392}%
\pgfsetfillcolor{currentfill}%
\pgfsetlinewidth{0.803000pt}%
\definecolor{currentstroke}{rgb}{0.180392,0.180392,0.180392}%
\pgfsetstrokecolor{currentstroke}%
\pgfsetdash{}{0pt}%
\pgfsys@defobject{currentmarker}{\pgfqpoint{0.000000in}{-0.048611in}}{\pgfqpoint{0.000000in}{0.000000in}}{%
\pgfpathmoveto{\pgfqpoint{0.000000in}{0.000000in}}%
\pgfpathlineto{\pgfqpoint{0.000000in}{-0.048611in}}%
\pgfusepath{stroke,fill}%
}%
\begin{pgfscope}%
\pgfsys@transformshift{4.186235in}{0.429287in}%
\pgfsys@useobject{currentmarker}{}%
\end{pgfscope}%
\end{pgfscope}%
\begin{pgfscope}%
\definecolor{textcolor}{rgb}{0.180392,0.180392,0.180392}%
\pgfsetstrokecolor{textcolor}%
\pgfsetfillcolor{textcolor}%
\pgftext[x=4.186235in,y=0.332064in,,top]{\color{textcolor}\rmfamily\fontsize{9.000000}{10.800000}\selectfont \(\displaystyle {4}\)}%
\end{pgfscope}%
\begin{pgfscope}%
\pgfpathrectangle{\pgfqpoint{0.564224in}{0.429287in}}{\pgfqpoint{5.634238in}{2.301989in}}%
\pgfusepath{clip}%
\pgfsetbuttcap%
\pgfsetroundjoin%
\pgfsetlinewidth{0.501875pt}%
\definecolor{currentstroke}{rgb}{0.698039,0.698039,0.698039}%
\pgfsetstrokecolor{currentstroke}%
\pgfsetdash{{1.850000pt}{0.800000pt}}{0.000000pt}%
\pgfpathmoveto{\pgfqpoint{4.991126in}{0.429287in}}%
\pgfpathlineto{\pgfqpoint{4.991126in}{2.731275in}}%
\pgfusepath{stroke}%
\end{pgfscope}%
\begin{pgfscope}%
\pgfsetbuttcap%
\pgfsetroundjoin%
\definecolor{currentfill}{rgb}{0.180392,0.180392,0.180392}%
\pgfsetfillcolor{currentfill}%
\pgfsetlinewidth{0.803000pt}%
\definecolor{currentstroke}{rgb}{0.180392,0.180392,0.180392}%
\pgfsetstrokecolor{currentstroke}%
\pgfsetdash{}{0pt}%
\pgfsys@defobject{currentmarker}{\pgfqpoint{0.000000in}{-0.048611in}}{\pgfqpoint{0.000000in}{0.000000in}}{%
\pgfpathmoveto{\pgfqpoint{0.000000in}{0.000000in}}%
\pgfpathlineto{\pgfqpoint{0.000000in}{-0.048611in}}%
\pgfusepath{stroke,fill}%
}%
\begin{pgfscope}%
\pgfsys@transformshift{4.991126in}{0.429287in}%
\pgfsys@useobject{currentmarker}{}%
\end{pgfscope}%
\end{pgfscope}%
\begin{pgfscope}%
\definecolor{textcolor}{rgb}{0.180392,0.180392,0.180392}%
\pgfsetstrokecolor{textcolor}%
\pgfsetfillcolor{textcolor}%
\pgftext[x=4.991126in,y=0.332064in,,top]{\color{textcolor}\rmfamily\fontsize{9.000000}{10.800000}\selectfont \(\displaystyle {5}\)}%
\end{pgfscope}%
\begin{pgfscope}%
\pgfpathrectangle{\pgfqpoint{0.564224in}{0.429287in}}{\pgfqpoint{5.634238in}{2.301989in}}%
\pgfusepath{clip}%
\pgfsetbuttcap%
\pgfsetroundjoin%
\pgfsetlinewidth{0.501875pt}%
\definecolor{currentstroke}{rgb}{0.698039,0.698039,0.698039}%
\pgfsetstrokecolor{currentstroke}%
\pgfsetdash{{1.850000pt}{0.800000pt}}{0.000000pt}%
\pgfpathmoveto{\pgfqpoint{5.796017in}{0.429287in}}%
\pgfpathlineto{\pgfqpoint{5.796017in}{2.731275in}}%
\pgfusepath{stroke}%
\end{pgfscope}%
\begin{pgfscope}%
\pgfsetbuttcap%
\pgfsetroundjoin%
\definecolor{currentfill}{rgb}{0.180392,0.180392,0.180392}%
\pgfsetfillcolor{currentfill}%
\pgfsetlinewidth{0.803000pt}%
\definecolor{currentstroke}{rgb}{0.180392,0.180392,0.180392}%
\pgfsetstrokecolor{currentstroke}%
\pgfsetdash{}{0pt}%
\pgfsys@defobject{currentmarker}{\pgfqpoint{0.000000in}{-0.048611in}}{\pgfqpoint{0.000000in}{0.000000in}}{%
\pgfpathmoveto{\pgfqpoint{0.000000in}{0.000000in}}%
\pgfpathlineto{\pgfqpoint{0.000000in}{-0.048611in}}%
\pgfusepath{stroke,fill}%
}%
\begin{pgfscope}%
\pgfsys@transformshift{5.796017in}{0.429287in}%
\pgfsys@useobject{currentmarker}{}%
\end{pgfscope}%
\end{pgfscope}%
\begin{pgfscope}%
\definecolor{textcolor}{rgb}{0.180392,0.180392,0.180392}%
\pgfsetstrokecolor{textcolor}%
\pgfsetfillcolor{textcolor}%
\pgftext[x=5.796017in,y=0.332064in,,top]{\color{textcolor}\rmfamily\fontsize{9.000000}{10.800000}\selectfont \(\displaystyle {6}\)}%
\end{pgfscope}%
\begin{pgfscope}%
\definecolor{textcolor}{rgb}{0.180392,0.180392,0.180392}%
\pgfsetstrokecolor{textcolor}%
\pgfsetfillcolor{textcolor}%
\pgftext[x=3.381343in,y=0.150823in,,top]{\color{textcolor}\rmfamily\fontsize{10.800000}{12.960000}\selectfont Time (Seconds)}%
\end{pgfscope}%
\begin{pgfscope}%
\pgfpathrectangle{\pgfqpoint{0.564224in}{0.429287in}}{\pgfqpoint{5.634238in}{2.301989in}}%
\pgfusepath{clip}%
\pgfsetbuttcap%
\pgfsetroundjoin%
\pgfsetlinewidth{0.501875pt}%
\definecolor{currentstroke}{rgb}{0.698039,0.698039,0.698039}%
\pgfsetstrokecolor{currentstroke}%
\pgfsetdash{{1.850000pt}{0.800000pt}}{0.000000pt}%
\pgfpathmoveto{\pgfqpoint{0.564224in}{0.533922in}}%
\pgfpathlineto{\pgfqpoint{6.198463in}{0.533922in}}%
\pgfusepath{stroke}%
\end{pgfscope}%
\begin{pgfscope}%
\pgfsetbuttcap%
\pgfsetroundjoin%
\definecolor{currentfill}{rgb}{0.180392,0.180392,0.180392}%
\pgfsetfillcolor{currentfill}%
\pgfsetlinewidth{0.803000pt}%
\definecolor{currentstroke}{rgb}{0.180392,0.180392,0.180392}%
\pgfsetstrokecolor{currentstroke}%
\pgfsetdash{}{0pt}%
\pgfsys@defobject{currentmarker}{\pgfqpoint{-0.048611in}{0.000000in}}{\pgfqpoint{-0.000000in}{0.000000in}}{%
\pgfpathmoveto{\pgfqpoint{-0.000000in}{0.000000in}}%
\pgfpathlineto{\pgfqpoint{-0.048611in}{0.000000in}}%
\pgfusepath{stroke,fill}%
}%
\begin{pgfscope}%
\pgfsys@transformshift{0.564224in}{0.533922in}%
\pgfsys@useobject{currentmarker}{}%
\end{pgfscope}%
\end{pgfscope}%
\begin{pgfscope}%
\definecolor{textcolor}{rgb}{0.180392,0.180392,0.180392}%
\pgfsetstrokecolor{textcolor}%
\pgfsetfillcolor{textcolor}%
\pgftext[x=0.206378in, y=0.488704in, left, base]{\color{textcolor}\rmfamily\fontsize{9.000000}{10.800000}\selectfont \(\displaystyle {\ensuremath{-}2.0}\)}%
\end{pgfscope}%
\begin{pgfscope}%
\pgfpathrectangle{\pgfqpoint{0.564224in}{0.429287in}}{\pgfqpoint{5.634238in}{2.301989in}}%
\pgfusepath{clip}%
\pgfsetbuttcap%
\pgfsetroundjoin%
\pgfsetlinewidth{0.501875pt}%
\definecolor{currentstroke}{rgb}{0.698039,0.698039,0.698039}%
\pgfsetstrokecolor{currentstroke}%
\pgfsetdash{{1.850000pt}{0.800000pt}}{0.000000pt}%
\pgfpathmoveto{\pgfqpoint{0.564224in}{0.795512in}}%
\pgfpathlineto{\pgfqpoint{6.198463in}{0.795512in}}%
\pgfusepath{stroke}%
\end{pgfscope}%
\begin{pgfscope}%
\pgfsetbuttcap%
\pgfsetroundjoin%
\definecolor{currentfill}{rgb}{0.180392,0.180392,0.180392}%
\pgfsetfillcolor{currentfill}%
\pgfsetlinewidth{0.803000pt}%
\definecolor{currentstroke}{rgb}{0.180392,0.180392,0.180392}%
\pgfsetstrokecolor{currentstroke}%
\pgfsetdash{}{0pt}%
\pgfsys@defobject{currentmarker}{\pgfqpoint{-0.048611in}{0.000000in}}{\pgfqpoint{-0.000000in}{0.000000in}}{%
\pgfpathmoveto{\pgfqpoint{-0.000000in}{0.000000in}}%
\pgfpathlineto{\pgfqpoint{-0.048611in}{0.000000in}}%
\pgfusepath{stroke,fill}%
}%
\begin{pgfscope}%
\pgfsys@transformshift{0.564224in}{0.795512in}%
\pgfsys@useobject{currentmarker}{}%
\end{pgfscope}%
\end{pgfscope}%
\begin{pgfscope}%
\definecolor{textcolor}{rgb}{0.180392,0.180392,0.180392}%
\pgfsetstrokecolor{textcolor}%
\pgfsetfillcolor{textcolor}%
\pgftext[x=0.206378in, y=0.750294in, left, base]{\color{textcolor}\rmfamily\fontsize{9.000000}{10.800000}\selectfont \(\displaystyle {\ensuremath{-}1.5}\)}%
\end{pgfscope}%
\begin{pgfscope}%
\pgfpathrectangle{\pgfqpoint{0.564224in}{0.429287in}}{\pgfqpoint{5.634238in}{2.301989in}}%
\pgfusepath{clip}%
\pgfsetbuttcap%
\pgfsetroundjoin%
\pgfsetlinewidth{0.501875pt}%
\definecolor{currentstroke}{rgb}{0.698039,0.698039,0.698039}%
\pgfsetstrokecolor{currentstroke}%
\pgfsetdash{{1.850000pt}{0.800000pt}}{0.000000pt}%
\pgfpathmoveto{\pgfqpoint{0.564224in}{1.057102in}}%
\pgfpathlineto{\pgfqpoint{6.198463in}{1.057102in}}%
\pgfusepath{stroke}%
\end{pgfscope}%
\begin{pgfscope}%
\pgfsetbuttcap%
\pgfsetroundjoin%
\definecolor{currentfill}{rgb}{0.180392,0.180392,0.180392}%
\pgfsetfillcolor{currentfill}%
\pgfsetlinewidth{0.803000pt}%
\definecolor{currentstroke}{rgb}{0.180392,0.180392,0.180392}%
\pgfsetstrokecolor{currentstroke}%
\pgfsetdash{}{0pt}%
\pgfsys@defobject{currentmarker}{\pgfqpoint{-0.048611in}{0.000000in}}{\pgfqpoint{-0.000000in}{0.000000in}}{%
\pgfpathmoveto{\pgfqpoint{-0.000000in}{0.000000in}}%
\pgfpathlineto{\pgfqpoint{-0.048611in}{0.000000in}}%
\pgfusepath{stroke,fill}%
}%
\begin{pgfscope}%
\pgfsys@transformshift{0.564224in}{1.057102in}%
\pgfsys@useobject{currentmarker}{}%
\end{pgfscope}%
\end{pgfscope}%
\begin{pgfscope}%
\definecolor{textcolor}{rgb}{0.180392,0.180392,0.180392}%
\pgfsetstrokecolor{textcolor}%
\pgfsetfillcolor{textcolor}%
\pgftext[x=0.206378in, y=1.011883in, left, base]{\color{textcolor}\rmfamily\fontsize{9.000000}{10.800000}\selectfont \(\displaystyle {\ensuremath{-}1.0}\)}%
\end{pgfscope}%
\begin{pgfscope}%
\pgfpathrectangle{\pgfqpoint{0.564224in}{0.429287in}}{\pgfqpoint{5.634238in}{2.301989in}}%
\pgfusepath{clip}%
\pgfsetbuttcap%
\pgfsetroundjoin%
\pgfsetlinewidth{0.501875pt}%
\definecolor{currentstroke}{rgb}{0.698039,0.698039,0.698039}%
\pgfsetstrokecolor{currentstroke}%
\pgfsetdash{{1.850000pt}{0.800000pt}}{0.000000pt}%
\pgfpathmoveto{\pgfqpoint{0.564224in}{1.318691in}}%
\pgfpathlineto{\pgfqpoint{6.198463in}{1.318691in}}%
\pgfusepath{stroke}%
\end{pgfscope}%
\begin{pgfscope}%
\pgfsetbuttcap%
\pgfsetroundjoin%
\definecolor{currentfill}{rgb}{0.180392,0.180392,0.180392}%
\pgfsetfillcolor{currentfill}%
\pgfsetlinewidth{0.803000pt}%
\definecolor{currentstroke}{rgb}{0.180392,0.180392,0.180392}%
\pgfsetstrokecolor{currentstroke}%
\pgfsetdash{}{0pt}%
\pgfsys@defobject{currentmarker}{\pgfqpoint{-0.048611in}{0.000000in}}{\pgfqpoint{-0.000000in}{0.000000in}}{%
\pgfpathmoveto{\pgfqpoint{-0.000000in}{0.000000in}}%
\pgfpathlineto{\pgfqpoint{-0.048611in}{0.000000in}}%
\pgfusepath{stroke,fill}%
}%
\begin{pgfscope}%
\pgfsys@transformshift{0.564224in}{1.318691in}%
\pgfsys@useobject{currentmarker}{}%
\end{pgfscope}%
\end{pgfscope}%
\begin{pgfscope}%
\definecolor{textcolor}{rgb}{0.180392,0.180392,0.180392}%
\pgfsetstrokecolor{textcolor}%
\pgfsetfillcolor{textcolor}%
\pgftext[x=0.206378in, y=1.273473in, left, base]{\color{textcolor}\rmfamily\fontsize{9.000000}{10.800000}\selectfont \(\displaystyle {\ensuremath{-}0.5}\)}%
\end{pgfscope}%
\begin{pgfscope}%
\pgfpathrectangle{\pgfqpoint{0.564224in}{0.429287in}}{\pgfqpoint{5.634238in}{2.301989in}}%
\pgfusepath{clip}%
\pgfsetbuttcap%
\pgfsetroundjoin%
\pgfsetlinewidth{0.501875pt}%
\definecolor{currentstroke}{rgb}{0.698039,0.698039,0.698039}%
\pgfsetstrokecolor{currentstroke}%
\pgfsetdash{{1.850000pt}{0.800000pt}}{0.000000pt}%
\pgfpathmoveto{\pgfqpoint{0.564224in}{1.580281in}}%
\pgfpathlineto{\pgfqpoint{6.198463in}{1.580281in}}%
\pgfusepath{stroke}%
\end{pgfscope}%
\begin{pgfscope}%
\pgfsetbuttcap%
\pgfsetroundjoin%
\definecolor{currentfill}{rgb}{0.180392,0.180392,0.180392}%
\pgfsetfillcolor{currentfill}%
\pgfsetlinewidth{0.803000pt}%
\definecolor{currentstroke}{rgb}{0.180392,0.180392,0.180392}%
\pgfsetstrokecolor{currentstroke}%
\pgfsetdash{}{0pt}%
\pgfsys@defobject{currentmarker}{\pgfqpoint{-0.048611in}{0.000000in}}{\pgfqpoint{-0.000000in}{0.000000in}}{%
\pgfpathmoveto{\pgfqpoint{-0.000000in}{0.000000in}}%
\pgfpathlineto{\pgfqpoint{-0.048611in}{0.000000in}}%
\pgfusepath{stroke,fill}%
}%
\begin{pgfscope}%
\pgfsys@transformshift{0.564224in}{1.580281in}%
\pgfsys@useobject{currentmarker}{}%
\end{pgfscope}%
\end{pgfscope}%
\begin{pgfscope}%
\definecolor{textcolor}{rgb}{0.180392,0.180392,0.180392}%
\pgfsetstrokecolor{textcolor}%
\pgfsetfillcolor{textcolor}%
\pgftext[x=0.310752in, y=1.535063in, left, base]{\color{textcolor}\rmfamily\fontsize{9.000000}{10.800000}\selectfont \(\displaystyle {0.0}\)}%
\end{pgfscope}%
\begin{pgfscope}%
\pgfpathrectangle{\pgfqpoint{0.564224in}{0.429287in}}{\pgfqpoint{5.634238in}{2.301989in}}%
\pgfusepath{clip}%
\pgfsetbuttcap%
\pgfsetroundjoin%
\pgfsetlinewidth{0.501875pt}%
\definecolor{currentstroke}{rgb}{0.698039,0.698039,0.698039}%
\pgfsetstrokecolor{currentstroke}%
\pgfsetdash{{1.850000pt}{0.800000pt}}{0.000000pt}%
\pgfpathmoveto{\pgfqpoint{0.564224in}{1.841871in}}%
\pgfpathlineto{\pgfqpoint{6.198463in}{1.841871in}}%
\pgfusepath{stroke}%
\end{pgfscope}%
\begin{pgfscope}%
\pgfsetbuttcap%
\pgfsetroundjoin%
\definecolor{currentfill}{rgb}{0.180392,0.180392,0.180392}%
\pgfsetfillcolor{currentfill}%
\pgfsetlinewidth{0.803000pt}%
\definecolor{currentstroke}{rgb}{0.180392,0.180392,0.180392}%
\pgfsetstrokecolor{currentstroke}%
\pgfsetdash{}{0pt}%
\pgfsys@defobject{currentmarker}{\pgfqpoint{-0.048611in}{0.000000in}}{\pgfqpoint{-0.000000in}{0.000000in}}{%
\pgfpathmoveto{\pgfqpoint{-0.000000in}{0.000000in}}%
\pgfpathlineto{\pgfqpoint{-0.048611in}{0.000000in}}%
\pgfusepath{stroke,fill}%
}%
\begin{pgfscope}%
\pgfsys@transformshift{0.564224in}{1.841871in}%
\pgfsys@useobject{currentmarker}{}%
\end{pgfscope}%
\end{pgfscope}%
\begin{pgfscope}%
\definecolor{textcolor}{rgb}{0.180392,0.180392,0.180392}%
\pgfsetstrokecolor{textcolor}%
\pgfsetfillcolor{textcolor}%
\pgftext[x=0.310752in, y=1.796652in, left, base]{\color{textcolor}\rmfamily\fontsize{9.000000}{10.800000}\selectfont \(\displaystyle {0.5}\)}%
\end{pgfscope}%
\begin{pgfscope}%
\pgfpathrectangle{\pgfqpoint{0.564224in}{0.429287in}}{\pgfqpoint{5.634238in}{2.301989in}}%
\pgfusepath{clip}%
\pgfsetbuttcap%
\pgfsetroundjoin%
\pgfsetlinewidth{0.501875pt}%
\definecolor{currentstroke}{rgb}{0.698039,0.698039,0.698039}%
\pgfsetstrokecolor{currentstroke}%
\pgfsetdash{{1.850000pt}{0.800000pt}}{0.000000pt}%
\pgfpathmoveto{\pgfqpoint{0.564224in}{2.103460in}}%
\pgfpathlineto{\pgfqpoint{6.198463in}{2.103460in}}%
\pgfusepath{stroke}%
\end{pgfscope}%
\begin{pgfscope}%
\pgfsetbuttcap%
\pgfsetroundjoin%
\definecolor{currentfill}{rgb}{0.180392,0.180392,0.180392}%
\pgfsetfillcolor{currentfill}%
\pgfsetlinewidth{0.803000pt}%
\definecolor{currentstroke}{rgb}{0.180392,0.180392,0.180392}%
\pgfsetstrokecolor{currentstroke}%
\pgfsetdash{}{0pt}%
\pgfsys@defobject{currentmarker}{\pgfqpoint{-0.048611in}{0.000000in}}{\pgfqpoint{-0.000000in}{0.000000in}}{%
\pgfpathmoveto{\pgfqpoint{-0.000000in}{0.000000in}}%
\pgfpathlineto{\pgfqpoint{-0.048611in}{0.000000in}}%
\pgfusepath{stroke,fill}%
}%
\begin{pgfscope}%
\pgfsys@transformshift{0.564224in}{2.103460in}%
\pgfsys@useobject{currentmarker}{}%
\end{pgfscope}%
\end{pgfscope}%
\begin{pgfscope}%
\definecolor{textcolor}{rgb}{0.180392,0.180392,0.180392}%
\pgfsetstrokecolor{textcolor}%
\pgfsetfillcolor{textcolor}%
\pgftext[x=0.310752in, y=2.058242in, left, base]{\color{textcolor}\rmfamily\fontsize{9.000000}{10.800000}\selectfont \(\displaystyle {1.0}\)}%
\end{pgfscope}%
\begin{pgfscope}%
\pgfpathrectangle{\pgfqpoint{0.564224in}{0.429287in}}{\pgfqpoint{5.634238in}{2.301989in}}%
\pgfusepath{clip}%
\pgfsetbuttcap%
\pgfsetroundjoin%
\pgfsetlinewidth{0.501875pt}%
\definecolor{currentstroke}{rgb}{0.698039,0.698039,0.698039}%
\pgfsetstrokecolor{currentstroke}%
\pgfsetdash{{1.850000pt}{0.800000pt}}{0.000000pt}%
\pgfpathmoveto{\pgfqpoint{0.564224in}{2.365050in}}%
\pgfpathlineto{\pgfqpoint{6.198463in}{2.365050in}}%
\pgfusepath{stroke}%
\end{pgfscope}%
\begin{pgfscope}%
\pgfsetbuttcap%
\pgfsetroundjoin%
\definecolor{currentfill}{rgb}{0.180392,0.180392,0.180392}%
\pgfsetfillcolor{currentfill}%
\pgfsetlinewidth{0.803000pt}%
\definecolor{currentstroke}{rgb}{0.180392,0.180392,0.180392}%
\pgfsetstrokecolor{currentstroke}%
\pgfsetdash{}{0pt}%
\pgfsys@defobject{currentmarker}{\pgfqpoint{-0.048611in}{0.000000in}}{\pgfqpoint{-0.000000in}{0.000000in}}{%
\pgfpathmoveto{\pgfqpoint{-0.000000in}{0.000000in}}%
\pgfpathlineto{\pgfqpoint{-0.048611in}{0.000000in}}%
\pgfusepath{stroke,fill}%
}%
\begin{pgfscope}%
\pgfsys@transformshift{0.564224in}{2.365050in}%
\pgfsys@useobject{currentmarker}{}%
\end{pgfscope}%
\end{pgfscope}%
\begin{pgfscope}%
\definecolor{textcolor}{rgb}{0.180392,0.180392,0.180392}%
\pgfsetstrokecolor{textcolor}%
\pgfsetfillcolor{textcolor}%
\pgftext[x=0.310752in, y=2.319832in, left, base]{\color{textcolor}\rmfamily\fontsize{9.000000}{10.800000}\selectfont \(\displaystyle {1.5}\)}%
\end{pgfscope}%
\begin{pgfscope}%
\pgfpathrectangle{\pgfqpoint{0.564224in}{0.429287in}}{\pgfqpoint{5.634238in}{2.301989in}}%
\pgfusepath{clip}%
\pgfsetbuttcap%
\pgfsetroundjoin%
\pgfsetlinewidth{0.501875pt}%
\definecolor{currentstroke}{rgb}{0.698039,0.698039,0.698039}%
\pgfsetstrokecolor{currentstroke}%
\pgfsetdash{{1.850000pt}{0.800000pt}}{0.000000pt}%
\pgfpathmoveto{\pgfqpoint{0.564224in}{2.626639in}}%
\pgfpathlineto{\pgfqpoint{6.198463in}{2.626639in}}%
\pgfusepath{stroke}%
\end{pgfscope}%
\begin{pgfscope}%
\pgfsetbuttcap%
\pgfsetroundjoin%
\definecolor{currentfill}{rgb}{0.180392,0.180392,0.180392}%
\pgfsetfillcolor{currentfill}%
\pgfsetlinewidth{0.803000pt}%
\definecolor{currentstroke}{rgb}{0.180392,0.180392,0.180392}%
\pgfsetstrokecolor{currentstroke}%
\pgfsetdash{}{0pt}%
\pgfsys@defobject{currentmarker}{\pgfqpoint{-0.048611in}{0.000000in}}{\pgfqpoint{-0.000000in}{0.000000in}}{%
\pgfpathmoveto{\pgfqpoint{-0.000000in}{0.000000in}}%
\pgfpathlineto{\pgfqpoint{-0.048611in}{0.000000in}}%
\pgfusepath{stroke,fill}%
}%
\begin{pgfscope}%
\pgfsys@transformshift{0.564224in}{2.626639in}%
\pgfsys@useobject{currentmarker}{}%
\end{pgfscope}%
\end{pgfscope}%
\begin{pgfscope}%
\definecolor{textcolor}{rgb}{0.180392,0.180392,0.180392}%
\pgfsetstrokecolor{textcolor}%
\pgfsetfillcolor{textcolor}%
\pgftext[x=0.310752in, y=2.581421in, left, base]{\color{textcolor}\rmfamily\fontsize{9.000000}{10.800000}\selectfont \(\displaystyle {2.0}\)}%
\end{pgfscope}%
\begin{pgfscope}%
\definecolor{textcolor}{rgb}{0.180392,0.180392,0.180392}%
\pgfsetstrokecolor{textcolor}%
\pgfsetfillcolor{textcolor}%
\pgftext[x=0.150823in,y=1.580281in,,bottom,rotate=90.000000]{\color{textcolor}\rmfamily\fontsize{10.800000}{12.960000}\selectfont Intensity (Unitless)}%
\end{pgfscope}%
\begin{pgfscope}%
\pgfpathrectangle{\pgfqpoint{0.564224in}{0.429287in}}{\pgfqpoint{5.634238in}{2.301989in}}%
\pgfusepath{clip}%
\pgfsetrectcap%
\pgfsetroundjoin%
\pgfsetlinewidth{2.007500pt}%
\definecolor{currentstroke}{rgb}{0.000000,0.501961,0.000000}%
\pgfsetstrokecolor{currentstroke}%
\pgfsetdash{}{0pt}%
\pgfpathmoveto{\pgfqpoint{0.563500in}{1.581203in}}%
\pgfpathlineto{\pgfqpoint{0.846878in}{1.583526in}}%
\pgfpathlineto{\pgfqpoint{1.029786in}{1.587126in}}%
\pgfpathlineto{\pgfqpoint{1.171475in}{1.592061in}}%
\pgfpathlineto{\pgfqpoint{1.287403in}{1.598229in}}%
\pgfpathlineto{\pgfqpoint{1.387873in}{1.605703in}}%
\pgfpathlineto{\pgfqpoint{1.478039in}{1.614570in}}%
\pgfpathlineto{\pgfqpoint{1.560476in}{1.624864in}}%
\pgfpathlineto{\pgfqpoint{1.637761in}{1.636766in}}%
\pgfpathlineto{\pgfqpoint{1.709894in}{1.650141in}}%
\pgfpathlineto{\pgfqpoint{1.776874in}{1.664773in}}%
\pgfpathlineto{\pgfqpoint{1.841279in}{1.681063in}}%
\pgfpathlineto{\pgfqpoint{1.903107in}{1.698932in}}%
\pgfpathlineto{\pgfqpoint{1.964935in}{1.719148in}}%
\pgfpathlineto{\pgfqpoint{2.024186in}{1.740855in}}%
\pgfpathlineto{\pgfqpoint{2.083438in}{1.764952in}}%
\pgfpathlineto{\pgfqpoint{2.142690in}{1.791517in}}%
\pgfpathlineto{\pgfqpoint{2.201942in}{1.820597in}}%
\pgfpathlineto{\pgfqpoint{2.261194in}{1.852203in}}%
\pgfpathlineto{\pgfqpoint{2.320446in}{1.886304in}}%
\pgfpathlineto{\pgfqpoint{2.382274in}{1.924459in}}%
\pgfpathlineto{\pgfqpoint{2.446678in}{1.966839in}}%
\pgfpathlineto{\pgfqpoint{2.516234in}{2.015350in}}%
\pgfpathlineto{\pgfqpoint{2.593519in}{2.072095in}}%
\pgfpathlineto{\pgfqpoint{2.686261in}{2.143151in}}%
\pgfpathlineto{\pgfqpoint{2.987673in}{2.376551in}}%
\pgfpathlineto{\pgfqpoint{3.054653in}{2.424136in}}%
\pgfpathlineto{\pgfqpoint{3.113905in}{2.463483in}}%
\pgfpathlineto{\pgfqpoint{3.165428in}{2.495147in}}%
\pgfpathlineto{\pgfqpoint{3.214376in}{2.522695in}}%
\pgfpathlineto{\pgfqpoint{3.260747in}{2.546257in}}%
\pgfpathlineto{\pgfqpoint{3.304541in}{2.566042in}}%
\pgfpathlineto{\pgfqpoint{3.345760in}{2.582320in}}%
\pgfpathlineto{\pgfqpoint{3.386979in}{2.596198in}}%
\pgfpathlineto{\pgfqpoint{3.425621in}{2.606936in}}%
\pgfpathlineto{\pgfqpoint{3.464264in}{2.615398in}}%
\pgfpathlineto{\pgfqpoint{3.502906in}{2.621527in}}%
\pgfpathlineto{\pgfqpoint{3.541549in}{2.625282in}}%
\pgfpathlineto{\pgfqpoint{3.577615in}{2.626620in}}%
\pgfpathlineto{\pgfqpoint{3.613681in}{2.625858in}}%
\pgfpathlineto{\pgfqpoint{3.649748in}{2.623001in}}%
\pgfpathlineto{\pgfqpoint{3.688390in}{2.617635in}}%
\pgfpathlineto{\pgfqpoint{3.727033in}{2.609920in}}%
\pgfpathlineto{\pgfqpoint{3.765675in}{2.599910in}}%
\pgfpathlineto{\pgfqpoint{3.804318in}{2.587673in}}%
\pgfpathlineto{\pgfqpoint{3.845537in}{2.572258in}}%
\pgfpathlineto{\pgfqpoint{3.886755in}{2.554521in}}%
\pgfpathlineto{\pgfqpoint{3.930550in}{2.533280in}}%
\pgfpathlineto{\pgfqpoint{3.976921in}{2.508295in}}%
\pgfpathlineto{\pgfqpoint{4.025868in}{2.479389in}}%
\pgfpathlineto{\pgfqpoint{4.079968in}{2.444759in}}%
\pgfpathlineto{\pgfqpoint{4.139220in}{2.404078in}}%
\pgfpathlineto{\pgfqpoint{4.206200in}{2.355317in}}%
\pgfpathlineto{\pgfqpoint{4.288637in}{2.292445in}}%
\pgfpathlineto{\pgfqpoint{4.445783in}{2.169019in}}%
\pgfpathlineto{\pgfqpoint{4.559135in}{2.081494in}}%
\pgfpathlineto{\pgfqpoint{4.641572in}{2.020609in}}%
\pgfpathlineto{\pgfqpoint{4.713705in}{1.970065in}}%
\pgfpathlineto{\pgfqpoint{4.780685in}{1.925847in}}%
\pgfpathlineto{\pgfqpoint{4.842513in}{1.887601in}}%
\pgfpathlineto{\pgfqpoint{4.904341in}{1.851980in}}%
\pgfpathlineto{\pgfqpoint{4.963593in}{1.820391in}}%
\pgfpathlineto{\pgfqpoint{5.022845in}{1.791328in}}%
\pgfpathlineto{\pgfqpoint{5.082097in}{1.764780in}}%
\pgfpathlineto{\pgfqpoint{5.141348in}{1.740700in}}%
\pgfpathlineto{\pgfqpoint{5.200600in}{1.719009in}}%
\pgfpathlineto{\pgfqpoint{5.262428in}{1.698809in}}%
\pgfpathlineto{\pgfqpoint{5.324256in}{1.680954in}}%
\pgfpathlineto{\pgfqpoint{5.388661in}{1.664678in}}%
\pgfpathlineto{\pgfqpoint{5.455641in}{1.650060in}}%
\pgfpathlineto{\pgfqpoint{5.525197in}{1.637136in}}%
\pgfpathlineto{\pgfqpoint{5.599906in}{1.625530in}}%
\pgfpathlineto{\pgfqpoint{5.679767in}{1.615394in}}%
\pgfpathlineto{\pgfqpoint{5.767357in}{1.606569in}}%
\pgfpathlineto{\pgfqpoint{5.865251in}{1.599038in}}%
\pgfpathlineto{\pgfqpoint{5.976027in}{1.592857in}}%
\pgfpathlineto{\pgfqpoint{6.104835in}{1.587995in}}%
\pgfpathlineto{\pgfqpoint{6.200153in}{1.585567in}}%
\pgfpathlineto{\pgfqpoint{6.200153in}{1.585567in}}%
\pgfusepath{stroke}%
\end{pgfscope}%
\begin{pgfscope}%
\pgfpathrectangle{\pgfqpoint{0.564224in}{0.429287in}}{\pgfqpoint{5.634238in}{2.301989in}}%
\pgfusepath{clip}%
\pgfsetrectcap%
\pgfsetroundjoin%
\pgfsetlinewidth{2.007500pt}%
\definecolor{currentstroke}{rgb}{0.000000,0.501961,0.000000}%
\pgfsetstrokecolor{currentstroke}%
\pgfsetdash{}{0pt}%
\pgfpathmoveto{\pgfqpoint{0.563500in}{1.579359in}}%
\pgfpathlineto{\pgfqpoint{0.846878in}{1.577036in}}%
\pgfpathlineto{\pgfqpoint{1.029786in}{1.573435in}}%
\pgfpathlineto{\pgfqpoint{1.171475in}{1.568501in}}%
\pgfpathlineto{\pgfqpoint{1.287403in}{1.562333in}}%
\pgfpathlineto{\pgfqpoint{1.387873in}{1.554858in}}%
\pgfpathlineto{\pgfqpoint{1.478039in}{1.545992in}}%
\pgfpathlineto{\pgfqpoint{1.560476in}{1.535697in}}%
\pgfpathlineto{\pgfqpoint{1.637761in}{1.523796in}}%
\pgfpathlineto{\pgfqpoint{1.709894in}{1.510420in}}%
\pgfpathlineto{\pgfqpoint{1.776874in}{1.495789in}}%
\pgfpathlineto{\pgfqpoint{1.841279in}{1.479499in}}%
\pgfpathlineto{\pgfqpoint{1.903107in}{1.461629in}}%
\pgfpathlineto{\pgfqpoint{1.964935in}{1.441413in}}%
\pgfpathlineto{\pgfqpoint{2.024186in}{1.419706in}}%
\pgfpathlineto{\pgfqpoint{2.083438in}{1.395610in}}%
\pgfpathlineto{\pgfqpoint{2.142690in}{1.369045in}}%
\pgfpathlineto{\pgfqpoint{2.201942in}{1.339965in}}%
\pgfpathlineto{\pgfqpoint{2.261194in}{1.308358in}}%
\pgfpathlineto{\pgfqpoint{2.320446in}{1.274258in}}%
\pgfpathlineto{\pgfqpoint{2.382274in}{1.236102in}}%
\pgfpathlineto{\pgfqpoint{2.446678in}{1.193722in}}%
\pgfpathlineto{\pgfqpoint{2.516234in}{1.145212in}}%
\pgfpathlineto{\pgfqpoint{2.593519in}{1.088466in}}%
\pgfpathlineto{\pgfqpoint{2.686261in}{1.017411in}}%
\pgfpathlineto{\pgfqpoint{2.987673in}{0.784011in}}%
\pgfpathlineto{\pgfqpoint{3.054653in}{0.736425in}}%
\pgfpathlineto{\pgfqpoint{3.113905in}{0.697079in}}%
\pgfpathlineto{\pgfqpoint{3.165428in}{0.665415in}}%
\pgfpathlineto{\pgfqpoint{3.214376in}{0.637867in}}%
\pgfpathlineto{\pgfqpoint{3.260747in}{0.614305in}}%
\pgfpathlineto{\pgfqpoint{3.304541in}{0.594520in}}%
\pgfpathlineto{\pgfqpoint{3.345760in}{0.578242in}}%
\pgfpathlineto{\pgfqpoint{3.386979in}{0.564364in}}%
\pgfpathlineto{\pgfqpoint{3.425621in}{0.553626in}}%
\pgfpathlineto{\pgfqpoint{3.464264in}{0.545164in}}%
\pgfpathlineto{\pgfqpoint{3.502906in}{0.539034in}}%
\pgfpathlineto{\pgfqpoint{3.541549in}{0.535280in}}%
\pgfpathlineto{\pgfqpoint{3.577615in}{0.533942in}}%
\pgfpathlineto{\pgfqpoint{3.613681in}{0.534704in}}%
\pgfpathlineto{\pgfqpoint{3.649748in}{0.537561in}}%
\pgfpathlineto{\pgfqpoint{3.688390in}{0.542927in}}%
\pgfpathlineto{\pgfqpoint{3.727033in}{0.550642in}}%
\pgfpathlineto{\pgfqpoint{3.765675in}{0.560652in}}%
\pgfpathlineto{\pgfqpoint{3.804318in}{0.572889in}}%
\pgfpathlineto{\pgfqpoint{3.845537in}{0.588304in}}%
\pgfpathlineto{\pgfqpoint{3.886755in}{0.606041in}}%
\pgfpathlineto{\pgfqpoint{3.930550in}{0.627282in}}%
\pgfpathlineto{\pgfqpoint{3.976921in}{0.652267in}}%
\pgfpathlineto{\pgfqpoint{4.025868in}{0.681173in}}%
\pgfpathlineto{\pgfqpoint{4.079968in}{0.715803in}}%
\pgfpathlineto{\pgfqpoint{4.139220in}{0.756484in}}%
\pgfpathlineto{\pgfqpoint{4.206200in}{0.805244in}}%
\pgfpathlineto{\pgfqpoint{4.288637in}{0.868116in}}%
\pgfpathlineto{\pgfqpoint{4.445783in}{0.991543in}}%
\pgfpathlineto{\pgfqpoint{4.559135in}{1.079068in}}%
\pgfpathlineto{\pgfqpoint{4.641572in}{1.139952in}}%
\pgfpathlineto{\pgfqpoint{4.713705in}{1.190497in}}%
\pgfpathlineto{\pgfqpoint{4.780685in}{1.234715in}}%
\pgfpathlineto{\pgfqpoint{4.842513in}{1.272961in}}%
\pgfpathlineto{\pgfqpoint{4.904341in}{1.308581in}}%
\pgfpathlineto{\pgfqpoint{4.963593in}{1.340171in}}%
\pgfpathlineto{\pgfqpoint{5.022845in}{1.369234in}}%
\pgfpathlineto{\pgfqpoint{5.082097in}{1.395782in}}%
\pgfpathlineto{\pgfqpoint{5.141348in}{1.419862in}}%
\pgfpathlineto{\pgfqpoint{5.200600in}{1.441553in}}%
\pgfpathlineto{\pgfqpoint{5.262428in}{1.461753in}}%
\pgfpathlineto{\pgfqpoint{5.324256in}{1.479608in}}%
\pgfpathlineto{\pgfqpoint{5.388661in}{1.495884in}}%
\pgfpathlineto{\pgfqpoint{5.455641in}{1.510502in}}%
\pgfpathlineto{\pgfqpoint{5.525197in}{1.523426in}}%
\pgfpathlineto{\pgfqpoint{5.599906in}{1.535032in}}%
\pgfpathlineto{\pgfqpoint{5.679767in}{1.545168in}}%
\pgfpathlineto{\pgfqpoint{5.767357in}{1.553993in}}%
\pgfpathlineto{\pgfqpoint{5.865251in}{1.561524in}}%
\pgfpathlineto{\pgfqpoint{5.976027in}{1.567705in}}%
\pgfpathlineto{\pgfqpoint{6.104835in}{1.572566in}}%
\pgfpathlineto{\pgfqpoint{6.200153in}{1.574995in}}%
\pgfpathlineto{\pgfqpoint{6.200153in}{1.574995in}}%
\pgfusepath{stroke}%
\end{pgfscope}%
\begin{pgfscope}%
\pgfpathrectangle{\pgfqpoint{0.564224in}{0.429287in}}{\pgfqpoint{5.634238in}{2.301989in}}%
\pgfusepath{clip}%
\pgfsetbuttcap%
\pgfsetroundjoin%
\pgfsetlinewidth{2.007500pt}%
\definecolor{currentstroke}{rgb}{1.000000,0.647059,0.000000}%
\pgfsetstrokecolor{currentstroke}%
\pgfsetdash{{2.000000pt}{3.300000pt}}{0.000000pt}%
\pgfpathmoveto{\pgfqpoint{0.563500in}{1.581203in}}%
\pgfpathlineto{\pgfqpoint{0.846878in}{1.583526in}}%
\pgfpathlineto{\pgfqpoint{1.029786in}{1.587126in}}%
\pgfpathlineto{\pgfqpoint{1.171475in}{1.592061in}}%
\pgfpathlineto{\pgfqpoint{1.287403in}{1.598229in}}%
\pgfpathlineto{\pgfqpoint{1.387873in}{1.605703in}}%
\pgfpathlineto{\pgfqpoint{1.478039in}{1.614570in}}%
\pgfpathlineto{\pgfqpoint{1.560476in}{1.624864in}}%
\pgfpathlineto{\pgfqpoint{1.637761in}{1.636766in}}%
\pgfpathlineto{\pgfqpoint{1.709894in}{1.650141in}}%
\pgfpathlineto{\pgfqpoint{1.776874in}{1.664773in}}%
\pgfpathlineto{\pgfqpoint{1.841279in}{1.681063in}}%
\pgfpathlineto{\pgfqpoint{1.903107in}{1.698932in}}%
\pgfpathlineto{\pgfqpoint{1.964935in}{1.719148in}}%
\pgfpathlineto{\pgfqpoint{2.024186in}{1.740855in}}%
\pgfpathlineto{\pgfqpoint{2.083438in}{1.764952in}}%
\pgfpathlineto{\pgfqpoint{2.142690in}{1.791517in}}%
\pgfpathlineto{\pgfqpoint{2.201942in}{1.820597in}}%
\pgfpathlineto{\pgfqpoint{2.261194in}{1.852203in}}%
\pgfpathlineto{\pgfqpoint{2.320446in}{1.886304in}}%
\pgfpathlineto{\pgfqpoint{2.382274in}{1.924459in}}%
\pgfpathlineto{\pgfqpoint{2.446678in}{1.966839in}}%
\pgfpathlineto{\pgfqpoint{2.516234in}{2.015350in}}%
\pgfpathlineto{\pgfqpoint{2.593519in}{2.072095in}}%
\pgfpathlineto{\pgfqpoint{2.686261in}{2.143151in}}%
\pgfpathlineto{\pgfqpoint{2.987673in}{2.376551in}}%
\pgfpathlineto{\pgfqpoint{3.054653in}{2.424136in}}%
\pgfpathlineto{\pgfqpoint{3.113905in}{2.463483in}}%
\pgfpathlineto{\pgfqpoint{3.165428in}{2.495147in}}%
\pgfpathlineto{\pgfqpoint{3.214376in}{2.522695in}}%
\pgfpathlineto{\pgfqpoint{3.260747in}{2.546257in}}%
\pgfpathlineto{\pgfqpoint{3.304541in}{2.566042in}}%
\pgfpathlineto{\pgfqpoint{3.345760in}{2.582320in}}%
\pgfpathlineto{\pgfqpoint{3.386979in}{2.596198in}}%
\pgfpathlineto{\pgfqpoint{3.425621in}{2.606936in}}%
\pgfpathlineto{\pgfqpoint{3.464264in}{2.615398in}}%
\pgfpathlineto{\pgfqpoint{3.502906in}{2.621527in}}%
\pgfpathlineto{\pgfqpoint{3.541549in}{2.625282in}}%
\pgfpathlineto{\pgfqpoint{3.577615in}{2.626620in}}%
\pgfpathlineto{\pgfqpoint{3.613681in}{2.625858in}}%
\pgfpathlineto{\pgfqpoint{3.649748in}{2.623001in}}%
\pgfpathlineto{\pgfqpoint{3.688390in}{2.617635in}}%
\pgfpathlineto{\pgfqpoint{3.727033in}{2.609920in}}%
\pgfpathlineto{\pgfqpoint{3.765675in}{2.599910in}}%
\pgfpathlineto{\pgfqpoint{3.804318in}{2.587673in}}%
\pgfpathlineto{\pgfqpoint{3.845537in}{2.572258in}}%
\pgfpathlineto{\pgfqpoint{3.886755in}{2.554521in}}%
\pgfpathlineto{\pgfqpoint{3.930550in}{2.533280in}}%
\pgfpathlineto{\pgfqpoint{3.976921in}{2.508295in}}%
\pgfpathlineto{\pgfqpoint{4.025868in}{2.479389in}}%
\pgfpathlineto{\pgfqpoint{4.079968in}{2.444759in}}%
\pgfpathlineto{\pgfqpoint{4.139220in}{2.404078in}}%
\pgfpathlineto{\pgfqpoint{4.206200in}{2.355317in}}%
\pgfpathlineto{\pgfqpoint{4.288637in}{2.292445in}}%
\pgfpathlineto{\pgfqpoint{4.445783in}{2.169019in}}%
\pgfpathlineto{\pgfqpoint{4.559135in}{2.081494in}}%
\pgfpathlineto{\pgfqpoint{4.641572in}{2.020609in}}%
\pgfpathlineto{\pgfqpoint{4.713705in}{1.970065in}}%
\pgfpathlineto{\pgfqpoint{4.780685in}{1.925847in}}%
\pgfpathlineto{\pgfqpoint{4.842513in}{1.887601in}}%
\pgfpathlineto{\pgfqpoint{4.904341in}{1.851980in}}%
\pgfpathlineto{\pgfqpoint{4.963593in}{1.820391in}}%
\pgfpathlineto{\pgfqpoint{5.022845in}{1.791328in}}%
\pgfpathlineto{\pgfqpoint{5.082097in}{1.764780in}}%
\pgfpathlineto{\pgfqpoint{5.141348in}{1.740700in}}%
\pgfpathlineto{\pgfqpoint{5.200600in}{1.719009in}}%
\pgfpathlineto{\pgfqpoint{5.262428in}{1.698809in}}%
\pgfpathlineto{\pgfqpoint{5.324256in}{1.680954in}}%
\pgfpathlineto{\pgfqpoint{5.388661in}{1.664678in}}%
\pgfpathlineto{\pgfqpoint{5.455641in}{1.650060in}}%
\pgfpathlineto{\pgfqpoint{5.525197in}{1.637136in}}%
\pgfpathlineto{\pgfqpoint{5.599906in}{1.625530in}}%
\pgfpathlineto{\pgfqpoint{5.679767in}{1.615394in}}%
\pgfpathlineto{\pgfqpoint{5.767357in}{1.606569in}}%
\pgfpathlineto{\pgfqpoint{5.865251in}{1.599038in}}%
\pgfpathlineto{\pgfqpoint{5.976027in}{1.592857in}}%
\pgfpathlineto{\pgfqpoint{6.104835in}{1.587995in}}%
\pgfpathlineto{\pgfqpoint{6.200153in}{1.585567in}}%
\pgfpathlineto{\pgfqpoint{6.200153in}{1.585567in}}%
\pgfusepath{stroke}%
\end{pgfscope}%
\begin{pgfscope}%
\pgfpathrectangle{\pgfqpoint{0.564224in}{0.429287in}}{\pgfqpoint{5.634238in}{2.301989in}}%
\pgfusepath{clip}%
\pgfsetbuttcap%
\pgfsetroundjoin%
\pgfsetlinewidth{2.007500pt}%
\definecolor{currentstroke}{rgb}{1.000000,0.647059,0.000000}%
\pgfsetstrokecolor{currentstroke}%
\pgfsetdash{{2.000000pt}{3.300000pt}}{0.000000pt}%
\pgfpathmoveto{\pgfqpoint{0.563500in}{1.579359in}}%
\pgfpathlineto{\pgfqpoint{0.846878in}{1.577036in}}%
\pgfpathlineto{\pgfqpoint{1.029786in}{1.573435in}}%
\pgfpathlineto{\pgfqpoint{1.171475in}{1.568501in}}%
\pgfpathlineto{\pgfqpoint{1.287403in}{1.562333in}}%
\pgfpathlineto{\pgfqpoint{1.387873in}{1.554858in}}%
\pgfpathlineto{\pgfqpoint{1.478039in}{1.545992in}}%
\pgfpathlineto{\pgfqpoint{1.560476in}{1.535697in}}%
\pgfpathlineto{\pgfqpoint{1.637761in}{1.523796in}}%
\pgfpathlineto{\pgfqpoint{1.709894in}{1.510420in}}%
\pgfpathlineto{\pgfqpoint{1.776874in}{1.495789in}}%
\pgfpathlineto{\pgfqpoint{1.841279in}{1.479499in}}%
\pgfpathlineto{\pgfqpoint{1.903107in}{1.461629in}}%
\pgfpathlineto{\pgfqpoint{1.964935in}{1.441413in}}%
\pgfpathlineto{\pgfqpoint{2.024186in}{1.419706in}}%
\pgfpathlineto{\pgfqpoint{2.083438in}{1.395610in}}%
\pgfpathlineto{\pgfqpoint{2.142690in}{1.369045in}}%
\pgfpathlineto{\pgfqpoint{2.201942in}{1.339965in}}%
\pgfpathlineto{\pgfqpoint{2.261194in}{1.308358in}}%
\pgfpathlineto{\pgfqpoint{2.320446in}{1.274258in}}%
\pgfpathlineto{\pgfqpoint{2.382274in}{1.236102in}}%
\pgfpathlineto{\pgfqpoint{2.446678in}{1.193722in}}%
\pgfpathlineto{\pgfqpoint{2.516234in}{1.145212in}}%
\pgfpathlineto{\pgfqpoint{2.593519in}{1.088466in}}%
\pgfpathlineto{\pgfqpoint{2.686261in}{1.017411in}}%
\pgfpathlineto{\pgfqpoint{2.987673in}{0.784011in}}%
\pgfpathlineto{\pgfqpoint{3.054653in}{0.736425in}}%
\pgfpathlineto{\pgfqpoint{3.113905in}{0.697079in}}%
\pgfpathlineto{\pgfqpoint{3.165428in}{0.665415in}}%
\pgfpathlineto{\pgfqpoint{3.214376in}{0.637867in}}%
\pgfpathlineto{\pgfqpoint{3.260747in}{0.614305in}}%
\pgfpathlineto{\pgfqpoint{3.304541in}{0.594520in}}%
\pgfpathlineto{\pgfqpoint{3.345760in}{0.578242in}}%
\pgfpathlineto{\pgfqpoint{3.386979in}{0.564364in}}%
\pgfpathlineto{\pgfqpoint{3.425621in}{0.553626in}}%
\pgfpathlineto{\pgfqpoint{3.464264in}{0.545164in}}%
\pgfpathlineto{\pgfqpoint{3.502906in}{0.539034in}}%
\pgfpathlineto{\pgfqpoint{3.541549in}{0.535280in}}%
\pgfpathlineto{\pgfqpoint{3.577615in}{0.533942in}}%
\pgfpathlineto{\pgfqpoint{3.613681in}{0.534704in}}%
\pgfpathlineto{\pgfqpoint{3.649748in}{0.537561in}}%
\pgfpathlineto{\pgfqpoint{3.688390in}{0.542927in}}%
\pgfpathlineto{\pgfqpoint{3.727033in}{0.550642in}}%
\pgfpathlineto{\pgfqpoint{3.765675in}{0.560652in}}%
\pgfpathlineto{\pgfqpoint{3.804318in}{0.572889in}}%
\pgfpathlineto{\pgfqpoint{3.845537in}{0.588304in}}%
\pgfpathlineto{\pgfqpoint{3.886755in}{0.606041in}}%
\pgfpathlineto{\pgfqpoint{3.930550in}{0.627282in}}%
\pgfpathlineto{\pgfqpoint{3.976921in}{0.652267in}}%
\pgfpathlineto{\pgfqpoint{4.025868in}{0.681173in}}%
\pgfpathlineto{\pgfqpoint{4.079968in}{0.715803in}}%
\pgfpathlineto{\pgfqpoint{4.139220in}{0.756484in}}%
\pgfpathlineto{\pgfqpoint{4.206200in}{0.805244in}}%
\pgfpathlineto{\pgfqpoint{4.288637in}{0.868116in}}%
\pgfpathlineto{\pgfqpoint{4.445783in}{0.991543in}}%
\pgfpathlineto{\pgfqpoint{4.559135in}{1.079068in}}%
\pgfpathlineto{\pgfqpoint{4.641572in}{1.139952in}}%
\pgfpathlineto{\pgfqpoint{4.713705in}{1.190497in}}%
\pgfpathlineto{\pgfqpoint{4.780685in}{1.234715in}}%
\pgfpathlineto{\pgfqpoint{4.842513in}{1.272961in}}%
\pgfpathlineto{\pgfqpoint{4.904341in}{1.308581in}}%
\pgfpathlineto{\pgfqpoint{4.963593in}{1.340171in}}%
\pgfpathlineto{\pgfqpoint{5.022845in}{1.369234in}}%
\pgfpathlineto{\pgfqpoint{5.082097in}{1.395782in}}%
\pgfpathlineto{\pgfqpoint{5.141348in}{1.419862in}}%
\pgfpathlineto{\pgfqpoint{5.200600in}{1.441553in}}%
\pgfpathlineto{\pgfqpoint{5.262428in}{1.461753in}}%
\pgfpathlineto{\pgfqpoint{5.324256in}{1.479608in}}%
\pgfpathlineto{\pgfqpoint{5.388661in}{1.495884in}}%
\pgfpathlineto{\pgfqpoint{5.455641in}{1.510502in}}%
\pgfpathlineto{\pgfqpoint{5.525197in}{1.523426in}}%
\pgfpathlineto{\pgfqpoint{5.599906in}{1.535032in}}%
\pgfpathlineto{\pgfqpoint{5.679767in}{1.545168in}}%
\pgfpathlineto{\pgfqpoint{5.767357in}{1.553993in}}%
\pgfpathlineto{\pgfqpoint{5.865251in}{1.561524in}}%
\pgfpathlineto{\pgfqpoint{5.976027in}{1.567705in}}%
\pgfpathlineto{\pgfqpoint{6.104835in}{1.572566in}}%
\pgfpathlineto{\pgfqpoint{6.200153in}{1.574995in}}%
\pgfpathlineto{\pgfqpoint{6.200153in}{1.574995in}}%
\pgfusepath{stroke}%
\end{pgfscope}%
\begin{pgfscope}%
\pgfsetrectcap%
\pgfsetmiterjoin%
\pgfsetlinewidth{0.803000pt}%
\definecolor{currentstroke}{rgb}{0.737255,0.737255,0.737255}%
\pgfsetstrokecolor{currentstroke}%
\pgfsetdash{}{0pt}%
\pgfpathmoveto{\pgfqpoint{0.564224in}{0.429287in}}%
\pgfpathlineto{\pgfqpoint{0.564224in}{2.731275in}}%
\pgfusepath{stroke}%
\end{pgfscope}%
\begin{pgfscope}%
\pgfsetrectcap%
\pgfsetmiterjoin%
\pgfsetlinewidth{0.803000pt}%
\definecolor{currentstroke}{rgb}{0.737255,0.737255,0.737255}%
\pgfsetstrokecolor{currentstroke}%
\pgfsetdash{}{0pt}%
\pgfpathmoveto{\pgfqpoint{6.198463in}{0.429287in}}%
\pgfpathlineto{\pgfqpoint{6.198463in}{2.731275in}}%
\pgfusepath{stroke}%
\end{pgfscope}%
\begin{pgfscope}%
\pgfsetrectcap%
\pgfsetmiterjoin%
\pgfsetlinewidth{0.803000pt}%
\definecolor{currentstroke}{rgb}{0.737255,0.737255,0.737255}%
\pgfsetstrokecolor{currentstroke}%
\pgfsetdash{}{0pt}%
\pgfpathmoveto{\pgfqpoint{0.564224in}{0.429287in}}%
\pgfpathlineto{\pgfqpoint{6.198463in}{0.429287in}}%
\pgfusepath{stroke}%
\end{pgfscope}%
\begin{pgfscope}%
\pgfsetrectcap%
\pgfsetmiterjoin%
\pgfsetlinewidth{0.803000pt}%
\definecolor{currentstroke}{rgb}{0.737255,0.737255,0.737255}%
\pgfsetstrokecolor{currentstroke}%
\pgfsetdash{}{0pt}%
\pgfpathmoveto{\pgfqpoint{0.564224in}{2.731275in}}%
\pgfpathlineto{\pgfqpoint{6.198463in}{2.731275in}}%
\pgfusepath{stroke}%
\end{pgfscope}%
\begin{pgfscope}%
\definecolor{textcolor}{rgb}{0.180392,0.180392,0.180392}%
\pgfsetstrokecolor{textcolor}%
\pgfsetfillcolor{textcolor}%
\pgftext[x=3.381343in,y=2.814609in,,base]{\color{textcolor}\rmfamily\fontsize{12.960000}{15.552000}\selectfont Envelope of the Signal}%
\end{pgfscope}%
\end{pgfpicture}%
\makeatother%
\endgroup%
}
				\caption{\centering{The difference between the continuous signal and discretely measured data-points of the signal}}
				\label{fig:envelope_continuous_discrete}
			\end{figure}	
			
			\textbf{TODO: something about having to use multiple matrices}
			
			
			\pagebreak
			
			
		\subsubsection{Dispersive Step}
			To solve for the dispersive step of the equation, we have to solve Equation \ref{eq:dispersive_solution}: 
			
			\begin{equation} \label{eq:dispersive_solution}
				\frac{\partial F(z, t)}{\partial z} = - j \frac{ \beta ''}{2} \frac{\partial^2}{\partial t^2}
			\end{equation}
			
			\vspace{5mm}
			
			To do this we must consider the Fourier transform of $F(z, t)$, given by Equation \ref{eq:dispersive_solution_ft}:
			
			\begin{equation} \label{eq:dispersive_solution_ft}
				\mathlarger{\tilde{F}(z, \omega) = \int_{-\infty}^{+\infty} F(z, t) e^{-j \omega} dt}
			\end{equation}
			
			
			\vspace{5mm}
			
			
			We are interested in change with propagation, so computing this integral produces and differentiating gives:
			\begin{equation}
				\mathlarger{\frac{\partial \tilde{F}(z, \omega)}{\partial z} = (+ j \frac{\omega^2}{2} \beta '') \tilde{F}(z, \omega)} \nonumber
			\end{equation}
			
			\vspace{5mm}
			
			And finally to form a general solution to calculate the new signal value after each dispersive portion of the step, $h$, the necessary calculations before completing the non-linear portion are shown in frequency and time domain as Equations \ref{eq:dispersive_solution_freq} and \ref{eq:dispersive_solution_time}:
			
			\begin{gather} \label{eq:dispersive_solution_freq} 
				\mathlarger{\tilde{F}(z+h, \omega) =  \tilde{F}(z, \omega) e^{+ j \frac{\omega^2}{2} \beta '' \cdot h} }\\
				\mathlarger{F(z+h, t) = \mathcal{FT}^{-1} \begin{bmatrix} \tilde{F}(z+h, \omega) \end{bmatrix}} \label{eq:dispersive_solution_time}
			\end{gather}
			
			\vspace{5mm}
		
		
		\subsubsection{Non-Linear Step}
			For the non-linear portion of our regime we begin with Equation \ref{eq:non_linear_solution}:
			
			\begin{equation} \label{eq:non_linear_solution}
				\mathlarger{\frac{\partial F(z, t)}{\partial z} = j \gamma |F|^2 F}
			\end{equation}
			
			\vspace{5mm}
			
			Since one can demonstrate that $|F|^2$ is z-independent, the final solution is given by Equation \ref{eq:non_linear_solution_final}:

			\begin{equation} \label{eq:non_linear_solution_final}
				\mathlarger{ F(z+h, t) =  F(z, t) e^{j \gamma |F(z, t)|^2 \cdot h}}
			\end{equation}
			
			{\footnotesize Recall that $h$ will be our step size in the z-axis. Also note this is no th}
			
			
			\pagebreak
			
			
	\subsection{Summary of the Scheme for Numeric Algorithm}
		The summary of the steps we will take in order to complete the numeric implementation of everything we have covered in this section is as follows: \linebreak
		
		\begin{enumerate}[leftmargin=1cm]
			\item Initialisation (of variables, matrices, and any necessary functions or classes)
			\item Definition of:
				\begin{itemize}
					\item Temporal and Spatial Co-Ordinates
					\item Frequency Domain Co-Ordinates
					\item Quantities relevant to our signal (intensities, envelope definition, etc.)
				\end{itemize}
			\item Perform the Split-Step Fourier Method:
				\begin{itemize}
					\item The Dispersive Step
					\item The Non-Linear Step
					\item Save the data for each $z$ step at each time index
				\end{itemize}
			\item Display the results appropriately
		\end{enumerate}
		
		
		\pagebreak
		
		
	\subsection{Numeric Implementation in Python}
		The code can be found below in Section \ref{code:Split_step_fourier_intro}, it has been purposefully made more difficult to copy and paste so that you will, at the very least, type the code out manually and perhaps absorb some of the essential information regarding the implementation of the Split-Step Fourier Method. \linebreak
		
		This is only introductory code, so we aren't actually going to look at any propagation that involves dispersion or non-linearity, but the two relevant terms, $\beta ''$ and $\gamma$, can be changed to observe this (in the next sections this is what will be done in order to produce the graphs to observe the various effects). \linebreak
		
		The first graph shown below, Figure \ref{fig:split-step-intro-2d-time}, displays the input envelope and final, output envelope in the time domain. In this case it is a simple Gaussian signal with a half-waist of 1 second and an absolute intensity of 1. \linebreak
		
		\begin{figure}[h]
			\centering
			\scalebox{0.85}{%% Creator: Matplotlib, PGF backend
%%
%% To include the figure in your LaTeX document, write
%%   \input{<filename>.pgf}
%%
%% Make sure the required packages are loaded in your preamble
%%   \usepackage{pgf}
%%
%% Also ensure that all the required font packages are loaded; for instance,
%% the lmodern package is sometimes necessary when using math font.
%%   \usepackage{lmodern}
%%
%% Figures using additional raster images can only be included by \input if
%% they are in the same directory as the main LaTeX file. For loading figures
%% from other directories you can use the `import` package
%%   \usepackage{import}
%%
%% and then include the figures with
%%   \import{<path to file>}{<filename>.pgf}
%%
%% Matplotlib used the following preamble
%%   \usepackage[T1]{fontenc} \usepackage{mathpazo}
%%
\begingroup%
\makeatletter%
\begin{pgfpicture}%
\pgfpathrectangle{\pgfpointorigin}{\pgfqpoint{6.098220in}{2.952062in}}%
\pgfusepath{use as bounding box, clip}%
\begin{pgfscope}%
\pgfsetbuttcap%
\pgfsetmiterjoin%
\definecolor{currentfill}{rgb}{1.000000,1.000000,1.000000}%
\pgfsetfillcolor{currentfill}%
\pgfsetlinewidth{0.000000pt}%
\definecolor{currentstroke}{rgb}{1.000000,1.000000,1.000000}%
\pgfsetstrokecolor{currentstroke}%
\pgfsetdash{}{0pt}%
\pgfpathmoveto{\pgfqpoint{0.000000in}{0.000000in}}%
\pgfpathlineto{\pgfqpoint{6.098220in}{0.000000in}}%
\pgfpathlineto{\pgfqpoint{6.098220in}{2.952062in}}%
\pgfpathlineto{\pgfqpoint{0.000000in}{2.952062in}}%
\pgfpathlineto{\pgfqpoint{0.000000in}{0.000000in}}%
\pgfpathclose%
\pgfusepath{fill}%
\end{pgfscope}%
\begin{pgfscope}%
\pgfsetbuttcap%
\pgfsetmiterjoin%
\definecolor{currentfill}{rgb}{0.933333,0.933333,0.933333}%
\pgfsetfillcolor{currentfill}%
\pgfsetlinewidth{0.000000pt}%
\definecolor{currentstroke}{rgb}{0.000000,0.000000,0.000000}%
\pgfsetstrokecolor{currentstroke}%
\pgfsetstrokeopacity{0.000000}%
\pgfsetdash{}{0pt}%
\pgfpathmoveto{\pgfqpoint{0.465477in}{0.429287in}}%
\pgfpathlineto{\pgfqpoint{6.035720in}{0.429287in}}%
\pgfpathlineto{\pgfqpoint{6.035720in}{2.738500in}}%
\pgfpathlineto{\pgfqpoint{0.465477in}{2.738500in}}%
\pgfpathlineto{\pgfqpoint{0.465477in}{0.429287in}}%
\pgfpathclose%
\pgfusepath{fill}%
\end{pgfscope}%
\begin{pgfscope}%
\pgfpathrectangle{\pgfqpoint{0.465477in}{0.429287in}}{\pgfqpoint{5.570243in}{2.309214in}}%
\pgfusepath{clip}%
\pgfsetbuttcap%
\pgfsetroundjoin%
\pgfsetlinewidth{0.501875pt}%
\definecolor{currentstroke}{rgb}{0.698039,0.698039,0.698039}%
\pgfsetstrokecolor{currentstroke}%
\pgfsetdash{{1.850000pt}{0.800000pt}}{0.000000pt}%
\pgfpathmoveto{\pgfqpoint{0.465478in}{0.429287in}}%
\pgfpathlineto{\pgfqpoint{0.465478in}{2.738500in}}%
\pgfusepath{stroke}%
\end{pgfscope}%
\begin{pgfscope}%
\pgfsetbuttcap%
\pgfsetroundjoin%
\definecolor{currentfill}{rgb}{0.180392,0.180392,0.180392}%
\pgfsetfillcolor{currentfill}%
\pgfsetlinewidth{0.803000pt}%
\definecolor{currentstroke}{rgb}{0.180392,0.180392,0.180392}%
\pgfsetstrokecolor{currentstroke}%
\pgfsetdash{}{0pt}%
\pgfsys@defobject{currentmarker}{\pgfqpoint{0.000000in}{-0.048611in}}{\pgfqpoint{0.000000in}{0.000000in}}{%
\pgfpathmoveto{\pgfqpoint{0.000000in}{0.000000in}}%
\pgfpathlineto{\pgfqpoint{0.000000in}{-0.048611in}}%
\pgfusepath{stroke,fill}%
}%
\begin{pgfscope}%
\pgfsys@transformshift{0.465478in}{0.429287in}%
\pgfsys@useobject{currentmarker}{}%
\end{pgfscope}%
\end{pgfscope}%
\begin{pgfscope}%
\definecolor{textcolor}{rgb}{0.180392,0.180392,0.180392}%
\pgfsetstrokecolor{textcolor}%
\pgfsetfillcolor{textcolor}%
\pgftext[x=0.465478in,y=0.332064in,,top]{\color{textcolor}\rmfamily\fontsize{9.000000}{10.800000}\selectfont \(\displaystyle {\ensuremath{-}20}\)}%
\end{pgfscope}%
\begin{pgfscope}%
\pgfpathrectangle{\pgfqpoint{0.465477in}{0.429287in}}{\pgfqpoint{5.570243in}{2.309214in}}%
\pgfusepath{clip}%
\pgfsetbuttcap%
\pgfsetroundjoin%
\pgfsetlinewidth{0.501875pt}%
\definecolor{currentstroke}{rgb}{0.698039,0.698039,0.698039}%
\pgfsetstrokecolor{currentstroke}%
\pgfsetdash{{1.850000pt}{0.800000pt}}{0.000000pt}%
\pgfpathmoveto{\pgfqpoint{1.161758in}{0.429287in}}%
\pgfpathlineto{\pgfqpoint{1.161758in}{2.738500in}}%
\pgfusepath{stroke}%
\end{pgfscope}%
\begin{pgfscope}%
\pgfsetbuttcap%
\pgfsetroundjoin%
\definecolor{currentfill}{rgb}{0.180392,0.180392,0.180392}%
\pgfsetfillcolor{currentfill}%
\pgfsetlinewidth{0.803000pt}%
\definecolor{currentstroke}{rgb}{0.180392,0.180392,0.180392}%
\pgfsetstrokecolor{currentstroke}%
\pgfsetdash{}{0pt}%
\pgfsys@defobject{currentmarker}{\pgfqpoint{0.000000in}{-0.048611in}}{\pgfqpoint{0.000000in}{0.000000in}}{%
\pgfpathmoveto{\pgfqpoint{0.000000in}{0.000000in}}%
\pgfpathlineto{\pgfqpoint{0.000000in}{-0.048611in}}%
\pgfusepath{stroke,fill}%
}%
\begin{pgfscope}%
\pgfsys@transformshift{1.161758in}{0.429287in}%
\pgfsys@useobject{currentmarker}{}%
\end{pgfscope}%
\end{pgfscope}%
\begin{pgfscope}%
\definecolor{textcolor}{rgb}{0.180392,0.180392,0.180392}%
\pgfsetstrokecolor{textcolor}%
\pgfsetfillcolor{textcolor}%
\pgftext[x=1.161758in,y=0.332064in,,top]{\color{textcolor}\rmfamily\fontsize{9.000000}{10.800000}\selectfont \(\displaystyle {\ensuremath{-}15}\)}%
\end{pgfscope}%
\begin{pgfscope}%
\pgfpathrectangle{\pgfqpoint{0.465477in}{0.429287in}}{\pgfqpoint{5.570243in}{2.309214in}}%
\pgfusepath{clip}%
\pgfsetbuttcap%
\pgfsetroundjoin%
\pgfsetlinewidth{0.501875pt}%
\definecolor{currentstroke}{rgb}{0.698039,0.698039,0.698039}%
\pgfsetstrokecolor{currentstroke}%
\pgfsetdash{{1.850000pt}{0.800000pt}}{0.000000pt}%
\pgfpathmoveto{\pgfqpoint{1.858038in}{0.429287in}}%
\pgfpathlineto{\pgfqpoint{1.858038in}{2.738500in}}%
\pgfusepath{stroke}%
\end{pgfscope}%
\begin{pgfscope}%
\pgfsetbuttcap%
\pgfsetroundjoin%
\definecolor{currentfill}{rgb}{0.180392,0.180392,0.180392}%
\pgfsetfillcolor{currentfill}%
\pgfsetlinewidth{0.803000pt}%
\definecolor{currentstroke}{rgb}{0.180392,0.180392,0.180392}%
\pgfsetstrokecolor{currentstroke}%
\pgfsetdash{}{0pt}%
\pgfsys@defobject{currentmarker}{\pgfqpoint{0.000000in}{-0.048611in}}{\pgfqpoint{0.000000in}{0.000000in}}{%
\pgfpathmoveto{\pgfqpoint{0.000000in}{0.000000in}}%
\pgfpathlineto{\pgfqpoint{0.000000in}{-0.048611in}}%
\pgfusepath{stroke,fill}%
}%
\begin{pgfscope}%
\pgfsys@transformshift{1.858038in}{0.429287in}%
\pgfsys@useobject{currentmarker}{}%
\end{pgfscope}%
\end{pgfscope}%
\begin{pgfscope}%
\definecolor{textcolor}{rgb}{0.180392,0.180392,0.180392}%
\pgfsetstrokecolor{textcolor}%
\pgfsetfillcolor{textcolor}%
\pgftext[x=1.858038in,y=0.332064in,,top]{\color{textcolor}\rmfamily\fontsize{9.000000}{10.800000}\selectfont \(\displaystyle {\ensuremath{-}10}\)}%
\end{pgfscope}%
\begin{pgfscope}%
\pgfpathrectangle{\pgfqpoint{0.465477in}{0.429287in}}{\pgfqpoint{5.570243in}{2.309214in}}%
\pgfusepath{clip}%
\pgfsetbuttcap%
\pgfsetroundjoin%
\pgfsetlinewidth{0.501875pt}%
\definecolor{currentstroke}{rgb}{0.698039,0.698039,0.698039}%
\pgfsetstrokecolor{currentstroke}%
\pgfsetdash{{1.850000pt}{0.800000pt}}{0.000000pt}%
\pgfpathmoveto{\pgfqpoint{2.554318in}{0.429287in}}%
\pgfpathlineto{\pgfqpoint{2.554318in}{2.738500in}}%
\pgfusepath{stroke}%
\end{pgfscope}%
\begin{pgfscope}%
\pgfsetbuttcap%
\pgfsetroundjoin%
\definecolor{currentfill}{rgb}{0.180392,0.180392,0.180392}%
\pgfsetfillcolor{currentfill}%
\pgfsetlinewidth{0.803000pt}%
\definecolor{currentstroke}{rgb}{0.180392,0.180392,0.180392}%
\pgfsetstrokecolor{currentstroke}%
\pgfsetdash{}{0pt}%
\pgfsys@defobject{currentmarker}{\pgfqpoint{0.000000in}{-0.048611in}}{\pgfqpoint{0.000000in}{0.000000in}}{%
\pgfpathmoveto{\pgfqpoint{0.000000in}{0.000000in}}%
\pgfpathlineto{\pgfqpoint{0.000000in}{-0.048611in}}%
\pgfusepath{stroke,fill}%
}%
\begin{pgfscope}%
\pgfsys@transformshift{2.554318in}{0.429287in}%
\pgfsys@useobject{currentmarker}{}%
\end{pgfscope}%
\end{pgfscope}%
\begin{pgfscope}%
\definecolor{textcolor}{rgb}{0.180392,0.180392,0.180392}%
\pgfsetstrokecolor{textcolor}%
\pgfsetfillcolor{textcolor}%
\pgftext[x=2.554318in,y=0.332064in,,top]{\color{textcolor}\rmfamily\fontsize{9.000000}{10.800000}\selectfont \(\displaystyle {\ensuremath{-}5}\)}%
\end{pgfscope}%
\begin{pgfscope}%
\pgfpathrectangle{\pgfqpoint{0.465477in}{0.429287in}}{\pgfqpoint{5.570243in}{2.309214in}}%
\pgfusepath{clip}%
\pgfsetbuttcap%
\pgfsetroundjoin%
\pgfsetlinewidth{0.501875pt}%
\definecolor{currentstroke}{rgb}{0.698039,0.698039,0.698039}%
\pgfsetstrokecolor{currentstroke}%
\pgfsetdash{{1.850000pt}{0.800000pt}}{0.000000pt}%
\pgfpathmoveto{\pgfqpoint{3.250599in}{0.429287in}}%
\pgfpathlineto{\pgfqpoint{3.250599in}{2.738500in}}%
\pgfusepath{stroke}%
\end{pgfscope}%
\begin{pgfscope}%
\pgfsetbuttcap%
\pgfsetroundjoin%
\definecolor{currentfill}{rgb}{0.180392,0.180392,0.180392}%
\pgfsetfillcolor{currentfill}%
\pgfsetlinewidth{0.803000pt}%
\definecolor{currentstroke}{rgb}{0.180392,0.180392,0.180392}%
\pgfsetstrokecolor{currentstroke}%
\pgfsetdash{}{0pt}%
\pgfsys@defobject{currentmarker}{\pgfqpoint{0.000000in}{-0.048611in}}{\pgfqpoint{0.000000in}{0.000000in}}{%
\pgfpathmoveto{\pgfqpoint{0.000000in}{0.000000in}}%
\pgfpathlineto{\pgfqpoint{0.000000in}{-0.048611in}}%
\pgfusepath{stroke,fill}%
}%
\begin{pgfscope}%
\pgfsys@transformshift{3.250599in}{0.429287in}%
\pgfsys@useobject{currentmarker}{}%
\end{pgfscope}%
\end{pgfscope}%
\begin{pgfscope}%
\definecolor{textcolor}{rgb}{0.180392,0.180392,0.180392}%
\pgfsetstrokecolor{textcolor}%
\pgfsetfillcolor{textcolor}%
\pgftext[x=3.250599in,y=0.332064in,,top]{\color{textcolor}\rmfamily\fontsize{9.000000}{10.800000}\selectfont \(\displaystyle {0}\)}%
\end{pgfscope}%
\begin{pgfscope}%
\pgfpathrectangle{\pgfqpoint{0.465477in}{0.429287in}}{\pgfqpoint{5.570243in}{2.309214in}}%
\pgfusepath{clip}%
\pgfsetbuttcap%
\pgfsetroundjoin%
\pgfsetlinewidth{0.501875pt}%
\definecolor{currentstroke}{rgb}{0.698039,0.698039,0.698039}%
\pgfsetstrokecolor{currentstroke}%
\pgfsetdash{{1.850000pt}{0.800000pt}}{0.000000pt}%
\pgfpathmoveto{\pgfqpoint{3.946879in}{0.429287in}}%
\pgfpathlineto{\pgfqpoint{3.946879in}{2.738500in}}%
\pgfusepath{stroke}%
\end{pgfscope}%
\begin{pgfscope}%
\pgfsetbuttcap%
\pgfsetroundjoin%
\definecolor{currentfill}{rgb}{0.180392,0.180392,0.180392}%
\pgfsetfillcolor{currentfill}%
\pgfsetlinewidth{0.803000pt}%
\definecolor{currentstroke}{rgb}{0.180392,0.180392,0.180392}%
\pgfsetstrokecolor{currentstroke}%
\pgfsetdash{}{0pt}%
\pgfsys@defobject{currentmarker}{\pgfqpoint{0.000000in}{-0.048611in}}{\pgfqpoint{0.000000in}{0.000000in}}{%
\pgfpathmoveto{\pgfqpoint{0.000000in}{0.000000in}}%
\pgfpathlineto{\pgfqpoint{0.000000in}{-0.048611in}}%
\pgfusepath{stroke,fill}%
}%
\begin{pgfscope}%
\pgfsys@transformshift{3.946879in}{0.429287in}%
\pgfsys@useobject{currentmarker}{}%
\end{pgfscope}%
\end{pgfscope}%
\begin{pgfscope}%
\definecolor{textcolor}{rgb}{0.180392,0.180392,0.180392}%
\pgfsetstrokecolor{textcolor}%
\pgfsetfillcolor{textcolor}%
\pgftext[x=3.946879in,y=0.332064in,,top]{\color{textcolor}\rmfamily\fontsize{9.000000}{10.800000}\selectfont \(\displaystyle {5}\)}%
\end{pgfscope}%
\begin{pgfscope}%
\pgfpathrectangle{\pgfqpoint{0.465477in}{0.429287in}}{\pgfqpoint{5.570243in}{2.309214in}}%
\pgfusepath{clip}%
\pgfsetbuttcap%
\pgfsetroundjoin%
\pgfsetlinewidth{0.501875pt}%
\definecolor{currentstroke}{rgb}{0.698039,0.698039,0.698039}%
\pgfsetstrokecolor{currentstroke}%
\pgfsetdash{{1.850000pt}{0.800000pt}}{0.000000pt}%
\pgfpathmoveto{\pgfqpoint{4.643159in}{0.429287in}}%
\pgfpathlineto{\pgfqpoint{4.643159in}{2.738500in}}%
\pgfusepath{stroke}%
\end{pgfscope}%
\begin{pgfscope}%
\pgfsetbuttcap%
\pgfsetroundjoin%
\definecolor{currentfill}{rgb}{0.180392,0.180392,0.180392}%
\pgfsetfillcolor{currentfill}%
\pgfsetlinewidth{0.803000pt}%
\definecolor{currentstroke}{rgb}{0.180392,0.180392,0.180392}%
\pgfsetstrokecolor{currentstroke}%
\pgfsetdash{}{0pt}%
\pgfsys@defobject{currentmarker}{\pgfqpoint{0.000000in}{-0.048611in}}{\pgfqpoint{0.000000in}{0.000000in}}{%
\pgfpathmoveto{\pgfqpoint{0.000000in}{0.000000in}}%
\pgfpathlineto{\pgfqpoint{0.000000in}{-0.048611in}}%
\pgfusepath{stroke,fill}%
}%
\begin{pgfscope}%
\pgfsys@transformshift{4.643159in}{0.429287in}%
\pgfsys@useobject{currentmarker}{}%
\end{pgfscope}%
\end{pgfscope}%
\begin{pgfscope}%
\definecolor{textcolor}{rgb}{0.180392,0.180392,0.180392}%
\pgfsetstrokecolor{textcolor}%
\pgfsetfillcolor{textcolor}%
\pgftext[x=4.643159in,y=0.332064in,,top]{\color{textcolor}\rmfamily\fontsize{9.000000}{10.800000}\selectfont \(\displaystyle {10}\)}%
\end{pgfscope}%
\begin{pgfscope}%
\pgfpathrectangle{\pgfqpoint{0.465477in}{0.429287in}}{\pgfqpoint{5.570243in}{2.309214in}}%
\pgfusepath{clip}%
\pgfsetbuttcap%
\pgfsetroundjoin%
\pgfsetlinewidth{0.501875pt}%
\definecolor{currentstroke}{rgb}{0.698039,0.698039,0.698039}%
\pgfsetstrokecolor{currentstroke}%
\pgfsetdash{{1.850000pt}{0.800000pt}}{0.000000pt}%
\pgfpathmoveto{\pgfqpoint{5.339440in}{0.429287in}}%
\pgfpathlineto{\pgfqpoint{5.339440in}{2.738500in}}%
\pgfusepath{stroke}%
\end{pgfscope}%
\begin{pgfscope}%
\pgfsetbuttcap%
\pgfsetroundjoin%
\definecolor{currentfill}{rgb}{0.180392,0.180392,0.180392}%
\pgfsetfillcolor{currentfill}%
\pgfsetlinewidth{0.803000pt}%
\definecolor{currentstroke}{rgb}{0.180392,0.180392,0.180392}%
\pgfsetstrokecolor{currentstroke}%
\pgfsetdash{}{0pt}%
\pgfsys@defobject{currentmarker}{\pgfqpoint{0.000000in}{-0.048611in}}{\pgfqpoint{0.000000in}{0.000000in}}{%
\pgfpathmoveto{\pgfqpoint{0.000000in}{0.000000in}}%
\pgfpathlineto{\pgfqpoint{0.000000in}{-0.048611in}}%
\pgfusepath{stroke,fill}%
}%
\begin{pgfscope}%
\pgfsys@transformshift{5.339440in}{0.429287in}%
\pgfsys@useobject{currentmarker}{}%
\end{pgfscope}%
\end{pgfscope}%
\begin{pgfscope}%
\definecolor{textcolor}{rgb}{0.180392,0.180392,0.180392}%
\pgfsetstrokecolor{textcolor}%
\pgfsetfillcolor{textcolor}%
\pgftext[x=5.339440in,y=0.332064in,,top]{\color{textcolor}\rmfamily\fontsize{9.000000}{10.800000}\selectfont \(\displaystyle {15}\)}%
\end{pgfscope}%
\begin{pgfscope}%
\pgfpathrectangle{\pgfqpoint{0.465477in}{0.429287in}}{\pgfqpoint{5.570243in}{2.309214in}}%
\pgfusepath{clip}%
\pgfsetbuttcap%
\pgfsetroundjoin%
\pgfsetlinewidth{0.501875pt}%
\definecolor{currentstroke}{rgb}{0.698039,0.698039,0.698039}%
\pgfsetstrokecolor{currentstroke}%
\pgfsetdash{{1.850000pt}{0.800000pt}}{0.000000pt}%
\pgfpathmoveto{\pgfqpoint{6.035720in}{0.429287in}}%
\pgfpathlineto{\pgfqpoint{6.035720in}{2.738500in}}%
\pgfusepath{stroke}%
\end{pgfscope}%
\begin{pgfscope}%
\pgfsetbuttcap%
\pgfsetroundjoin%
\definecolor{currentfill}{rgb}{0.180392,0.180392,0.180392}%
\pgfsetfillcolor{currentfill}%
\pgfsetlinewidth{0.803000pt}%
\definecolor{currentstroke}{rgb}{0.180392,0.180392,0.180392}%
\pgfsetstrokecolor{currentstroke}%
\pgfsetdash{}{0pt}%
\pgfsys@defobject{currentmarker}{\pgfqpoint{0.000000in}{-0.048611in}}{\pgfqpoint{0.000000in}{0.000000in}}{%
\pgfpathmoveto{\pgfqpoint{0.000000in}{0.000000in}}%
\pgfpathlineto{\pgfqpoint{0.000000in}{-0.048611in}}%
\pgfusepath{stroke,fill}%
}%
\begin{pgfscope}%
\pgfsys@transformshift{6.035720in}{0.429287in}%
\pgfsys@useobject{currentmarker}{}%
\end{pgfscope}%
\end{pgfscope}%
\begin{pgfscope}%
\definecolor{textcolor}{rgb}{0.180392,0.180392,0.180392}%
\pgfsetstrokecolor{textcolor}%
\pgfsetfillcolor{textcolor}%
\pgftext[x=6.035720in,y=0.332064in,,top]{\color{textcolor}\rmfamily\fontsize{9.000000}{10.800000}\selectfont \(\displaystyle {20}\)}%
\end{pgfscope}%
\begin{pgfscope}%
\definecolor{textcolor}{rgb}{0.180392,0.180392,0.180392}%
\pgfsetstrokecolor{textcolor}%
\pgfsetfillcolor{textcolor}%
\pgftext[x=3.250599in,y=0.150823in,,top]{\color{textcolor}\rmfamily\fontsize{10.800000}{12.960000}\selectfont Time, t}%
\end{pgfscope}%
\begin{pgfscope}%
\pgfpathrectangle{\pgfqpoint{0.465477in}{0.429287in}}{\pgfqpoint{5.570243in}{2.309214in}}%
\pgfusepath{clip}%
\pgfsetbuttcap%
\pgfsetroundjoin%
\pgfsetlinewidth{0.501875pt}%
\definecolor{currentstroke}{rgb}{0.698039,0.698039,0.698039}%
\pgfsetstrokecolor{currentstroke}%
\pgfsetdash{{1.850000pt}{0.800000pt}}{0.000000pt}%
\pgfpathmoveto{\pgfqpoint{0.465477in}{0.534251in}}%
\pgfpathlineto{\pgfqpoint{6.035720in}{0.534251in}}%
\pgfusepath{stroke}%
\end{pgfscope}%
\begin{pgfscope}%
\pgfsetbuttcap%
\pgfsetroundjoin%
\definecolor{currentfill}{rgb}{0.180392,0.180392,0.180392}%
\pgfsetfillcolor{currentfill}%
\pgfsetlinewidth{0.803000pt}%
\definecolor{currentstroke}{rgb}{0.180392,0.180392,0.180392}%
\pgfsetstrokecolor{currentstroke}%
\pgfsetdash{}{0pt}%
\pgfsys@defobject{currentmarker}{\pgfqpoint{-0.048611in}{0.000000in}}{\pgfqpoint{-0.000000in}{0.000000in}}{%
\pgfpathmoveto{\pgfqpoint{-0.000000in}{0.000000in}}%
\pgfpathlineto{\pgfqpoint{-0.048611in}{0.000000in}}%
\pgfusepath{stroke,fill}%
}%
\begin{pgfscope}%
\pgfsys@transformshift{0.465477in}{0.534251in}%
\pgfsys@useobject{currentmarker}{}%
\end{pgfscope}%
\end{pgfscope}%
\begin{pgfscope}%
\definecolor{textcolor}{rgb}{0.180392,0.180392,0.180392}%
\pgfsetstrokecolor{textcolor}%
\pgfsetfillcolor{textcolor}%
\pgftext[x=0.212005in, y=0.489033in, left, base]{\color{textcolor}\rmfamily\fontsize{9.000000}{10.800000}\selectfont \(\displaystyle {0.0}\)}%
\end{pgfscope}%
\begin{pgfscope}%
\pgfpathrectangle{\pgfqpoint{0.465477in}{0.429287in}}{\pgfqpoint{5.570243in}{2.309214in}}%
\pgfusepath{clip}%
\pgfsetbuttcap%
\pgfsetroundjoin%
\pgfsetlinewidth{0.501875pt}%
\definecolor{currentstroke}{rgb}{0.698039,0.698039,0.698039}%
\pgfsetstrokecolor{currentstroke}%
\pgfsetdash{{1.850000pt}{0.800000pt}}{0.000000pt}%
\pgfpathmoveto{\pgfqpoint{0.465477in}{0.952019in}}%
\pgfpathlineto{\pgfqpoint{6.035720in}{0.952019in}}%
\pgfusepath{stroke}%
\end{pgfscope}%
\begin{pgfscope}%
\pgfsetbuttcap%
\pgfsetroundjoin%
\definecolor{currentfill}{rgb}{0.180392,0.180392,0.180392}%
\pgfsetfillcolor{currentfill}%
\pgfsetlinewidth{0.803000pt}%
\definecolor{currentstroke}{rgb}{0.180392,0.180392,0.180392}%
\pgfsetstrokecolor{currentstroke}%
\pgfsetdash{}{0pt}%
\pgfsys@defobject{currentmarker}{\pgfqpoint{-0.048611in}{0.000000in}}{\pgfqpoint{-0.000000in}{0.000000in}}{%
\pgfpathmoveto{\pgfqpoint{-0.000000in}{0.000000in}}%
\pgfpathlineto{\pgfqpoint{-0.048611in}{0.000000in}}%
\pgfusepath{stroke,fill}%
}%
\begin{pgfscope}%
\pgfsys@transformshift{0.465477in}{0.952019in}%
\pgfsys@useobject{currentmarker}{}%
\end{pgfscope}%
\end{pgfscope}%
\begin{pgfscope}%
\definecolor{textcolor}{rgb}{0.180392,0.180392,0.180392}%
\pgfsetstrokecolor{textcolor}%
\pgfsetfillcolor{textcolor}%
\pgftext[x=0.212005in, y=0.906801in, left, base]{\color{textcolor}\rmfamily\fontsize{9.000000}{10.800000}\selectfont \(\displaystyle {0.2}\)}%
\end{pgfscope}%
\begin{pgfscope}%
\pgfpathrectangle{\pgfqpoint{0.465477in}{0.429287in}}{\pgfqpoint{5.570243in}{2.309214in}}%
\pgfusepath{clip}%
\pgfsetbuttcap%
\pgfsetroundjoin%
\pgfsetlinewidth{0.501875pt}%
\definecolor{currentstroke}{rgb}{0.698039,0.698039,0.698039}%
\pgfsetstrokecolor{currentstroke}%
\pgfsetdash{{1.850000pt}{0.800000pt}}{0.000000pt}%
\pgfpathmoveto{\pgfqpoint{0.465477in}{1.369787in}}%
\pgfpathlineto{\pgfqpoint{6.035720in}{1.369787in}}%
\pgfusepath{stroke}%
\end{pgfscope}%
\begin{pgfscope}%
\pgfsetbuttcap%
\pgfsetroundjoin%
\definecolor{currentfill}{rgb}{0.180392,0.180392,0.180392}%
\pgfsetfillcolor{currentfill}%
\pgfsetlinewidth{0.803000pt}%
\definecolor{currentstroke}{rgb}{0.180392,0.180392,0.180392}%
\pgfsetstrokecolor{currentstroke}%
\pgfsetdash{}{0pt}%
\pgfsys@defobject{currentmarker}{\pgfqpoint{-0.048611in}{0.000000in}}{\pgfqpoint{-0.000000in}{0.000000in}}{%
\pgfpathmoveto{\pgfqpoint{-0.000000in}{0.000000in}}%
\pgfpathlineto{\pgfqpoint{-0.048611in}{0.000000in}}%
\pgfusepath{stroke,fill}%
}%
\begin{pgfscope}%
\pgfsys@transformshift{0.465477in}{1.369787in}%
\pgfsys@useobject{currentmarker}{}%
\end{pgfscope}%
\end{pgfscope}%
\begin{pgfscope}%
\definecolor{textcolor}{rgb}{0.180392,0.180392,0.180392}%
\pgfsetstrokecolor{textcolor}%
\pgfsetfillcolor{textcolor}%
\pgftext[x=0.212005in, y=1.324569in, left, base]{\color{textcolor}\rmfamily\fontsize{9.000000}{10.800000}\selectfont \(\displaystyle {0.4}\)}%
\end{pgfscope}%
\begin{pgfscope}%
\pgfpathrectangle{\pgfqpoint{0.465477in}{0.429287in}}{\pgfqpoint{5.570243in}{2.309214in}}%
\pgfusepath{clip}%
\pgfsetbuttcap%
\pgfsetroundjoin%
\pgfsetlinewidth{0.501875pt}%
\definecolor{currentstroke}{rgb}{0.698039,0.698039,0.698039}%
\pgfsetstrokecolor{currentstroke}%
\pgfsetdash{{1.850000pt}{0.800000pt}}{0.000000pt}%
\pgfpathmoveto{\pgfqpoint{0.465477in}{1.787555in}}%
\pgfpathlineto{\pgfqpoint{6.035720in}{1.787555in}}%
\pgfusepath{stroke}%
\end{pgfscope}%
\begin{pgfscope}%
\pgfsetbuttcap%
\pgfsetroundjoin%
\definecolor{currentfill}{rgb}{0.180392,0.180392,0.180392}%
\pgfsetfillcolor{currentfill}%
\pgfsetlinewidth{0.803000pt}%
\definecolor{currentstroke}{rgb}{0.180392,0.180392,0.180392}%
\pgfsetstrokecolor{currentstroke}%
\pgfsetdash{}{0pt}%
\pgfsys@defobject{currentmarker}{\pgfqpoint{-0.048611in}{0.000000in}}{\pgfqpoint{-0.000000in}{0.000000in}}{%
\pgfpathmoveto{\pgfqpoint{-0.000000in}{0.000000in}}%
\pgfpathlineto{\pgfqpoint{-0.048611in}{0.000000in}}%
\pgfusepath{stroke,fill}%
}%
\begin{pgfscope}%
\pgfsys@transformshift{0.465477in}{1.787555in}%
\pgfsys@useobject{currentmarker}{}%
\end{pgfscope}%
\end{pgfscope}%
\begin{pgfscope}%
\definecolor{textcolor}{rgb}{0.180392,0.180392,0.180392}%
\pgfsetstrokecolor{textcolor}%
\pgfsetfillcolor{textcolor}%
\pgftext[x=0.212005in, y=1.742337in, left, base]{\color{textcolor}\rmfamily\fontsize{9.000000}{10.800000}\selectfont \(\displaystyle {0.6}\)}%
\end{pgfscope}%
\begin{pgfscope}%
\pgfpathrectangle{\pgfqpoint{0.465477in}{0.429287in}}{\pgfqpoint{5.570243in}{2.309214in}}%
\pgfusepath{clip}%
\pgfsetbuttcap%
\pgfsetroundjoin%
\pgfsetlinewidth{0.501875pt}%
\definecolor{currentstroke}{rgb}{0.698039,0.698039,0.698039}%
\pgfsetstrokecolor{currentstroke}%
\pgfsetdash{{1.850000pt}{0.800000pt}}{0.000000pt}%
\pgfpathmoveto{\pgfqpoint{0.465477in}{2.205324in}}%
\pgfpathlineto{\pgfqpoint{6.035720in}{2.205324in}}%
\pgfusepath{stroke}%
\end{pgfscope}%
\begin{pgfscope}%
\pgfsetbuttcap%
\pgfsetroundjoin%
\definecolor{currentfill}{rgb}{0.180392,0.180392,0.180392}%
\pgfsetfillcolor{currentfill}%
\pgfsetlinewidth{0.803000pt}%
\definecolor{currentstroke}{rgb}{0.180392,0.180392,0.180392}%
\pgfsetstrokecolor{currentstroke}%
\pgfsetdash{}{0pt}%
\pgfsys@defobject{currentmarker}{\pgfqpoint{-0.048611in}{0.000000in}}{\pgfqpoint{-0.000000in}{0.000000in}}{%
\pgfpathmoveto{\pgfqpoint{-0.000000in}{0.000000in}}%
\pgfpathlineto{\pgfqpoint{-0.048611in}{0.000000in}}%
\pgfusepath{stroke,fill}%
}%
\begin{pgfscope}%
\pgfsys@transformshift{0.465477in}{2.205324in}%
\pgfsys@useobject{currentmarker}{}%
\end{pgfscope}%
\end{pgfscope}%
\begin{pgfscope}%
\definecolor{textcolor}{rgb}{0.180392,0.180392,0.180392}%
\pgfsetstrokecolor{textcolor}%
\pgfsetfillcolor{textcolor}%
\pgftext[x=0.212005in, y=2.160105in, left, base]{\color{textcolor}\rmfamily\fontsize{9.000000}{10.800000}\selectfont \(\displaystyle {0.8}\)}%
\end{pgfscope}%
\begin{pgfscope}%
\pgfpathrectangle{\pgfqpoint{0.465477in}{0.429287in}}{\pgfqpoint{5.570243in}{2.309214in}}%
\pgfusepath{clip}%
\pgfsetbuttcap%
\pgfsetroundjoin%
\pgfsetlinewidth{0.501875pt}%
\definecolor{currentstroke}{rgb}{0.698039,0.698039,0.698039}%
\pgfsetstrokecolor{currentstroke}%
\pgfsetdash{{1.850000pt}{0.800000pt}}{0.000000pt}%
\pgfpathmoveto{\pgfqpoint{0.465477in}{2.623092in}}%
\pgfpathlineto{\pgfqpoint{6.035720in}{2.623092in}}%
\pgfusepath{stroke}%
\end{pgfscope}%
\begin{pgfscope}%
\pgfsetbuttcap%
\pgfsetroundjoin%
\definecolor{currentfill}{rgb}{0.180392,0.180392,0.180392}%
\pgfsetfillcolor{currentfill}%
\pgfsetlinewidth{0.803000pt}%
\definecolor{currentstroke}{rgb}{0.180392,0.180392,0.180392}%
\pgfsetstrokecolor{currentstroke}%
\pgfsetdash{}{0pt}%
\pgfsys@defobject{currentmarker}{\pgfqpoint{-0.048611in}{0.000000in}}{\pgfqpoint{-0.000000in}{0.000000in}}{%
\pgfpathmoveto{\pgfqpoint{-0.000000in}{0.000000in}}%
\pgfpathlineto{\pgfqpoint{-0.048611in}{0.000000in}}%
\pgfusepath{stroke,fill}%
}%
\begin{pgfscope}%
\pgfsys@transformshift{0.465477in}{2.623092in}%
\pgfsys@useobject{currentmarker}{}%
\end{pgfscope}%
\end{pgfscope}%
\begin{pgfscope}%
\definecolor{textcolor}{rgb}{0.180392,0.180392,0.180392}%
\pgfsetstrokecolor{textcolor}%
\pgfsetfillcolor{textcolor}%
\pgftext[x=0.212005in, y=2.577873in, left, base]{\color{textcolor}\rmfamily\fontsize{9.000000}{10.800000}\selectfont \(\displaystyle {1.0}\)}%
\end{pgfscope}%
\begin{pgfscope}%
\definecolor{textcolor}{rgb}{0.180392,0.180392,0.180392}%
\pgfsetstrokecolor{textcolor}%
\pgfsetfillcolor{textcolor}%
\pgftext[x=0.156450in,y=1.583893in,,bottom,rotate=90.000000]{\color{textcolor}\rmfamily\fontsize{10.800000}{12.960000}\selectfont \(\displaystyle F(t)\)}%
\end{pgfscope}%
\begin{pgfscope}%
\pgfpathrectangle{\pgfqpoint{0.465477in}{0.429287in}}{\pgfqpoint{5.570243in}{2.309214in}}%
\pgfusepath{clip}%
\pgfsetrectcap%
\pgfsetroundjoin%
\pgfsetlinewidth{2.007500pt}%
\definecolor{currentstroke}{rgb}{0.000000,0.501961,0.000000}%
\pgfsetstrokecolor{currentstroke}%
\pgfsetdash{}{0pt}%
\pgfpathmoveto{\pgfqpoint{0.465478in}{0.534251in}}%
\pgfpathlineto{\pgfqpoint{2.693575in}{0.534952in}}%
\pgfpathlineto{\pgfqpoint{2.749277in}{0.537455in}}%
\pgfpathlineto{\pgfqpoint{2.777128in}{0.540703in}}%
\pgfpathlineto{\pgfqpoint{2.804979in}{0.546734in}}%
\pgfpathlineto{\pgfqpoint{2.832831in}{0.557456in}}%
\pgfpathlineto{\pgfqpoint{2.860682in}{0.575696in}}%
\pgfpathlineto{\pgfqpoint{2.888533in}{0.605371in}}%
\pgfpathlineto{\pgfqpoint{2.916384in}{0.651507in}}%
\pgfpathlineto{\pgfqpoint{2.944235in}{0.719994in}}%
\pgfpathlineto{\pgfqpoint{2.972087in}{0.816945in}}%
\pgfpathlineto{\pgfqpoint{2.999938in}{0.947630in}}%
\pgfpathlineto{\pgfqpoint{3.027789in}{1.115026in}}%
\pgfpathlineto{\pgfqpoint{3.055640in}{1.318216in}}%
\pgfpathlineto{\pgfqpoint{3.083491in}{1.550999in}}%
\pgfpathlineto{\pgfqpoint{3.139194in}{2.051061in}}%
\pgfpathlineto{\pgfqpoint{3.167045in}{2.278997in}}%
\pgfpathlineto{\pgfqpoint{3.194896in}{2.462494in}}%
\pgfpathlineto{\pgfqpoint{3.222748in}{2.581730in}}%
\pgfpathlineto{\pgfqpoint{3.250599in}{2.623092in}}%
\pgfpathlineto{\pgfqpoint{3.278450in}{2.581730in}}%
\pgfpathlineto{\pgfqpoint{3.306301in}{2.462494in}}%
\pgfpathlineto{\pgfqpoint{3.334152in}{2.278997in}}%
\pgfpathlineto{\pgfqpoint{3.362004in}{2.051061in}}%
\pgfpathlineto{\pgfqpoint{3.417706in}{1.550999in}}%
\pgfpathlineto{\pgfqpoint{3.445557in}{1.318216in}}%
\pgfpathlineto{\pgfqpoint{3.473408in}{1.115026in}}%
\pgfpathlineto{\pgfqpoint{3.501260in}{0.947630in}}%
\pgfpathlineto{\pgfqpoint{3.529111in}{0.816945in}}%
\pgfpathlineto{\pgfqpoint{3.556962in}{0.719994in}}%
\pgfpathlineto{\pgfqpoint{3.584813in}{0.651507in}}%
\pgfpathlineto{\pgfqpoint{3.612665in}{0.605371in}}%
\pgfpathlineto{\pgfqpoint{3.640516in}{0.575696in}}%
\pgfpathlineto{\pgfqpoint{3.668367in}{0.557456in}}%
\pgfpathlineto{\pgfqpoint{3.696218in}{0.546734in}}%
\pgfpathlineto{\pgfqpoint{3.724069in}{0.540703in}}%
\pgfpathlineto{\pgfqpoint{3.751921in}{0.537455in}}%
\pgfpathlineto{\pgfqpoint{3.807623in}{0.534952in}}%
\pgfpathlineto{\pgfqpoint{3.919028in}{0.534272in}}%
\pgfpathlineto{\pgfqpoint{6.035720in}{0.534251in}}%
\pgfpathlineto{\pgfqpoint{6.035720in}{0.534251in}}%
\pgfusepath{stroke}%
\end{pgfscope}%
\begin{pgfscope}%
\pgfpathrectangle{\pgfqpoint{0.465477in}{0.429287in}}{\pgfqpoint{5.570243in}{2.309214in}}%
\pgfusepath{clip}%
\pgfsetrectcap%
\pgfsetroundjoin%
\pgfsetlinewidth{2.007500pt}%
\definecolor{currentstroke}{rgb}{1.000000,0.647059,0.000000}%
\pgfsetstrokecolor{currentstroke}%
\pgfsetdash{}{0pt}%
\pgfpathmoveto{\pgfqpoint{0.465478in}{0.534251in}}%
\pgfpathlineto{\pgfqpoint{2.693575in}{0.534955in}}%
\pgfpathlineto{\pgfqpoint{2.749277in}{0.537471in}}%
\pgfpathlineto{\pgfqpoint{2.777128in}{0.540735in}}%
\pgfpathlineto{\pgfqpoint{2.804979in}{0.546796in}}%
\pgfpathlineto{\pgfqpoint{2.832831in}{0.557572in}}%
\pgfpathlineto{\pgfqpoint{2.860682in}{0.575903in}}%
\pgfpathlineto{\pgfqpoint{2.888533in}{0.605726in}}%
\pgfpathlineto{\pgfqpoint{2.916384in}{0.652094in}}%
\pgfpathlineto{\pgfqpoint{2.944235in}{0.720923in}}%
\pgfpathlineto{\pgfqpoint{2.972087in}{0.818358in}}%
\pgfpathlineto{\pgfqpoint{2.999938in}{0.949697in}}%
\pgfpathlineto{\pgfqpoint{3.027789in}{1.117930in}}%
\pgfpathlineto{\pgfqpoint{3.055640in}{1.322136in}}%
\pgfpathlineto{\pgfqpoint{3.083491in}{1.556083in}}%
\pgfpathlineto{\pgfqpoint{3.139194in}{2.058645in}}%
\pgfpathlineto{\pgfqpoint{3.167045in}{2.287721in}}%
\pgfpathlineto{\pgfqpoint{3.194896in}{2.472135in}}%
\pgfpathlineto{\pgfqpoint{3.222748in}{2.591967in}}%
\pgfpathlineto{\pgfqpoint{3.250599in}{2.633536in}}%
\pgfpathlineto{\pgfqpoint{3.278450in}{2.591967in}}%
\pgfpathlineto{\pgfqpoint{3.306301in}{2.472135in}}%
\pgfpathlineto{\pgfqpoint{3.334152in}{2.287721in}}%
\pgfpathlineto{\pgfqpoint{3.362004in}{2.058645in}}%
\pgfpathlineto{\pgfqpoint{3.417706in}{1.556083in}}%
\pgfpathlineto{\pgfqpoint{3.445557in}{1.322136in}}%
\pgfpathlineto{\pgfqpoint{3.473408in}{1.117930in}}%
\pgfpathlineto{\pgfqpoint{3.501260in}{0.949697in}}%
\pgfpathlineto{\pgfqpoint{3.529111in}{0.818358in}}%
\pgfpathlineto{\pgfqpoint{3.556962in}{0.720923in}}%
\pgfpathlineto{\pgfqpoint{3.584813in}{0.652094in}}%
\pgfpathlineto{\pgfqpoint{3.612665in}{0.605726in}}%
\pgfpathlineto{\pgfqpoint{3.640516in}{0.575903in}}%
\pgfpathlineto{\pgfqpoint{3.668367in}{0.557572in}}%
\pgfpathlineto{\pgfqpoint{3.696218in}{0.546796in}}%
\pgfpathlineto{\pgfqpoint{3.724069in}{0.540735in}}%
\pgfpathlineto{\pgfqpoint{3.751921in}{0.537471in}}%
\pgfpathlineto{\pgfqpoint{3.807623in}{0.534955in}}%
\pgfpathlineto{\pgfqpoint{3.919028in}{0.534272in}}%
\pgfpathlineto{\pgfqpoint{6.035720in}{0.534251in}}%
\pgfpathlineto{\pgfqpoint{6.035720in}{0.534251in}}%
\pgfusepath{stroke}%
\end{pgfscope}%
\begin{pgfscope}%
\pgfsetrectcap%
\pgfsetmiterjoin%
\pgfsetlinewidth{0.803000pt}%
\definecolor{currentstroke}{rgb}{0.737255,0.737255,0.737255}%
\pgfsetstrokecolor{currentstroke}%
\pgfsetdash{}{0pt}%
\pgfpathmoveto{\pgfqpoint{0.465477in}{0.429287in}}%
\pgfpathlineto{\pgfqpoint{0.465477in}{2.738500in}}%
\pgfusepath{stroke}%
\end{pgfscope}%
\begin{pgfscope}%
\pgfsetrectcap%
\pgfsetmiterjoin%
\pgfsetlinewidth{0.803000pt}%
\definecolor{currentstroke}{rgb}{0.737255,0.737255,0.737255}%
\pgfsetstrokecolor{currentstroke}%
\pgfsetdash{}{0pt}%
\pgfpathmoveto{\pgfqpoint{6.035720in}{0.429287in}}%
\pgfpathlineto{\pgfqpoint{6.035720in}{2.738500in}}%
\pgfusepath{stroke}%
\end{pgfscope}%
\begin{pgfscope}%
\pgfsetrectcap%
\pgfsetmiterjoin%
\pgfsetlinewidth{0.803000pt}%
\definecolor{currentstroke}{rgb}{0.737255,0.737255,0.737255}%
\pgfsetstrokecolor{currentstroke}%
\pgfsetdash{}{0pt}%
\pgfpathmoveto{\pgfqpoint{0.465478in}{0.429287in}}%
\pgfpathlineto{\pgfqpoint{6.035720in}{0.429287in}}%
\pgfusepath{stroke}%
\end{pgfscope}%
\begin{pgfscope}%
\pgfsetrectcap%
\pgfsetmiterjoin%
\pgfsetlinewidth{0.803000pt}%
\definecolor{currentstroke}{rgb}{0.737255,0.737255,0.737255}%
\pgfsetstrokecolor{currentstroke}%
\pgfsetdash{}{0pt}%
\pgfpathmoveto{\pgfqpoint{0.465478in}{2.738500in}}%
\pgfpathlineto{\pgfqpoint{6.035720in}{2.738500in}}%
\pgfusepath{stroke}%
\end{pgfscope}%
\begin{pgfscope}%
\definecolor{textcolor}{rgb}{0.180392,0.180392,0.180392}%
\pgfsetstrokecolor{textcolor}%
\pgfsetfillcolor{textcolor}%
\pgftext[x=3.250599in,y=2.821834in,,base]{\color{textcolor}\rmfamily\fontsize{12.960000}{15.552000}\selectfont Input and Output Envelope - Time Domain}%
\end{pgfscope}%
\begin{pgfscope}%
\pgfsetbuttcap%
\pgfsetmiterjoin%
\definecolor{currentfill}{rgb}{0.933333,0.933333,0.933333}%
\pgfsetfillcolor{currentfill}%
\pgfsetfillopacity{0.800000}%
\pgfsetlinewidth{0.501875pt}%
\definecolor{currentstroke}{rgb}{0.800000,0.800000,0.800000}%
\pgfsetstrokecolor{currentstroke}%
\pgfsetstrokeopacity{0.800000}%
\pgfsetdash{}{0pt}%
\pgfpathmoveto{\pgfqpoint{0.552977in}{2.262129in}}%
\pgfpathlineto{\pgfqpoint{1.654592in}{2.262129in}}%
\pgfpathquadraticcurveto{\pgfqpoint{1.679592in}{2.262129in}}{\pgfqpoint{1.679592in}{2.287129in}}%
\pgfpathlineto{\pgfqpoint{1.679592in}{2.651000in}}%
\pgfpathquadraticcurveto{\pgfqpoint{1.679592in}{2.676000in}}{\pgfqpoint{1.654592in}{2.676000in}}%
\pgfpathlineto{\pgfqpoint{0.552977in}{2.676000in}}%
\pgfpathquadraticcurveto{\pgfqpoint{0.527977in}{2.676000in}}{\pgfqpoint{0.527977in}{2.651000in}}%
\pgfpathlineto{\pgfqpoint{0.527977in}{2.287129in}}%
\pgfpathquadraticcurveto{\pgfqpoint{0.527977in}{2.262129in}}{\pgfqpoint{0.552977in}{2.262129in}}%
\pgfpathlineto{\pgfqpoint{0.552977in}{2.262129in}}%
\pgfpathclose%
\pgfusepath{stroke,fill}%
\end{pgfscope}%
\begin{pgfscope}%
\pgfsetrectcap%
\pgfsetroundjoin%
\pgfsetlinewidth{2.007500pt}%
\definecolor{currentstroke}{rgb}{0.000000,0.501961,0.000000}%
\pgfsetstrokecolor{currentstroke}%
\pgfsetdash{}{0pt}%
\pgfpathmoveto{\pgfqpoint{0.577977in}{2.579314in}}%
\pgfpathlineto{\pgfqpoint{0.702977in}{2.579314in}}%
\pgfpathlineto{\pgfqpoint{0.827977in}{2.579314in}}%
\pgfusepath{stroke}%
\end{pgfscope}%
\begin{pgfscope}%
\definecolor{textcolor}{rgb}{0.180392,0.180392,0.180392}%
\pgfsetstrokecolor{textcolor}%
\pgfsetfillcolor{textcolor}%
\pgftext[x=0.927977in,y=2.535564in,left,base]{\color{textcolor}\rmfamily\fontsize{9.000000}{10.800000}\selectfont Initial Signal}%
\end{pgfscope}%
\begin{pgfscope}%
\pgfsetrectcap%
\pgfsetroundjoin%
\pgfsetlinewidth{2.007500pt}%
\definecolor{currentstroke}{rgb}{1.000000,0.647059,0.000000}%
\pgfsetstrokecolor{currentstroke}%
\pgfsetdash{}{0pt}%
\pgfpathmoveto{\pgfqpoint{0.577977in}{2.391128in}}%
\pgfpathlineto{\pgfqpoint{0.702977in}{2.391128in}}%
\pgfpathlineto{\pgfqpoint{0.827977in}{2.391128in}}%
\pgfusepath{stroke}%
\end{pgfscope}%
\begin{pgfscope}%
\definecolor{textcolor}{rgb}{0.180392,0.180392,0.180392}%
\pgfsetstrokecolor{textcolor}%
\pgfsetfillcolor{textcolor}%
\pgftext[x=0.927977in,y=2.347378in,left,base]{\color{textcolor}\rmfamily\fontsize{9.000000}{10.800000}\selectfont Final Signal}%
\end{pgfscope}%
\end{pgfpicture}%
\makeatother%
\endgroup%
}
			\caption{\centering{Split-Step Fourier Method Introduction - 2D Graph of the Initial and Final Envelope in the Time Domain}}
			\label{fig:split-step-intro-2d-time}
		\end{figure}
		
		The next important graph, Figure \ref{fig:split-step-intro-2d-freq}, is the input and output envelopes in the frequency domain. \linebreak
		
		\begin{figure}[h]
			\centering
			\scalebox{0.8}{%% Creator: Matplotlib, PGF backend
%%
%% To include the figure in your LaTeX document, write
%%   \input{<filename>.pgf}
%%
%% Make sure the required packages are loaded in your preamble
%%   \usepackage{pgf}
%%
%% Also ensure that all the required font packages are loaded; for instance,
%% the lmodern package is sometimes necessary when using math font.
%%   \usepackage{lmodern}
%%
%% Figures using additional raster images can only be included by \input if
%% they are in the same directory as the main LaTeX file. For loading figures
%% from other directories you can use the `import` package
%%   \usepackage{import}
%%
%% and then include the figures with
%%   \import{<path to file>}{<filename>.pgf}
%%
%% Matplotlib used the following preamble
%%   \usepackage[T1]{fontenc} \usepackage{mathpazo}
%%
\begingroup%
\makeatletter%
\begin{pgfpicture}%
\pgfpathrectangle{\pgfpointorigin}{\pgfqpoint{6.086066in}{2.987301in}}%
\pgfusepath{use as bounding box, clip}%
\begin{pgfscope}%
\pgfsetbuttcap%
\pgfsetmiterjoin%
\definecolor{currentfill}{rgb}{1.000000,1.000000,1.000000}%
\pgfsetfillcolor{currentfill}%
\pgfsetlinewidth{0.000000pt}%
\definecolor{currentstroke}{rgb}{1.000000,1.000000,1.000000}%
\pgfsetstrokecolor{currentstroke}%
\pgfsetdash{}{0pt}%
\pgfpathmoveto{\pgfqpoint{0.000000in}{0.000000in}}%
\pgfpathlineto{\pgfqpoint{6.086066in}{0.000000in}}%
\pgfpathlineto{\pgfqpoint{6.086066in}{2.987301in}}%
\pgfpathlineto{\pgfqpoint{0.000000in}{2.987301in}}%
\pgfpathlineto{\pgfqpoint{0.000000in}{0.000000in}}%
\pgfpathclose%
\pgfusepath{fill}%
\end{pgfscope}%
\begin{pgfscope}%
\pgfsetbuttcap%
\pgfsetmiterjoin%
\definecolor{currentfill}{rgb}{0.933333,0.933333,0.933333}%
\pgfsetfillcolor{currentfill}%
\pgfsetlinewidth{0.000000pt}%
\definecolor{currentstroke}{rgb}{0.000000,0.000000,0.000000}%
\pgfsetstrokecolor{currentstroke}%
\pgfsetstrokeopacity{0.000000}%
\pgfsetdash{}{0pt}%
\pgfpathmoveto{\pgfqpoint{0.477328in}{0.429287in}}%
\pgfpathlineto{\pgfqpoint{6.086066in}{0.429287in}}%
\pgfpathlineto{\pgfqpoint{6.086066in}{2.773739in}}%
\pgfpathlineto{\pgfqpoint{0.477328in}{2.773739in}}%
\pgfpathlineto{\pgfqpoint{0.477328in}{0.429287in}}%
\pgfpathclose%
\pgfusepath{fill}%
\end{pgfscope}%
\begin{pgfscope}%
\pgfpathrectangle{\pgfqpoint{0.477328in}{0.429287in}}{\pgfqpoint{5.608738in}{2.344453in}}%
\pgfusepath{clip}%
\pgfsetbuttcap%
\pgfsetroundjoin%
\pgfsetlinewidth{0.501875pt}%
\definecolor{currentstroke}{rgb}{0.698039,0.698039,0.698039}%
\pgfsetstrokecolor{currentstroke}%
\pgfsetdash{{1.850000pt}{0.800000pt}}{0.000000pt}%
\pgfpathmoveto{\pgfqpoint{1.038201in}{0.429287in}}%
\pgfpathlineto{\pgfqpoint{1.038201in}{2.773739in}}%
\pgfusepath{stroke}%
\end{pgfscope}%
\begin{pgfscope}%
\pgfsetbuttcap%
\pgfsetroundjoin%
\definecolor{currentfill}{rgb}{0.180392,0.180392,0.180392}%
\pgfsetfillcolor{currentfill}%
\pgfsetlinewidth{0.803000pt}%
\definecolor{currentstroke}{rgb}{0.180392,0.180392,0.180392}%
\pgfsetstrokecolor{currentstroke}%
\pgfsetdash{}{0pt}%
\pgfsys@defobject{currentmarker}{\pgfqpoint{0.000000in}{-0.048611in}}{\pgfqpoint{0.000000in}{0.000000in}}{%
\pgfpathmoveto{\pgfqpoint{0.000000in}{0.000000in}}%
\pgfpathlineto{\pgfqpoint{0.000000in}{-0.048611in}}%
\pgfusepath{stroke,fill}%
}%
\begin{pgfscope}%
\pgfsys@transformshift{1.038201in}{0.429287in}%
\pgfsys@useobject{currentmarker}{}%
\end{pgfscope}%
\end{pgfscope}%
\begin{pgfscope}%
\definecolor{textcolor}{rgb}{0.180392,0.180392,0.180392}%
\pgfsetstrokecolor{textcolor}%
\pgfsetfillcolor{textcolor}%
\pgftext[x=1.038201in,y=0.332064in,,top]{\color{textcolor}\rmfamily\fontsize{9.000000}{10.800000}\selectfont \(\displaystyle {\ensuremath{-}40}\)}%
\end{pgfscope}%
\begin{pgfscope}%
\pgfpathrectangle{\pgfqpoint{0.477328in}{0.429287in}}{\pgfqpoint{5.608738in}{2.344453in}}%
\pgfusepath{clip}%
\pgfsetbuttcap%
\pgfsetroundjoin%
\pgfsetlinewidth{0.501875pt}%
\definecolor{currentstroke}{rgb}{0.698039,0.698039,0.698039}%
\pgfsetstrokecolor{currentstroke}%
\pgfsetdash{{1.850000pt}{0.800000pt}}{0.000000pt}%
\pgfpathmoveto{\pgfqpoint{2.159949in}{0.429287in}}%
\pgfpathlineto{\pgfqpoint{2.159949in}{2.773739in}}%
\pgfusepath{stroke}%
\end{pgfscope}%
\begin{pgfscope}%
\pgfsetbuttcap%
\pgfsetroundjoin%
\definecolor{currentfill}{rgb}{0.180392,0.180392,0.180392}%
\pgfsetfillcolor{currentfill}%
\pgfsetlinewidth{0.803000pt}%
\definecolor{currentstroke}{rgb}{0.180392,0.180392,0.180392}%
\pgfsetstrokecolor{currentstroke}%
\pgfsetdash{}{0pt}%
\pgfsys@defobject{currentmarker}{\pgfqpoint{0.000000in}{-0.048611in}}{\pgfqpoint{0.000000in}{0.000000in}}{%
\pgfpathmoveto{\pgfqpoint{0.000000in}{0.000000in}}%
\pgfpathlineto{\pgfqpoint{0.000000in}{-0.048611in}}%
\pgfusepath{stroke,fill}%
}%
\begin{pgfscope}%
\pgfsys@transformshift{2.159949in}{0.429287in}%
\pgfsys@useobject{currentmarker}{}%
\end{pgfscope}%
\end{pgfscope}%
\begin{pgfscope}%
\definecolor{textcolor}{rgb}{0.180392,0.180392,0.180392}%
\pgfsetstrokecolor{textcolor}%
\pgfsetfillcolor{textcolor}%
\pgftext[x=2.159949in,y=0.332064in,,top]{\color{textcolor}\rmfamily\fontsize{9.000000}{10.800000}\selectfont \(\displaystyle {\ensuremath{-}20}\)}%
\end{pgfscope}%
\begin{pgfscope}%
\pgfpathrectangle{\pgfqpoint{0.477328in}{0.429287in}}{\pgfqpoint{5.608738in}{2.344453in}}%
\pgfusepath{clip}%
\pgfsetbuttcap%
\pgfsetroundjoin%
\pgfsetlinewidth{0.501875pt}%
\definecolor{currentstroke}{rgb}{0.698039,0.698039,0.698039}%
\pgfsetstrokecolor{currentstroke}%
\pgfsetdash{{1.850000pt}{0.800000pt}}{0.000000pt}%
\pgfpathmoveto{\pgfqpoint{3.281697in}{0.429287in}}%
\pgfpathlineto{\pgfqpoint{3.281697in}{2.773739in}}%
\pgfusepath{stroke}%
\end{pgfscope}%
\begin{pgfscope}%
\pgfsetbuttcap%
\pgfsetroundjoin%
\definecolor{currentfill}{rgb}{0.180392,0.180392,0.180392}%
\pgfsetfillcolor{currentfill}%
\pgfsetlinewidth{0.803000pt}%
\definecolor{currentstroke}{rgb}{0.180392,0.180392,0.180392}%
\pgfsetstrokecolor{currentstroke}%
\pgfsetdash{}{0pt}%
\pgfsys@defobject{currentmarker}{\pgfqpoint{0.000000in}{-0.048611in}}{\pgfqpoint{0.000000in}{0.000000in}}{%
\pgfpathmoveto{\pgfqpoint{0.000000in}{0.000000in}}%
\pgfpathlineto{\pgfqpoint{0.000000in}{-0.048611in}}%
\pgfusepath{stroke,fill}%
}%
\begin{pgfscope}%
\pgfsys@transformshift{3.281697in}{0.429287in}%
\pgfsys@useobject{currentmarker}{}%
\end{pgfscope}%
\end{pgfscope}%
\begin{pgfscope}%
\definecolor{textcolor}{rgb}{0.180392,0.180392,0.180392}%
\pgfsetstrokecolor{textcolor}%
\pgfsetfillcolor{textcolor}%
\pgftext[x=3.281697in,y=0.332064in,,top]{\color{textcolor}\rmfamily\fontsize{9.000000}{10.800000}\selectfont \(\displaystyle {0}\)}%
\end{pgfscope}%
\begin{pgfscope}%
\pgfpathrectangle{\pgfqpoint{0.477328in}{0.429287in}}{\pgfqpoint{5.608738in}{2.344453in}}%
\pgfusepath{clip}%
\pgfsetbuttcap%
\pgfsetroundjoin%
\pgfsetlinewidth{0.501875pt}%
\definecolor{currentstroke}{rgb}{0.698039,0.698039,0.698039}%
\pgfsetstrokecolor{currentstroke}%
\pgfsetdash{{1.850000pt}{0.800000pt}}{0.000000pt}%
\pgfpathmoveto{\pgfqpoint{4.403444in}{0.429287in}}%
\pgfpathlineto{\pgfqpoint{4.403444in}{2.773739in}}%
\pgfusepath{stroke}%
\end{pgfscope}%
\begin{pgfscope}%
\pgfsetbuttcap%
\pgfsetroundjoin%
\definecolor{currentfill}{rgb}{0.180392,0.180392,0.180392}%
\pgfsetfillcolor{currentfill}%
\pgfsetlinewidth{0.803000pt}%
\definecolor{currentstroke}{rgb}{0.180392,0.180392,0.180392}%
\pgfsetstrokecolor{currentstroke}%
\pgfsetdash{}{0pt}%
\pgfsys@defobject{currentmarker}{\pgfqpoint{0.000000in}{-0.048611in}}{\pgfqpoint{0.000000in}{0.000000in}}{%
\pgfpathmoveto{\pgfqpoint{0.000000in}{0.000000in}}%
\pgfpathlineto{\pgfqpoint{0.000000in}{-0.048611in}}%
\pgfusepath{stroke,fill}%
}%
\begin{pgfscope}%
\pgfsys@transformshift{4.403444in}{0.429287in}%
\pgfsys@useobject{currentmarker}{}%
\end{pgfscope}%
\end{pgfscope}%
\begin{pgfscope}%
\definecolor{textcolor}{rgb}{0.180392,0.180392,0.180392}%
\pgfsetstrokecolor{textcolor}%
\pgfsetfillcolor{textcolor}%
\pgftext[x=4.403444in,y=0.332064in,,top]{\color{textcolor}\rmfamily\fontsize{9.000000}{10.800000}\selectfont \(\displaystyle {20}\)}%
\end{pgfscope}%
\begin{pgfscope}%
\pgfpathrectangle{\pgfqpoint{0.477328in}{0.429287in}}{\pgfqpoint{5.608738in}{2.344453in}}%
\pgfusepath{clip}%
\pgfsetbuttcap%
\pgfsetroundjoin%
\pgfsetlinewidth{0.501875pt}%
\definecolor{currentstroke}{rgb}{0.698039,0.698039,0.698039}%
\pgfsetstrokecolor{currentstroke}%
\pgfsetdash{{1.850000pt}{0.800000pt}}{0.000000pt}%
\pgfpathmoveto{\pgfqpoint{5.525192in}{0.429287in}}%
\pgfpathlineto{\pgfqpoint{5.525192in}{2.773739in}}%
\pgfusepath{stroke}%
\end{pgfscope}%
\begin{pgfscope}%
\pgfsetbuttcap%
\pgfsetroundjoin%
\definecolor{currentfill}{rgb}{0.180392,0.180392,0.180392}%
\pgfsetfillcolor{currentfill}%
\pgfsetlinewidth{0.803000pt}%
\definecolor{currentstroke}{rgb}{0.180392,0.180392,0.180392}%
\pgfsetstrokecolor{currentstroke}%
\pgfsetdash{}{0pt}%
\pgfsys@defobject{currentmarker}{\pgfqpoint{0.000000in}{-0.048611in}}{\pgfqpoint{0.000000in}{0.000000in}}{%
\pgfpathmoveto{\pgfqpoint{0.000000in}{0.000000in}}%
\pgfpathlineto{\pgfqpoint{0.000000in}{-0.048611in}}%
\pgfusepath{stroke,fill}%
}%
\begin{pgfscope}%
\pgfsys@transformshift{5.525192in}{0.429287in}%
\pgfsys@useobject{currentmarker}{}%
\end{pgfscope}%
\end{pgfscope}%
\begin{pgfscope}%
\definecolor{textcolor}{rgb}{0.180392,0.180392,0.180392}%
\pgfsetstrokecolor{textcolor}%
\pgfsetfillcolor{textcolor}%
\pgftext[x=5.525192in,y=0.332064in,,top]{\color{textcolor}\rmfamily\fontsize{9.000000}{10.800000}\selectfont \(\displaystyle {40}\)}%
\end{pgfscope}%
\begin{pgfscope}%
\definecolor{textcolor}{rgb}{0.180392,0.180392,0.180392}%
\pgfsetstrokecolor{textcolor}%
\pgfsetfillcolor{textcolor}%
\pgftext[x=3.281697in,y=0.150823in,,top]{\color{textcolor}\rmfamily\fontsize{10.800000}{12.960000}\selectfont Angular Frequency, \(\displaystyle \omega\)}%
\end{pgfscope}%
\begin{pgfscope}%
\pgfpathrectangle{\pgfqpoint{0.477328in}{0.429287in}}{\pgfqpoint{5.608738in}{2.344453in}}%
\pgfusepath{clip}%
\pgfsetbuttcap%
\pgfsetroundjoin%
\pgfsetlinewidth{0.501875pt}%
\definecolor{currentstroke}{rgb}{0.698039,0.698039,0.698039}%
\pgfsetstrokecolor{currentstroke}%
\pgfsetdash{{1.850000pt}{0.800000pt}}{0.000000pt}%
\pgfpathmoveto{\pgfqpoint{0.477328in}{0.535853in}}%
\pgfpathlineto{\pgfqpoint{6.086066in}{0.535853in}}%
\pgfusepath{stroke}%
\end{pgfscope}%
\begin{pgfscope}%
\pgfsetbuttcap%
\pgfsetroundjoin%
\definecolor{currentfill}{rgb}{0.180392,0.180392,0.180392}%
\pgfsetfillcolor{currentfill}%
\pgfsetlinewidth{0.803000pt}%
\definecolor{currentstroke}{rgb}{0.180392,0.180392,0.180392}%
\pgfsetstrokecolor{currentstroke}%
\pgfsetdash{}{0pt}%
\pgfsys@defobject{currentmarker}{\pgfqpoint{-0.048611in}{0.000000in}}{\pgfqpoint{-0.000000in}{0.000000in}}{%
\pgfpathmoveto{\pgfqpoint{-0.000000in}{0.000000in}}%
\pgfpathlineto{\pgfqpoint{-0.048611in}{0.000000in}}%
\pgfusepath{stroke,fill}%
}%
\begin{pgfscope}%
\pgfsys@transformshift{0.477328in}{0.535853in}%
\pgfsys@useobject{currentmarker}{}%
\end{pgfscope}%
\end{pgfscope}%
\begin{pgfscope}%
\definecolor{textcolor}{rgb}{0.180392,0.180392,0.180392}%
\pgfsetstrokecolor{textcolor}%
\pgfsetfillcolor{textcolor}%
\pgftext[x=0.223855in, y=0.490634in, left, base]{\color{textcolor}\rmfamily\fontsize{9.000000}{10.800000}\selectfont \(\displaystyle {0.0}\)}%
\end{pgfscope}%
\begin{pgfscope}%
\pgfpathrectangle{\pgfqpoint{0.477328in}{0.429287in}}{\pgfqpoint{5.608738in}{2.344453in}}%
\pgfusepath{clip}%
\pgfsetbuttcap%
\pgfsetroundjoin%
\pgfsetlinewidth{0.501875pt}%
\definecolor{currentstroke}{rgb}{0.698039,0.698039,0.698039}%
\pgfsetstrokecolor{currentstroke}%
\pgfsetdash{{1.850000pt}{0.800000pt}}{0.000000pt}%
\pgfpathmoveto{\pgfqpoint{0.477328in}{0.962117in}}%
\pgfpathlineto{\pgfqpoint{6.086066in}{0.962117in}}%
\pgfusepath{stroke}%
\end{pgfscope}%
\begin{pgfscope}%
\pgfsetbuttcap%
\pgfsetroundjoin%
\definecolor{currentfill}{rgb}{0.180392,0.180392,0.180392}%
\pgfsetfillcolor{currentfill}%
\pgfsetlinewidth{0.803000pt}%
\definecolor{currentstroke}{rgb}{0.180392,0.180392,0.180392}%
\pgfsetstrokecolor{currentstroke}%
\pgfsetdash{}{0pt}%
\pgfsys@defobject{currentmarker}{\pgfqpoint{-0.048611in}{0.000000in}}{\pgfqpoint{-0.000000in}{0.000000in}}{%
\pgfpathmoveto{\pgfqpoint{-0.000000in}{0.000000in}}%
\pgfpathlineto{\pgfqpoint{-0.048611in}{0.000000in}}%
\pgfusepath{stroke,fill}%
}%
\begin{pgfscope}%
\pgfsys@transformshift{0.477328in}{0.962117in}%
\pgfsys@useobject{currentmarker}{}%
\end{pgfscope}%
\end{pgfscope}%
\begin{pgfscope}%
\definecolor{textcolor}{rgb}{0.180392,0.180392,0.180392}%
\pgfsetstrokecolor{textcolor}%
\pgfsetfillcolor{textcolor}%
\pgftext[x=0.223855in, y=0.916898in, left, base]{\color{textcolor}\rmfamily\fontsize{9.000000}{10.800000}\selectfont \(\displaystyle {0.2}\)}%
\end{pgfscope}%
\begin{pgfscope}%
\pgfpathrectangle{\pgfqpoint{0.477328in}{0.429287in}}{\pgfqpoint{5.608738in}{2.344453in}}%
\pgfusepath{clip}%
\pgfsetbuttcap%
\pgfsetroundjoin%
\pgfsetlinewidth{0.501875pt}%
\definecolor{currentstroke}{rgb}{0.698039,0.698039,0.698039}%
\pgfsetstrokecolor{currentstroke}%
\pgfsetdash{{1.850000pt}{0.800000pt}}{0.000000pt}%
\pgfpathmoveto{\pgfqpoint{0.477328in}{1.388381in}}%
\pgfpathlineto{\pgfqpoint{6.086066in}{1.388381in}}%
\pgfusepath{stroke}%
\end{pgfscope}%
\begin{pgfscope}%
\pgfsetbuttcap%
\pgfsetroundjoin%
\definecolor{currentfill}{rgb}{0.180392,0.180392,0.180392}%
\pgfsetfillcolor{currentfill}%
\pgfsetlinewidth{0.803000pt}%
\definecolor{currentstroke}{rgb}{0.180392,0.180392,0.180392}%
\pgfsetstrokecolor{currentstroke}%
\pgfsetdash{}{0pt}%
\pgfsys@defobject{currentmarker}{\pgfqpoint{-0.048611in}{0.000000in}}{\pgfqpoint{-0.000000in}{0.000000in}}{%
\pgfpathmoveto{\pgfqpoint{-0.000000in}{0.000000in}}%
\pgfpathlineto{\pgfqpoint{-0.048611in}{0.000000in}}%
\pgfusepath{stroke,fill}%
}%
\begin{pgfscope}%
\pgfsys@transformshift{0.477328in}{1.388381in}%
\pgfsys@useobject{currentmarker}{}%
\end{pgfscope}%
\end{pgfscope}%
\begin{pgfscope}%
\definecolor{textcolor}{rgb}{0.180392,0.180392,0.180392}%
\pgfsetstrokecolor{textcolor}%
\pgfsetfillcolor{textcolor}%
\pgftext[x=0.223855in, y=1.343163in, left, base]{\color{textcolor}\rmfamily\fontsize{9.000000}{10.800000}\selectfont \(\displaystyle {0.4}\)}%
\end{pgfscope}%
\begin{pgfscope}%
\pgfpathrectangle{\pgfqpoint{0.477328in}{0.429287in}}{\pgfqpoint{5.608738in}{2.344453in}}%
\pgfusepath{clip}%
\pgfsetbuttcap%
\pgfsetroundjoin%
\pgfsetlinewidth{0.501875pt}%
\definecolor{currentstroke}{rgb}{0.698039,0.698039,0.698039}%
\pgfsetstrokecolor{currentstroke}%
\pgfsetdash{{1.850000pt}{0.800000pt}}{0.000000pt}%
\pgfpathmoveto{\pgfqpoint{0.477328in}{1.814645in}}%
\pgfpathlineto{\pgfqpoint{6.086066in}{1.814645in}}%
\pgfusepath{stroke}%
\end{pgfscope}%
\begin{pgfscope}%
\pgfsetbuttcap%
\pgfsetroundjoin%
\definecolor{currentfill}{rgb}{0.180392,0.180392,0.180392}%
\pgfsetfillcolor{currentfill}%
\pgfsetlinewidth{0.803000pt}%
\definecolor{currentstroke}{rgb}{0.180392,0.180392,0.180392}%
\pgfsetstrokecolor{currentstroke}%
\pgfsetdash{}{0pt}%
\pgfsys@defobject{currentmarker}{\pgfqpoint{-0.048611in}{0.000000in}}{\pgfqpoint{-0.000000in}{0.000000in}}{%
\pgfpathmoveto{\pgfqpoint{-0.000000in}{0.000000in}}%
\pgfpathlineto{\pgfqpoint{-0.048611in}{0.000000in}}%
\pgfusepath{stroke,fill}%
}%
\begin{pgfscope}%
\pgfsys@transformshift{0.477328in}{1.814645in}%
\pgfsys@useobject{currentmarker}{}%
\end{pgfscope}%
\end{pgfscope}%
\begin{pgfscope}%
\definecolor{textcolor}{rgb}{0.180392,0.180392,0.180392}%
\pgfsetstrokecolor{textcolor}%
\pgfsetfillcolor{textcolor}%
\pgftext[x=0.223855in, y=1.769427in, left, base]{\color{textcolor}\rmfamily\fontsize{9.000000}{10.800000}\selectfont \(\displaystyle {0.6}\)}%
\end{pgfscope}%
\begin{pgfscope}%
\pgfpathrectangle{\pgfqpoint{0.477328in}{0.429287in}}{\pgfqpoint{5.608738in}{2.344453in}}%
\pgfusepath{clip}%
\pgfsetbuttcap%
\pgfsetroundjoin%
\pgfsetlinewidth{0.501875pt}%
\definecolor{currentstroke}{rgb}{0.698039,0.698039,0.698039}%
\pgfsetstrokecolor{currentstroke}%
\pgfsetdash{{1.850000pt}{0.800000pt}}{0.000000pt}%
\pgfpathmoveto{\pgfqpoint{0.477328in}{2.240909in}}%
\pgfpathlineto{\pgfqpoint{6.086066in}{2.240909in}}%
\pgfusepath{stroke}%
\end{pgfscope}%
\begin{pgfscope}%
\pgfsetbuttcap%
\pgfsetroundjoin%
\definecolor{currentfill}{rgb}{0.180392,0.180392,0.180392}%
\pgfsetfillcolor{currentfill}%
\pgfsetlinewidth{0.803000pt}%
\definecolor{currentstroke}{rgb}{0.180392,0.180392,0.180392}%
\pgfsetstrokecolor{currentstroke}%
\pgfsetdash{}{0pt}%
\pgfsys@defobject{currentmarker}{\pgfqpoint{-0.048611in}{0.000000in}}{\pgfqpoint{-0.000000in}{0.000000in}}{%
\pgfpathmoveto{\pgfqpoint{-0.000000in}{0.000000in}}%
\pgfpathlineto{\pgfqpoint{-0.048611in}{0.000000in}}%
\pgfusepath{stroke,fill}%
}%
\begin{pgfscope}%
\pgfsys@transformshift{0.477328in}{2.240909in}%
\pgfsys@useobject{currentmarker}{}%
\end{pgfscope}%
\end{pgfscope}%
\begin{pgfscope}%
\definecolor{textcolor}{rgb}{0.180392,0.180392,0.180392}%
\pgfsetstrokecolor{textcolor}%
\pgfsetfillcolor{textcolor}%
\pgftext[x=0.223855in, y=2.195691in, left, base]{\color{textcolor}\rmfamily\fontsize{9.000000}{10.800000}\selectfont \(\displaystyle {0.8}\)}%
\end{pgfscope}%
\begin{pgfscope}%
\pgfpathrectangle{\pgfqpoint{0.477328in}{0.429287in}}{\pgfqpoint{5.608738in}{2.344453in}}%
\pgfusepath{clip}%
\pgfsetbuttcap%
\pgfsetroundjoin%
\pgfsetlinewidth{0.501875pt}%
\definecolor{currentstroke}{rgb}{0.698039,0.698039,0.698039}%
\pgfsetstrokecolor{currentstroke}%
\pgfsetdash{{1.850000pt}{0.800000pt}}{0.000000pt}%
\pgfpathmoveto{\pgfqpoint{0.477328in}{2.667173in}}%
\pgfpathlineto{\pgfqpoint{6.086066in}{2.667173in}}%
\pgfusepath{stroke}%
\end{pgfscope}%
\begin{pgfscope}%
\pgfsetbuttcap%
\pgfsetroundjoin%
\definecolor{currentfill}{rgb}{0.180392,0.180392,0.180392}%
\pgfsetfillcolor{currentfill}%
\pgfsetlinewidth{0.803000pt}%
\definecolor{currentstroke}{rgb}{0.180392,0.180392,0.180392}%
\pgfsetstrokecolor{currentstroke}%
\pgfsetdash{}{0pt}%
\pgfsys@defobject{currentmarker}{\pgfqpoint{-0.048611in}{0.000000in}}{\pgfqpoint{-0.000000in}{0.000000in}}{%
\pgfpathmoveto{\pgfqpoint{-0.000000in}{0.000000in}}%
\pgfpathlineto{\pgfqpoint{-0.048611in}{0.000000in}}%
\pgfusepath{stroke,fill}%
}%
\begin{pgfscope}%
\pgfsys@transformshift{0.477328in}{2.667173in}%
\pgfsys@useobject{currentmarker}{}%
\end{pgfscope}%
\end{pgfscope}%
\begin{pgfscope}%
\definecolor{textcolor}{rgb}{0.180392,0.180392,0.180392}%
\pgfsetstrokecolor{textcolor}%
\pgfsetfillcolor{textcolor}%
\pgftext[x=0.223855in, y=2.621955in, left, base]{\color{textcolor}\rmfamily\fontsize{9.000000}{10.800000}\selectfont \(\displaystyle {1.0}\)}%
\end{pgfscope}%
\begin{pgfscope}%
\definecolor{textcolor}{rgb}{0.180392,0.180392,0.180392}%
\pgfsetstrokecolor{textcolor}%
\pgfsetfillcolor{textcolor}%
\pgftext[x=0.168300in,y=1.601513in,,bottom,rotate=90.000000]{\color{textcolor}\rmfamily\fontsize{10.800000}{12.960000}\selectfont \(\displaystyle \tilde{F}(\omega)\)}%
\end{pgfscope}%
\begin{pgfscope}%
\pgfpathrectangle{\pgfqpoint{0.477328in}{0.429287in}}{\pgfqpoint{5.608738in}{2.344453in}}%
\pgfusepath{clip}%
\pgfsetrectcap%
\pgfsetroundjoin%
\pgfsetlinewidth{2.007500pt}%
\definecolor{currentstroke}{rgb}{0.000000,0.501961,0.000000}%
\pgfsetstrokecolor{currentstroke}%
\pgfsetdash{}{0pt}%
\pgfpathmoveto{\pgfqpoint{2.396273in}{0.535853in}}%
\pgfpathlineto{\pgfqpoint{3.118708in}{0.536631in}}%
\pgfpathlineto{\pgfqpoint{3.127519in}{0.537683in}}%
\pgfpathlineto{\pgfqpoint{3.136329in}{0.539951in}}%
\pgfpathlineto{\pgfqpoint{3.145139in}{0.544593in}}%
\pgfpathlineto{\pgfqpoint{3.153949in}{0.553603in}}%
\pgfpathlineto{\pgfqpoint{3.162759in}{0.570181in}}%
\pgfpathlineto{\pgfqpoint{3.171569in}{0.599078in}}%
\pgfpathlineto{\pgfqpoint{3.180380in}{0.646747in}}%
\pgfpathlineto{\pgfqpoint{3.189190in}{0.721080in}}%
\pgfpathlineto{\pgfqpoint{3.198000in}{0.830488in}}%
\pgfpathlineto{\pgfqpoint{3.206810in}{0.982172in}}%
\pgfpathlineto{\pgfqpoint{3.215620in}{1.179706in}}%
\pgfpathlineto{\pgfqpoint{3.224431in}{1.420376in}}%
\pgfpathlineto{\pgfqpoint{3.250861in}{2.246513in}}%
\pgfpathlineto{\pgfqpoint{3.259671in}{2.468760in}}%
\pgfpathlineto{\pgfqpoint{3.268482in}{2.615738in}}%
\pgfpathlineto{\pgfqpoint{3.277292in}{2.667173in}}%
\pgfpathlineto{\pgfqpoint{3.286102in}{2.615738in}}%
\pgfpathlineto{\pgfqpoint{3.294912in}{2.468760in}}%
\pgfpathlineto{\pgfqpoint{3.303722in}{2.246513in}}%
\pgfpathlineto{\pgfqpoint{3.338963in}{1.179706in}}%
\pgfpathlineto{\pgfqpoint{3.347773in}{0.982172in}}%
\pgfpathlineto{\pgfqpoint{3.356583in}{0.830488in}}%
\pgfpathlineto{\pgfqpoint{3.365394in}{0.721080in}}%
\pgfpathlineto{\pgfqpoint{3.374204in}{0.646747in}}%
\pgfpathlineto{\pgfqpoint{3.383014in}{0.599078in}}%
\pgfpathlineto{\pgfqpoint{3.391824in}{0.570181in}}%
\pgfpathlineto{\pgfqpoint{3.400634in}{0.553603in}}%
\pgfpathlineto{\pgfqpoint{3.409444in}{0.544593in}}%
\pgfpathlineto{\pgfqpoint{3.418255in}{0.539951in}}%
\pgfpathlineto{\pgfqpoint{3.427065in}{0.537683in}}%
\pgfpathlineto{\pgfqpoint{3.444685in}{0.536168in}}%
\pgfpathlineto{\pgfqpoint{3.506357in}{0.535853in}}%
\pgfpathlineto{\pgfqpoint{4.158310in}{0.535853in}}%
\pgfpathlineto{\pgfqpoint{4.158310in}{0.535853in}}%
\pgfusepath{stroke}%
\end{pgfscope}%
\begin{pgfscope}%
\pgfpathrectangle{\pgfqpoint{0.477328in}{0.429287in}}{\pgfqpoint{5.608738in}{2.344453in}}%
\pgfusepath{clip}%
\pgfsetrectcap%
\pgfsetroundjoin%
\pgfsetlinewidth{2.007500pt}%
\definecolor{currentstroke}{rgb}{1.000000,0.647059,0.000000}%
\pgfsetstrokecolor{currentstroke}%
\pgfsetdash{}{0pt}%
\pgfpathmoveto{\pgfqpoint{2.396273in}{0.535853in}}%
\pgfpathlineto{\pgfqpoint{3.118708in}{0.536631in}}%
\pgfpathlineto{\pgfqpoint{3.127519in}{0.537683in}}%
\pgfpathlineto{\pgfqpoint{3.136329in}{0.539951in}}%
\pgfpathlineto{\pgfqpoint{3.145139in}{0.544593in}}%
\pgfpathlineto{\pgfqpoint{3.153949in}{0.553603in}}%
\pgfpathlineto{\pgfqpoint{3.162759in}{0.570181in}}%
\pgfpathlineto{\pgfqpoint{3.171569in}{0.599078in}}%
\pgfpathlineto{\pgfqpoint{3.180380in}{0.646747in}}%
\pgfpathlineto{\pgfqpoint{3.189190in}{0.721080in}}%
\pgfpathlineto{\pgfqpoint{3.198000in}{0.830488in}}%
\pgfpathlineto{\pgfqpoint{3.206810in}{0.982172in}}%
\pgfpathlineto{\pgfqpoint{3.215620in}{1.179706in}}%
\pgfpathlineto{\pgfqpoint{3.224431in}{1.420376in}}%
\pgfpathlineto{\pgfqpoint{3.250861in}{2.246513in}}%
\pgfpathlineto{\pgfqpoint{3.259671in}{2.468760in}}%
\pgfpathlineto{\pgfqpoint{3.268482in}{2.615738in}}%
\pgfpathlineto{\pgfqpoint{3.277292in}{2.667173in}}%
\pgfpathlineto{\pgfqpoint{3.286102in}{2.615738in}}%
\pgfpathlineto{\pgfqpoint{3.294912in}{2.468760in}}%
\pgfpathlineto{\pgfqpoint{3.303722in}{2.246513in}}%
\pgfpathlineto{\pgfqpoint{3.338963in}{1.179706in}}%
\pgfpathlineto{\pgfqpoint{3.347773in}{0.982172in}}%
\pgfpathlineto{\pgfqpoint{3.356583in}{0.830488in}}%
\pgfpathlineto{\pgfqpoint{3.365394in}{0.721080in}}%
\pgfpathlineto{\pgfqpoint{3.374204in}{0.646747in}}%
\pgfpathlineto{\pgfqpoint{3.383014in}{0.599078in}}%
\pgfpathlineto{\pgfqpoint{3.391824in}{0.570181in}}%
\pgfpathlineto{\pgfqpoint{3.400634in}{0.553603in}}%
\pgfpathlineto{\pgfqpoint{3.409444in}{0.544593in}}%
\pgfpathlineto{\pgfqpoint{3.418255in}{0.539951in}}%
\pgfpathlineto{\pgfqpoint{3.427065in}{0.537683in}}%
\pgfpathlineto{\pgfqpoint{3.444685in}{0.536168in}}%
\pgfpathlineto{\pgfqpoint{3.506357in}{0.535853in}}%
\pgfpathlineto{\pgfqpoint{4.158310in}{0.535853in}}%
\pgfpathlineto{\pgfqpoint{4.158310in}{0.535853in}}%
\pgfusepath{stroke}%
\end{pgfscope}%
\begin{pgfscope}%
\pgfsetrectcap%
\pgfsetmiterjoin%
\pgfsetlinewidth{0.803000pt}%
\definecolor{currentstroke}{rgb}{0.737255,0.737255,0.737255}%
\pgfsetstrokecolor{currentstroke}%
\pgfsetdash{}{0pt}%
\pgfpathmoveto{\pgfqpoint{0.477328in}{0.429287in}}%
\pgfpathlineto{\pgfqpoint{0.477328in}{2.773739in}}%
\pgfusepath{stroke}%
\end{pgfscope}%
\begin{pgfscope}%
\pgfsetrectcap%
\pgfsetmiterjoin%
\pgfsetlinewidth{0.803000pt}%
\definecolor{currentstroke}{rgb}{0.737255,0.737255,0.737255}%
\pgfsetstrokecolor{currentstroke}%
\pgfsetdash{}{0pt}%
\pgfpathmoveto{\pgfqpoint{6.086066in}{0.429287in}}%
\pgfpathlineto{\pgfqpoint{6.086066in}{2.773739in}}%
\pgfusepath{stroke}%
\end{pgfscope}%
\begin{pgfscope}%
\pgfsetrectcap%
\pgfsetmiterjoin%
\pgfsetlinewidth{0.803000pt}%
\definecolor{currentstroke}{rgb}{0.737255,0.737255,0.737255}%
\pgfsetstrokecolor{currentstroke}%
\pgfsetdash{}{0pt}%
\pgfpathmoveto{\pgfqpoint{0.477328in}{0.429287in}}%
\pgfpathlineto{\pgfqpoint{6.086066in}{0.429287in}}%
\pgfusepath{stroke}%
\end{pgfscope}%
\begin{pgfscope}%
\pgfsetrectcap%
\pgfsetmiterjoin%
\pgfsetlinewidth{0.803000pt}%
\definecolor{currentstroke}{rgb}{0.737255,0.737255,0.737255}%
\pgfsetstrokecolor{currentstroke}%
\pgfsetdash{}{0pt}%
\pgfpathmoveto{\pgfqpoint{0.477328in}{2.773739in}}%
\pgfpathlineto{\pgfqpoint{6.086066in}{2.773739in}}%
\pgfusepath{stroke}%
\end{pgfscope}%
\begin{pgfscope}%
\definecolor{textcolor}{rgb}{0.180392,0.180392,0.180392}%
\pgfsetstrokecolor{textcolor}%
\pgfsetfillcolor{textcolor}%
\pgftext[x=3.281697in,y=2.857072in,,base]{\color{textcolor}\rmfamily\fontsize{12.960000}{15.552000}\selectfont Input and Output Envelope - Frequency Domain}%
\end{pgfscope}%
\begin{pgfscope}%
\pgfsetbuttcap%
\pgfsetmiterjoin%
\definecolor{currentfill}{rgb}{0.933333,0.933333,0.933333}%
\pgfsetfillcolor{currentfill}%
\pgfsetfillopacity{0.800000}%
\pgfsetlinewidth{0.501875pt}%
\definecolor{currentstroke}{rgb}{0.800000,0.800000,0.800000}%
\pgfsetstrokecolor{currentstroke}%
\pgfsetstrokeopacity{0.800000}%
\pgfsetdash{}{0pt}%
\pgfpathmoveto{\pgfqpoint{0.564828in}{2.297367in}}%
\pgfpathlineto{\pgfqpoint{1.854315in}{2.297367in}}%
\pgfpathquadraticcurveto{\pgfqpoint{1.879315in}{2.297367in}}{\pgfqpoint{1.879315in}{2.322367in}}%
\pgfpathlineto{\pgfqpoint{1.879315in}{2.686239in}}%
\pgfpathquadraticcurveto{\pgfqpoint{1.879315in}{2.711239in}}{\pgfqpoint{1.854315in}{2.711239in}}%
\pgfpathlineto{\pgfqpoint{0.564828in}{2.711239in}}%
\pgfpathquadraticcurveto{\pgfqpoint{0.539828in}{2.711239in}}{\pgfqpoint{0.539828in}{2.686239in}}%
\pgfpathlineto{\pgfqpoint{0.539828in}{2.322367in}}%
\pgfpathquadraticcurveto{\pgfqpoint{0.539828in}{2.297367in}}{\pgfqpoint{0.564828in}{2.297367in}}%
\pgfpathlineto{\pgfqpoint{0.564828in}{2.297367in}}%
\pgfpathclose%
\pgfusepath{stroke,fill}%
\end{pgfscope}%
\begin{pgfscope}%
\pgfsetrectcap%
\pgfsetroundjoin%
\pgfsetlinewidth{2.007500pt}%
\definecolor{currentstroke}{rgb}{0.000000,0.501961,0.000000}%
\pgfsetstrokecolor{currentstroke}%
\pgfsetdash{}{0pt}%
\pgfpathmoveto{\pgfqpoint{0.589828in}{2.614553in}}%
\pgfpathlineto{\pgfqpoint{0.714828in}{2.614553in}}%
\pgfpathlineto{\pgfqpoint{0.839828in}{2.614553in}}%
\pgfusepath{stroke}%
\end{pgfscope}%
\begin{pgfscope}%
\definecolor{textcolor}{rgb}{0.180392,0.180392,0.180392}%
\pgfsetstrokecolor{textcolor}%
\pgfsetfillcolor{textcolor}%
\pgftext[x=0.939828in,y=2.570803in,left,base]{\color{textcolor}\rmfamily\fontsize{9.000000}{10.800000}\selectfont Initial Spectrum}%
\end{pgfscope}%
\begin{pgfscope}%
\pgfsetrectcap%
\pgfsetroundjoin%
\pgfsetlinewidth{2.007500pt}%
\definecolor{currentstroke}{rgb}{1.000000,0.647059,0.000000}%
\pgfsetstrokecolor{currentstroke}%
\pgfsetdash{}{0pt}%
\pgfpathmoveto{\pgfqpoint{0.589828in}{2.426367in}}%
\pgfpathlineto{\pgfqpoint{0.714828in}{2.426367in}}%
\pgfpathlineto{\pgfqpoint{0.839828in}{2.426367in}}%
\pgfusepath{stroke}%
\end{pgfscope}%
\begin{pgfscope}%
\definecolor{textcolor}{rgb}{0.180392,0.180392,0.180392}%
\pgfsetstrokecolor{textcolor}%
\pgfsetfillcolor{textcolor}%
\pgftext[x=0.939828in,y=2.382617in,left,base]{\color{textcolor}\rmfamily\fontsize{9.000000}{10.800000}\selectfont Final Spectrum}%
\end{pgfscope}%
\end{pgfpicture}%
\makeatother%
\endgroup%
}
			\caption{\centering{Split-Step Fourier Method Introduction - 2D Graph of the Initial and Final Envelope in the Frequency Domain}}
			\label{fig:split-step-intro-2d-freq}
		\end{figure}	
		
		\pagebreak
		
		Then finally we have the two 3D plots of the propagation in both the time domain and frequency domain, Figures \ref{fig:split-step-intro-3d-time} and \ref{fig:split-step-intro-3d-freq}: \linebreak
				
		\begin{figure}[h]
			\centering
			\scalebox{0.7}{\input{./Graphs/split-step-intro-3d-time.pgf}}
			\caption{\centering{Split-Step Fourier Method Introduction - 3D Propagation Graph Envelope in the Time Domain}}
			\label{fig:split-step-intro-3d-time}
		\end{figure}	
		
		\vspace{5mm}
		
		\begin{figure}[h]
			\centering
			\scalebox{0.7}{\input{./Graphs/split-step-intro-3d-freq.pgf}}
			\caption{\centering{Split-Step Fourier Method Introduction - 3D Propagation Graph Envelope in the Frequency Domain}}
			\label{fig:split-step-intro-3d-freq}
		\end{figure}	
		
		
		\pagebreak
		
		
		\subsection{Split-Step Fourier Method in Python - Introduction} \label{code:Split_step_fourier_intro}
			\lstinputlisting{./Python-Opt-Comm-Net/Split-Step-Fourier-Method-Intro.py}
		
		
		\pagebreak
		
		
\section{Group Velocity Dispersion, GVD}

Also known as Dispersion Induced Pulse Broadening, the effects of GVD on optical pulses propagating in an optical fibre are studied by setting the non-linear term to zero, $\gamma = 0$, $F(z, t)$ then satisfies Equation \ref{eq:gvd_satisfaction}:

	\begin{equation} \label{eq:gvd_satisfaction}
		\mathlarger{ j \frac{\partial F(z, t)}{\partial z} \approx \frac{ \beta_2}{2} \frac{\partial^2 F}{\partial t^2} }
	\end{equation}
	
	{\footnotesize Be careful, $\beta_2 \neq \beta ''$} \linebreak	
	
	\vspace{5mm}

Equation \ref{eq:gvd_satisfaction} is trivially solved using the Fourier Transform, if $\tilde{F}(z, \omega$ is the signal $F(z, t)$ in frequency domain, such that it satisfies Equation \ref{eq:gvd_satisfaction_fourier}, then it also satisfies Equation \ref{eq:ode_gvd_satisfaction}:
	
	\begin{equation} \label{eq:gvd_satisfaction_fourier}
		\mathlarger{F(z, t) = \frac{1}{2 \pi} \int_{-\infty}^{+\infty} \tilde{F}(z, \omega) e^{-j \omega t} d \omega}
	\end{equation}
	
	\vspace{5mm}

	\begin{equation} \label{eq:ode_gvd_satisfaction}
		\mathlarger{ j \frac{\partial F(z, t)}{\partial z} = \frac{1}{2} \beta_2 \omega^2 \tilde{F} }
	\end{equation}
	
	\vspace{5mm}
	
The solution to this ODE is given by Equation \ref{eq:gvd_ode_solution}, it is clear from this solution that each spectral component of the signal will have it's phase changed differently for a given propagation distance, $z$, and frequency, $\omega$.
	
	\begin{equation} \label{eq:gvd_ode_solution}
		\mathlarger{ \tilde{F}(z, \omega) = \tilde{F}(z=0, \omega) e^{j \frac{1}{2} \beta_2 \omega^2 z} }
	\end{equation}

	\vspace{5mm}

If one substitutes Equation \ref{eq:gvd_ode_solution} into Equation \ref{eq:gvd_satisfaction_fourier} the general solution to Equation \ref{eq:gvd_satisfaction} is obtained, shown below in Equation \ref{eq:dispersive_solution_general}:

	\begin{equation} \label{eq:dispersive_solution_general}
		\mathlarger{F(z, t) = \frac{1}{2 \pi} \int_{-\infty}^{+\infty} \tilde{F(0, \omega)} e^{-j \frac{1}{2} \beta_2 \omega^2 z - j \omega t} d \omega}
	\end{equation}

	{\footnotesize Where $\tilde{F(0, \omega)}$ is the Fourier Transform of the input signal.} \linebreak	

	\vspace{5mm}
	
	This Equation (\ref{eq:dispersive_solution_general}) can be used for input functions of almost any shape, making this equation rather powerful. \linebreak
	
	
	\subsection{GVD with a Gaussian Pulse Input - Analytical Solution}
	
	We will first analytically consider the propagation of a Gaussian Pulse with GVD, the Gaussian equation was given in Section \ref{sec:detour_gaussian} as Equation \ref{eq:gaussian_time_def}, repeated again the Gaussian is defined here as:
	
	\begin{equation*}
		\mathlarger{ \mathlarger{ f(t) = I e^{-\frac{t^2}{2t^2_0}} } }
	\end{equation*}
	
	Where $t_0$ is the half-width at $\frac{1}{e}$ intensity, we could also use the Full Width at Half Maximum (FWHM), we can define for the FWHM:
	
	\begin{gather*}
		\mathlarger{ T_{FWHM} = 1.665 t_0 = 2 \sqrt{ln(2)} t_0 }
	\end{gather*}
	
	\vspace{5mm}
	
	Now, if we combine Equations \ref{eq:dispersive_solution_general} and \ref{eq:gaussian_time_def} then perform the integration, we will obtain Equation \ref{eq:gauss_in_dispersive_solution}:
	
	\begin{equation} \label{eq:gauss_in_dispersive_solution}
		\mathlarger{F(z, t) = \frac{t_0}{(t_0^2 - j \beta_2 z)^{\frac{1}{2}}} e^{- \frac{t^2}{2(t_0^2 - j \beta_2 z)}} }
	\end{equation}
	
	\vspace{5mm}
	
	This equation displays that a Gaussian signal will clearly maintain it's shape upon propagation, however it's intensity will lessen, and it's width in time will increase, we can define the equation for the time domain change after a given propagation distance as Equation \ref{eq:waist_change_gauss_with_dispersion}:
	
		\begin{equation} \label{eq:waist_change_gauss_with_dispersion}
			\mathlarger{T_1(z) = t_0(1 + (\frac{z}{L_D})^2)^{\frac{1}{2}} }
		\end{equation}
		
	With $L_D = \frac{T_0^2}{|\beta_2|}$ and it is usually called the ``Dispersion Length''. This value defines how, for a given fibre length, shorter pulses will broaden more than longer ones as the dispersion length will be larger. In other words the dispersion length governs the extent of broadening. At a propagation distance equal to the dispersion length, $z = L_D$, a Gaussian will broaden by $\sqrt{2}$. \linebreak
		
	Comparing our Gaussian input, and calculated output after propagation (Equations \ref{eq:gaussian_time_def} and \ref{eq:gauss_in_dispersive_solution}) indicates that even though our input is un-chirped (with no phase modulation), we have a chirped output, this is probably easier to observe if we define our propagation in different terms, shown it Equations \ref{eq:waist_change_gauss_using_phi} and \ref{eq:waist_change_gauss_phi_definition}:
		
		\begin{equation} \label{eq:waist_change_gauss_using_phi}
			\mathlarger{ F(z, t) = |F(z, t)| e^{j \phi(z,t)} }
		\end{equation}
			
		\vspace{5mm}
			
		\begin{equation} \label{eq:waist_change_gauss_phi_definition}
			\mathlarger{ \phi(z,t)  = \frac{-sin(\beta_2)(\frac{z}{L_D})}{1 + (\frac{z}{L_D})^2} \frac{t^2}{t_0^2} + \frac{1}{2} tan^{-2}(\frac{z}{L_D}) }
		\end{equation}
			
		\vspace{5mm}
			
	As the phase changes with respect to time, the change in the frequency domain of the signal due to dispersion after propagation will be different for a given frequency. This difference can be obtained by differentiating with respect to time, given in Equation \ref{eq:waist_change_gauss_phi_wrt_time}, $\partial \omega$ is the chirp and clearly depends on the sign and value of $\beta_2$:
						
		\begin{equation} \label{eq:waist_change_gauss_phi_wrt_time}
			\mathlarger{ \partial \omega (t) = -\frac{\partial \phi(z,t)}{\partial t}  = \frac{-sin(\beta_2)(\frac{2z}{L_D}) t}{1 + (\frac{z}{L_D})^2 t_0^2} }
		\end{equation}
			
		\vspace{5mm}
		
	Dispersion-Induced pulse broadening is an important phenomena to understand, there is also some unmentioned (thus far) terminology attached to it; when the dispersive term is greater than zero, $\beta_2 > 0$, long wavelengths (e.g. red light) travel faster than short wavelengths (e.g. blue light) and we term this type of dispersion normal, whilst the opposite ($\beta_2 < 0$) is termed anomalous dispersion. \linebreak
	
	
	
	
	\subsection{GVD with a Gaussian Pulse Input - Numerical Simulation}
	



	\pagebreak




	\subsection{GVD with a Chirped Gaussian Pulse Input - Analytical Solution}
	
	
	\subsection{GVD with a Chirped Gaussian Pulse Input - Numerical Simulation}




	\subsection{GVD with Hyperbolic-Secant Pulse Input - Analytical Solution}
	
	
	\subsection{GVD with Hyperbolic-Secant Pulse Input - Numerical Simulation}




	\subsection{GVD Induced Limitations - Dispersion Management}




\section{Self Phase Modulation, SPM}



	\subsection{Changes in Pulse Spectra due to SPM}




\section{Competition of SPM and GVD}







\newpage
\setstretch{1}  % Reduce bibliography line spacing
\bibliographystyle{IEEEtran}
\bibliography{references.bib}
\end{document}
