\documentclass[colorlinks,11pt,a4paper,normalphoto,withhyper,ragged2e]{altareport}


%%%%%%%%%%%%%%%%%%%%%%%%%%%%%%%%%%%%%%%%%%%%%%%%%%%%%%%%%%%%%%%%%%%%%%%%%%%%%%%%%%%%%%%%%%%%%%%%%%%%%%%%%%%%%%%%%%%%%%%%%%%%%%%%%%%%%%%%%%%%%%%%%%%%%%%%%%%%%%%%%%%%%%%%%%%%
%%%%%%%%%% DEFAULT PACKAGES & SETTINGS %%%%%%%%%%
\usepackage{setspace} %1.5 line spacing
\usepackage{notoccite} %% Citation numbering
\usepackage{lscape} %% Landscape table
\usepackage{caption} %% Adds a newline in the table caption
\usepackage[rgb]{xcolor}

%% The paracol package lets you typeset columns of text in parallel
\usepackage{paracol}
\usepackage[none]{hyphenat}

%%% Document/Theme Fonts, Space and Text Settings
\usepackage{fontspec}
\setmainfont{Roboto Slab}
\setsansfont{Lato}
\renewcommand{\familydefault}{\sfdefault}
\captionsetup{font=footnotesize} % Make Captions a sensible size
\setlength{\intextsep}{4pt} % Set defualt spacing around floats
% \captionsetup{aboveskip=5pt, belowskip=5pt} % Reduce space around captions
\geometry{left=1.25cm,right=1.25cm,top=2.5cm,bottom=2.5cm,columnsep=8mm} % Change the page layout
\setstretch{1.5}   % 1.5 line spacing
\definecolor{CommentGreen}{HTML}{228B22}
\justifying

%% Math Env Text Settings
\usepackage{mathtools}
\usepackage{unicode-math}
\setmathfont{XITS Math}
\usepackage{amsmath}
\usepackage{bm}
\everymath=\expandafter{\the\everymath\displaystyle}
%%%%%%%%%%%%%%%%%%%%%%%%%%%%%%%%%%%%%%%%%%%%%%%%%%%%%%%%%%%%%%%%%%%%%%%%%%%%%%%%%%%%%%%%%%%%%%%%%%%%%%%%%%%%%%%%%%%%%%%%%%%%%%%%%%%%%%%%%%%%%%%%%%%%%%%%%%%%%%%%%%%%%%%%%%%%


%%%%%%%%%%%%%%%%%%%%%%%%%%%%%%%%%%%%%%%%%%%%%%%%%%%%%%%%%%%%%%%%%%%%%%%%%%%%%%%%%%%%%%%%%%%%%%%%%%%%%%%%%%%%%%%%%%%%%%%%%%%%%%%%%%%%%%%%%%%%%%%%%%%%%%%%%%%%%%%%%%%%%%%%%%%%
%%%%%%%%%% DOCUMENT SPECIFIC PACKAGES AND SETTINGS %%%%%%%%%%
\usepackage{relsize}

\usepackage{pythontex} % Run python code in this latex doc

%%%%% Settings for python pgf graphs %%%%%
\usepackage{pgfplots}
\usetikzlibrary{arrows.meta}

\pgfplotsset{compat=newest,
    width=6cm,
    height=3cm,
    scale only axis=true,
    max space between ticks=25pt,
    try min ticks=5,
    every axis/.style={
        axis y line=left,
        axis x line=bottom,
        axis line style={thick,->,>=latex, shorten >=-.4cm}
    },
    every axis plot/.append style={thick},
    tick style={black, thick}
}
\tikzset{
    semithick/.style={line width=0.8pt},
}

\usepgfplotslibrary{groupplots}
\usepgfplotslibrary{dateplot}
%%%%%%%%%%%%%%%%%%%%%%%%%%%%%%%%%%%%%%%%%%%%%%%%%%%%%%%%%%%%%%%%%%%%%%%%%%%%%%%%%%%%%%%%%%%%%%%%%%%%%%%%%%%%%%%%%%%%%%%%%%%%%%%%%%%%%%%%%%%%%%%%%%%%%%%%%%%%%%%%%%%%%%%%%%%%


%%%%%%%%%%%%%%%%%%%%%%%%%%%%%%%%%%%%%%%%%%%%%%%%%%%%%%%%%%%%%%%%%%%%%%%%%%%%%%%%%%%%%%%%%%%%%%%%%%%%%%%%%%%%%%%%%%%%%%%%%%%%%%%%%%%%%%%%%%%%%%%%%%%%%%%%%%%%%%%%%%%%%%%%%%%%
%%%%%%%%%% USEFUL SETTINGS %%%%%%%%%%
%% Change some font sizes, this will override the defaults
\renewcommand{\ReportTitleFont}{\Huge\rmfamily\bfseries} %% Title Page - Main Title
\renewcommand{\ReportSubTitleFont}{\huge\bfseries} %% Title Page - Sub-Title
\renewcommand{\ReportSectionFont}{\LARGE\rmfamily\bfseries} %% Section Title
\renewcommand{\ReportSubSectionFont}{\large\bfseries} %% SubSection Title
\renewcommand{\FootNoteFont}{\footnotesize} %% Footnotes and Header/Footer
%%%%%%%%%%%%%%%%%%%%%%%%%%%%%%%%%%%%%%%%%%%%%%%%%%%%%%%%%%%%%%%%%%%%%%%%%%%%%%%%%%%%%%%%%%%%%%%%%%%%%%%%%%%%%%%%%%%%%%%%%%%%%%%%%%%%%%%%%%%%%%%%%%%%%%%%%%%%%%%%%%%%%%%%%%%%


%%%%%%%%%%%%%%%%%%%%%%%%%%%%%%%%%%%%%%%%%%%%%%%%%%%%%%%%%%%%%%%%%%%%%%%%%%%%%%%%%%%%%%%%%%%%%%%%%%%%%%%%%%%%%%%%%%%%%%%%%%%%%%%%%%%%%%%%%%%%%%%%%%%%%%%%%%%%%%%%%%%%%%%%%%%%
%%%%%%%%%% THEMES %%%%%%%%%%
%% Standard theme options are below, leave blank for B&W / no colours (BoringDefault). Note the theme will be set to default if you enter a non-exsistant theme name.
\SetTheme{UNIBS}
%% UNIBS
%% UNILIM
%% PastelBlue
%% GreenAndGold
%% Purple
%% PastelRed
%% BoringDefault (Leave blank / enter anything not found above)
%%%%%%%%%%%%%%%%%%%%%%%%%%%%%%%%%%%%%%%%%%%%%%%%%%%%%%%%%%%%%%%%%%%%%%%%%%%%%%%%%%%%%%%%%%%%%%%%%%%%%%%%%%%%%%%%%%%%%%%%%%%%%%%%%%%%%%%%%%%%%%%%%%%%%%%%%%%%%%%%%%%%%%%%%%%%


%%%%%%%%%%%%%%%%%%%%%%%%%%%%%%%%%%%%%%%%%%%%%%%%%%%%%%%%%%%%%%%%%%%%%%%%%%%%%%%%%%%%%%%%%%%%%%%%%%%%%%%%%%%%%%%%%%%%%%%%%%%%%%%%%%%%%%%%%%%%%%%%%%%%%%%%%%%%%%%%%%%%%%%%%%%%
%%%%%%%%%% TITLE PAGE INFO %%%%%%%%%%
\ReportTitle{Quantum Technologies}
\SubTitle{Homework Solutions}
\Author{Andrew Simon Wilson}
\ReportDate{\today}
\FacultyOrLocation{EMIMEO Programme}
\ModCoord{Prof. Artoni Maurizio}

%%%%%%%%%%%%%%%%%%%%%%%%%%%%%%%%%%%%%%%%%%%%%%%%%%%%%%%%%%%%%%%%%%%%%%%%%%%%%%%%%%%%%%%%%%%%%%%%%%%%%%%%%%%%%%%%%%%%%%%%%%%%%%%%%%%%%%%%%%%%%%%%%%%%%%%%%%%%%%%%%%%%%%%%%%%%


%%%%%%%%%%%%%%%%%%%%%%%%%%%%%%%%%%%%%%%%%%%%%%%%%%%%%%%%%%%%%%%%%%%%%%%%%%%%%%%%%%%%%%%%%%%%%%%%%%%%%%%%%%%%%%%%%%%%%%%%%%%%%%%%%%%%%%%%%%%%%%%%%%%%%%%%%%%%%%%%%%%%%%%%%%%%
%%%%%%%%%% TEST AREA %%%%%%%%%%
\selectcolormodel{natural}
\usepackage{booktabs}
\usepackage{ninecolors}
\selectcolormodel{rgb}

\usepackage{tabularray}

\UseTblrLibrary{booktabs,siunitx}




%%%% https://tikz.net/blackbody/ %%%%
\usepackage{tikz}
\usetikzlibrary{decorations.pathmorphing,decorations.markings,calc} % for random steps & snake
\usetikzlibrary{arrows.meta} % for arrow size
\tikzset{>=latex} % for LaTeX arrow head
\tikzstyle{radiation}=[-{Latex[length=2,width=1.5]},red!95!black!50,opacity=0.7,very thin,decorate,decoration={snake,amplitude=0.7,segment length=2,post length=2}]




%%%% https://tikz.net/blackbody_plots/ %%%%
% CUSTOM COLORS
% See https://tikz.net/blackbody_color/
\definecolor{1000K}{rgb}{1,0.0337,0}
\definecolor{2000K}{rgb}{1,0.2647,0.0033}
\definecolor{3000K}{rgb}{1,0.4870,0.1411}
\definecolor{4000K}{rgb}{1,0.6636,0.3583}
\definecolor{5000K}{rgb}{1,0.7992,0.6045}
\definecolor{6000K}{rgb}{1,0.9019,0.8473}
\definecolor{8000K}{rgb}{0.7874,0.8187,1}
\definecolor{10000K}{rgb}{0.6268,0.7039,1}
\pgfdeclareverticalshading{rainbow}{100bp}{
  color(0bp)=(red); color(25bp)=(red); color(35bp)=(yellow);
  color(45bp)=(green); color(55bp)=(cyan); color(65bp)=(blue);
  color(75bp)=(violet); color(100bp)=(violet)
}
\colorlet{myred}{red!70!black}
\colorlet{mygreen}{green!70!black}
\colorlet{mydarkgreen}{green!55!black}

% PLANCK & RAYLEIGH-JEANS
% 2hc^2/lambda^5 = 2 * 6.62607015e-34 * 299792458^2
%                = 1.191042972e-16
%    W.m -> kW.nm: 1.191042972e26
%  hc/k lambda T = 6.62607015e-34*299792458/(1.38064852e-23)
%                = 0.01438777378
%         m -> nm: 0.01438777378e9
% 2ckT/lambda^4  = 2 * 299792458 * 1.38064852e-23
%                = 8.278160269e-15
%    W.m -> kW.nm: 8.278160269e18
\pgfmathdeclarefunction{planck}{2}{%
  \pgfmathparse{1.191042972e26/(#1^5)/(exp(0.01439e9/(#1*#2))-1)}%
}
\pgfmathdeclarefunction{rayleighjeans}{2}{%
  \pgfmathparse{8.278160269e18*#2/(#1^4)}%
}
\pgfmathdeclarefunction{wien}{2}{%
  \pgfmathparse{1.191042972e26/(#1^5)*exp(-0.01439e9/(#1*#2))}%
}
\pgfmathdeclarefunction{lampeak}{1}{% % Wien's displacement law
  \pgfmathparse{2.898e6/#1}%
}




%%%% https://tikz.net/photoelectric_effect/ %%%%
% Circuits
\usepackage{circuitikz}
%% Specifications
\ctikzset{bipoles/thickness=1.2}

% Styles
\tikzset{>=latex}

% Tikz Library
\usetikzlibrary{angles,quotes}

% Define Color
\tikzstyle{bigphoton}=[-{Latex[length=8,width=6]},red!95!black!50,opacity=0.85,very thin,decorate,decoration={snake,amplitude=2.8,segment length=8,post length=8}]




%%%% https://tikz.net/function_average/ %%%%
\usepackage{physics}
\usepackage[outline]{contour} % glow around text
\contourlength{1.0pt}

\tikzset{>=latex} % for LaTeX arrow head
\colorlet{myred_}{red!85!black}
\colorlet{myblue_}{blue!80!black}
\colorlet{mydarkred_}{myred_!80!black}
\colorlet{mydarkblue_}{myblue_!60!black}
\tikzstyle{xline}=[myblue_,thick]
\def\tick#1#2{\draw[thick] (#1) ++ (#2:0.09) --++ (#2-180:0.18)}
\tikzstyle{myarr_}=[myblue_!50,-{Latex[length=3,width=2]}]
\def\N{100}
%%%%%%%%%%%%%%%%%%%%%%%%%%%%%%%%%%%%%%%%%%%%%%%%%%%%%%%%%%%%%%%%%%%%%%%%%%%%%%%%%%%%%%%%%%%%%%%%%%%%%%%%%%%%%%%%%%%%%%%%%%%%%%%%%%%%%%%%%%%%%%%%%%%%%%%%%%%%%%%%%%%%%%%%%%%%




\begin{document}

\MakeReportTitlePage


%%%%% CONTENTS %%%%%
\pagenumbering{roman} % Start roman numbering
\setcounter{page}{1}


%%%%%%%%%% YOUR NAME, PROFESSION, PORTRAIT, CONTACT INFO, SOCIAL MEDIA ETC. %%%%%%%%%%
\name{Andrew Simon Wilson, BEng}
\tagline{Post-graduate Master's Student - EMIMEO Programme}

\personalinfo{
  \email{andrew.wilson@protonmail.com}
  \linkedin{andrew-simon-wilson} 
  \github{AS-Wilson}
  \phone{+44 7930 560 383}
}

%% You can add multiple photos on the left or right
% \photoR{3cm}{Images/a-wilson-potrait.jpg}
% \photoL{3cm}{Yacht_High,Suitcase_High}

\section*{Author Details}
\makeauthordetails

%% Table of contents print level -1: part, 0: chapter, 1: section, 2:sub-section, 3:sub-sub-section, etc.
\setcounter{tocdepth}{2} 
\tableofcontents %% Prints a list of all sections based on the above command
%\listoffigures %% Prints a list of all figures in the report
%\listoftables %% Prints a list of all tables in the report




%%%%%%%%%% DOCUMENT CONTENT BEGINS HERE %%%%%%%%%%

%%%%% INTRO %%%%%
\section*{Explanation and Introduction of this Document}
I wrote this document for the students studying Quantum Technologies to have a nice set of notes, and correct reference code and graphs for the module. I hope that it is sufficient for this task and it helps all of your studies. \linebreak
I spent have spent a lot of time developing the template used to make this {\LaTeX} document, I want others to benefit from this work so the source code for this template is available on GitHub \cite{latex_template_github}.
\newpage
\pagenumbering{arabic} % Start document numbering - roman numbering





\section{Homework One}

	\begin{enumerate}[leftmargin=1cm]
		\item In the Bohr’s model of the Hydrogen atom, give an expression for: \\
		
		\begin{enumerate}
			\item The electron speed as a function of the orbit radius. \\
			\item The total energy as a function of the orbit radius. \\
		\end{enumerate}
		
		
		\item How many different photons could be emitted upon a transition from the n=5 down to the fundamental n=1 of an H atom. \\
		
		\begin{enumerate}
			\item Compute the exact frequency for one of the transitions. \\
		\end{enumerate}
		
		
		\item A commercial (green) laser pointer has a max/output power, $P \leq 1mW$, and a beam spot, $w_0 = 1.1mm$, if $\lambda = 532nm$ is the wavelength of the light source, compute the following: \\
		
		\begin{enumerate}
			\item The pointer photon’s flux; and,   \\
			\item The number of photons emitted in 10 sec when you purposely cover half of the exit hole with your finger. (Neglect divergence). \\
		\end{enumerate}
		
		
		\item
		\begin{enumerate}
			\item Give the potential energy of a charge, $q_2$, located at point, $r_2$, due to the presence of a charge, $q_1$, located at a point, $r_1$. \\
			\item Derive the force exerted on the charge, $q_2$, as due to the charge, $q_1$. \\
			\item Discuss the two charge possibilities. \\
			\item Apply the above results to the case of an electron placed at a distance, $r$, from a nucleus of a Hydrogen atom (Bohr’s model). \\
			\item Redo part (d.) for a nucleus of charge $Z_e$. \\
		\end{enumerate}
		
		
		 \item[\textbf{Extra:}] In a photoelectric experiment $Ca$ is used as photo-cathode and the following values of stopping potential, $V_s$, vs wavelength, $\lambda$, are measured, using these values calculate the Planck constant, $\hbar$, and the work function, $\Phi$.\\
		 
	\end{enumerate}
	
	TODO - add table 



	\pagebreak
	
	
	
%\section{Homework Two}
%
%- [ ]  5. Refraction (Snell’s law) is commonly understood within the context of classical electrodynamics in terns of “waves”.
%    - [ ]  Should one adopt a description of electrodynamics in terms of “particles”, could such refraction still be described in terms of “photons”?
%    - [ ]  Explain your reasoning.
%- [ ]  6. a. Compute the De Broglie wavelength ($\lambda_{DB}$) associated with a molecule of air (~$N_2$) in your room (STP). Compute the average distance ($d$) between molecules (STP) and compare $d$ and $\lambda_{DB}$.
%    - [ ]  b. Is this any different from the case discussed in class for a conduction band electron (STP)? Briefly explain.
%- [ ]  7. In a single-photon double-slit experiment the (single-photon “sp”) source has been generated via a time-correlated photon parametric down-conversion process whereby a photon (pump) of energy $\hbar \omega_p$ is converted into a “pair” of photon each having energy $\hbar \omega_{sp}$.
%    - [ ]  a. If $\lambda_p = 335nm$ is the pump photon wavelength, compute the photon $\lambda_{sp}$ wavelength of the down-converted single-photon impinging onto the double slit.
%    - [ ]  b. In a real experiment we need to place a “lens” between the image plane of the two slits and the far-field screen where the interference pattern is generated. Draw a scheme of the double slit interferometer showing all its components.
%    - [ ]  c. Derive an expression for the distance between the fringes in terms of the wavelength, $\lambda_{sp}$, the lens focal length, $f$, and the distance, $d$, between the two slits.
%    - [ ]  d. Assuming that the between centres distance between the two slits is $d \approx 500\mu m$, the focal length $f \approx 500mm$, “estimate” the fringes distance. Compare your result with the observed fringe spacing that you may infer from the data collected (photon by photon) in the attached video.
%




	\pagebreak
	
	
	
	
	
%## Homework Three
%
%- [ ]  8. a. Suppose a particle is moving freely inside a carbon nanotube. If the particle if moving with a given momentum, $p$ (eigenvalue), find its quantum state (eigenfunction) by directly solving the corresponding eigenvalue equation for the momentum operator, given by:
%    
%    $$
%    \hat{p} = -i\hbar \hat{\nabla}
%    $$
%    
%    - [ ]  b. Show that these particle states are orthogonal (in the Dirac sense)
%    - [ ]  c. Show that such states are also complete with respect to any square integrable function.
%    - [ ]  d. Apply part (8. c.) to a Gaussian function of width and centred at the origin and one centred at some position, $x = a$
%    
%    ![Untitled](https://s3-us-west-2.amazonaws.com/secure.notion-static.com/7af4a034-6139-48ea-96e2-70ecaf9a31df/Untitled.png)
%    
%- [ ]  9. a. For an electron moving inside a tiny nanotube (1D) give the general solution of the Schrödinger equation.
%    - [ ]  b. Discuss (briefly) whether such a solution is acceptable. In the negative what would you foresee as a way out?
%- [ ]  10. Work out eigenfunctions and eigenvalues of a free electron inside a box of sides $\{a,b,c\}$, with impenetrable walls using the method of separation of variables for (partial) differential equations.
%- [ ]  11. Find under what condition the solutions of the Schrödinger equation, shown below, can be factorised as $ψ(x,t) = ψ(x) T(t)$. Determine the form of $T(t)$.
	
	
	
	\pagebreak
	
	
	
\section{Homework Four}
	
	\subsection{•}
	
	
	\pagebreak
	
	
	
	
\section{Homework Five}
	
	\subsection{•}
	
	
	\pagebreak
	
	
	
	
\section{Homework Six}
	
	\subsection{•}
	
	
	\pagebreak
	
	
	
	
\section{Homework Seven}
	
	\subsection{•}
	
	
	\pagebreak
	
	
	
	
	
	
	
	
\end{document}